\documentclass[12pt]{article}
\usepackage{pmmeta}
\pmcanonicalname{PropertiesOfTheLegendreSymbol}
\pmcreated{2013-03-22 16:17:52}
\pmmodified{2013-03-22 16:17:52}
\pmowner{alozano}{2414}
\pmmodifier{alozano}{2414}
\pmtitle{properties of the Legendre symbol}
\pmrecord{4}{38418}
\pmprivacy{1}
\pmauthor{alozano}{2414}
\pmtype{Theorem}
\pmcomment{trigger rebuild}
\pmclassification{msc}{11-00}
\pmrelated{EulersCriterion}

\endmetadata

% this is the default PlanetMath preamble.  as your knowledge
% of TeX increases, you will probably want to edit this, but
% it should be fine as is for beginners.

% almost certainly you want these
\usepackage{amssymb}
\usepackage{amsmath}
\usepackage{amsthm}
\usepackage{amsfonts}

% used for TeXing text within eps files
%\usepackage{psfrag}
% need this for including graphics (\includegraphics)
%\usepackage{graphicx}
% for neatly defining theorems and propositions
%\usepackage{amsthm}
% making logically defined graphics
%%%\usepackage{xypic}

% there are many more packages, add them here as you need them

% define commands here

\newtheorem{thm}{Theorem}
\newtheorem{defn}{Definition}
\newtheorem*{prop}{Proposition}
\newtheorem{lemma}{Lemma}
\newtheorem{cor}{Corollary}

\theoremstyle{definition}
\newtheorem*{rem}{Remark}

% Some sets
\newcommand{\Nats}{\mathbb{N}}
\newcommand{\Ints}{\mathbb{Z}}
\newcommand{\Reals}{\mathbb{R}}
\newcommand{\Complex}{\mathbb{C}}
\newcommand{\Rats}{\mathbb{Q}}
\newcommand{\Gal}{\operatorname{Gal}}
\newcommand{\Cl}{\operatorname{Cl}}
\begin{document}
Let $p$ be an odd prime and let $a$ be an arbitrary integer. Let $\displaystyle \left(\frac{a}{p}\right)$ be the Legendre symbol of $a$ modulo $p$. Then:

\begin{prop} The following properties are satisfied:
\begin{enumerate}
\item If $a\equiv b \mod p$ then $\displaystyle \left(\frac{a}{p}\right)=\left(\frac{b}{p}\right)$.

\item If $a\neq 0 \mod p$ then $\displaystyle \left(\frac{a^2}{p}\right)=1$.

\item If $a\neq 0 \mod p$ and $b\in \mathbb{Z}$ then $\displaystyle \left(\frac{a^2b}{p}\right)=\left(\frac{b}{p}\right)$.

\item $\displaystyle \left(\frac{a}{p}\right)\left(\frac{b}{p}\right)=\left(\frac{ab}{p}\right)$.
\end{enumerate}
\end{prop}

\begin{proof}
The first three properties are immediate from the definition of the Legendre symbol. Remember that $(a/p)$ is $1$ if $x^2\equiv a \mod p$ has solutions, the value is $-1$ if there are no solutions, and equals $0$ if $a\equiv 0 \mod p$.

The fourth property is a consequence of Euler's criterion. Indeed,
$$\left(\frac{a}{p}\right)\equiv a^{(p-1)/2},\quad \left(\frac{b}{p}\right)\equiv b^{(p-1)/2},\quad
\text{and } \left(\frac{ab}{p}\right)\equiv (ab)^{(p-1)/2} \mod
p.$$ It is clear then that $(a/p)(b/p)\equiv (ab/p) \mod p$. Since
the numbers involved are all $\pm 1$ or $0$, the congruence also
holds with equality in $\mathbb{Z}$.
\end{proof}

\begin{rem}
Property (4) is somewhat surprising because, in particular, it says that the product of two quadratic non-residues modulo $p$ is a quadratic residue modulo $p$, which is not at all obvious.
\end{rem}
%%%%%
%%%%%
\end{document}
