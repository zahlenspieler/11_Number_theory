\documentclass[12pt]{article}
\usepackage{pmmeta}
\pmcanonicalname{ABCConjecture}
\pmcreated{2013-03-22 11:45:23}
\pmmodified{2013-03-22 11:45:23}
\pmowner{yark}{2760}
\pmmodifier{yark}{2760}
\pmtitle{ABC conjecture}
\pmrecord{21}{30201}
\pmprivacy{1}
\pmauthor{yark}{2760}
\pmtype{Conjecture}
\pmcomment{trigger rebuild}
\pmclassification{msc}{11A99}
\pmclassification{msc}{55-00}
\pmclassification{msc}{82-00}
\pmclassification{msc}{83-00}
\pmclassification{msc}{81-00}
\pmclassification{msc}{18-00}
\pmclassification{msc}{18C10}
%\pmkeywords{number theory}

\endmetadata

\usepackage{amssymb}
\usepackage{amsmath}
\usepackage{amsfonts}

\newcommand{\rad}{\operatorname{rad}}
\begin{document}
The \emph{ABC conjecture} states that given any $\epsilon > 0$,
there is a constant $\kappa ( \epsilon )$ such that
\[
  \max(|A|,|B|,|C|) \leq \kappa ( \epsilon ) ( \rad (ABC))^{1 + \epsilon}
\]
for all mutually coprime integers $A$, $B$, $C$ with $A+B=C$,
where $\rad$ is the radical of an integer.
This conjecture was formulated by Masser and Oesterl\'{e} in 1980.

The ABC conjecture is considered 
one of the most important unsolved problems in number \PMlinkescapetext{theory},
as many results would follow directly from this conjecture.
For example, Fermat's Last Theorem could be proved (for sufficiently large exponents)
with about one page worth of proof.

\section*{Further Reading}

\PMlinkexternal{The Amazing ABC Conjecture}{http://www.maa.org/mathland/mathtrek_12_8.html}
--- an article on the ABC conjecture by Ivars Peterson.

\PMlinkexternal{The ABC's of Number Theory}{http://www.hcs.harvard.edu/hcmr/issue1/elkies.pdf}
--- an article on the ABC conjecture by Noam Elkies. (PDF file)

%%%%%
%%%%%
%%%%%
%%%%%
\end{document}
