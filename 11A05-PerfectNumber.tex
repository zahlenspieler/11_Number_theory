\documentclass[12pt]{article}
\usepackage{pmmeta}
\pmcanonicalname{PerfectNumber}
\pmcreated{2013-03-22 11:45:29}
\pmmodified{2013-03-22 11:45:29}
\pmowner{Wkbj79}{1863}
\pmmodifier{Wkbj79}{1863}
\pmtitle{perfect number}
\pmrecord{24}{30206}
\pmprivacy{1}
\pmauthor{Wkbj79}{1863}
\pmtype{Definition}
\pmcomment{trigger rebuild}
\pmclassification{msc}{11A05}
\pmclassification{msc}{20D99}
\pmclassification{msc}{20D06}
\pmclassification{msc}{18-00}
%\pmkeywords{number theory}

\endmetadata

\usepackage{amssymb}
\usepackage{amsmath}
\usepackage{amsfonts}
\usepackage{amsthm}
\usepackage{graphicx}
%%%%\usepackage{xypic}
\newtheorem{lemma*}{Lemma}
\begin{document}
\PMlinkescapeword{forces}
\PMlinkescapeword{formulas}
\PMlinkescapeword{perfect}

An positive integer $n$ is called \emph{perfect} if it is the sum of all positive divisors of $n$ less than $n$ itself.  It is not known if there are any odd perfect numbers, but all even perfect numbers have been classified according to the following lemma:

\begin{lemma*}
An even number is perfect if and only if it equals $2^{k-1}(2^k-1)$ for some integer $k>1$ and $2^k-1$ is prime.\\
\end{lemma*}

\begin{proof}
Let $\sigma$ denote the sum of divisors function.  Recall that this function is multiplicative.

Necessity: Let $p=2^k-1$ be prime and $n=2^{k-1}p$.  We have that
\begin{eqnarray*}
\sigma(n) & = & \sigma(2^{k-1}p)\\
& = & \sigma(2^{k-1}) \sigma(p)\\
& = & (2^k-1)(p+1)\\
& = & (2^k-1)2^k\\
& = & 2n,
\end{eqnarray*}
which shows that $n$ is perfect.

Sufficiency: Assume $n$ is an even perfect number.  Write $n=2^{k-1}m$ for some odd $m$ and some $k>1$.  Then we have $\gcd(2^{k-1},m)=1$.  Thus,
\[ \sigma(n)=\sigma(2^{k-1}m)=\sigma(2^{k-1})\sigma(m)=(2^k-1)\sigma(m). \]
Since $n$ is perfect, $\sigma(n)=2n$ by definition.  Therefore, $\sigma(n)=2n=2^km$.  Piecing together the two formulas for $\sigma(n)$ yields
\[ 2^km=(2^k-1)\sigma(m). \]
Thus, $(2^k-1)\mid 2^km$, which forces $(2^k-1)\mid m$.  Write $m=(2^k-1)M$.  Note that $1\le M<m$.  From above, we have:
\begin{eqnarray*}
2^km & = & (2^k-1)\sigma(m) \\
2^k(2^k-1)M & = & (2^k-1)\sigma(m) \\
2^kM & = & \sigma(m)
\end{eqnarray*}
Since $m\mid m$ by definition of \PMlinkname{divides}{Divides} and $M\mid m$ by assumption, we have
\[ 2^kM=\sigma(m)\geq m+M=2^kM, \]
which forces $\sigma(m)=m+M$. Therefore, $m$ has only two positive divisors, $m$ and $M$.  Hence, $m$ must be prime, $M=1$, and $m=(2^k-1)M=2^k-1$, from which the result follows.
\end{proof}

The lemma can be used to produce examples of (even) perfect numbers:

\begin{itemize}
\item If $k=2$, then $2^k-1=2^2-1=3$, which is prime.  According to the lemma, $2^{k-1}(2^k-1)=2^{2-1} \cdot 3=6$ is perfect.  Indeed, $1+2+3=6$.
\item If $k=3$, then $2^k-1=2^3-1=7$, which is prime.  According to the lemma, $2^{k-1}(2^k-1)=2^{3-1} \cdot 7=28$ is perfect.  Indeed, $1+2+4+7+14=28$.
\item If $k=5$, then $2^k-1=2^5-1=31$, which is prime.  According to the lemma, $2^{k-1}(2^k-1)=2^{5-1} \cdot 31=496$ is perfect.  Indeed, $1+2+4+8+16+31+62+124+248=496$.
\end{itemize}

Note that $k=4$ yields that $2^k-1=2^4-1=15$, which is not prime.

The sequence of known perfect numbers appears in the OEIS as sequence \PMlinkexternal{A000396}{http://www.research.att.com/~njas/sequences/?q=A000396}.
%%%%%
%%%%%
%%%%%
%%%%%
\end{document}
