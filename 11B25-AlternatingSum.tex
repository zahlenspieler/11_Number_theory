\documentclass[12pt]{article}
\usepackage{pmmeta}
\pmcanonicalname{AlternatingSum}
\pmcreated{2013-03-22 17:35:30}
\pmmodified{2013-03-22 17:35:30}
\pmowner{PrimeFan}{13766}
\pmmodifier{PrimeFan}{13766}
\pmtitle{alternating sum}
\pmrecord{7}{40004}
\pmprivacy{1}
\pmauthor{PrimeFan}{13766}
\pmtype{Definition}
\pmcomment{trigger rebuild}
\pmclassification{msc}{11B25}

\endmetadata

% this is the default PlanetMath preamble.  as your knowledge
% of TeX increases, you will probably want to edit this, but
% it should be fine as is for beginners.

% almost certainly you want these
\usepackage{amssymb}
\usepackage{amsmath}
\usepackage{amsfonts}

% used for TeXing text within eps files
%\usepackage{psfrag}
% need this for including graphics (\includegraphics)
%\usepackage{graphicx}
% for neatly defining theorems and propositions
%\usepackage{amsthm}
% making logically defined graphics
%%%\usepackage{xypic}

% there are many more packages, add them here as you need them

% define commands here

\begin{document}
An {\em alternating sum} is a sequence of arithmetic operations in which each addition is followed by a subtraction, and viceversa, applied to a sequence of numerical entities. For example, $$\log 2 = 1 - \frac{1}{2} + \frac{1}{3} - \frac{1}{4} + \frac{1}{5} - \frac{1}{6} + \frac{1}{7} - \ldots$$ An alternating sum is also called an alternating series.

Alternating sums are often expressed in summation notation with the iterated expression involving multiplication by negative one raised to the iterator. Since a negative number raised to an odd number gives a negative number while raised to an even number gives a positive number (see: factors with minus sign), $(-1)^i$ essentially has the effect of turning the odd-indexed terms of the sequence negative but keeping their absolute values the same. Our previous example would thus be restated $$\log 2 = \sum_{i = 1}^\infty (-1)^{i - 1} \frac{1}{i}.$$

If the operands in an alternating sum decrease in value as the iterator increases, and approach zero, then the alternating sum converges to a specific value. This fact is used in many of the best-known expression for $\pi$ or fractions thereof, such as the Gregory series: $$\frac{\pi}{4} = \sum_{i = 0}^\infty (-1)^i \frac{1}{2i + 1}$$

Other constants also find expression as alternating sums, such as Cahen's constant.

An alternating sum need not necessarily involve an infinity of operands. For example, the alternating factorial of $n$ is computed by an alternating sum stopping at $i = n$.

\begin{thebibliography}{1}
\bibitem{td} Tobias Dantzig, {\it Number: The Language of Science}, ed. Joseph Mazur. New York: Pi Press (2005): 166
\end{thebibliography}
%%%%%
%%%%%
\end{document}
