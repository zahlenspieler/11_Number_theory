\documentclass[12pt]{article}
\usepackage{pmmeta}
\pmcanonicalname{MangoldtSummatoryFunction}
\pmcreated{2013-03-22 13:27:16}
\pmmodified{2013-03-22 13:27:16}
\pmowner{mathcam}{2727}
\pmmodifier{mathcam}{2727}
\pmtitle{Mangoldt summatory function}
\pmrecord{9}{34020}
\pmprivacy{1}
\pmauthor{mathcam}{2727}
\pmtype{Definition}
\pmcomment{trigger rebuild}
\pmclassification{msc}{11A41}
\pmsynonym{von Mangoldt summatory function}{MangoldtSummatoryFunction}
\pmrelated{ChebyshevFunctions}

\endmetadata

% this is the default PlanetMath preamble.  as your knowledge
% of TeX increases, you will probably want to edit this, but
% it should be fine as is for beginners.

% almost certainly you want these
\usepackage{amssymb}
\usepackage{amsmath}
\usepackage{amsfonts}

% used for TeXing text within eps files
%\usepackage{psfrag}
% need this for including graphics (\includegraphics)
%\usepackage{graphicx}
% for neatly defining theorems and propositions
%\usepackage{amsthm}
% making logically defined graphics
%%%\usepackage{xypic}

% there are many more packages, add them here as you need them

% define commands here
\begin{document}
A number theoretic function used in the study of prime numbers; specifically it was used in the proof of the prime number theorem.

It is defined thus:

$$
\psi(x) = \sum_{r \leq x} \Lambda(r)
$$

where $\Lambda$ is the Mangoldt function.

The Mangoldt summatory function is valid for all positive real x.

Note that we do not have to worry that the inequality above is ambiguous, because $\Lambda(x)$ is only non-zero for natural $x$. So no matter whether we take it to mean r is real, integer or natural, the result is the same because we just get a lot of zeros added to our answer.

The prime number theorem, which states:

$$
\pi(x) \sim \frac{x}{\ln(x)}
$$

where $\pi(x)$ is the prime counting function, is equivalent to the statement that:

$$
\psi(x) \sim x
$$

We can also define a ``smoothing function'' for the summatory function, defined as:

$$
\psi_1(x) = \int_0^x \psi(t) dt
$$

and then the prime number theorem is also equivalent to:

$$
\psi_1(x) \sim \frac{1}{2} x^2
$$

which turns out to be easier to work with than the original form.
%%%%%
%%%%%
\end{document}
