\documentclass[12pt]{article}
\usepackage{pmmeta}
\pmcanonicalname{FactorsOfNAndXn1}
\pmcreated{2013-03-22 16:35:05}
\pmmodified{2013-03-22 16:35:05}
\pmowner{pahio}{2872}
\pmmodifier{pahio}{2872}
\pmtitle{factors of $n$ and $x^n-1$}
\pmrecord{7}{38776}
\pmprivacy{1}
\pmauthor{pahio}{2872}
\pmtype{Theorem}
\pmcomment{trigger rebuild}
\pmclassification{msc}{11R60}
\pmclassification{msc}{11C08}
\pmclassification{msc}{11R18}
\pmrelated{PrimeFactorsOfXn1}

\endmetadata

% this is the default PlanetMath preamble.  as your knowledge
% of TeX increases, you will probably want to edit this, but
% it should be fine as is for beginners.

% almost certainly you want these
\usepackage{amssymb}
\usepackage{amsmath}
\usepackage{amsfonts}

% used for TeXing text within eps files
%\usepackage{psfrag}
% need this for including graphics (\includegraphics)
%\usepackage{graphicx}
% for neatly defining theorems and propositions
 \usepackage{amsthm}
% making logically defined graphics
%%%\usepackage{xypic}

% there are many more packages, add them here as you need them

% define commands here

\theoremstyle{definition}
\newtheorem*{thmplain}{Theorem}

\begin{document}
\PMlinkescapeword{root}


Let $n$ be a positive integer.\, Then the binomial\, $x^n\!-\!1$\, has as many \PMlinkname{prime factors}{PrimeFactorsOfXn1} with integer coefficients as the integer $n$ has positive divisors, both numbers thus being \PMlinkname{$\tau(n)$}{TauFunction}.

{\em Proof.}\, If\, $\Phi_d(x)$ generally means the $d$th cyclotomic polynomial
$$\Phi_d(x) := (x-\zeta_1)(x-\zeta_2)\ldots(x-\zeta_{\varphi(d)}),$$
where the $\zeta_j$s are the primitive $d$th roots of unity, then the equation
           $$\prod_{d|n,\,\,d>0}\!\Phi_d(x) = x^n\!-\!1$$
is true, because each $n$th root of unity is also a \PMlinkname{primitive}{RootOfUnity} $d$th root of unity for one and only one positive divisor of $n$.\, The cyclotomic factor polynomials $\Phi_d(x)$ have integer coefficients and are \PMlinkname{irreducible}{IrreduciblePolynomial2}.\, Thus the number of them is same as the number $\tau(n)$ of positive divisors of $n$.

For illustrating the proof, let\, $n = 6$ (divisors 1, 2, 3, 6); think the sixth roots of unity:\, $\zeta^0$, $\zeta^1$, $\zeta^2$, $\zeta^3$, $\zeta^4$, $\zeta^5$
(where\, $\zeta = e^{i\pi/3} = \frac{1+i\sqrt{3}}{2}$).\, From them, $\zeta^0 = 1$\, is the primitive 1st root, $\zeta^3$ the primitive 2nd root, $\zeta^2$ and $\zeta^4$ the primitive 3rd roots, $\zeta^1$ and $\zeta^5$ the primitive 6th roots of unity.


%%%%%
%%%%%
\end{document}
