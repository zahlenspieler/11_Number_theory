\documentclass[12pt]{article}
\usepackage{pmmeta}
\pmcanonicalname{CullenNumber}
\pmcreated{2013-03-22 17:21:34}
\pmmodified{2013-03-22 17:21:34}
\pmowner{PrimeFan}{13766}
\pmmodifier{PrimeFan}{13766}
\pmtitle{Cullen number}
\pmrecord{5}{39719}
\pmprivacy{1}
\pmauthor{PrimeFan}{13766}
\pmtype{Definition}
\pmcomment{trigger rebuild}
\pmclassification{msc}{11A51}
\pmrelated{WoodallNumber}

% this is the default PlanetMath preamble.  as your knowledge
% of TeX increases, you will probably want to edit this, but
% it should be fine as is for beginners.

% almost certainly you want these
\usepackage{amssymb}
\usepackage{amsmath}
\usepackage{amsfonts}

% used for TeXing text within eps files
%\usepackage{psfrag}
% need this for including graphics (\includegraphics)
%\usepackage{graphicx}
% for neatly defining theorems and propositions
%\usepackage{amsthm}
% making logically defined graphics
%%%\usepackage{xypic}

% there are many more packages, add them here as you need them

% define commands here

\begin{document}
A {\em Cullen number} $C_m$ is a number of the form $2^m m + 1$, where $m$ is an integer. Almost all Cullen numbers are composite, and they have $2m - 1$ as a factor if that number is a prime of the form $8k - 3$ (with $k$ a positive integer). The first few Cullen numbers are 1, 3, 9, 25, 65, 161, 385, 897, 2049, 4609 (listed in A002064 of Sloane's OEIS). With the exception of 3, no Proth number is also a Cullen number. 
%%%%%
%%%%%
\end{document}
