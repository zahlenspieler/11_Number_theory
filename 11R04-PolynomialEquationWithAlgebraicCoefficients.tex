\documentclass[12pt]{article}
\usepackage{pmmeta}
\pmcanonicalname{PolynomialEquationWithAlgebraicCoefficients}
\pmcreated{2013-03-22 19:07:37}
\pmmodified{2013-03-22 19:07:37}
\pmowner{pahio}{2872}
\pmmodifier{pahio}{2872}
\pmtitle{polynomial equation with algebraic coefficients}
\pmrecord{10}{42022}
\pmprivacy{1}
\pmauthor{pahio}{2872}
\pmtype{Theorem}
\pmcomment{trigger rebuild}
\pmclassification{msc}{11R04}
\pmsynonym{monic equation with algebraic coefficients}{PolynomialEquationWithAlgebraicCoefficients}

% this is the default PlanetMath preamble.  as your knowledge
% of TeX increases, you will probably want to edit this, but
% it should be fine as is for beginners.

% almost certainly you want these
\usepackage{amssymb}
\usepackage{amsmath}
\usepackage{amsfonts}

% used for TeXing text within eps files
%\usepackage{psfrag}
% need this for including graphics (\includegraphics)
%\usepackage{graphicx}
% for neatly defining theorems and propositions
 \usepackage{amsthm}
% making logically defined graphics
%%%\usepackage{xypic}

% there are many more packages, add them here as you need them

% define commands here

\theoremstyle{definition}
\newtheorem*{thmplain}{Theorem}

\begin{document}
If $\alpha_1,\,\ldots,\,\alpha_k$ are algebraic numbers [resp. algebraic integers] and 
$$f_1(\alpha_1,\,\ldots,\,\alpha_k),\;\ldots,\;f_n(\alpha_1,\,\ldots,\,\alpha_k)$$ 
polynomials in $\alpha_1,\,\ldots,\,\alpha_k$ with rational [resp. integer] coefficients, then all complex roots of the equation
\begin{align}
x^n+f_1(\alpha_1,\,\ldots,\,\alpha_k)x^{n-1}+\ldots+f_n(\alpha_1,\,\ldots,\,\alpha_k) \;=\; 0
\end{align}
are algebraic numbers [resp. algebraic integers].\\


\emph{Proof.}\, Let the minimal polynomial $x^m+a_1x^{m-1}+\ldots+a_m$ of $\alpha_1$ over $\mathbb{Z}$ have the \PMlinkname{zeros}{ZeroOfAFunction}
$$\alpha_1^{(1)} = \alpha_1,\; \alpha_1^{(2)},\; \ldots,\; \alpha_1^{(m)}$$
and denote by\, $F(x;\,\alpha_1,\,\alpha_2,\,\ldots,\,\alpha_k)$\, the left hand side of the equation (1).\, Consider the equation
\begin{align}
G(x;\,\alpha_2,\,\ldots,\,\alpha_k) 
\;:=\; \prod_{i=1}^m F(x;\,\alpha_1^{(i)},\,\alpha_2,\,\ldots,\,\alpha_k) \;=\; 0.
\end{align}
Here, the coefficients of the polynomial $G$ are polynomials in the numbers
$$\alpha_1^{(1)},\; \alpha_1^{(2)},\; \ldots,\; \alpha_1^{(m)},\; \alpha_2,\; \ldots,\; \alpha_k$$
with rational [resp. integer] coefficients.\, Thus the coefficients of $G$ are symmetric polynomials $g_j$ in the numbers $\alpha_1^{(i)}$:
$$G \;=\; \sum_jg_jx^j$$
By the fundamental theorem of symmetric polynomials, the coefficients $g_j$ of $G$ are polynomials in 
$\alpha_2,\,\ldots,\,\alpha_k$ with rational [resp. integer] coefficients.\, Consequently, $G$ has the form
$$G \;=\; x^h+a_1'(\alpha_2,\,\ldots,\,\alpha_k)x^{h-1}+\ldots+a_h'(\alpha_2,\,\ldots,\,\alpha_k)$$
where the coefficients $a_i'(\alpha_2,\,\ldots,\,\alpha_k)$ are polynoms in the numbers $\alpha_j$ with rational [resp. integer] coefficients.\, As one continues similarly, removing one by one also $\alpha_2,\,\ldots,\,\alpha_k$ which go back to the rational [resp. integer] coefficients of the corresponding minimal polynomials, one shall finally arrive at an equation
\begin{align}
x^s+A_1x^{s-1}+\ldots+A_s \;=\; 0,
\end{align}
among the roots of which there are the roots of (1); the coefficients $A_\nu$ do no more explicitely depend on the algebraic numbers $\alpha_1,\,\ldots,\,\alpha_k$ but are rational numbers [resp. integers].\\
Accordingly, the roots of (1) are algebraic numbers [resp. algebraic integers], Q.E.D.

%%%%%
%%%%%
\end{document}
