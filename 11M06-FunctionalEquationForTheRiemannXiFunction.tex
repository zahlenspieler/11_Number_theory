\documentclass[12pt]{article}
\usepackage{pmmeta}
\pmcanonicalname{FunctionalEquationForTheRiemannXiFunction}
\pmcreated{2013-03-22 13:24:15}
\pmmodified{2013-03-22 13:24:15}
\pmowner{rspuzio}{6075}
\pmmodifier{rspuzio}{6075}
\pmtitle{functional equation for the Riemann Xi function}
\pmrecord{7}{33946}
\pmprivacy{1}
\pmauthor{rspuzio}{6075}
\pmtype{Theorem}
\pmcomment{trigger rebuild}
\pmclassification{msc}{11M06}

% this is the default PlanetMath preamble.  as your knowledge
% of TeX increases, you will probably want to edit this, but
% it should be fine as is for beginners.

% almost certainly you want these
\usepackage{amssymb}
\usepackage{amsmath}
\usepackage{amsfonts}

% used for TeXing text within eps files
%\usepackage{psfrag}
% need this for including graphics (\includegraphics)
%\usepackage{graphicx}
% for neatly defining theorems and propositions
%\usepackage{amsthm}
% making logically defined graphics
%%%\usepackage{xypic}

% there are many more packages, add them here as you need them

% define commands here
\begin{document}
The Riemann Xi Function satisfies the following functional equation:
 $$\Xi(s) = \Xi(1-s)$$
This equation  directly implies the Riemann Zeta function's functional equation. 

This equation plays an important role in the theory of the Riemann Zeta function.  It allows one to analytically continue the Zeta and the Xi functions to the whole complex plane.  The definition of the Zeta function as a series is only valid when $\Re(s) > 1$.  By using this equation, one can express the values of these two functions when $\Re(s) < 1$ in terms of the values when $\Re(s) > 1$.  As an illustration of its importance, one can cite the fact that there are no zeros of the Zeta function with real part greater than 1, so without this functional equation the study of the Zeta function would be very limited.
%%%%%
%%%%%
\end{document}
