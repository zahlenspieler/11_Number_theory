\documentclass[12pt]{article}
\usepackage{pmmeta}
\pmcanonicalname{RamanujansFormulaForPi}
\pmcreated{2013-03-22 15:53:41}
\pmmodified{2013-03-22 15:53:41}
\pmowner{alozano}{2414}
\pmmodifier{alozano}{2414}
\pmtitle{Ramanujan's formula for pi}
\pmrecord{7}{37896}
\pmprivacy{1}
\pmauthor{alozano}{2414}
\pmtype{Theorem}
\pmcomment{trigger rebuild}
\pmclassification{msc}{11-00}
\pmclassification{msc}{51-00}
\pmrelated{CyclometricFunctions}

\endmetadata

% this is the default PlanetMath preamble.  as your knowledge
% of TeX increases, you will probably want to edit this, but
% it should be fine as is for beginners.

% almost certainly you want these
\usepackage{amssymb}
\usepackage{amsmath}
\usepackage{amsthm}
\usepackage{amsfonts}

% used for TeXing text within eps files
%\usepackage{psfrag}
% need this for including graphics (\includegraphics)
%\usepackage{graphicx}
% for neatly defining theorems and propositions
%\usepackage{amsthm}
% making logically defined graphics
%%%\usepackage{xypic}

% there are many more packages, add them here as you need them

% define commands here

\newtheorem*{thm}{Theorem}
\newtheorem{defn}{Definition}
\newtheorem{prop}{Proposition}
\newtheorem{lemma}{Lemma}
\newtheorem{cor}{Corollary}

\theoremstyle{definition}
\newtheorem{exa}{Example}

% Some sets
\newcommand{\Nats}{\mathbb{N}}
\newcommand{\Ints}{\mathbb{Z}}
\newcommand{\Reals}{\mathbb{R}}
\newcommand{\Complex}{\mathbb{C}}
\newcommand{\Rats}{\mathbb{Q}}
\newcommand{\Gal}{\operatorname{Gal}}
\newcommand{\Cl}{\operatorname{Cl}}
\begin{document}
Around $1910$, Ramanujan proved the following formula:

\begin{thm}
The following series converges and the sum equals $\frac{1}{\pi}$:
$$\frac{1}{\pi}=\frac{2\sqrt{2}}{9801}\sum_{n=0}^\infty \frac{(4n)!(1103+26390n)}{(n!)^4396^{4n}}.$$
\end{thm} 

Needless to say, the convergence is extremely fast. For example, if we only use the term $n=0$ we obtain the following approximation:
$$\pi \approx \frac{9801}{2\cdot 1103\cdot \sqrt{2}}=3.14159273001\ldots$$
and the error is (in absolute value) equal to $0.0000000764235\ldots$ In $1985$, William Gosper used this formula to calculate the first 17 million digits of $\pi$. 

Another similar formula can be easily obtained from the power series of $\arctan x$. Although the convergence is good, it is not as impressive as in Ramanujan's formula:

$$\pi=2\sqrt{3}\sum_{n=0}^\infty \frac{(-1)^n}{(2n+1)3^n}.$$ 
%%%%%
%%%%%
\end{document}
