\documentclass[12pt]{article}
\usepackage{pmmeta}
\pmcanonicalname{TheSumOfTheValuesOfACharacterOfAFiniteGroupIs0}
\pmcreated{2013-03-22 14:10:30}
\pmmodified{2013-03-22 14:10:30}
\pmowner{alozano}{2414}
\pmmodifier{alozano}{2414}
\pmtitle{the sum of the values of a character of a finite group is $0$}
\pmrecord{6}{35601}
\pmprivacy{1}
\pmauthor{alozano}{2414}
\pmtype{Theorem}
\pmcomment{trigger rebuild}
\pmclassification{msc}{11A25}
%\pmkeywords{sum}
%\pmkeywords{vanishing}
%\pmkeywords{character}

% this is the default PlanetMath preamble.  as your knowledge
% of TeX increases, you will probably want to edit this, but
% it should be fine as is for beginners.

% almost certainly you want these
\usepackage{amssymb}
\usepackage{amsmath}
\usepackage{amsthm}
\usepackage{amsfonts}

% used for TeXing text within eps files
%\usepackage{psfrag}
% need this for including graphics (\includegraphics)
%\usepackage{graphicx}
% for neatly defining theorems and propositions
%\usepackage{amsthm}
% making logically defined graphics
%%%\usepackage{xypic}

% there are many more packages, add them here as you need them

% define commands here

\newtheorem{thm}{Theorem}
\newtheorem{defn}{Definition}
\newtheorem{prop}{Proposition}
\newtheorem{lemma}{Lemma}
\newtheorem{cor}{Corollary}

% Some sets
\newcommand{\Nats}{\mathbb{N}}
\newcommand{\Ints}{\mathbb{Z}}
\newcommand{\Reals}{\mathbb{R}}
\newcommand{\Complex}{\mathbb{C}}
\newcommand{\Rats}{\mathbb{Q}}
\begin{document}
The following is an argument that occurs in many proofs involving characters of groups. Here we use additive notation for the group $G$, however this group is not assumed to be abelian.

\begin{lemma}
Let $G$ be a finite group, and let $K$ be a field. Let $\chi\colon G\to K^{\times}$ be a character, where $K^{\times}$ denotes the multiplicative group of $K$. Then:
$$\sum_{g\in G} \chi(g) =\begin{cases}
\mid G \mid, \text{ if } \chi \text{ is trivial,}\\
0_K, \text{ otherwise}
\end{cases}$$
where $0_K$ is the zero element in $K$, and $\mid G \mid$ is the order of the group $G$.
\end{lemma}
\begin{proof}
First assume that $\chi$ is trivial, i.e. for all $g\in G$ we have $\chi(g)=1\in K$. Then the result is clear.

Thus, let us assume that there exists $g_1$ in $G$ such that $\chi(g_1)=h\neq 1\in K$. Notice that for any element $g_1\in G$ the map:
$$ G \to G,\quad g\mapsto g_1+g $$
is clearly a bijection. Define $\mathcal{S}=\sum_{g\in G} \chi(g)\in K$. Then:
\begin{eqnarray*}
h\cdot\mathcal{S} &=& \chi(g_1)\cdot\mathcal{S}\\
&=& \chi(g_1)\cdot \sum_{g\in G}\chi(g)\\
&=& \sum_{g\in G}\chi(g_1)\cdot\chi(g)\\
&=& \sum_{g\in G}\chi(g_1 + g),\quad (1)\\
&=& \sum_{j\in G}\chi(j),\quad (2)\\
&=& \mathcal{S}
\end{eqnarray*}
By the remark above, sums $(1)$ and $(2)$ are equal, since both run over all possible values of $\chi$ over elements of $G$. Thus, we have proved that:
$$h\cdot \mathcal{S}=\mathcal{S}$$
and $h\neq 1\in K$. Since $K$ is a field, it follows that $\mathcal{S}=0\in K$, as desired.

\end{proof}
%%%%%
%%%%%
\end{document}
