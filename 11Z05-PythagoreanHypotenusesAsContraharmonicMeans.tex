\documentclass[12pt]{article}
\usepackage{pmmeta}
\pmcanonicalname{PythagoreanHypotenusesAsContraharmonicMeans}
\pmcreated{2013-11-20 20:23:01}
\pmmodified{2013-11-20 20:23:01}
\pmowner{pahio}{2872}
\pmmodifier{pahio}{2872}
\pmtitle{Pythagorean hypotenuses as contraharmonic means}
\pmrecord{10}{41278}
\pmprivacy{1}
\pmauthor{pahio}{2872}
\pmtype{Data Structure}
\pmcomment{trigger rebuild}
\pmclassification{msc}{11Z05}
\pmclassification{msc}{11D45}
\pmclassification{msc}{11D09}
\pmclassification{msc}{11A05}
\pmrelated{PythagoreanTriple}
\pmrelated{ContraharmonicMean}
\pmrelated{FirstPrimitivePythagoreanTriplets}
\pmrelated{ContraharmonicProportion}

\endmetadata

% this is the default PlanetMath preamble.  as your knowledge
% of TeX increases, you will probably want to edit this, but
% it should be fine as is for beginners.

% almost certainly you want these
\usepackage{amssymb}
\usepackage{amsmath}
\usepackage{amsfonts}

% used for TeXing text within eps files
%\usepackage{psfrag}
% need this for including graphics (\includegraphics)
%\usepackage{graphicx}
% for neatly defining theorems and propositions
 \usepackage{amsthm}
% making logically defined graphics
%%%\usepackage{xypic}

% there are many more packages, add them here as you need them

% define commands here

\theoremstyle{definition}
\newtheorem*{thmplain}{Theorem}

\begin{document}
\begin{center}
\begin{tabular}{|c|c|c|}
\hline\hline
$c^2 \;=\; a^2+b^2$ & $\displaystyle c \;=\; \frac{u^2+v^2}{u+v}$ & N.B. \\
\hline\hline
$5^2 = 3^2+4^2$ & $5 = (2^2+6^2)/(2+6)$ & 3\\
\hline 
$10^2 = 6^2+8^2$ & $10 = (6^2+12^2)/(6+12)$ & \\
\hline
$13^2 = 5^2+12^2$ & $13 = (3^2+15^2)/(3+15)$ & 5\\
\hline
$15^2 = 9^2+12^2$ & $15 = (9^2+18^2)/(9+18)$ & \\
\hline
$17^2 = 8^2+15^2$ & $17 = (5^2+20^2)/(5+20)$ & 5\\
\hline
$20^2 = 12^2+16^2$ & $20 = (8^2+24^2)/(8+24)$ & \\
\hline
$25^2 = 7^2+24^2$ & $25 = (4^2+28^2)/(4+28)$ & 7\\
\hline
$26^2 = 10^2+24^2$ & $26 = (6^2+30^2)/(6+30)$ & \\
\hline
$29^2 = 20^2+21^2$ & $29 = (14^2+35^2)/(14+35)$ & 7\\
\hline
$30^2 = 18^2+24^2$ & $30 = (12^2+36^2)/(12+36)$ & \\
\hline
$34^2 = 16^2+30^2$ & $34 = (10^2+40^2)/(10+40)$ & 5\\
\hline
$35^2 = 21^2+28^2$ & $35 = (14^2+42^2)/(14+42)$ & \\
\hline
$37^2 = 12^2+35^2$ & $37 = (30^2+42^2)/(30+42)$ & 7\\
\hline
$39^2 = 15^2+36^2$ & $39 = (9^2+45^2)/(9+45)$ & \\
\hline
$40^2 = 24^2+32^2$ & $40 = (16^2+48^2)/(16+48)$ & \\
\hline
$41^2 = 9^2+40^2$ & $41 = (36^2+45^2)/(36+45)$ & 3\\
\hline
$50^2 = 14^2+48^2$ & $50 = (42^2+56^2)/(42+56)$ & 7\\
\hline
$51^2 = 24^2+45^2$ & $51 = (15^2+60^2)/(15+60)$ & \\
\hline
$52^2 = 20^2+48^2$ & $52 = (12^2+60^2)/(12+60)$ & 3\\
\hline
$53^2 = 28^2+45^2$ & $53 = (18^2+63^2)/(18+63)$ & 7\\
\hline
$55^2 = 33^2+44^2$ & $55 = (22^2+66^2)/(22+66)$ & \\
\hline
$58^2 = 40^2+42^2$ & $58 = (30^2+70^2)/(30+70)$ & 7\\
\hline
$60^2 = 36^2+48^2$ & $60 = (24^2+72^2)/(24+72)$ & \\
\hline
$61^2 = 11^2+60^2$ & $61 = (6^2+66^2)/(6+66)$ & 11\\
\hline
$61^2 = 11^2+60^2$ & $61 = (55^2+66^2)/(55+66)$ & 11\\

\hline
$65^2 = 39^2+52^2$ & $65 = (26^2+78^2)/(26+78)$ & \\
\hline
$68^2 = 32^2+60^2$ & $68 = (20^2+80^2)/(20+80)$ & \\
\hline
$70^2 = 42^2+56^2$ & $70 = (28^2+84^2)/(28+84)$ & \\
\hline
$73^2 = 48^2+55^2$ & $73 = (40^2+88^2)/(40+88)$ & 11\\
\hline
$74^2 = 24^2+70^2$ & $74 = (14^2+84^2)/(14+84)$ & 7\\
\hline
$75^2 = 21^2+72^2$ & $75 = (12^2+84^2)/(12+84)$ & 7\\
\hline
$78^2 = 30^2+72^2$ & $78 = (18^2+90^2)/(18+90)$ & \\
\hline
$80^2 = 48^2+64^2$ & $80 = (32^2+96^2)/(32+96)$ & \\
\hline
$82^2 = 18^2+80^2$ & $82 = (10^2+90^2)/(10+90)$ & 3\\
\hline
$85^2 = 40^2+75^2$ & $85 = (25^2+100^2)/(25+100)$ & \\
\hline
$87^2 = 60^2+63^2$ & $87 = (42^2+105^2)/(42+105)$ & 7\\
\hline
$89^2 = 39^2+80^2$ & $89 = (24^2+104^2)/(24+104)$ & 13\\
\hline
$90^2 = 54^2+72^2$ & $90 = (36^2+108^2)/(36+108)$ & \\
\hline
$91^2 = 35^2+84^2$ & $91 = (21^2+105^2)/(21+105)$ & \\
\hline
$95^2 = 57^2+76^2$ & $95 = (38^2+114^2)/(38+114)$ & 19\\
\hline
$97^2 = 72^2+65^2$ & $97 = (45^2+117^2)/(45+117)$ & 13\\
\hline
$100^2 = 60^2+80^2$ & $100 = (40^2+120^2)/(40+120)$ & \\

\end{tabular}
\end{center}


%%%%%
%%%%%
\end{document}
