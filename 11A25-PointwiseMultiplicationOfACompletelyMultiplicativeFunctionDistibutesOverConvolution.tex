\documentclass[12pt]{article}
\usepackage{pmmeta}
\pmcanonicalname{PointwiseMultiplicationOfACompletelyMultiplicativeFunctionDistibutesOverConvolution}
\pmcreated{2013-03-22 15:59:49}
\pmmodified{2013-03-22 15:59:49}
\pmowner{Wkbj79}{1863}
\pmmodifier{Wkbj79}{1863}
\pmtitle{pointwise multiplication of a completely multiplicative function distibutes over convolution}
\pmrecord{8}{38022}
\pmprivacy{1}
\pmauthor{Wkbj79}{1863}
\pmtype{Theorem}
\pmcomment{trigger rebuild}
\pmclassification{msc}{11A25}
\pmrelated{ArithmeticFunction}
\pmrelated{CompletelyMultiplicative}
\pmrelated{MultiplicativeFunction}

\endmetadata

\usepackage{amssymb}
\usepackage{amsmath}
\usepackage{amsfonts}

\usepackage{psfrag}
\usepackage{graphicx}
\usepackage{amsthm}
%%\usepackage{xypic}

\newtheorem*{thm*}{Theorem}
\begin{document}
\begin{thm*}
Let $f$ be a completely multiplicative function and $g$ and $h$ be arithmetic functions.  Then $f(g*h)=(fg)*(fh)$.
\end{thm*}

\begin{proof}
Let $n$ be a positive integer.  Then

\begin{center}
\begin{tabular}{ll}
$\displaystyle (f(g*h))(n)$ & $\displaystyle = f(n)(g*h)(n)$ \\
& $\displaystyle = f(n) \sum_{d|n} g(d)h\left( \frac{n}{d} \right)$ \\
& $\displaystyle = \sum_{d|n} f(n)g(d)h\left( \frac{n}{d} \right)$ \\
& $\displaystyle = \sum_{d|n} f\left( d \cdot \frac{n}{d} \right) g(d)h\left( \frac{n}{d} \right)$ \\
& $\displaystyle = \sum_{d|n} f(d)f\left( \frac{n}{d} \right) g(d)h\left( \frac{n}{d} \right)$ \\
& $\displaystyle = \sum_{d|n} (fg)(d)(fh)\left( \frac{n}{d} \right)$ \\
& $\displaystyle = ((fg)*(fh))(n)$. \end{tabular}
\end{center}
\end{proof}
%%%%%
%%%%%
\end{document}
