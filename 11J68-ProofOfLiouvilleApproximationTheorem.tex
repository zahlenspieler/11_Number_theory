\documentclass[12pt]{article}
\usepackage{pmmeta}
\pmcanonicalname{ProofOfLiouvilleApproximationTheorem}
\pmcreated{2013-03-22 13:19:22}
\pmmodified{2013-03-22 13:19:22}
\pmowner{lieven}{1075}
\pmmodifier{lieven}{1075}
\pmtitle{proof of Liouville approximation theorem}
\pmrecord{6}{33833}
\pmprivacy{1}
\pmauthor{lieven}{1075}
\pmtype{Proof}
\pmcomment{trigger rebuild}
\pmclassification{msc}{11J68}

\endmetadata

% this is the default PlanetMath preamble.  as your knowledge
% of TeX increases, you will probably want to edit this, but
% it should be fine as is for beginners.

% almost certainly you want these
\usepackage{amssymb}
\usepackage{amsmath}
\usepackage{amsfonts}

% used for TeXing text within eps files
%\usepackage{psfrag}
% need this for including graphics (\includegraphics)
%\usepackage{graphicx}
% for neatly defining theorems and propositions
%\usepackage{amsthm}
% making logically defined graphics
%%%\usepackage{xypic}

% there are many more packages, add them here as you need them

% define commands here
\begin{document}
Let $\alpha$ satisfy the equation $f(\alpha)=a_n\alpha^n+a_{n-1}\alpha^{n-1}+\dots+a_0=0$ where the $a_i$ are integers. Choose $M$ such that $M>\max_{\alpha-1\leq x\leq\alpha+1}|f'(x)|$.

Suppose $\frac{p}{q}$ lies in $(\alpha-1,\alpha+1)$ and $f\left(\frac{p}{q}\right)\neq 0$.

$$\left|f\left(\frac{p}{q}\right)\right|=\frac{\left|a^np^n+a_{n-1}p^{n-1}q+\dots+a_0q^n\right|}{q^n}\geq\frac{1} {q^n}$$

since the numerator is a non-zero integer.

By the mean-value theorem 

$$\frac{1}{q^n}\leq \left|f\left(\frac{p}{q}\right)-f(\alpha)\right|=\left|\left(\frac{p}{q}-\alpha\right)f'(x)\right|<M\left|\frac{p}{q}-\alpha\right|.$$
%%%%%
%%%%%
\end{document}
