\documentclass[12pt]{article}
\usepackage{pmmeta}
\pmcanonicalname{AnalyticNumberTheory}
\pmcreated{2013-03-22 15:59:55}
\pmmodified{2013-03-22 15:59:55}
\pmowner{Wkbj79}{1863}
\pmmodifier{Wkbj79}{1863}
\pmtitle{analytic number theory}
\pmrecord{8}{38025}
\pmprivacy{1}
\pmauthor{Wkbj79}{1863}
\pmtype{Topic}
\pmcomment{trigger rebuild}
\pmclassification{msc}{11N37}
\pmclassification{msc}{11M06}
\pmclassification{msc}{11N05}
\pmclassification{msc}{11-01}

\endmetadata

% this is the default PlanetMath preamble.  as your knowledge
% of TeX increases, you will probably want to edit this, but
% it should be fine as is for beginners.

% almost certainly you want these
\usepackage{amssymb}
\usepackage{amsmath}
\usepackage{amsfonts}

% used for TeXing text within eps files
%\usepackage{psfrag}
% need this for including graphics (\includegraphics)
%\usepackage{graphicx}
% for neatly defining theorems and propositions
%\usepackage{amsthm}
% making logically defined graphics
%%%\usepackage{xypic}

% there are many more packages, add them here as you need them

% define commands here

\begin{document}
Analytic number theory uses the machinery of analysis to tackle questions related to integers and transcendence. One of its most famous achievements is the proof of the prime number theorem.

One concept that is important in analytic number theory is asymptotic estimates.  Tools that are used to obtain asymptotic estimates for sums include the \PMlinkname{Euler-Maclaurin summation formula}{EulerMaclaurinSummationFormula}, Abel's lemma (summation by parts), the convolution method, and the Dirichlet hyperbola method.  Asymptotic estimates are important for determining asymptotic densities of certain subsets of the natural numbers.

Another one of Dirichlet's contributions to analytic number theory is the Dirichlet series.  As an example, the Dirichlet series of a Dirichlet character is a Dirichlet L-series.  A tool that is helpful for studying any Dirichlet series is the Euler product.  The most famous Dirichlet series is the Riemann zeta function, which is the Dirichlet series of the completely multiplicative function $1$.  This leads up to what is possibly the most important unsolved problem in analytic number theory:  the Riemann hypothesis.  This \PMlinkescapetext{states} that all nontrivial zeros of the Riemann zeta function have real part equal to $\frac{1}{2}$.  Its \PMlinkescapetext{connection} to prime numbers is made clearer by the Euler product formula.
%%%%%
%%%%%
\end{document}
