\documentclass[12pt]{article}
\usepackage{pmmeta}
\pmcanonicalname{RothsTheorem}
\pmcreated{2013-03-22 15:02:23}
\pmmodified{2013-03-22 15:02:23}
\pmowner{alozano}{2414}
\pmmodifier{alozano}{2414}
\pmtitle{Roth's theorem}
\pmrecord{7}{36753}
\pmprivacy{1}
\pmauthor{alozano}{2414}
\pmtype{Theorem}
\pmcomment{trigger rebuild}
\pmclassification{msc}{11J81}
\pmclassification{msc}{11J68}
%\pmkeywords{transcendental}
%\pmkeywords{algebraic}
%\pmkeywords{rational approximation}
\pmrelated{ExampleOfTranscendentalNumber}

% this is the default PlanetMath preamble.  as your knowledge
% of TeX increases, you will probably want to edit this, but
% it should be fine as is for beginners.

% almost certainly you want these
\usepackage{amssymb}
\usepackage{amsmath}
\usepackage{amsthm}
\usepackage{amsfonts}

% used for TeXing text within eps files
%\usepackage{psfrag}
% need this for including graphics (\includegraphics)
%\usepackage{graphicx}
% for neatly defining theorems and propositions
%\usepackage{amsthm}
% making logically defined graphics
%%%\usepackage{xypic}

% there are many more packages, add them here as you need them

% define commands here

\newtheorem{thm}{Theorem}
\newtheorem{defn}{Definition}
\newtheorem{prop}{Proposition}
\newtheorem{lemma}{Lemma}
\newtheorem{cor}{Corollary}

% Some sets
\newcommand{\Nats}{\mathbb{N}}
\newcommand{\Ints}{\mathbb{Z}}
\newcommand{\Reals}{\mathbb{R}}
\newcommand{\Complex}{\mathbb{C}}
\newcommand{\Rats}{\mathbb{Q}}
\begin{document}
The following theorem is due to Klaus Roth and it is a generalization of a previous theorem of Liouville (see Liouville approximation theorem). Roth was awarded the Fields Medal for his work on the geometry of numbers. W. M. Schmidt generalized the result even further. The result is widely used to prove that a certain number is transcendental. Here, for a rational number $t$ in reduced form, the denominator of $t$ is denoted by $d(t)$.

\begin{thm}
For any algebraic number $\alpha$ and for any $\epsilon>0$ there are only finitely many rational numbers $t$ with:
$$|\alpha - t| < \frac{1}{d(t)^{2+\epsilon}}.$$
In other words, the equation:
$$\left|\alpha - \frac{p}{q}\right| < \frac{1}{q^{2+\epsilon}}$$
has only finitely many solutions with $p\in \Ints$ and $q\in \Ints^+$.
\end{thm}
%%%%%
%%%%%
\end{document}
