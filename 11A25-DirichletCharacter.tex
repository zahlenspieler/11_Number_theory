\documentclass[12pt]{article}
\usepackage{pmmeta}
\pmcanonicalname{DirichletCharacter}
\pmcreated{2013-03-22 13:22:31}
\pmmodified{2013-03-22 13:22:31}
\pmowner{bbukh}{348}
\pmmodifier{bbukh}{348}
\pmtitle{Dirichlet character}
\pmrecord{10}{33906}
\pmprivacy{1}
\pmauthor{bbukh}{348}
\pmtype{Definition}
\pmcomment{trigger rebuild}
\pmclassification{msc}{11A25}
\pmrelated{CharacterOfAFiniteGroup}
\pmdefines{trivial character}
\pmdefines{primitive character}
\pmdefines{conductor}
\pmdefines{induced character}

\endmetadata

\usepackage{amssymb}
\usepackage{amsmath}
\usepackage{amsfonts}

%\usepackage{psfrag}
%\usepackage{graphicx}
%\usepackage{amsthm}
%%%\usepackage{xypic}
\newcommand*{\legsym}{\genfrac{(}{)}{}{}}
\begin{document}
\PMlinkescapeword{induced}
\PMlinkescapeword{primitive}

A \emph{Dirichlet character} modulo $m$ is a group homomorphism from $\left(\frac{\mathbb{Z}}{m\mathbb{Z}}\right)^*$ to $\mathbb{C^*}$. Dirichlet characters are usually denoted by the Greek letter $\chi$. The function \begin{equation*}
\gamma(n)=\begin{cases}
\chi(n\bmod m),&\text{if }\gcd(n,m)=1,\\
0,&\text{if }\gcd(n,m)>1. \end{cases}
\end{equation*}
is also referred to as a Dirichlet character.
The Dirichlet characters modulo $m$ form a group if one defines $\chi\chi'$ to be the function which takes $a\in\left(\frac{\mathbb{Z}}{m\mathbb{Z}}\right)^*$ to $\chi(a)\chi'(a)$.  It turns out that this resulting group is isomorphic to $\left(\frac{\mathbb{Z}}{m\mathbb{Z}}\right)^*$.  The trivial character is given by $\chi(a)=1$ for all $a\in\left(\frac{\mathbb{Z}}{m\mathbb{Z}}\right)^*$, and it acts as the identity element for the group.  
A character $\chi$ modulo $m$ is said to be \emph{induced} by a character $\chi'$ modulo $m'$ if $m'\mid m$ and $\chi(n)=\chi'(n\bmod m')$. A character which is not induced by any other character is called \emph{primitive}.
%A character is said to be primitive if it does not arise as the composite
%\[
%\left(\frac{\mathbb{Z}}{m\mathbb{Z}}\right)^* \rightarrow %\left(\frac{\mathbb{Z}}{m'\mathbb{Z}}\right)^* \rightarrow \mathbb{C^*},
%\]
%for any proper divisor $m'\mid m$, where the first map is the natural mapping %and the second map is a character mod $m'$.  
If $\chi$ is non-primitive, the $\gcd$ of all such $m'$ is called the conductor of $\chi$.

\emph{Examples:}
\begin{itemize}
\item Legendre symbol $\legsym{n}{p}$ is a Dirichlet character modulo $p$ for any odd prime $p$. More generally, Jacobi symbol $\legsym{n}{m}$ is a Dirichlet character modulo $m$.
\item The character modulo $4$ given by $\chi(1)=1$ and $\chi(3)=-1$ is a primitive character modulo $4$. The only other character modulo $4$ is the trivial character.
\end{itemize}
%%%%%
%%%%%
\end{document}
