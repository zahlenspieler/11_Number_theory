\documentclass[12pt]{article}
\usepackage{pmmeta}
\pmcanonicalname{ProductOfDivisorsFunction}
\pmcreated{2013-03-22 18:55:45}
\pmmodified{2013-03-22 18:55:45}
\pmowner{pahio}{2872}
\pmmodifier{pahio}{2872}
\pmtitle{product of divisors function}
\pmrecord{7}{41782}
\pmprivacy{1}
\pmauthor{pahio}{2872}
\pmtype{Definition}
\pmcomment{trigger rebuild}
\pmclassification{msc}{11A25}
\pmsynonym{divisor product}{ProductOfDivisorsFunction}

\endmetadata

% this is the default PlanetMath preamble.  as your knowledge
% of TeX increases, you will probably want to edit this, but
% it should be fine as is for beginners.

% almost certainly you want these
\usepackage{amssymb}
\usepackage{amsmath}
\usepackage{amsfonts}

% used for TeXing text within eps files
%\usepackage{psfrag}
% need this for including graphics (\includegraphics)
%\usepackage{graphicx}
% for neatly defining theorems and propositions
 \usepackage{amsthm}
% making logically defined graphics
%%%\usepackage{xypic}

% there are many more packages, add them here as you need them

% define commands here

\theoremstyle{definition}
\newtheorem*{thmplain}{Theorem}

\begin{document}
The product of all positive divisors of a nonzero integer $n$ is equal $\sqrt{n^{\tau(n)}}$, where tau function $\tau(n)$ expresses the number of the positive divisors of $n$.\\

\emph{Proof.}\, Let\, $t = \tau(n)$\, and the positive divisors of $n$ be\, 
$a_1 < a_2 < \ldots < a_t.$

If $n$ is not a square of an integer, $t$ is even (see \PMlinkid{parity of $\tau$ function}{11781}), whence
\begin{align*}
\begin{cases}
 a_1a_t \;=\; n\\
 a_2a_{t-1} \;=\; n\\
 \quad\cdots\\
 a_{\frac{t}{2}}a_{\frac{t+2}{2}} \;=\; n.
\end{cases}
\end{align*}
Thus
$$\prod_{d \mid n}d \;=\; a_1a_2\cdots a_t \;=\; n^{\frac{t}{2}}.$$
If $n$ is a square of an integer, $t$ is odd, and we have
\begin{align*}
\begin{cases}
 a_1a_t \;=\; n\\
 a_2a_{t-1} \;=\; n\\
 \quad\cdots\\
 a_{\frac{t-1}{2}}a_{\frac{t+3}{2}} \;=\; n\\
 \;\;a_{\frac{t+1}{2}} \;=\; n^{\frac{1}{2}}.
\end{cases}
\end{align*}
In this case we obtain a \PMlinkescapetext{similar} result:
$$\prod_{d \mid n}d \;=\; a_1a_2\cdots a_t \;=\; n^{\frac{t-1}{2}+\frac{1}{2}} \;=\; n^{\frac{t}{2}}$$\\

\textbf{Note.}\, The absolute value of the product of all divisors is $n^{\tau(n)}.$



%%%%%
%%%%%
\end{document}
