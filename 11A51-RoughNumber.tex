\documentclass[12pt]{article}
\usepackage{pmmeta}
\pmcanonicalname{RoughNumber}
\pmcreated{2013-03-22 18:09:46}
\pmmodified{2013-03-22 18:09:46}
\pmowner{PrimeFan}{13766}
\pmmodifier{PrimeFan}{13766}
\pmtitle{rough number}
\pmrecord{7}{40722}
\pmprivacy{1}
\pmauthor{PrimeFan}{13766}
\pmtype{Definition}
\pmcomment{trigger rebuild}
\pmclassification{msc}{11A51}
\pmrelated{SmoothNumber}

% this is the default PlanetMath preamble.  as your knowledge
% of TeX increases, you will probably want to edit this, but
% it should be fine as is for beginners.

% almost certainly you want these
\usepackage{amssymb}
\usepackage{amsmath}
\usepackage{amsfonts}

% used for TeXing text within eps files
%\usepackage{psfrag}
% need this for including graphics (\includegraphics)
%\usepackage{graphicx}
% for neatly defining theorems and propositions
%\usepackage{amsthm}
% making logically defined graphics
%%%\usepackage{xypic}

% there are many more packages, add them here as you need them

% define commands here

\begin{document}
\PMlinkescapeword{addition}

A $k$-{\em rough number} is an integer $n$ having only prime factors greater than or equal to $k$. A number that is $k$-rough is thus not a $k$-smooth number. However, a prime $p$ is both $p$-smooth and $p$-rough because neither inequality is strict.

For example, 771166905830363 is 7-rough since, in addition to having 7 as a prime factor, it has greater primes as factors. By comparison, 93386641873154605056 is not 7-rough, and is in fact 7-smooth, since all its divisors are 2, 3,  and powers of those small primes.
%%%%%
%%%%%
\end{document}
