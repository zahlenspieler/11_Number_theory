\documentclass[12pt]{article}
\usepackage{pmmeta}
\pmcanonicalname{TranscendentalNumber}
\pmcreated{2013-03-22 11:55:54}
\pmmodified{2013-03-22 11:55:54}
\pmowner{yark}{2760}
\pmmodifier{yark}{2760}
\pmtitle{transcendental number}
\pmrecord{13}{30656}
\pmprivacy{1}
\pmauthor{yark}{2760}
\pmtype{Definition}
\pmcomment{trigger rebuild}
\pmclassification{msc}{11J82}
\pmclassification{msc}{11J81}
\pmrelated{Pi}
\pmrelated{Irrational}

\usepackage{amssymb}
\usepackage{amsmath}
\usepackage{amsfonts}

\def\Q{\mathbb{Q}}
\def\Z{\mathbb{Z}}
\begin{document}
A \emph{transcendental number} is a complex number
that is not an algebraic number.
That is, it is a complex number that is transcendental over $\Q$
(or, equivalently, over $\Z$).

Well known transcendental numbers include $\pi$ and $e$
(the base of natural logarithms).

Cantor showed that, in a sense,
``almost all'' complex numbers are transcendental:
there are uncountably many complex numbers, but
\PMlinkname{only countably many algebraic numbers}{AlgebraicNumbersAreCountable}.

%%%%%
%%%%%
%%%%%
%%%%%
\end{document}
