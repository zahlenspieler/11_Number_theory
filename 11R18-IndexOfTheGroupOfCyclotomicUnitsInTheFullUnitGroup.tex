\documentclass[12pt]{article}
\usepackage{pmmeta}
\pmcanonicalname{IndexOfTheGroupOfCyclotomicUnitsInTheFullUnitGroup}
\pmcreated{2013-03-22 15:42:49}
\pmmodified{2013-03-22 15:42:49}
\pmowner{alozano}{2414}
\pmmodifier{alozano}{2414}
\pmtitle{index of the group of cyclotomic units in the full unit group}
\pmrecord{4}{37661}
\pmprivacy{1}
\pmauthor{alozano}{2414}
\pmtype{Theorem}
\pmcomment{trigger rebuild}
\pmclassification{msc}{11R18}

% this is the default PlanetMath preamble.  as your knowledge
% of TeX increases, you will probably want to edit this, but
% it should be fine as is for beginners.

% almost certainly you want these
\usepackage{amssymb}
\usepackage{amsmath}
\usepackage{amsthm}
\usepackage{amsfonts}

% used for TeXing text within eps files
%\usepackage{psfrag}
% need this for including graphics (\includegraphics)
%\usepackage{graphicx}
% for neatly defining theorems and propositions
%\usepackage{amsthm}
% making logically defined graphics
%%%\usepackage{xypic}

% there are many more packages, add them here as you need them

% define commands here

\newtheorem{thm1}{Theorem}
\newtheorem{defn1}{Definition}
\newtheorem{prop1}{Proposition}
\newtheorem{lemma1}{Lemma}
\newtheorem{cor1}{Corollary}

\theoremstyle{definition}
\newtheorem{exa}{Example}
\newtheorem{remark}{Remark}

% Some sets
\newcommand{\Nats}{\mathbb{N}}
\newcommand{\Ints}{\mathbb{Z}}
\newcommand{\Reals}{\mathbb{R}}
\newcommand{\Complex}{\mathbb{C}}
\newcommand{\Rats}{\mathbb{Q}}
\newcommand{\Gal}{\operatorname{Gal}}
\newcommand{\Cl}{\operatorname{Cl}}
\begin{document}
Let $K_n=\Rats(\zeta_{p^n})$ where $\zeta_{p^n}$ is a primitive
$p^n$th root of unity, let $h_n$ be the class number of $K_n$ and
let $\mathcal{O}_n=\mathcal{O}_{K_n}$ be the ring of integers in
$K_n$. Let $E_n=\mathcal{O}_n^\times$ be the group of units in
$K_n$. The cyclotomic units are a subgroup $C_n$ of $E_n$ which
satisfy:
\begin{itemize}
\item The elements of $C_n$ are defined analytically.

\item The subgroup $C_n$ is of finite index in $E_n$. Furthermore,
the index is  $h_{n}^+$: Let $E^+_n$ be the group of units in
$K^+_n$ and let $C^+_n=C_n\cap E^+_n$. Then
$[E_n^+:C_n^+]=h_{n}^+$. Moreover, it can be shown that
$[E_n:C_n]=[E_n^+:C_n^+]$ because $E_n=\mu_{p^n}E_n^+$ (this is
exercise 8.5 in \cite{wash}).

\item The subgroups $C_n$ behave ``well'' in towers. More
precisely, the norm of $C_{n+1}$ down to $K_{n}$ is $C_{n}$. This
follows from the fact that the norm of $\zeta_{p^{n+1}}$ down to
$K_{n}$ is $\zeta_{p^n}$.
\end{itemize}

\begin{defn1}
Let $p$ be prime and let $n\geq 1$. Let $\zeta_{p^n}$ be a
primitive $p^n$th root of unity.
\begin{enumerate}
\item The cyclotomic unit group $C_n^+\subset
K_n^+=\Rats(\zeta_{p^n})^+$ is the group of units generated by
$-1$ and the units
$$\xi_a=\zeta_{p^n}^{(1-a)/2}\frac{1-\zeta_{p^n}^a}{1-\zeta_{p^n}}=\pm \frac{\sin(\pi a/p^n)}{\sin(\pi/p^n)}$$
with $1<a<\frac{p^n}{2}$ and $\gcd(a,p)=1$.

\item The cyclotomic unit group $C_n\subset
K_n=\Rats(\zeta_{p^n})$ is the group generated by $\zeta_{p^n}$
and the cyclotomic units $C_n^+$ of $K_n^+$.
\end{enumerate}
\end{defn1}
\begin{remark}
Let $\sigma_a:\zeta_{p^n}\to \zeta_{p^n}^a$ be an element of
$\Gal(K_n/\Rats)$. Then:
$$\xi_a=\zeta_{p^n}^{(1-a)/2}\frac{1-\zeta_{p^n}^a}{1-\zeta_{p^n}}=
\frac{(\zeta_{p^n}^{-1/2}(1-\zeta_{p^n}))^{\sigma_a}}{\zeta_{p^n}^{-1/2}(1-\zeta_{p^n})}.$$
\end{remark}
\begin{remark}
\label{rem1} Let $g$ be a primitive root modulo $p^n$. Let
$a\equiv g^r \mod p^n$. Then one can rewrite $\xi_a$ as:
$$\xi_a=
\prod_{i=0}^{r-1}\xi_g^{\sigma_g^i}.$$ In particular $\xi_g$
generates $C_n^+/\{\pm 1\}$ as a module over
$\Ints[\Gal(\Rats(\zeta_{p^n})^+/\Rats)]$.
\end{remark}

Notice that in order to show that the index of $C_n$ in $K_n$ is
finite it suffices to show that the index of $C_n^+$ in $K_n^+$ is
finite. Indeed, let $[K_n:\Rats]=2d$. Since $K_n$ is a totally
imaginary field and by Dirichlet's unit theorem the free rank of
$E_n$ is $r_1+r_2-1=d-1$. On the other hand, $[K_n^+:\Rats]=d$ and
$K_n^+$ is totally real, thus the free rank of $E_n^+$ is also
$d-1$. Therefore the free rank of $E_n^+$ and $E_n$ are equal. As
we claimed before, the index $[E_n^+:C_n^+]$ is rather interesting
to us.
\begin{thm1}[\cite{wash},Thm. 8.2]
\label{cycloindex} Let $p$ be a prime and $n\geq 1$. Let $h^+_{n}$
be the class number of $\Rats(\zeta_{p^n})^+$. The cyclotomic
units $C_{n}^+$ of $\Rats(\zeta_{p^n})^+$ are a subgroup of finite
index in the full unit group $E_{n}^+$. Furthermore:
$$h^+_{n}=[E_{n}^+:C^+_{n}]=[E_n:C_n].$$
\end{thm1}
In the proof of the previous theorem one calculates the regulator
of the units $\xi_a$ in terms of values of Dirichlet L-functions
with even characters. In particular, one calculates:
$$R(\{\xi_a\})=\pm\prod_{\chi\neq
\chi_0}\frac{1}{2}\tau(\chi)L(1,\overline{\chi})=h^+_{n}\cdot
R^+$$ where in the last equality one uses the properties of Gauss
sums and the class number formula in terms of Dirichlet
L-functions evaluated at $s=1$. This yields that $R(\{\xi_a\})$ in
non-zero, therefore the index in $E_n^+$ is finite and moreover
$$h^+_{n}=\frac{R(\{\xi_a\})}{R^+}=[E_{n}^+:C^+_{n}]=[E_n:C_n].$$
An immediate consequence of this is that if $p$ divides $h^+_{n}$
then there exists a cyclotomic unit $\gamma \in C_n^+$ such that
$\gamma$ is a $p$th power in $E_{n}^+$ but not in $C_n^+$.

\begin{thebibliography}{00}
\bibitem{wash} L. C. Washington, {\em Introduction to Cyclotomic
Fields}, Second Edition, Springer-Verlag, New York.

\end{thebibliography}
%%%%%
%%%%%
\end{document}
