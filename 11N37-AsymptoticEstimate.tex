\documentclass[12pt]{article}
\usepackage{pmmeta}
\pmcanonicalname{AsymptoticEstimate}
\pmcreated{2013-03-22 16:00:01}
\pmmodified{2013-03-22 16:00:01}
\pmowner{Wkbj79}{1863}
\pmmodifier{Wkbj79}{1863}
\pmtitle{asymptotic estimate}
\pmrecord{13}{38027}
\pmprivacy{1}
\pmauthor{Wkbj79}{1863}
\pmtype{Definition}
\pmcomment{trigger rebuild}
\pmclassification{msc}{11N37}
\pmrelated{AsymptoticEstimatesForRealValuedNonnegativeMultiplicativeFunctions}
\pmrelated{DisplaystyleYOmeganOleftFracxlogXy12YRightFor1LeY2}
\pmrelated{DisplaystyleXlog2xOleftsum_nLeX2OmeganRight}
\pmrelated{DisplaystyleSum_nLeXYomeganO_yxlogXy1ForYGe0}

% this is the default PlanetMath preamble.  as your knowledge
% of TeX increases, you will probably want to edit this, but
% it should be fine as is for beginners.

% almost certainly you want these
\usepackage{amssymb}
\usepackage{amsmath}
\usepackage{amsfonts}

% used for TeXing text within eps files
%\usepackage{psfrag}
% need this for including graphics (\includegraphics)
%\usepackage{graphicx}
% for neatly defining theorems and propositions
%\usepackage{amsthm}
% making logically defined graphics
%%%\usepackage{xypic}

% there are many more packages, add them here as you need them

% define commands here

\begin{document}
\PMlinkescapephrase{characteristic function}

An \emph{asymptotic estimate} is an \PMlinkescapetext{estimate} that involves the use of $O$, $o$, or $\sim$.  These are all defined in the entry Landau notation.  Examples of asymptotic \PMlinkescapetext{estimates} are:

\begin{center}
\begin{tabular}{rll}
$\displaystyle \sum_{n \le x} \mu^2(n)$ & $\displaystyle = \frac{6}{\pi^2}x+O(\sqrt{x})$ & (see convolution method for more details) \\
$\displaystyle \pi(x)$ & $\displaystyle \sim \frac{x}{\log x}$ & (see prime number theorem for more details) \end{tabular}
\end{center}

Unless otherwise specified, asymptotic \PMlinkescapetext{estimates} are typically valid for $x \to \infty$.  An example of an asymptotic \PMlinkescapetext{estimate} that is different from those above in this aspect is
\[
\cos x=1-\frac{x^2}{2}+O(x^4) \text{ for } |x|<1.
\]
Note that the above \PMlinkescapetext{estimate} would be undesirable for $x \to \infty$, as the \PMlinkescapetext{error} would be larger than the \PMlinkescapetext{estimate}.  Such is not the case for $|x|<1$, though.

Tools that are useful for obtaining asymptotic \PMlinkescapetext{estimates} include:
\begin{itemize}
\item the Euler-Maclaurin summation formula
\item Abel's lemma
\item the \PMlinkname{convolution method}{ConvolutionMethod}
\item the Dirichlet hyperbola method
\end{itemize}

If $A \subseteq \mathbb{N}$, then an asymptotic \PMlinkescapetext{estimate} for $\displaystyle \sum_{n \le x} \chi_A(x)$, where $\chi_A$ denotes the \PMlinkname{characteristic function}{CharacteristicFunction} of $A$, enables one to determine the asymptotic density of $A$ using the \PMlinkescapetext{formula}
\[
\lim_{x \to \infty} \frac{1}{x} \sum_{n \le x} \chi_A(x)
\]
provided the limit exists.  The upper asymptotic density of $A$ and the lower asymptotic density of $A$ can be computed in a \PMlinkescapetext{similar} manner using $\limsup$ and $\liminf$, respectively.  (See \PMlinkname{asymptotic density}{AsymptoticDensity} for more details.)

For example, $\mu^2$ is the characteristic function of the squarefree natural numbers.  Using the asymptotic \PMlinkescapetext{estimate} above yields the asymptotic density of the squarefree natural numbers:

\begin{center}
$\begin{array}{ll}
\displaystyle \lim_{x \to \infty} \frac{1}{x} \sum_{n \le x} \mu^2(n) & \displaystyle =\lim_{x \to \infty} \frac{1}{x} \left( \frac{6}{\pi^2}x+O(\sqrt{x}) \right) \\
\\
& \displaystyle =\lim_{x \to \infty} \frac{6}{\pi^2}+O \left( \frac{\sqrt{x}}{x} \right) \\
\\
& \displaystyle =\frac{6}{\pi^2} \end{array}$
\end{center}
%%%%%
%%%%%
\end{document}
