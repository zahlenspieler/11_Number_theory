\documentclass[12pt]{article}
\usepackage{pmmeta}
\pmcanonicalname{ExpressingFractionsInFactorialBase}
\pmcreated{2013-03-22 16:46:04}
\pmmodified{2013-03-22 16:46:04}
\pmowner{rspuzio}{6075}
\pmmodifier{rspuzio}{6075}
\pmtitle{expressing fractions in factorial base}
\pmrecord{13}{38995}
\pmprivacy{1}
\pmauthor{rspuzio}{6075}
\pmtype{Algorithm}
\pmcomment{trigger rebuild}
\pmclassification{msc}{11A63}

% this is the default PlanetMath preamble.  as your knowledge
% of TeX increases, you will probably want to edit this, but
% it should be fine as is for beginners.

% almost certainly you want these
\usepackage{amssymb}
\usepackage{amsmath}
\usepackage{amsfonts}

% used for TeXing text within eps files
%\usepackage{psfrag}
% need this for including graphics (\includegraphics)
%\usepackage{graphicx}
% for neatly defining theorems and propositions
%\usepackage{amsthm}
% making logically defined graphics
%%%\usepackage{xypic}

% there are many more packages, add them here as you need them

% define commands here

\begin{document}
Given a fraction $p/q$, one may re-express it in factorial
base as follows.  For simplicity, we assume that the 
fraction is already written in lowest terms, i.e. that
$p$ and $q$ have no common factor.  As in the \PMlinkname{parent}{FactorialBaseRepresentationOfFractions}
entry, we assume the both $p$ and $q$ are positive and
that $p < q$, so our fraction is less than 1.

To begin, determine the smallest integer $d(q)$ such that $q$ 
\PMlinkname{divides}{Divisibility} $d(q)!$.

Rewrite the fraction with $d(q)!$ as denominator:
\[
{p \over q} = {p \cdot d(q)!/q \over d(q)!}
\]

Successively split off terms as follows: given a
fraction $m/n!$, write\, $m = k \cdot n + r$\, where
$r < n$ and then write
\[
{m \over n!} := {k \over (n-1)!} + {r \over n!}
\]
Initially, we choose $m = p \cdot d(q)!/q$ and
$n = d(q)$.  At each successive repetition of the
procedure, set $m$ to be the value of $k$ from
the previous step and decrease $n$ by 1.

Let us illustrate this with an example.  We
will rewrite $7/8$ in factorial base.  Looking 
at factorials, we see that $8$ does not divide
either $2!$ or $3!$, but it does divide $4!$,
so we have $d(8) = 4$.

Next we rewrite the fraction with $4!$ as
denominator.  Since $4! = 24$ and $24/8 = 3$,
we should multiply both numerator and denominator
of our fraction by $3$ to obtain $7/8 = 21/24 = 
21/4!$.

Now, we split off terms.  We have $21 = 5 \cdot 4 +
1$, so
\[
\frac{21}{4!} = \frac{5}{3!} + \frac{1}{4!}.
\] 
Next, $5 = 1 \cdot 3 + 2$, so
\begin{eqnarray*}
\frac{5}{3!} &=& \frac{1}{2!} + \frac{2}{3!} \\
\frac{21}{4!} &=& \frac{1}{2!} + \frac{2}{3!} +
\frac{1}{4!}.
\end{eqnarray*}
Since $1 < 2$, we are done.  Thus, we see that the
factorial base representation of $7/8$ is 
$0 \, . \, 1 \, 2 \, 1$ as in the table in the 
parent entry.

We can make the calculation more concise by noting
that the splitting off of terms can also be described 
as follows:  the factorial base representation of
$m/n!$ is the factorial base representation of
$k/(n-1)!$ followed by $r$, where\, $m = k \cdot n
+ r$\, as above.  For example, let us compute the
factorial base representation of $7/9$ using this
trick.  Looking at factorials, we see that $d(9) = 6$.
We express our fraction with a denominator of $6! = 720$:
\[
\frac{7}{9} = \frac{80 \cdot 7}{80 \cdot 9} =
\frac{560}{720} = \frac{560}{6!}
\]
We now split of digits like follows:
\begin{eqnarray*}
\frac{560}{6!} &=& \frac{93}{5!} \, 2 \\
&=& \frac{18}{4!} \, 3 \, 2 \\
&=& \frac{4}{3!} \, 2 \, 3 \, 2 \\
&=& \frac{1}{2!} \, 1 \, 2 \, 3 \, 2 
\end{eqnarray*}
Hence, we see that the factorial base representation
of $7/9$ is $0 \, . \, 1 \, 1 \, 2 \, 3 \, 2$.

The following LISP program computes factorial base
representations of rational numbers using this method.
It was used to compute the table of factorial base
representations in the parent entry.

\begin{flushleft}
\texttt{
(defun factorial) (n) \\
\ \ (cond ((= n 0) 1) \\
\ \ \ \ \ \ \ \ (t (* n (factorial (- n 1))))))}
\end{flushleft}

\begin{flushleft}
\texttt{
(defun d-tilde (n m) \\
\ \ (cond ((= (\% (factorial m) n) 0) m) \\
\ \ \ \ \ \ \ \ (t (d-tilde n (+ m 1))))) }
\end{flushleft}

\begin{flushleft}
\texttt{
(defun d (n) (d-tilde n 1))}
\end{flushleft}

\begin{flushleft}
\texttt{
(defun s (p q) \\
\ \ (cond ((= q 1) nil) \\
\ \ \ \ \ \ \ \ (t (append \\
\ \ \ \ \ \ \ \ \ \ \ \ (s (/ p q) (- q 1)) \\
\ \ \ \ \ \ \ \ \ \ \ \ (list (\% p q)))))) }
\end{flushleft}

\begin{flushleft}
\texttt{
(defun r (p q)
\ \ (s \\
\ \ \ (* p (/ (f (d q)) q)) \\
\ \ \ (d q))) }
\end{flushleft}

Since LISP can look rather confusing to someone who is not familiar
with its notational conventions --- all functions are written as
prefixes and parentheses are used a bit differently than usually, 
a translation into more typical mathematical notation may be helpful.  
To do this, let us define a few symbols.  Let `$\ast$' denote the 
concatenaton of tuplets --- given an $n$-tuplet $(x1, x_2, \ldots, x_n)$
and an $m$-tuplet $(y_1, y_2, \ldots, y_m)$, we set
$(x1, x_2, \ldots, x_n) \ast (y_1, y_2, \ldots, y_m)$ to be the
$m+n$-tuplet $(x1, x_2, \ldots, x_n, y_1, y_2, \ldots y_m)$.
Let ``$m \div n$'' denote the integer part of $m/n$ and let
``$m \operatorname{\%} n$ denote the remainder of dividing $m$ by $n$.  For
instance, $13 \div 5 = 2$ and $13 \operatorname{\%} 5 = 3$.  With these notational
conventions, we may re-express our program as follows:

\[
n! := 
\begin{cases}
1 & n = 0 \\ 
n \cdot (n-1)! & \textrm{otherwise} 
\end{cases}
\]

\[
{\tilde d} (n,m) := 
\begin{cases}
m & m! \operatorname{\%} n = 0 \\
{\tilde d} (n, m+1) & \textrm{otherwise} 
\end{cases}
\]

\[
d(n) := {\tilde d} (n,1)
\]

\[
s(p,q) := 
\begin{cases}
() & q = 1 \\
s (p \div q, q - 1) \ast (p \operatorname{\%} q) &
\mathrm{otherwise}
\end{cases}
\]

\[
r(p,q) := s (p \cdot f(d(q)) / q, d(q)
\]

%%%%%
%%%%%
\end{document}
