\documentclass[12pt]{article}
\usepackage{pmmeta}
\pmcanonicalname{RationalAlgebraicIntegers}
\pmcreated{2013-03-22 19:07:34}
\pmmodified{2013-03-22 19:07:34}
\pmowner{pahio}{2872}
\pmmodifier{pahio}{2872}
\pmtitle{rational algebraic integers}
\pmrecord{5}{42021}
\pmprivacy{1}
\pmauthor{pahio}{2872}
\pmtype{Theorem}
\pmcomment{trigger rebuild}
\pmclassification{msc}{11R04}
\pmrelated{MultiplesOfAnAlgebraicNumber}

\endmetadata

% this is the default PlanetMath preamble.  as your knowledge
% of TeX increases, you will probably want to edit this, but
% it should be fine as is for beginners.

% almost certainly you want these
\usepackage{amssymb}
\usepackage{amsmath}
\usepackage{amsfonts}

% used for TeXing text within eps files
%\usepackage{psfrag}
% need this for including graphics (\includegraphics)
%\usepackage{graphicx}
% for neatly defining theorems and propositions
 \usepackage{amsthm}
% making logically defined graphics
%%%\usepackage{xypic}

% there are many more packages, add them here as you need them

% define commands here

\theoremstyle{definition}
\newtheorem*{thmplain}{Theorem}

\begin{document}
\textbf{Theorem.}\, A rational number is an algebraic integer iff it is a rational integer.\\

\emph{Proof.}\, $1^\circ$.\, Any rational integer $m$ has the minimal polynomial $x\!-\!m$, whence it is an algebraic integer.\\
$2^\circ$.\, Let the rational number\, $\alpha = \frac{m}{n}$\, be an algebraic integer where $m,\,n$ are coprime integers and\, $n > 0$.\, Then there is a polynomial
$$f(x) \;=\; x^k+a_1x^{k-1}+\ldots+a_k$$
with\, $a_1,\,\ldots,\,a_k \in \mathbb{Z}$\, such that
$$f(\alpha) \;=\; \left(\frac{m}{n}\right)^k+a_1\left(\frac{m}{n}\right)^{k-1}+\ldots+a_k \;=\; 0.$$
Multiplying this equation termwise by $n^k$ implies
$$m^k \;=\; -a_1m^{k-1}n-\ldots-a_kn^k,$$
which says that\, $n \mid m^k$ (see divisibility in rings).\, Since $m$ and $n$ are coprime and $n$ positive, it follows that\, $n = 1$.\, Therefore,\, $\alpha = m \in \mathbb{Z}$.

%%%%%
%%%%%
\end{document}
