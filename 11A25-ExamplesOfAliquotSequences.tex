\documentclass[12pt]{article}
\usepackage{pmmeta}
\pmcanonicalname{ExamplesOfAliquotSequences}
\pmcreated{2013-03-22 16:09:21}
\pmmodified{2013-03-22 16:09:21}
\pmowner{PrimeFan}{13766}
\pmmodifier{PrimeFan}{13766}
\pmtitle{examples of aliquot sequences}
\pmrecord{6}{38237}
\pmprivacy{1}
\pmauthor{PrimeFan}{13766}
\pmtype{Example}
\pmcomment{trigger rebuild}
\pmclassification{msc}{11A25}

\endmetadata

% this is the default PlanetMath preamble.  as your knowledge
% of TeX increases, you will probably want to edit this, but
% it should be fine as is for beginners.

% almost certainly you want these
\usepackage{amssymb}
\usepackage{amsmath}
\usepackage{amsfonts}

% used for TeXing text within eps files
%\usepackage{psfrag}
% need this for including graphics (\includegraphics)
%\usepackage{graphicx}
% for neatly defining theorems and propositions
%\usepackage{amsthm}
% making logically defined graphics
%%%\usepackage{xypic}

% there are many more packages, add them here as you need them

% define commands here

\begin{document}
The shortest aliquot sequence with distinct elements is that for 1, simply: 1, 0.

A prime number $p > 1$ has an aliquot sequences that is almost as short, namely: $p$, 1, 0. For example, 47, 1, 0.

Technically, all aliquot sequences are infinite, but some cease to be interesting sooner than others. Throughout this article, length will refer to the length of the aliquot sequence from its first element up to the first instance of the fixed point.

Some aliquot sequences zag all over the place before finally settling down on 0. The aliquot sequence for 60, for example: 60, 108, 172, 136, 134, 70, 74, 40, 50, 43, 1, 0. Or for a more dramatic example: 138, 150, 222, 234, 312, 528, 960, 2088, 3762, 5598, 6570, 10746, 13254, 13830, 19434, 20886, 21606, 25098, 26742, 26754, 40446, 63234, 77406, 110754, 171486, 253458, 295740, 647748, 1077612, 1467588, 1956812, 2109796, 1889486, 953914, 668966, 353578, 176792, 254128, 308832, 502104, 753216, 1240176, 2422288, 2697920, 3727264, 3655076, 2760844, 2100740, 2310856, 2455544, 3212776, 3751064, 3282196, 2723020, 3035684, 2299240, 2988440, 5297320, 8325080, 11222920, 15359480, 19199440, 28875608, 25266172, 19406148, 26552604, 40541052, 54202884, 72270540, 147793668, 228408732, 348957876, 508132204, 404465636, 303708376, 290504024, 312058216, 294959384, 290622016, 286081174, 151737434, 75868720, 108199856, 101437396, 76247552, 76099654, 42387146, 21679318, 12752594, 7278382, 3660794, 1855066, 927536, 932464, 1013592, 1546008, 2425752, 5084088, 8436192, 13709064, 20563656, 33082104, 57142536, 99483384, 245978376, 487384824, 745600776, 1118401224, 1677601896, 2538372504, 4119772776, 8030724504, 14097017496, 21148436904, 40381357656, 60572036544, 100039354704, 179931895322, 94685963278, 51399021218, 28358080762, 18046051430, 17396081338, 8698040672, 8426226964, 6319670230, 5422685354, 3217383766, 1739126474, 996366646, 636221402, 318217798, 195756362, 101900794, 54202694, 49799866, 24930374, 17971642, 11130830, 8904682, 4913018, 3126502, 1574810, 1473382, 736694, 541162, 312470, 249994, 127286, 69898, 34952, 34708, 26038, 13994, 7000, 11720, 14740, 19532, 16588, 
18692, 14026, 7016, 6154, 3674, 2374, 1190, 1402, 704, 820, 944, 916, 694, 350, 394, 200, 265, 59, 1, 0.

Despite its length, 138's aliquot sequence does eventually settle down. One doesn't have to look too far up to find a much, much longer aliquot sequence: 276, 396, 696, 1104, 1872, 3770, 3790, 3050, 2716, 2772, 5964, 10164, 19628, 19684, 22876, 26404, 30044, 33796, 38780, 54628, 54684, 111300, 263676, 465668, 465724, 465780, 1026060, 2325540, 5335260, 11738916, 23117724, 45956820, 121129260, 266485716, 558454764, 1092873236, 1470806764, 1471882804, 1642613196, 2737688884, 2740114636, 2791337780, 4946860492, 4946860548, 9344070652, 9344070708, 15573451404, 27078171764, 27284104204, 27410152084, 27410152140, 76787720100, 220578719452, 254903331620, 361672366300, 603062136740, 921203207260, 1381419996068, 1395444575644, 1395478688996, 1395546402460, 2069258468900, 3065057872156, 3277068463844, 
3429776547484, 3597527970596, 4028517592540, 5641400009252, 5641400009308, 5641400009364, 9709348326636, 16331909651988, 31948891146732, 54770416120644, 100509779504316, 208751080955844, 388416032284476, 749365894850244, 1414070378301756, 2556878765995204, 2556878765995260, 6726041614128900, 15626498692840700, 23762659088671300, 35168735451235260, 78257359358590020, 186897487211247036, 340813223900632644, 592585414385033916, 1326583294186844484, 2594892903616159356, 4946738730471899844, 8244565422068579772, 13740942370114299844, 13780400058385352252, 13780400058385352308, 14272557426581383244, 14272557426581383300, 21155073391000330684, 21374326697892540932, 22138822441861473292, ... (A008892 of Sloane's OEIS provides links to far more extensive listings). It keeps going like that for at least another thousand terms, and the numbers get so large that their factorizations begin to take noticeable amounts of time. Even so, this aliquot sequence is not in strict ascending order. In fact, it is an open question if it's even possible for an aliquot sequence to be in ascending order.

However, there are aliquot sequences in descending order: 135, 105, 87, 33, 15, 9, 4, 3, 1, 0.

In exploring aliquot sequences with a computer, one needs to be careful not just with aliquot sequences of unknown length, but also with those where even the human operator already knows how short the sequence is. Even a sophisticated computer algebra system like Mathematica can get stuck on the aliquot sequence for 220 or 284 (amicable numbers) if the operator neglects to program in the ability to recognize cycles.
%%%%%
%%%%%
\end{document}
