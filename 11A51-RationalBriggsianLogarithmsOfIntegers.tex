\documentclass[12pt]{article}
\usepackage{pmmeta}
\pmcanonicalname{RationalBriggsianLogarithmsOfIntegers}
\pmcreated{2013-03-22 17:39:55}
\pmmodified{2013-03-22 17:39:55}
\pmowner{pahio}{2872}
\pmmodifier{pahio}{2872}
\pmtitle{rational Briggsian logarithms of integers}
\pmrecord{14}{40101}
\pmprivacy{1}
\pmauthor{pahio}{2872}
\pmtype{Theorem}
\pmcomment{trigger rebuild}
\pmclassification{msc}{11A51}
\pmrelated{Transcendental}
\pmrelated{RationalSineAndCosine}
\pmrelated{AllUnnaturalSquareRootsAreIrrational}
\pmrelated{BriggsianLogarithms}

\endmetadata

% this is the default PlanetMath preamble.  as your knowledge
% of TeX increases, you will probably want to edit this, but
% it should be fine as is for beginners.

% almost certainly you want these
\usepackage{amssymb}
\usepackage{amsmath}
\usepackage{amsfonts}

% used for TeXing text within eps files
%\usepackage{psfrag}
% need this for including graphics (\includegraphics)
%\usepackage{graphicx}
% for neatly defining theorems and propositions
 \usepackage{amsthm}
% making logically defined graphics
%%%\usepackage{xypic}

% there are many more packages, add them here as you need them

% define commands here

\theoremstyle{definition}
\newtheorem*{thmplain}{Theorem}
\DeclareMathOperator{\lb}{lb}
\begin{document}
\textbf{Theorem.}\, The only positive integers, whose Briggsian logarithms are rational, are the \PMlinkname{powers}{GeneralAssociativity}\, $1,\,10,\,100,\,\ldots$\, of ten.\, The logarithms of other positive integers are thus irrational (in fact, transcendental numbers).\, The same concerns also the Briggsian logarithms of the positive fractional numbers.\\

{\em Proof.}\, Let $a$ be a positive integer such that
$$\lg{a} = \frac{m}{n} \in \mathbb{Q},$$
where $m$ and $n$ are positive integers.\, By the definition of logarithm, we have\, 
$\displaystyle10^{\frac{m}{n}} = a$,\, which is \PMlinkname{equivalent}{Equivalent3} to
$$10^m = 2^m\cdot 5^m = a^n.$$
According to the fundamental theorem of arithmetics, the integer $a^n$ must have exactly $m$ prime divisors $2$ and equally many prime divisors $5$.\, This is not possible otherwise than that also $a$ itself consists of a same amount of prime divisors 2 and 5, i.e. the number $a$ is an integer power of 10.

As for any rational number $\displaystyle\frac{a}{b}$ (with\, $a,\,b \in \mathbb{Z}_+$), if one had
$$\lg{\frac{a}{b}} = \frac{m}{n} \in \mathbb{Q},$$
then
$$\left(\frac{a}{b}\right)^n = 10^m,$$
and it is apparent that the rational number $\displaystyle\frac{a}{b}$ has to be an integer, more accurately a power of ten.\, Therefore the logarithms of all fractional numbers are irrational.\\



\textbf{Note.}\, An analogous theorem concerns e.g. the binary logarithms ($\lb{a}$).\, As for the natural logarithms of positive rationals ($\ln{a}$), they all are transcendental numbers except\, $\ln1 = 0$.
%%%%%
%%%%%
\end{document}
