\documentclass[12pt]{article}
\usepackage{pmmeta}
\pmcanonicalname{ProofThatTheSumOfTheIteratedTotientFunctionIsAlwaysOdd}
\pmcreated{2013-03-22 16:34:26}
\pmmodified{2013-03-22 16:34:26}
\pmowner{PrimeFan}{13766}
\pmmodifier{PrimeFan}{13766}
\pmtitle{proof that the sum of the iterated totient function is always odd}
\pmrecord{6}{38764}
\pmprivacy{1}
\pmauthor{PrimeFan}{13766}
\pmtype{Proof}
\pmcomment{trigger rebuild}
\pmclassification{msc}{11A25}

\endmetadata

% this is the default PlanetMath preamble.  as your knowledge
% of TeX increases, you will probably want to edit this, but
% it should be fine as is for beginners.

% almost certainly you want these
\usepackage{amssymb}
\usepackage{amsmath}
\usepackage{amsfonts}

% used for TeXing text within eps files
%\usepackage{psfrag}
% need this for including graphics (\includegraphics)
%\usepackage{graphicx}
% for neatly defining theorems and propositions
%\usepackage{amsthm}
% making logically defined graphics
%%%\usepackage{xypic}

% there are many more packages, add them here as you need them

% define commands here

\begin{document}
Given a positive integer $n$, it is always the case that $$2 \not\vert \sum_{i = 1}^{c + 1} \phi^i(n),$$ where $\phi^i(x)$ is the iterated totient function and $c$ is the integer such that $\phi^c(n) = 2$.

Accepting as proven that $n > \phi(n)$ and $2 | \phi(n)$ for $n > 2$, it is clear that summing up the iterates of the totient function up to $c$ is summing up a series of even numbers in descending order and that this sum is therefore itself even. Then, when we add the $c + 1$ iterate, the sum turns odd.

As a bonus, this proves that no even number can be a perfect totient number.
%%%%%
%%%%%
\end{document}
