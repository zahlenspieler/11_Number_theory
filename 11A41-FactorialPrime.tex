\documentclass[12pt]{article}
\usepackage{pmmeta}
\pmcanonicalname{FactorialPrime}
\pmcreated{2013-03-22 16:19:19}
\pmmodified{2013-03-22 16:19:19}
\pmowner{PrimeFan}{13766}
\pmmodifier{PrimeFan}{13766}
\pmtitle{factorial prime}
\pmrecord{6}{38449}
\pmprivacy{1}
\pmauthor{PrimeFan}{13766}
\pmtype{Definition}
\pmcomment{trigger rebuild}
\pmclassification{msc}{11A41}
\pmclassification{msc}{05A10}
\pmclassification{msc}{11B65}

\endmetadata

% this is the default PlanetMath preamble.  as your knowledge
% of TeX increases, you will probably want to edit this, but
% it should be fine as is for beginners.

% almost certainly you want these
\usepackage{amssymb}
\usepackage{amsmath}
\usepackage{amsfonts}

% used for TeXing text within eps files
%\usepackage{psfrag}
% need this for including graphics (\includegraphics)
%\usepackage{graphicx}
% for neatly defining theorems and propositions
%\usepackage{amsthm}
% making logically defined graphics
%%%\usepackage{xypic}

% there are many more packages, add them here as you need them

% define commands here

\begin{document}
A {\em factorial prime} is a number that is one less or one more than a factorial and is also a prime number. The first few factorial primes are: 2, 3, 5, 7, 23, 719, 5039, 39916801, 479001599, 87178291199 (sequence A088054 in the OEIS). It is conjectured that only for $n = 3$ are both $n! - 1$ and $n! + 1$ both primes.

Factorial primes have a r\^ole in an argument that 1 is not a prime number. If $n$ is a positive integer and $p$ is a prime number, $n! + p$ is never a prime for $p < n$, because obviously it will be a multiple of $p$, just as $n!$ is. But $n! + 1$, even though it certainly is a multiple of 1, can be a prime, specifically, a factorial prime. (The same is also true if we subtract instead of add).
%%%%%
%%%%%
\end{document}
