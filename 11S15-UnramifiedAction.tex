\documentclass[12pt]{article}
\usepackage{pmmeta}
\pmcanonicalname{UnramifiedAction}
\pmcreated{2013-03-22 13:56:26}
\pmmodified{2013-03-22 13:56:26}
\pmowner{alozano}{2414}
\pmmodifier{alozano}{2414}
\pmtitle{unramified action}
\pmrecord{5}{34702}
\pmprivacy{1}
\pmauthor{alozano}{2414}
\pmtype{Definition}
\pmcomment{trigger rebuild}
\pmclassification{msc}{11S15}
\pmsynonym{set is unramified at a valuation}{UnramifiedAction}
%\pmkeywords{unramifies}
%\pmkeywords{action}
%\pmkeywords{inertia}
%\pmkeywords{valuation}
\pmrelated{InfiniteGaloisTheory}
\pmrelated{DecompositionGroup}
\pmrelated{Valuation}

% this is the default PlanetMath preamble.  as your knowledge
% of TeX increases, you will probably want to edit this, but
% it should be fine as is for beginners.

% almost certainly you want these
\usepackage{amssymb}
\usepackage{amsmath}
\usepackage{amsthm}
\usepackage{amsfonts}

% used for TeXing text within eps files
%\usepackage{psfrag}
% need this for including graphics (\includegraphics)
%\usepackage{graphicx}
% for neatly defining theorems and propositions
%\usepackage{amsthm}
% making logically defined graphics
%%%\usepackage{xypic}

% there are many more packages, add them here as you need them

% define commands here

\newtheorem{thm}{Theorem}
\newtheorem{defn}{Definition}
\newtheorem{prop}{Proposition}
\newtheorem{lemma}{Lemma}
\newtheorem{cor}{Corollary}

% Some sets
\newcommand{\Nats}{\mathbb{N}}
\newcommand{\Ints}{\mathbb{Z}}
\newcommand{\Reals}{\mathbb{R}}
\newcommand{\Complex}{\mathbb{C}}
\newcommand{\Rats}{\mathbb{Q}}
\begin{document}
Let $K$ be a number field and let $\nu$ be a discrete valuation on
$K$ (this might be, for example, the valuation attached to a prime
ideal $\mathfrak{P}$ of $K$).

Let $K_{\nu}$ be the completion of $K$ at $\nu$, and let
$\mathcal{O}_{\nu}$ be the ring of integers of $K_{\nu}$, i.e.
$$\mathcal{O}_{\nu}=\{ k\in K_{\nu} \mid \nu(k)\geq 0 \}$$
The maximal ideal of $\mathcal{O}_{\nu}$ will be denoted by
$$\mathcal{M}=\{ k\in K_{\nu} \mid \nu(k)>0 \}$$
and we denote by $k_{\nu}$ the residue field of $K_{\nu}$, which
is
$$k_{\nu}=\mathcal{O}_{\nu}/\mathcal{M}$$
We will consider three different global Galois groups, namely
$$G_{\overline{K}/K}=\operatorname{Gal}(\overline{K}/K)$$
$$G_{\overline{K_{\nu}}/K_{\nu}}=\operatorname{Gal}(\overline{K_{\nu}}/K_{\nu})$$
$$G_{\overline{k_{\nu}}/k_{\nu}}=\operatorname{Gal}(\overline{k_{\nu}}/k_{\nu})$$
where $\overline{K},\overline{K_{\nu}},\overline{k_{\nu}}$ are \PMlinkescapetext{separable} algebraic closures of the corresponding field. We also
define notation for the inertia group of
$G_{\overline{K_{\nu}}/K_{\nu}}$
$$I_{\nu} \subseteq G_{\overline{K_{\nu}}/K_{\nu}}$$

\begin{defn}
Let $\mathcal{S}$ be a set and suppose there is a group action of
$Gal(\overline{K_{\nu}}/K_{\nu})$ on $\mathcal{S}$. We say that
$\mathcal{S}$ is \emph{unramified} at $\nu$, or the action of
$G_{\overline{K_{\nu}}/K_{\nu}}$ on $\mathcal{S}$ is unramified at
$\nu$, if the action of $I_{\nu}$ on $\mathcal{S}$ is trivial,
i.e.
$$\sigma(s)=s\quad \forall \sigma \in I_{\nu},\quad \forall s\in \mathcal{S}$$
\end{defn}

{\bf Remark}: By Galois theory we know that,
$K_{\nu}^{\operatorname{nr}}$, the fixed field of $I_{\nu}$, the
inertia subgroup, is the maximal unramified extension of
$K_{\nu}$, so
$$I_{\nu}\cong
\operatorname{Gal}(\overline{K_{\nu}}/K_{\nu}^{\operatorname{nr}})$$
%%%%%
%%%%%
\end{document}
