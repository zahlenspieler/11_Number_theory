\documentclass[12pt]{article}
\usepackage{pmmeta}
\pmcanonicalname{PushdownTheoremOnClassNumbers}
\pmcreated{2013-03-22 15:05:19}
\pmmodified{2013-03-22 15:05:19}
\pmowner{alozano}{2414}
\pmmodifier{alozano}{2414}
\pmtitle{push-down theorem on class numbers}
\pmrecord{6}{36815}
\pmprivacy{1}
\pmauthor{alozano}{2414}
\pmtype{Theorem}
\pmcomment{trigger rebuild}
\pmclassification{msc}{11R37}
\pmclassification{msc}{11R32}
\pmclassification{msc}{11R29}
\pmrelated{IdealClass}
\pmrelated{PExtension}
\pmrelated{ExtensionsWithoutUnramifiedSubextensionsAndClassNumberDivisibility}
\pmrelated{ClassNumberDivisibilityInPExtensions}
\pmrelated{ClassNumbersAndDiscriminantsTopicsOnClassGroups}

% this is the default PlanetMath preamble.  as your knowledge
% of TeX increases, you will probably want to edit this, but
% it should be fine as is for beginners.

% almost certainly you want these
\usepackage{amssymb}
\usepackage{amsmath}
\usepackage{amsthm}
\usepackage{amsfonts}

% used for TeXing text within eps files
%\usepackage{psfrag}
% need this for including graphics (\includegraphics)
%\usepackage{graphicx}
% for neatly defining theorems and propositions
%\usepackage{amsthm}
% making logically defined graphics
%%%\usepackage{xypic}

% there are many more packages, add them here as you need them

% define commands here

\newtheorem*{thm}{Theorem}
\newtheorem{defn}{Definition}
\newtheorem{prop}{Proposition}
\newtheorem{lemma}{Lemma}
\newtheorem{cor}{Corollary}

% Some sets
\newcommand{\Nats}{\mathbb{N}}
\newcommand{\Ints}{\mathbb{Z}}
\newcommand{\Reals}{\mathbb{R}}
\newcommand{\Complex}{\mathbb{C}}
\newcommand{\Rats}{\mathbb{Q}}
\begin{document}
As in the \PMlinkname{parent}{ClassNumberDivisibilityInExtensions} entry, given a number field $K$, the class number of $K$ is denoted by $h_K$.

\begin{thm}[Pushing-Down Theorem]
Let $E/F$ be a $p$-extension of number fields and suppose that only one prime ideal of $F$ is ramified in $E$ and that this prime is totally ramified. Then $p|h_E$ implies $p|h_F$.
\end{thm}

\begin{thebibliography}{HGM02} % '2nd argument contains the widest acronym'

\bibitem[Fr\"oh]{hgm}
A. Fr\"ohlich, \emph{On a method for the determination of class number factors in number fields}, Mathematika, 4 (1957), 113-121.

\bibitem[Iwas]{green}
K. Iwasawa, \emph{A note on Class Numbers of Algebraic Number Fields}, Abh. Math. Sem. Univ. Hamburg, 20 (1956), 257-258.

\end{thebibliography}
%%%%%
%%%%%
\end{document}
