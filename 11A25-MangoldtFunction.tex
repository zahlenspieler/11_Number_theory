\documentclass[12pt]{article}
\usepackage{pmmeta}
\pmcanonicalname{MangoldtFunction}
\pmcreated{2013-03-22 11:47:06}
\pmmodified{2013-03-22 11:47:06}
\pmowner{KimJ}{5}
\pmmodifier{KimJ}{5}
\pmtitle{Mangoldt function}
\pmrecord{10}{30256}
\pmprivacy{1}
\pmauthor{KimJ}{5}
\pmtype{Definition}
\pmcomment{trigger rebuild}
\pmclassification{msc}{11A25}
\pmclassification{msc}{18E30}
\pmclassification{msc}{81-00}
\pmsynonym{von Mangoldt function}{MangoldtFunction}
%\pmkeywords{number theory}

\endmetadata

\usepackage{amssymb}
\usepackage{amsmath}
\usepackage{amsfonts}
\usepackage{graphicx}
%%%%\usepackage{xypic}
\begin{document}
The Mangoldt function $\Lambda$ is defined by
\[ \Lambda (n) = 
\begin{cases}
\ln p, &\text{if $n=p^k$, where $p$ is a prime and $k$ is a natural number $\geq 1$}\\
0, &\text{otherwise}
\end{cases}
\]

The Moebius Inversion Formula leads to the identity $\Lambda (n) = \sum_{d|n} \mu (n/d) \ln d = - \sum_{d|n} \mu (d) \ln d$.
%%%%%
%%%%%
%%%%%
%%%%%
\end{document}
