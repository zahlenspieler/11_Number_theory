\documentclass[12pt]{article}
\usepackage{pmmeta}
\pmcanonicalname{TheCyclotomicUnitsAreAlgebraicUnits}
\pmcreated{2013-03-22 14:13:17}
\pmmodified{2013-03-22 14:13:17}
\pmowner{alozano}{2414}
\pmmodifier{alozano}{2414}
\pmtitle{the cyclotomic units are algebraic units}
\pmrecord{4}{35657}
\pmprivacy{1}
\pmauthor{alozano}{2414}
\pmtype{Theorem}
\pmcomment{trigger rebuild}
\pmclassification{msc}{11R18}
\pmrelated{AlgebraicInteger}

\endmetadata

% this is the default PlanetMath preamble.  as your knowledge
% of TeX increases, you will probably want to edit this, but
% it should be fine as is for beginners.

% almost certainly you want these
\usepackage{amssymb}
\usepackage{amsmath}
\usepackage{amsthm}
\usepackage{amsfonts}

% used for TeXing text within eps files
%\usepackage{psfrag}
% need this for including graphics (\includegraphics)
%\usepackage{graphicx}
% for neatly defining theorems and propositions
%\usepackage{amsthm}
% making logically defined graphics
%%%\usepackage{xypic}

% there are many more packages, add them here as you need them

% define commands here

\newtheorem{thm}{Theorem}
\newtheorem{defn}{Definition}
\newtheorem{prop}{Proposition}
\newtheorem{lemma}{Lemma}
\newtheorem{cor}{Corollary}

% Some sets
\newcommand{\Nats}{\mathbb{N}}
\newcommand{\Ints}{\mathbb{Z}}
\newcommand{\Reals}{\mathbb{R}}
\newcommand{\Complex}{\mathbb{C}}
\newcommand{\Rats}{\mathbb{Q}}
\begin{document}
Let $L=\Rats(\zeta_m)$ be a cyclotomic extension of $\Rats$ with $m$ chosen to be minimal and let $\mathcal{O}_L$ be the ring of integers ($=\Ints(\zeta_m)$), recall that \emph{the cyclotomic units} are the elements of the form
\begin{align*}
\eta=\frac{\zeta^r-1}{\zeta^s-1}
\end{align*}
with $r$ and $s$ relatively prime to $m$ (where $\zeta=\zeta_m$). Here we prove that these elements are indeed algebraic units, i.e. $\eta \in \mathcal{O}_L^\times$.

\begin{lemma}
The cyclotomic units are algebraic units. 
\end{lemma}
\begin{proof}
In order to prove the lemma, we will check that both $\eta$ and $\eta^{-1}$ are algebraic integers, thus $\eta$ is a unit. Notice that it suffices to prove that $\eta$ is an algebraic integer, because the rest follows from interchanging the role of $r$ and $s$.

Let $r,s\in \Ints$ be relatively prime to $m$, thus $r \mod m, s\mod m$ are units in $\Ints/m\Ints$ and we can find an integer $a$ such that:
$$a\cdot s \equiv r \mod m$$ 
Note also that it follows that $\zeta^r=\zeta^{as}$. Moreover, using the equality of polynomials:
$$x^{as}-1=(x^s-1)\cdot(x^{s(a-1)}+x^{s(a-2)}+\ldots+x^s+1)$$
we get:
\begin{eqnarray*}
\eta &=& \frac{\zeta^r-1}{\zeta^s-1}=\frac{\zeta^{as}-1}{\zeta^s-1}\\
&=& \frac{(\zeta^s-1)\cdot(\zeta^{s(a-1)}+\zeta^{s(a-2)}+\ldots+\zeta^s+1)}{\zeta^s-1}\\
&=& \zeta^{s(a-1)}+\zeta^{s(a-2)}+\ldots+\zeta^s+1 \in \mathcal{O}_L=\Ints[\zeta]
\end{eqnarray*}
Hence the result.
\end{proof}
%%%%%
%%%%%
\end{document}
