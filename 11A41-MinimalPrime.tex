\documentclass[12pt]{article}
\usepackage{pmmeta}
\pmcanonicalname{MinimalPrime}
\pmcreated{2013-03-22 16:52:23}
\pmmodified{2013-03-22 16:52:23}
\pmowner{PrimeFan}{13766}
\pmmodifier{PrimeFan}{13766}
\pmtitle{minimal prime}
\pmrecord{4}{39123}
\pmprivacy{1}
\pmauthor{PrimeFan}{13766}
\pmtype{Definition}
\pmcomment{trigger rebuild}
\pmclassification{msc}{11A41}
\pmclassification{msc}{11A63}

% this is the default PlanetMath preamble.  as your knowledge
% of TeX increases, you will probably want to edit this, but
% it should be fine as is for beginners.

% almost certainly you want these
\usepackage{amssymb}
\usepackage{amsmath}
\usepackage{amsfonts}

% used for TeXing text within eps files
%\usepackage{psfrag}
% need this for including graphics (\includegraphics)
%\usepackage{graphicx}
% for neatly defining theorems and propositions
%\usepackage{amsthm}
% making logically defined graphics
%%%\usepackage{xypic}

% there are many more packages, add them here as you need them

% define commands here

\begin{document}
A {\em minimal prime} is a prime number $p$ that when written in a given base $b$, no smaller prime $q < p$ can be formed from a substring of the digits of $p$ (the digits need not be consecutive, but they must be in the same order). For example, in base 10, the prime 991 is a minimal prime because all of its possible substrings (9, 9, 1, 99, 91, 91) are either composite or not considered prime. A071062 of Sloane's OEIS lists the twenty-six base 10 minimal primes.

Clearly, all primes $p < b$ are minimal primes in that base. Such primes are obviously finite, but so are those minimal primes $p > b$, per Michel Lothaire's findings. In binary, there are only exactly two minimal primes: 2 and 3, written 10 and 11 respectively. Every larger prime will have 1 as its most significant digit and possibly a 0 somewhere; the 1 and 0 can then be brought together to form 10 (2 in decimal). The exception to this are the Mersenne primes $2^q - 1$ (or binary repunits), but it is even more elegant to prove these are not minimal primes in binary: they contain all smaller Mersenne primes as substrings!

\begin{thebibliography}{1}
\bibitem{ml} M. Lothaire ``Combinatorics on words'' in {\it Encylopedia of mathematics and its applications} {\bf 17} New York: Addison-Wesley (1983): 238 - 247
\end{thebibliography}
%%%%%
%%%%%
\end{document}
