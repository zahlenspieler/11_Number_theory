\documentclass[12pt]{article}
\usepackage{pmmeta}
\pmcanonicalname{PropertiesOfmathbbQvarthetaconjugates}
\pmcreated{2013-03-22 19:09:14}
\pmmodified{2013-03-22 19:09:14}
\pmowner{pahio}{2872}
\pmmodifier{pahio}{2872}
\pmtitle{properties of $\mathbb{Q}(\vartheta)$-conjugates}
\pmrecord{7}{42058}
\pmprivacy{1}
\pmauthor{pahio}{2872}
\pmtype{Topic}
\pmcomment{trigger rebuild}
\pmclassification{msc}{11R04}
\pmclassification{msc}{11C08}
\pmclassification{msc}{12E05}
\pmclassification{msc}{12F05}
\pmrelated{ConjugateFields}
\pmrelated{IndependenceOfCharacteristicPolynomialOnPrimitiveElement}

\endmetadata

% this is the default PlanetMath preamble.  as your knowledge
% of TeX increases, you will probably want to edit this, but
% it should be fine as is for beginners.

% almost certainly you want these
\usepackage{amssymb}
\usepackage{amsmath}
\usepackage{amsfonts}

% used for TeXing text within eps files
%\usepackage{psfrag}
% need this for including graphics (\includegraphics)
%\usepackage{graphicx}
% for neatly defining theorems and propositions
 \usepackage{amsthm}
% making logically defined graphics
%%%\usepackage{xypic}

% there are many more packages, add them here as you need them

% define commands here

\theoremstyle{definition}
\newtheorem*{thmplain}{Theorem}

\begin{document}
\textbf{Lemma.}\, Let $\alpha_1,\,\alpha_2,\,\ldots,\,\alpha_s$ be algebraic numbers belonging to the number field 
$\mathbb{Q}(\vartheta)$ of \PMlinkname{degree}{NumberField} $n$ and $\alpha_i^{(j)}$ their \PMlinkid{$\mathbb{Q}(\vartheta)$-conjugates}{12046}.\, If 
$P(x_1,\,x_2,\,\ldots,\,x_s)$ is a polynomial with rational coefficients and if
$$P(\alpha_1,\,\alpha_2,\,\ldots,\,\alpha_s) \;=\; 0,$$
then also
$$P(\alpha_1^{(j)},\,\alpha_2^{(j)},\,\ldots,\,\alpha_s^{(j)}) \;=\; 0$$
for each\, $j = 1,\,2,\,\ldots,\,n$.\, In the special case of two elements $\alpha$ and $\beta$ of 
$\mathbb{Q}(\vartheta)$ one may infer the formulae
\begin{align}
(\alpha\beta)^{(j)} \;=\; \alpha^{(j)}\beta^{(j)}, \quad (\alpha\!+\!\beta)^{(j)} \;=\; \alpha^{(j)}\!+\!\beta^{(j)}.
\end{align}


The lemma implies easily the following theorems.\\

\textbf{Theorem 1.}\, All conjugate fields of $\mathbb{Q}(\vartheta)$ are isomorphic.\\


\textbf{Theorem 2.}\, The norm and the trace in the field $\mathbb{Q}(\vartheta)$ satisfy
$$\mbox{N}(\alpha\beta) \;=\; \mbox{N}(\alpha)\mbox{N}(\beta), \quad 
\mbox{S}(\alpha\!+\!\beta) \;=\; \mbox{S}(\alpha)\!+\!\mbox{S}(\beta).$$

Cf. the entry norm and trace of algebraic number.



%%%%%
%%%%%
\end{document}
