\documentclass[12pt]{article}
\usepackage{pmmeta}
\pmcanonicalname{KroneckerSymbol}
\pmcreated{2013-03-22 14:33:21}
\pmmodified{2013-03-22 14:33:21}
\pmowner{mathwizard}{128}
\pmmodifier{mathwizard}{128}
\pmtitle{Kronecker symbol}
\pmrecord{6}{36108}
\pmprivacy{1}
\pmauthor{mathwizard}{128}
\pmtype{Definition}
\pmcomment{trigger rebuild}
\pmclassification{msc}{11A07}
\pmclassification{msc}{11A15}
\pmsynonym{Kronecker-Jacobi symbol}{KroneckerSymbol}
\pmrelated{JacobiSymbol}
\pmrelated{LegendreSymbol}

\endmetadata

% this is the default PlanetMath preamble.  as your knowledge
% of TeX increases, you will probably want to edit this, but
% it should be fine as is for beginners.

% almost certainly you want these
\usepackage{amssymb}
\usepackage{amsmath}
\usepackage{amsfonts}

% used for TeXing text within eps files
%\usepackage{psfrag}
% need this for including graphics (\includegraphics)
%\usepackage{graphicx}
% for neatly defining theorems and propositions
%\usepackage{amsthm}
% making logically defined graphics
%%%\usepackage{xypic}

% there are many more packages, add them here as you need them

% define commands here
\begin{document}
The {\bf Kronecker symbol} is a generalization of the Jacobi symbol to all integers.

Let $n$ be an integer, with prime factorization $u \cdot {p_1}^{e_1} \cdots {p_k}^{e_k}$, where $u$ is a unit and the $p_i$ are primes.  Let $a \geq 0$ be an integer.  The Kronecker symbol  $\left(\frac{a}{n}\right)$ is defined to be

\[ \left(\frac{a}{n}\right) =  \left(\frac{a}{u}\right) \prod_{i=1}^k \left(\frac{a}{p_i}\right)^{e_i} \]

For odd $p_i$, the number $\left(\frac{a}{p_i}\right)$ is simply the usual Legendre symbol.  This leaves the case when $p_i=2$.  We define $\left(\frac{a}{2}\right)$ by

\[ \left(\frac{a}{2}\right) = \begin{cases}
0 &\text{if $a$ is even}\\
1 & \text{if $a$ is odd and $n \equiv 1$ or $n \equiv  7 \pmod{8}$} \\
-1 & \text{if $a$ is odd and $n \equiv 3$ or $n \equiv 5 \pmod{8}$} \\
\end{cases} \]

Since it extends the Jacobi symbol, the quantity $\left(\frac{a}{u}\right)$ is simply 1 when $u=1$.  When $u=-1$, we define it by

\[ \left(\frac{a}{-1}\right) = \begin{cases}
-1 & \text{if $a < 0$} \\
1 & \text{if $a > 0$} \\
\end{cases} \]

These extensions suffice to define the Kronecker symbol for all integer values $n$.
%%%%%
%%%%%
\end{document}
