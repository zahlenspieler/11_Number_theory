\documentclass[12pt]{article}
\usepackage{pmmeta}
\pmcanonicalname{RiemannZetaFunction}
\pmcreated{2013-03-22 12:38:01}
\pmmodified{2013-03-22 12:38:01}
\pmowner{alozano}{2414}
\pmmodifier{alozano}{2414}
\pmtitle{Riemann zeta function}
\pmrecord{18}{32896}
\pmprivacy{1}
\pmauthor{alozano}{2414}
\pmtype{Definition}
\pmcomment{trigger rebuild}
\pmclassification{msc}{11M06}
\pmsynonym{$\zeta$ function}{RiemannZetaFunction}
\pmrelated{AnalyticContinuationOfRiemannZeta}
\pmrelated{DedekindZetaFunction}
\pmrelated{DirichletSeries}
\pmrelated{EulerProduct}
\pmrelated{Complex}
\pmrelated{EulerProductFormula2}
\pmrelated{HarmonicSeriesOfPrimes}
\pmdefines{Euler product formula}
\pmdefines{Riemann hypothesis}

% this is the default PlanetMath preamble.  as your knowledge
% of TeX increases, you will probably want to edit this, but
% it should be fine as is for beginners.

% almost certainly you want these
\usepackage{amssymb}
\usepackage{amsmath}
\usepackage{amsfonts}

% used for TeXing text within eps files
%\usepackage{psfrag}
% need this for including graphics (\includegraphics)
%\usepackage{graphicx}
% for neatly defining theorems and propositions
\usepackage{amsthm}
% making logically defined graphics
%%%\usepackage{xypic}

% there are many more packages, add them here as you need them

% define commands here
\newcommand{\p}{{\mathfrak{p}}}
\newcommand{\C}{\mathbb{C}}
\newcommand{\R}{\mathbb{R}}
\newcommand{\Z}{\mathbb{Z}}
\newcommand{\N}{\mathbb{N}}
\renewcommand{\H}{\mathcal{H}}
\newcommand{\A}{\mathcal{A}}
\renewcommand{\c}{\mathcal{C}}
\renewcommand{\O}{\mathcal{O}}
\newcommand{\D}{\mathcal{D}}
\newcommand{\lra}{\longrightarrow}
\newcommand{\res}{\operatorname{res}}
\newcommand{\id}{\operatorname{id}}
\newcommand{\diff}{\operatorname{diff}}
\newcommand{\incl}{\operatorname{incl}}
\newcommand{\Hom}{\operatorname{Hom}}
\renewcommand{\Re}{\operatorname{Re}}

\newtheorem{theorem}{Theorem}
\newtheorem{proposition}[theorem]{Proposition}
\newtheorem{lemma}[theorem]{Lemma}
\newtheorem{corollary}[theorem]{Corollary}

\theoremstyle{definition}
\newtheorem{definition}[theorem]{Definition}
\newtheorem{example}[theorem]{Example}
\begin{document}
\section{Definition}

The \emph{Riemann zeta function} is defined to be the complex valued
function given by the series
\begin{equation}\label{def}
\zeta(s) := \sum_{n=1}^\infty \frac{1}{n^s},
\end{equation}
which is valid (in fact, absolutely convergent) for all complex
numbers $s$ with $\Re(s) > 1$. We list here some of the key
properties~\cite{ahlfors} of the zeta function.
\begin{enumerate}
\item For all $s$ with $\Re(s) > 1$, the zeta function satisfies the
\emph{Euler product formula}
\begin{equation}\label{product}
\zeta(s) = \prod_{p} \frac{1}{1 - p^{-s}},
\end{equation}
where the product is taken over all positive integer primes $p$, and converges
uniformly in a neighborhood of $s$.
\item The zeta function has a meromorphic continuation to the entire
complex plane with a simple pole at $s=1$, of residue $1$, and no
other singularities.
\item The zeta function satisfies the \emph{functional equation}
\begin{equation}\label{functional}
\zeta(s) = 2^s \pi^{s-1} \sin \frac{\pi s}{2} \Gamma(1-s) \zeta(1-s),
\end{equation}
for any $s \in \C$ (where $\Gamma$ denotes the Gamma function).
\end{enumerate}

\section{Distribution of primes}

The Euler product formula~\eqref{product} given above expresses the
zeta function as a product over the primes $p \in \Z$, and
consequently provides a link between the analytic properties of the
zeta function and the distribution of primes in the integers. As the
simplest possible illustration of this link, we show how the
properties of the zeta function given above can be used to prove that
there are infinitely many primes.

If the set $S$ of primes in $\Z$ were finite, then the Euler product
formula
$$
\zeta(s) = \prod_{p \in S} \frac{1}{1 - p^{-s}}
$$
would be a finite product, and consequently $\lim_{s \to 1} \zeta(s)$
would exist and would equal
$$
\lim_{s \to 1} \zeta(s) = \prod_{p \in S} \frac{1}{1 - p^{-1}}.
$$
But the existence of this limit contradicts the fact that $\zeta(s)$
has a pole at $s=1$, so the set $S$ of primes cannot be finite.

A more sophisticated analysis of the zeta function along these lines
can be used to prove both the analytic prime number theorem and
Dirichlet's theorem on primes in arithmetic progressions\footnote{In the case of arithmetic progressions, one also needs to examine the closely related Dirichlet $L$--functions in addition to the zeta function itself.}. Proofs of
the prime number theorem can be found in~\cite{bak-newman}
and~\cite{patterson}, and for proofs of Dirichlet's theorem on primes
in arithmetic progressions the reader may look in~\cite{janusz}
and~\cite{serre}.

\section{Zeros of the zeta function}

A \emph{nontrivial zero} of the Riemann zeta function is defined to be
a root $\zeta(s) = 0$ of the zeta function with the property that $0
\leq \Re(s) \leq 1$. Any other zero is called \emph{trivial zero} of
the zeta function.

The reason behind the terminology is as follows. For complex numbers
$s$ with real part greater than 1, the series definition~\eqref{def}
immediately shows that no zeros of the zeta function exist in this
region. It is then an easy matter to use the functional
equation~\eqref{functional} to find all zeros of the zeta function
with real part less than 0 (it turns out they are exactly the values
$-2n$, for $n$ a positive integer). However, for values of $s$ with
real part between 0 and 1, the situation is quite different, since we
have neither a series definition nor a functional equation to fall
back upon; and indeed to this day very little is known about the
behavior of the zeta function inside this critical strip of the
complex plane.

It is known that the prime number theorem is equivalent to the
assertion that the zeta function has no zeros $s$ with $\Re(s) = 0$ or
$\Re(s) = 1$. The celebrated \emph{Riemann hypothesis} asserts that all nontrivial zeros $s$ of the zeta function satisfy the much more precise equation $\Re(s) = 1/2$. If true, the hypothesis would have profound
consequences on the distribution of primes in the
integers~\cite{patterson}.

\newpage

\begin{thebibliography}{9}
\bibitem{ahlfors} Lars Ahlfors, \emph{Complex Analysis, Third Edition},
McGraw--Hill, Inc., 1979.
\bibitem{bak-newman} Joseph Bak \& Donald Newman, \emph{Complex
Analysis, Second Edition}, Springer--Verlag, 1991.
\bibitem{janusz} Gerald Janusz, \emph{Algebraic Number Fields, Second
Edition}, American Mathematical Society, 1996.
\bibitem{lang} Serge Lang, \emph{Algebraic Number Theory, Second
Edition}, Springer--Verlag, 1994.
\bibitem{patterson} Stephen Patterson, \emph{Introduction to the Theory
of the Riemann Zeta Function}, Cambridge University Press, 1988.
\bibitem{riemann} B. Riemann, \emph{Ueber die Anzahl der Primzahlen unter einer gegebenen Gr\"osse}, \PMlinkexternal{http://www.maths.tcd.ie/pub/HistMath/People/Riemann/Zeta/}{http://www.maths.tcd.ie/pub/HistMath/People/Riemann/Zeta/}
\bibitem{serre} Jean--Pierre Serre, \emph{A Course in Arithmetic},
Springer--Verlag, 1973.
\end{thebibliography}

%%%%%
%%%%%
\end{document}
