\documentclass[12pt]{article}
\usepackage{pmmeta}
\pmcanonicalname{DirichletEtaFunction}
\pmcreated{2013-03-22 14:31:28}
\pmmodified{2013-03-22 14:31:28}
\pmowner{Mathprof}{13753}
\pmmodifier{Mathprof}{13753}
\pmtitle{Dirichlet eta function}
\pmrecord{9}{36066}
\pmprivacy{1}
\pmauthor{Mathprof}{13753}
\pmtype{Definition}
\pmcomment{trigger rebuild}
\pmclassification{msc}{11M41}
\pmrelated{ZerosOfDirichletEtaFunction}

% this is the default PlanetMath preamble.  as your knowledge
% of TeX increases, you will probably want to edit this, but
% it should be fine as is for beginners.

% almost certainly you want these
\usepackage{amssymb}
\usepackage{amsmath}
\usepackage{amsfonts}

% used for TeXing text within eps files
%\usepackage{psfrag}
% need this for including graphics (\includegraphics)
%\usepackage{graphicx}
% for neatly defining theorems and propositions
%\usepackage{amsthm}
% making logically defined graphics
%%%\usepackage{xypic}

% there are many more packages, add them here as you need them

% define commands here
\begin{document}
For $s\in\mathbb{C}$, the \emph{Dirichlet eta function} is defined as

\begin{equation}
\eta(s) := \sum_{n=1}^{\infty} \frac{\left( -1 \right)^{n-1}}{n^{s}}\,.
\end{equation}

Let  $s=\sigma + it$.  For $s$ a positive real number  the series converges by the alternating series test, by the second \PMlinkescapetext{property} listed in the entry on   Dirichlet series it converges for all $s$ with $\sigma > 0$.

It can be shown that $\eta(s) = (1-2^{1-s})\zeta(s)$, where $\zeta(s)$ is the Riemann zeta function. The pole of $\zeta(s)$ at $s=1$ is cancelled by the zero
of $1-2^{1-s}$. 
%%%%%
%%%%%
\end{document}
