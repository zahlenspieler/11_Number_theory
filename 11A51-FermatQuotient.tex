\documentclass[12pt]{article}
\usepackage{pmmeta}
\pmcanonicalname{FermatQuotient}
\pmcreated{2013-03-22 19:34:22}
\pmmodified{2013-03-22 19:34:22}
\pmowner{pahio}{2872}
\pmmodifier{pahio}{2872}
\pmtitle{Fermat quotient}
\pmrecord{12}{42559}
\pmprivacy{1}
\pmauthor{pahio}{2872}
\pmtype{Definition}
\pmcomment{trigger rebuild}
\pmclassification{msc}{11A51}
\pmclassification{msc}{11A41}
\pmdefines{Lerch's formula}
\pmdefines{Fermat--Wilson quotient}

\endmetadata

% this is the default PlanetMath preamble.  as your knowledge
% of TeX increases, you will probably want to edit this, but
% it should be fine as is for beginners.

% almost certainly you want these
\usepackage{amssymb}
\usepackage{amsmath}
\usepackage{amsfonts}

% used for TeXing text within eps files
%\usepackage{psfrag}
% need this for including graphics (\includegraphics)
%\usepackage{graphicx}
% for neatly defining theorems and propositions
 \usepackage{amsthm}
% making logically defined graphics
%%%\usepackage{xypic}

% there are many more packages, add them here as you need them

% define commands here

\theoremstyle{definition}
\newtheorem*{thmplain}{Theorem}

\begin{document}
\PMlinkescapeword{forms}

If $a$ is an integer not divisible by a positive prime $p$, then Fermat's little theorem (a.k.a. Fermat's theorem) guarantees that the difference $a^{p-1}\!-\!1$ is divisible by $p$.\, The integer
$$q_p(a) \;:=\; \frac{a^{p-1}\!-\!1}{p}$$
is called the {\it Fermat quotient of $a$ modulo $p$}.\, Compare it with the Wilson quotient $w_p$, which is similarly related to Wilson's theorem.\\

{\it Lerch's formula}
$$\sum_{a=1}^{p-1}q_p(a) \;\equiv\; w_p \;\, \pmod p$$
for an odd prime $p$ connects the Fermat quotients and the Wilson quotient.\\


If $p$ is a positive prime but not a Wilson prime, and $w_p$ is its Wilson quotient, then the expression
$$q_p(w_p) \;=\; \frac{w_p^{p-1}\!-\!1}{p}$$
is called the {\it Fermat--Wilson quotient} of $p$.\, Sondow proves in [1] that the greatest common divisor of all Fermat--Wilson quotients is 24.


\begin{thebibliography}{8}
\bibitem{JS}{\sc Jonathan Sondow}:\, \emph{Lerch Quotients, Lerch Primes,
Fermat--Wilson Quotients, and the Wieferich-non-Wilson Primes 2, 3, 14771}. Available at \PMlinkexternal{arXiv}{http://arxiv.org/pdf/1110.3113v3.pdf}.
\end{thebibliography}

%%%%%
%%%%%
\end{document}
