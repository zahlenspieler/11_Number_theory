\documentclass[12pt]{article}
\usepackage{pmmeta}
\pmcanonicalname{PropertiesOfDiscriminantInAlgebraicNumberField}
\pmcreated{2013-03-22 19:09:28}
\pmmodified{2013-03-22 19:09:28}
\pmowner{pahio}{2872}
\pmmodifier{pahio}{2872}
\pmtitle{properties of discriminant in algebraic number field}
\pmrecord{12}{42062}
\pmprivacy{1}
\pmauthor{pahio}{2872}
\pmtype{Result}
\pmcomment{trigger rebuild}
\pmclassification{msc}{11R29}
%\pmkeywords{discriminant}
\pmrelated{MinimalityOfIntegralBasis}
\pmrelated{ConditionForPowerBasis}

% this is the default PlanetMath preamble.  as your knowledge
% of TeX increases, you will probably want to edit this, but
% it should be fine as is for beginners.

% almost certainly you want these
\usepackage{amssymb}
\usepackage{amsmath}
\usepackage{amsfonts}

% used for TeXing text within eps files
%\usepackage{psfrag}
% need this for including graphics (\includegraphics)
%\usepackage{graphicx}
% for neatly defining theorems and propositions
 \usepackage{amsthm}
% making logically defined graphics
%%%\usepackage{xypic}

% there are many more packages, add them here as you need them

% define commands here

\theoremstyle{definition}
\newtheorem*{thmplain}{Theorem}

\begin{document}
\textbf{Theorem 1.}\, Let $\alpha_1,\,\alpha_2,\,\ldots,\,\alpha_n$ and $\beta_1,\,\beta_2,\,\ldots,\,\beta_n$ be elements of the algebraic number field $\mathbb{Q}(\vartheta)$ of \PMlinkname{degree}{NumberField} $n$.\, If they satisfy the equations
$$\alpha_i \;=\; \sum_{j=1}^nc_{ij}\beta_j \quad \mbox{for} \quad i \,=\, 1,\,2,\,\ldots,\,n,$$
where all coefficients $c_{ij}$ are rational numbers, then the \PMlinkid{discriminants}{12060} are \PMlinkescapetext{connected} via the equation
$$\Delta(\alpha_1,\,\alpha_2,\,\ldots,\,\alpha_n) 
  \;=\; \det\!(c_{ij})^2\cdot\Delta(\beta_1,\,\beta_2,\,\ldots,\,\beta_n).$$\\

As a special case one obtains the

\textbf{Theorem 2.}\, If
\begin{align}
\alpha_i \;=\; c_{i1}+c_{i2}\vartheta+\ldots+c_{in}\vartheta^{n-1} \quad \mbox{for} \quad i \,=\, 1,\,2,\,\ldots,\,n
\end{align}
are the canonical forms of the elements $\alpha_i$ in $\mathbb{Q}(\vartheta)$, then 
$$\Delta(\alpha_1,\,\alpha_2,\,\ldots,\,\alpha_n) 
  \;=\; \det\!(c_{ij})^2\cdot\Delta(1,\,\vartheta,\,\ldots,\,\vartheta^{n-1}),$$
where $\Delta(1,\,\vartheta,\,\ldots,\,\vartheta^{n-1})$ is a Vandermonde determinant thus
having the product form
\begin{align}
\Delta(1,\,\vartheta,\,\ldots,\,\vartheta^{n-1}) 
\;=\; \left(\prod_{1 \le i \le j}(\vartheta_j-\vartheta_i)\right)^{\!2} 
\;=\; \prod_{1 \le i \le j}(\vartheta_i-\vartheta_j)^2
\end{align}
where $\vartheta_1 = \vartheta,\,\vartheta_2,\,\ldots,\,\vartheta_n$ are the algebraic conjugates of $\vartheta$.\\

Since the \PMlinkescapetext{discriminant} (2) is also the polynomial discriminant of the irreducible minimal polynomial of $\vartheta$, the numbers $\vartheta_i$ are inequal.\, It follows the

\textbf{Theorem 3.}\, When (1) are the canonical forms of the numbers $\alpha_i$, one has
$$\Delta(\alpha_1,\,\alpha_2,\,\ldots,\,\alpha_n) \;=\; 0 \quad 
\Longleftrightarrow \quad \det(c_{ij}) \;=\; 0.$$\\

The powers $1,\,\vartheta,\,\ldots,\,\vartheta^{n-1}$ of the \PMlinkname{primitive element}{SimpleFieldExtension} form a \PMlinkname{basis}{Basis} of the field extension $\mathbb{Q}(\vartheta)/\mathbb{Q}$ (see the canonical form of element of number field).\, By the theorem 3 we may write the

\textbf{Theorem 4.}\, The numbers $\alpha_1,\,\alpha_2,\,\ldots,\,\alpha_n$ of $\mathbb{Q}(\vartheta)$ are linearly independent over $\mathbb{Q}$ if and only if\, $\Delta(\alpha_1,\,\alpha_2,\,\ldots,\,\alpha_n) \;\neq\; 0$.\\

\textbf{Theorem 5.}\quad $\displaystyle \mathbb{Q}(\alpha) = \mathbb{Q}(\vartheta)
\;\; \Longleftrightarrow \;\; \Delta(1,\,\alpha,\,\alpha^2,\,\ldots,\,\alpha^{n-1}) \neq 0$.\; Here, the the discriminant is the \PMlinkname{discriminant of the algebraic number}{DiscriminantOfAlgebraicNumber} $\alpha$.
  

%%%%%
%%%%%
\end{document}
