\documentclass[12pt]{article}
\usepackage{pmmeta}
\pmcanonicalname{RefactorableNumber}
\pmcreated{2013-03-22 17:40:41}
\pmmodified{2013-03-22 17:40:41}
\pmowner{CompositeFan}{12809}
\pmmodifier{CompositeFan}{12809}
\pmtitle{refactorable number}
\pmrecord{4}{40117}
\pmprivacy{1}
\pmauthor{CompositeFan}{12809}
\pmtype{Definition}
\pmcomment{trigger rebuild}
\pmclassification{msc}{11A25}

% this is the default PlanetMath preamble.  as your knowledge
% of TeX increases, you will probably want to edit this, but
% it should be fine as is for beginners.

% almost certainly you want these
\usepackage{amssymb}
\usepackage{amsmath}
\usepackage{amsfonts}

% used for TeXing text within eps files
%\usepackage{psfrag}
% need this for including graphics (\includegraphics)
%\usepackage{graphicx}
% for neatly defining theorems and propositions
%\usepackage{amsthm}
% making logically defined graphics
%%%\usepackage{xypic}

% there are many more packages, add them here as you need them

% define commands here

\begin{document}
A {\em refactorable number} or {\em tau number} is an integer $n$ that is divisible by the count of its divisors, or to put it algebraically, $n$ is such that $\tau(n)|n$, with $\tau(n)$ being the divisor function. The first few refactorable numbers are listed in the OEIS A033950 are 1, 2, 8, 9, 12, 18, 24, 36, 40, 56, 60, 72, 80, 84, 88, 96. The equation $\gcd(n, x) = \tau(n)$ has solutions only if $n$ is a refactorable number.
%%%%%
%%%%%
\end{document}
