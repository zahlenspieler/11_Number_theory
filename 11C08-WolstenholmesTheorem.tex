\documentclass[12pt]{article}
\usepackage{pmmeta}
\pmcanonicalname{WolstenholmesTheorem}
\pmcreated{2013-03-22 19:14:06}
\pmmodified{2013-03-22 19:14:06}
\pmowner{pahio}{2872}
\pmmodifier{pahio}{2872}
\pmtitle{Wolstenholme's theorem}
\pmrecord{10}{42159}
\pmprivacy{1}
\pmauthor{pahio}{2872}
\pmtype{Theorem}
\pmcomment{trigger rebuild}
\pmclassification{msc}{11C08}
\pmclassification{msc}{11A07}
\pmrelated{HarmonicNumber}

% this is the default PlanetMath preamble.  as your knowledge
% of TeX increases, you will probably want to edit this, but
% it should be fine as is for beginners.

% almost certainly you want these
\usepackage{amssymb}
\usepackage{amsmath}
\usepackage{amsfonts}

% used for TeXing text within eps files
%\usepackage{psfrag}
% need this for including graphics (\includegraphics)
%\usepackage{graphicx}
% for neatly defining theorems and propositions
 \usepackage{amsthm}
% making logically defined graphics
%%%\usepackage{xypic}

% there are many more packages, add them here as you need them

% define commands here

\theoremstyle{definition}
\newtheorem*{thmplain}{Theorem}

\begin{document}
We want to show first that the harmonic number
$$H_n \;=:\; 1+\frac{1}{2}+\frac{1}{3}+\ldots+\frac{1}{n}\, \;=\; \int_0^1\frac{1\!-\!x^n}{1\!-\!x}\,dx$$
is never an integer ($n > 1$).

Denote by $p$ the greatest prime number not exceeding $n$.\, By Bertrand's postulate there is a prime $q$ with\, 
$p < q < 2p$.\, Therefore we have\, $n < 2p$.\, If $H_n$ were an integer, then the sum
$$n!H_n \;=\; \sum_{i=1}^n\frac{n!}{i}$$
had to be divisible by $p$.\, However its addend $\displaystyle\frac{n!}{p}$ is not divisible by $p$ but all other addends are, whence the sum cannot be divisible by $p$.\, The contradictory situation means that $H_n$ is not integer when\, $n > 1$.\\


\textbf{Theorem (Wolstenholme).}\, If $p$ is a prime number greater than 3, then the numerator of the harmonic number
$$H_{p-1} \;=\; 1+\frac{1}{2}+\frac{1}{3}+\ldots+\frac{1}{p-1}$$
is always divisible by $p^2$.\\

\emph{Proof.}\, Consider the polynomial
$$f(x) \;=:\; (x\!-\!1)(x\!-\!2)\cdots(x\!-\!p\!+\!1).$$
One has
\begin{align}
f(0) \;=\; (p\!-\!1)! \;=\; f(p)
\end{align}
and
\begin{align}
f(x) \;=\; x^{p-1}+a_1x^{p-2}+a_2x^{p-3}+\ldots+a_{p-2}x+(p\!-1\!)!
\end{align}
where $a_1,\,a_2,\,\ldots,\,a_{p-2}$ are integers.\, Because $1,\,2,\,\ldots,\,p\!-\!1$ form a set of all modulo $p$ incongruent roots of the \PMlinkname{Fermat's congruence}{FermatsTheorem} \,$x^{p-1} \equiv 1 \pmod{p}$,\, one may write the identical congruence
\begin{align}
x^{p-1}\!-\!1 \;\equiv\; x^{p-1}+a_1x^{p-2}+a_2x^{p-3}+\ldots+a_{p-2}x+(p\!-1\!)! \pmod{p}.
\end{align}
It may be written by Wilson's theorem \,$(p\!-1\!)! \equiv -1 \pmod{p}$\, as
\begin{align}
a_1x^{p-2}+a_2x^{p-3}+\ldots+a_{p-2}x \;\equiv\; 0 \pmod{p},
\end{align}
being thus true for any integer $x$.\, From (4) one can successively infer that $p$ divides all coefficients $a_i$, i.e. that (4) actually is a formal congruence.

For the derivative of the polynomial $f(x)$ one has
$$f'(x) \;=\; (x\!-\!2)\cdots(x\!-\!p\!+\!1)+\ldots+(x\!-\!1)\cdots(x\!-\!p\!+\!2)$$
and thus
\begin{align}
f'(0) \;=\; -2\cdot3\cdots(p\!-\!1)-\ldots-1\cdot2\cdots(p\!-\!2).
\end{align}
The Taylor series (Taylor polynomial) of $f(x)$ coincides with $f(x)$:
$$f(x) \;=\; f(0)+\frac{f'(0)}{1!}x+\frac{f''(0)}{2!}x^2+\ldots+\frac{f^{(p-1)}(0)}{(p\!-\!1)!}x^{p-1}$$
By (1), this equation implies
\begin{align}
0 \;=\; f'(0)+\frac{f''(0)}{2}p+\ldots+\frac{f^{(p-1)}(0)}{(p\!-\!1)!}p^{p-2}
\end{align}
Since\, $p \mid a_{p-3} = \frac{f''(0)}{2}$,\, one has\, $p \mid f''(0)$.\, It then follows by (6) that\, $p^2 \mid f'(0)$.\, And since (5) divided by $-(p\!-\!1)!$ gives
$$1+\frac{1}{2}+\frac{1}{3}+\ldots+\frac{1}{p-1} \;=\; -\frac{f'(0)}{(p\!-\!1)!},$$
the assertion has been proved.

\begin{thebibliography}{8}
\bibitem{K}{\sc L. Kuipers}: ``Der Wolstenholmesche Satz''.\, -- \emph{Elemente der Mathematik} \textbf{35} (1980).
\end{thebibliography}
%%%%%
%%%%%
\end{document}
