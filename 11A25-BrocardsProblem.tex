\documentclass[12pt]{article}
\usepackage{pmmeta}
\pmcanonicalname{BrocardsProblem}
\pmcreated{2013-03-22 18:09:59}
\pmmodified{2013-03-22 18:09:59}
\pmowner{PrimeFan}{13766}
\pmmodifier{PrimeFan}{13766}
\pmtitle{Brocard's problem}
\pmrecord{4}{40727}
\pmprivacy{1}
\pmauthor{PrimeFan}{13766}
\pmtype{Definition}
\pmcomment{trigger rebuild}
\pmclassification{msc}{11A25}

% this is the default PlanetMath preamble.  as your knowledge
% of TeX increases, you will probably want to edit this, but
% it should be fine as is for beginners.

% almost certainly you want these
\usepackage{amssymb}
\usepackage{amsmath}
\usepackage{amsfonts}

% used for TeXing text within eps files
%\usepackage{psfrag}
% need this for including graphics (\includegraphics)
%\usepackage{graphicx}
% for neatly defining theorems and propositions
%\usepackage{amsthm}
% making logically defined graphics
%%%\usepackage{xypic}

% there are many more packages, add them here as you need them

% define commands here

\begin{document}
{\em Brocard's problem}, first posed by Henri Brocard in 1876, asks for factorials that are one less than a square, that is, solutions to the equation $n! + 1 = m^2$. Only three solutions are known: $4! + 1 = 5^2$, $5! + 1 = 11^2$ and $7! + 1 = 71^2$. Srinivasa Ramanujan also pondered the problem, in 1913. Erd\H{o}s believed that there are no other solutions, and no more have been found for $n$ up to $10^9$.

\begin{thebibliography}{1}
\bibitem{eh} P. Erd\H{o}s, \& R. Obláth, ``\"Uber diophantische Gleichungen der Form $n! = x^p \pm y^p$ und $n! \pm m! = x^p$'' {\it Acta Szeged.} {\bf 8} (1937): 241 - 255
\end{thebibliography}
%%%%%
%%%%%
\end{document}
