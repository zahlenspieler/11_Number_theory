\documentclass[12pt]{article}
\usepackage{pmmeta}
\pmcanonicalname{PadicIntegers}
\pmcreated{2013-03-22 12:48:04}
\pmmodified{2013-03-22 12:48:04}
\pmowner{djao}{24}
\pmmodifier{djao}{24}
\pmtitle{$p$-adic integers}
\pmrecord{10}{33118}
\pmprivacy{1}
\pmauthor{djao}{24}
\pmtype{Definition}
\pmcomment{trigger rebuild}
\pmclassification{msc}{11S99}
\pmclassification{msc}{12J12}
\pmsynonym{$p$-adic numbers}{PadicIntegers}
\pmsynonym{$\mathbb{Z}_p$}{PadicIntegers}
\pmrelated{InverseLimit}

% this is the default PlanetMath preamble.  as your knowledge
% of TeX increases, you will probably want to edit this, but
% it should be fine as is for beginners.

% almost certainly you want these
\usepackage{amssymb}
\usepackage{amsmath}
\usepackage{amsfonts}

% used for TeXing text within eps files
%\usepackage{psfrag}
% need this for including graphics (\includegraphics)
%\usepackage{graphicx}
% for neatly defining theorems and propositions
%\usepackage{amsthm}
% making logically defined graphics
\usepackage[all]{xypic}

% there are many more packages, add them here as you need them

% define commands here
\newcommand{\ilim}{\,\underset{\longleftarrow}{\lim}\,}
\newcommand{\Z}{\mathbb{Z}}
\newcommand{\Q}{\mathbb{Q}}
\newcommand{\p}{\mathfrak{p}}
\begin{document}
\section{Basic construction}

For any prime $p$, the {\em $p$--adic integers} is the ring obtained by taking the completion of the integers $\Z$ with respect to the metric induced by the norm
\begin{equation}\label{valuation}
|x| := \frac{1}{p^{\nu_p(x)}},\ \ x \in \Z,
\end{equation}
where $\nu_p(x)$ denotes the largest integer $e$ such that $p^e$ divides $x$. The induced metric $d(x,y) := |x-y|$ is called the {\em $p$--adic metric} on $\Z$. The ring of $p$--adic integers is usually denoted by $\Z_p$, and its fraction field by $\Q_p$.

\section{Profinite viewpoint}

The ring $\Z_p$ of $p$--adic integers can also be constructed by taking the inverse limit
$$
\Z_p := \ilim \Z/p^n\Z
$$
over the inverse system $\cdots \to \Z/p^2\Z \to \Z/p\Z \to 0$ consisting of the rings $\Z/p^n\Z$, for all $n \geq 0$, with the projection maps defined to be the unique maps such that the diagram
$$
\xymatrix{
& \Z \ar[dl] \ar[dr]\\
\Z/p^{n+1}\Z \ar[rr] & & \Z/p^n\Z
}
$$
commutes. An algebraic and topological isomorphism between the two constructions is obtained by taking the coordinatewise projection map $\Z \to \ilim \Z/p^n\Z$, extended to the completion of $\Z$ under the $p$--adic metric.

This alternate characterization shows that $\Z_p$ is compact, since it is a closed subspace of the space
$$
\prod_{n \geq 0} \Z/p^n\Z
$$
which is an infinite product of finite topological spaces and hence compact under the product topology.

\section{Generalizations}

If we interpret the prime $p$ as an equivalence class of valuations on $\Q$, then the field $\Q_p$ is simply the completion of the topological field $\Q$ with respect to the metric induced by any member valuation of $p$ (indeed, the valuation defined in Equation~\eqref{valuation}, extended to $\Q$, may serve as the representative). This notion easily generalizes to other fields and valuations; namely, if $K$ is any field, and $\p$ is any prime of $K$, then the $\p$--adic field $K_\p$ is defined to be the completion of $K$ with respect to any valuation in $\p$. The analogue of the $p$--adic integers in this case can be obtained by taking the subset (and subring) of $K_\p$ consisting of all elements of absolute value less than or equal to $1$, which is well defined independent of the choice of valuation representing $\p$.

In the special case where $K$ is a number field, the $\p$--adic ring $K_\p$ is always a finite extension of $\Q_p$ whenever $\p$ is a finite prime, and is always equal to either $\mathbb{R}$ or $\mathbb{C}$ whenever $\p$ is an infinite prime.
%%%%%
%%%%%
\end{document}
