\documentclass[12pt]{article}
\usepackage{pmmeta}
\pmcanonicalname{HappyNumber}
\pmcreated{2013-03-22 16:19:56}
\pmmodified{2013-03-22 16:19:56}
\pmowner{PrimeFan}{13766}
\pmmodifier{PrimeFan}{13766}
\pmtitle{happy number}
\pmrecord{4}{38462}
\pmprivacy{1}
\pmauthor{PrimeFan}{13766}
\pmtype{Definition}
\pmcomment{trigger rebuild}
\pmclassification{msc}{11A63}

% this is the default PlanetMath preamble.  as your knowledge
% of TeX increases, you will probably want to edit this, but
% it should be fine as is for beginners.

% almost certainly you want these
\usepackage{amssymb}
\usepackage{amsmath}
\usepackage{amsfonts}

% used for TeXing text within eps files
%\usepackage{psfrag}
% need this for including graphics (\includegraphics)
%\usepackage{graphicx}
% for neatly defining theorems and propositions
%\usepackage{amsthm}
% making logically defined graphics
%%%\usepackage{xypic}

% there are many more packages, add them here as you need them

% define commands here

\begin{document}
Given a base $b$ integer $$n = \sum_{i = 1}^k d_ib^{i - 1}$$ where $d_1$ is the least significant digit and $d_k$ is the most significant, and the function $$f(m) = \sum_{i = 1}^k {d_i}^2,$$ if computing $f(n)$ and iterating that function on the result eventually leads to a fixed point of 1, then $n$ is said to be a {\em happy number} in base $b$.

For $b = 2$ and $b = 4$, all numbers are happy numbers. All standard positional bases have happy numbers.
%%%%%
%%%%%
\end{document}
