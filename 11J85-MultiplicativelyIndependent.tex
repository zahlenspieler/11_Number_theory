\documentclass[12pt]{article}
\usepackage{pmmeta}
\pmcanonicalname{MultiplicativelyIndependent}
\pmcreated{2013-03-22 19:36:03}
\pmmodified{2013-03-22 19:36:03}
\pmowner{pahio}{2872}
\pmmodifier{pahio}{2872}
\pmtitle{multiplicatively independent}
\pmrecord{6}{42592}
\pmprivacy{1}
\pmauthor{pahio}{2872}
\pmtype{Definition}
\pmcomment{trigger rebuild}
\pmclassification{msc}{11J85}
\pmclassification{msc}{12F05}

% this is the default PlanetMath preamble.  as your knowledge
% of TeX increases, you will probably want to edit this, but
% it should be fine as is for beginners.

% almost certainly you want these
\usepackage{amssymb}
\usepackage{amsmath}
\usepackage{amsfonts}

% used for TeXing text within eps files
%\usepackage{psfrag}
% need this for including graphics (\includegraphics)
%\usepackage{graphicx}
% for neatly defining theorems and propositions
 \usepackage{amsthm}
% making logically defined graphics
%%%\usepackage{xypic}

% there are many more packages, add them here as you need them

% define commands here

\theoremstyle{definition}
\newtheorem*{thmplain}{Theorem}

\begin{document}
A set $X$ of nonzero complex numbers is said to be {\it multiplicatively independent} iff every equation
$$x_1^{\nu_1}x_2^{\nu_2}\cdots x_n^{\nu_n} \;=\; 1$$
with\, $x_1,\,x_2,\,\ldots,\,x_n \in X$\, and\, $\nu_1,\,\nu_2,\,\ldots,\,\nu_n \in \mathbb{Z}$\, implies that
$$\nu_1 \;=\; \nu_2 \;= \ldots =\; \nu_n \;=\; 0.$$

For example, the set of prime numbers is multiplicatively independent, by the fundamental theorem of arithmetics.\\

Any algebraically independent set is also multiplicatively independent.\\

Evidently, $\{x_1,\,x_2,\,\ldots,\,x_n\}$ is multiplicatively
independent if and only if the numbers $\log x_1$,\,$\log x_2$,\,...,\,$\log x_n$ are linearly independent over $\mathbb{Q}$.\, Thus the Schanuel's conjecture may be formulated as the

\textbf{Conjecture.}\, If\, $\{x_1,\,x_2,\,\ldots,\,x_n\}$\, is multiplicatively independent, then the transcendence degree of the set
$$\{x_1,\,x_2,\,\ldots,\,x_n,\,\log x_1,\,\log x_2,\,\ldots,\,\log x_n\}$$
is at least $n$.

\begin{thebibliography}{8}
\bibitem{MS}{\sc Diego Marques \& Jonathan Sondow}: {\it Schanuel's conjecture and algebraic powers $z^w$ and $w^z$ with $z$ and $w$ transcendental} (2011).  Available \PMlinkexternal{here}{http://arxiv.org/pdf/1010.6216.pdf}.
\end{thebibliography}

%%%%%
%%%%%
\end{document}
