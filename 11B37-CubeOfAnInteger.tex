\documentclass[12pt]{article}
\usepackage{pmmeta}
\pmcanonicalname{CubeOfAnInteger}
\pmcreated{2013-03-22 19:34:33}
\pmmodified{2013-03-22 19:34:33}
\pmowner{pahio}{2872}
\pmmodifier{pahio}{2872}
\pmtitle{cube of an integer}
\pmrecord{11}{42562}
\pmprivacy{1}
\pmauthor{pahio}{2872}
\pmtype{Theorem}
\pmcomment{trigger rebuild}
\pmclassification{msc}{11B37}
\pmclassification{msc}{11A25}
\pmrelated{NicomachusTheorem}
\pmrelated{TriangularNumbers}
\pmrelated{DifferenceOfSquares}

\endmetadata

% this is the default PlanetMath preamble.  as your knowledge
% of TeX increases, you will probably want to edit this, but
% it should be fine as is for beginners.

% almost certainly you want these
\usepackage{amssymb}
\usepackage{amsmath}
\usepackage{amsfonts}

% used for TeXing text within eps files
%\usepackage{psfrag}
% need this for including graphics (\includegraphics)
%\usepackage{graphicx}
% for neatly defining theorems and propositions
 \usepackage{amsthm}
% making logically defined graphics
%%%\usepackage{xypic}

% there are many more packages, add them here as you need them

% define commands here

\theoremstyle{definition}
\newtheorem*{thmplain}{Theorem}

\begin{document}
\textbf{Theorem.}\, Any cube of integer is a difference of two squares, which in the case 
of a positive cube are the squares of two successive triangular numbers.\\

For proving the assertion, one needs only to check the identity
$$a^3 \;\equiv\; \left(\frac{(a\!+\!1)a}{2}\right)^{\!2}-\left(\frac{(a\!-\!1)a}{2}\right)^{\!2}.$$


For example we have\, $(-2)^3 = 1^2\!-\!3^2$\, and\, $4^3 = 64 = 10^2\!-\!6^2$.\\

Summing the first $n$ positive cubes, the identity allows \PMlinkexternal{telescoping}{http://planetmath.org/encyclopedia/TelescopingSum.html} between consecutive brackets,
\begin{align*}
1^3\!+\!2^3\!+\!3^3\!+\!4^3\!+\ldots+\!n^3 &\;=\; 
[1^2\!-\!0^2]\!+\![3^2\!-\!1^2]\!+\![6^2\!-\!3^2]\!+\![10^2\!-\!6^2]\!+\ldots
+\!\left[\!\left(\frac{(n\!+\!1)n}{2}\right)^{\!2}\!-\!\left(\frac{(n\!-\!1)n}{2}\right)^{\!2}\right]\\ 
&\;=\; \left(\frac{n^2\!+\!n}{2}\right)^{\!2},
\end{align*}
saving only the square $\left(\frac{(n+1)n}{2}\right)^2$.\, Thus we have this expression presenting the sum of the first $n$ positive cubes (cf. the Nicomachus theorem).

%%%%%
%%%%%
\end{document}
