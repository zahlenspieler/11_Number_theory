\documentclass[12pt]{article}
\usepackage{pmmeta}
\pmcanonicalname{ColossallyAbundantNumber}
\pmcreated{2013-03-22 17:37:52}
\pmmodified{2013-03-22 17:37:52}
\pmowner{CompositeFan}{12809}
\pmmodifier{CompositeFan}{12809}
\pmtitle{colossally abundant number}
\pmrecord{4}{40052}
\pmprivacy{1}
\pmauthor{CompositeFan}{12809}
\pmtype{Definition}
\pmcomment{trigger rebuild}
\pmclassification{msc}{11A05}

% this is the default PlanetMath preamble.  as your knowledge
% of TeX increases, you will probably want to edit this, but
% it should be fine as is for beginners.

% almost certainly you want these
\usepackage{amssymb}
\usepackage{amsmath}
\usepackage{amsfonts}

% used for TeXing text within eps files
%\usepackage{psfrag}
% need this for including graphics (\includegraphics)
%\usepackage{graphicx}
% for neatly defining theorems and propositions
%\usepackage{amsthm}
% making logically defined graphics
%%%\usepackage{xypic}

% there are many more packages, add them here as you need them

% define commands here

\begin{document}
An integer $n$ is a {\em colossally abundant number} if there is an exponent $\epsilon > 1$ such that the sum of divisors of $n$ divided by $n$ raised to that exponent is greater than or equal to the sum of divisors of any other integer $k > 1$ divided by $k$ raised to that same exponent. That is, $$\frac{\sigma(n)}{n^\epsilon} \geq \frac{\sigma(k)}{k^\epsilon},$$ with $\sigma(n)$ being the sum of divisors function.

The first few colossally abundant numbers are 1, 2, 6, 12, 60, 120, 360, 2520, 5040, 55440, 720720, 1441440, 4324320. The index of a colossally abundant number is equal to the number of its nondistinct prime factors, that is to say that for the $i$th colossally abundant number $c_i$ the equality $i = \Omega(c_i)$ is true.
%%%%%
%%%%%
\end{document}
