\documentclass[12pt]{article}
\usepackage{pmmeta}
\pmcanonicalname{QuadraticReciprocityRule}
\pmcreated{2013-03-22 11:42:27}
\pmmodified{2013-03-22 11:42:27}
\pmowner{alozano}{2414}
\pmmodifier{alozano}{2414}
\pmtitle{quadratic reciprocity rule}
\pmrecord{33}{30012}
\pmprivacy{1}
\pmauthor{alozano}{2414}
\pmtype{Theorem}
\pmcomment{trigger rebuild}
\pmclassification{msc}{11A15}
\pmsynonym{quadratic reciprocity}{QuadraticReciprocityRule}
\pmrelated{EulersCriterion}
\pmrelated{CubicReciprocityLaw}
\pmrelated{QuadraticReciprocityForPolynomials}
\pmrelated{LegendreSymbol}

\usepackage{amssymb}
\usepackage{amsmath}
\usepackage{amsthm}
\usepackage{amsfonts}
\usepackage{graphicx}
%%%%%%%%%%%%%%%%%%%%\usepackage{xypic}

\newtheorem*{thm}{Theorem}
\begin{document}
\begin{thm}[Law of Quadratic Reciprocity]
Let $p$ and $q$ be two distinct odd primes. Then:

$$ \left(\frac{q}{p}\right)\left(\frac{p}{q}\right)=(-1)^{(p-1)(q-1)/4} $$

where $\left(\frac{\cdot}{\cdot}\right)$ is the \PMlinkname{Jacobi}{JacobiSymbol}  symbol (or Legendre symbol).
\end{thm}

The following is an equivalent formulation of the Law of Quadratic Reciprocity: 

\begin{thm}[Quadratic Reciprocity (second form)]
Let $p,q$ be distinct odd primes. Then:
\begin{enumerate}
\item $\displaystyle \left(\frac{p}{q}\right) = \left(\frac{q}{p}\right)$ if one of $p$ or $q$ is congruent to $1$ modulo $4$;

\item $\displaystyle \left(\frac{p}{q}\right) = - \left(\frac{q}{p}\right)$ if both $p$ and $q$ are congruent to $3$ modulo $4$.
\end{enumerate}
\end{thm}
%%%%%
%%%%%
%%%%%
%%%%%
%%%%%
%%%%%
%%%%%
%%%%%
%%%%%
%%%%%
%%%%%
%%%%%
%%%%%
%%%%%
%%%%%
%%%%%
%%%%%
%%%%%
%%%%%
%%%%%
\end{document}
