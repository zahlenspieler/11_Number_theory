\documentclass[12pt]{article}
\usepackage{pmmeta}
\pmcanonicalname{SuranyisTheorem}
\pmcreated{2013-03-22 13:43:00}
\pmmodified{2013-03-22 13:43:00}
\pmowner{mathcam}{2727}
\pmmodifier{mathcam}{2727}
\pmtitle{Suranyi's theorem}
\pmrecord{10}{34398}
\pmprivacy{1}
\pmauthor{mathcam}{2727}
\pmtype{Theorem}
\pmcomment{trigger rebuild}
\pmclassification{msc}{11A99}
%\pmkeywords{Erdos sum squares}

\endmetadata

% this is the default PlanetMath preamble.  as your knowledge
% of TeX increases, you will probably want to edit this, but
% it should be fine as is for beginners.

% almost certainly you want these
\usepackage{amssymb}
\usepackage{amsmath}
\usepackage{amsfonts}

% used for TeXing text within eps files
%\usepackage{psfrag}
% need this for including graphics (\includegraphics)
%\usepackage{graphicx}
% for neatly defining theorems and propositions
%\usepackage{amsthm}
% making logically defined graphics
%%%\usepackage{xypic}

% there are many more packages, add them here as you need them

% define commands here
\begin{document}
Suranyi's theorem states that every integer $k$ can be expressed as the following sum:
$$ k=\pm 1^2 \pm 2^2 \pm \cdots \pm m^2 $$ 
for some $m \in \mathbb{Z}^+$.\\
We prove this by induction, taking the first four whole numbers as our \PMlinkescapetext{base} cases:
$$ 0 = 1^2 + 2^2 - 3^2 + 4^2 - 5^2 - 6^2 + 7^2 $$
$$ 1 = 1^2 $$
$$ 2 = -1^2 - 2^2 - 3^2 + 4^2 $$
$$ 3 = -1^2 + 2^2$$
\\Now it suffices to prove that if the theorem is true for $k$ then
it is also true for $k+4$.
\\As
$$ (m+1)^2 - (m+2)^2 - (m+3)^2 + (m+4)^2 = 4 $$
it's simple to finish the proof:
\\if $ k=\pm 1^2 \pm \cdots \pm m^2 $ then
$$ (k+4)=\pm 1^2 \pm \cdots \pm m^2 + (m+1)^2 - (m+2)^2 - (m+3)^2 + (m+4)^2$$
and we are done.
%%%%%
%%%%%
\end{document}
