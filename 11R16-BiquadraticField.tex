\documentclass[12pt]{article}
\usepackage{pmmeta}
\pmcanonicalname{BiquadraticField}
\pmcreated{2013-03-22 15:56:24}
\pmmodified{2013-03-22 15:56:24}
\pmowner{Wkbj79}{1863}
\pmmodifier{Wkbj79}{1863}
\pmtitle{biquadratic field}
\pmrecord{16}{37949}
\pmprivacy{1}
\pmauthor{Wkbj79}{1863}
\pmtype{Definition}
\pmcomment{trigger rebuild}
\pmclassification{msc}{11R16}
\pmsynonym{biquadratic number field}{BiquadraticField}
\pmrelated{BiquadraticExtension}
\pmrelated{BiquadraticEquation2}

\endmetadata

% this is the default PlanetMath preamble.  as your knowledge
% of TeX increases, you will probably want to edit this, but
% it should be fine as is for beginners.

% almost certainly you want these
\usepackage{amssymb}
\usepackage{amsmath}
\usepackage{amsfonts}

% used for TeXing text within eps files
%\usepackage{psfrag}
% need this for including graphics (\includegraphics)
%\usepackage{graphicx}
% for neatly defining theorems and propositions
%\usepackage{amsthm}
% making logically defined graphics
%%%\usepackage{xypic}

% there are many more packages, add them here as you need them

% define commands here

\begin{document}
A {\sl biquadratic field\/} (or {\sl biquadratic number field\/}) is a biquadratic extension of $\mathbb{Q}$.  To discuss these fields more easily, the set $S$ will be defined to be the set of all squarefree integers not equal to $1$.  Thus, any biquadratic field is of the form $\mathbb{Q}(\sqrt{m}, \sqrt{n})$ for distinct elements $m$ and $n$ of $S$.

Let $\displaystyle k=\frac{mn}{(\gcd(m,n))^2}$.  It can easily be verified that $k \in S$, $k \neq m$, and $k \neq n$.  Since $\displaystyle \sqrt{k}=\frac{\sqrt{mn}}{\gcd(m,n)} \in \mathbb{Q}(\sqrt{m}, \sqrt{n})$, the three distinct quadratic subfields of $\mathbb{Q}(\sqrt{m}, \sqrt{n})$ are $\mathbb{Q}(\sqrt{m})$, $\mathbb{Q}(\sqrt{n})$, and $\mathbb{Q}(\sqrt{k})$.  Note that $\mathbb{Q}(\sqrt{k})=\mathbb{Q}(\sqrt{mn})$.

Of the three cyclotomic fields of \PMlinkname{degree}{ExtensionField} four over $\mathbb{Q}$, $\mathbb{Q}(\omega_8)$ and $\mathbb{Q}(\omega_{12})$ are biquadratic fields.  The quadratic subfields of $\mathbb{Q}(\omega_8)$ are $\mathbb{Q}(\sqrt{2})$, $\mathbb{Q}(\sqrt{-1})$, and $\mathbb{Q}(\sqrt{-2})$; the quadratic subfields of $\mathbb{Q}(\omega_{12})$ are $\mathbb{Q}(\sqrt{3})$, $\mathbb{Q}(\sqrt{-1})$, and $\mathbb{Q}(\sqrt{-3})$.

Note that the only rational prime $p$ for which $e(P|p)=4$ is possible in a biquadratic field is $p=2$.  (The notation $e(P|p)$ refers to the ramification index of the prime ideal $P$ over $p$.)  This occurs for biquadratic fields $\mathbb{Q}(\sqrt{m}, \sqrt{n})$ in which exactly two of $m$, $n$, and $k$ are \PMlinkname{equivalent}{Congruences} to $2 \operatorname{mod} 4$ and the other is \PMlinkescapetext{equivalent} to $3 \operatorname{mod} 4$.  For example, in $\mathbb{Q}(\omega_8)=\mathbb{Q}(\sqrt{2}, \sqrt{-1})$, we have that $e(P|2)=4$.

Certain biquadratic fields provide excellent counterexamples to statements that some people might think to be true.  For example, the biquadratic field $K=\mathbb{Q}(\sqrt{2}, \sqrt{-3})$ is useful for demonstrating that a subring of a principal ideal domain need not be a principal ideal domain.  It can easily be verified that $\mathcal{O}_K$ (the ring of integers of $K$) is a principal ideal domain, but $\mathbb{Z}[\sqrt{-6}]$, which is a subring of $\mathcal{O}_K$, is not a principal ideal domain.  Also, biquadratic fields of the form $L=\mathbb{Q}(\sqrt{m}, \sqrt{n})$ with $m$ and $n$ distinct elements of $S$ such that $m \equiv 1 \operatorname{mod} 3$ and $n \equiv 1 \operatorname{mod} 3$ are useful for demonstrating that rings of integers need not have \PMlinkname{power bases over $\mathbb{Z}$}{PowerBasisOverMathbbZ}.  Note that $3$ splits completely in both $\mathbb{Q}(\sqrt{m})$ and $\mathbb{Q}(\sqrt{n})$ and thus in $L$.  Therefore, $3\mathcal{O}_L=P_1P_2P_3P_4$ for distinct \PMlinkname{prime ideals}{PrimeIdeal} $P_1$, $P_2$, $P_3$, and $P_4$ of $\mathcal{O}_L$.  Now suppose $\mathcal{O}_L=\mathbb{Z}[\alpha]$ for some $\alpha \in L$.  Then $L=\mathbb{Q}(\alpha)$, and the minimal polynomial $f$ for $\alpha$ over $\mathbb{Q}$ has degree $4$.  This yields that $f$, considered as a polynomial over $\mathbb{F}_3$, is supposed to factor into four distinct monic polynomials of degree $1$, which is a contradiction.
%%%%%
%%%%%
\end{document}
