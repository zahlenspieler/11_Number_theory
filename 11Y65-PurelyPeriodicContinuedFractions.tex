\documentclass[12pt]{article}
\usepackage{pmmeta}
\pmcanonicalname{PurelyPeriodicContinuedFractions}
\pmcreated{2013-03-22 18:04:44}
\pmmodified{2013-03-22 18:04:44}
\pmowner{rm50}{10146}
\pmmodifier{rm50}{10146}
\pmtitle{purely periodic continued fractions}
\pmrecord{6}{40615}
\pmprivacy{1}
\pmauthor{rm50}{10146}
\pmtype{Theorem}
\pmcomment{trigger rebuild}
\pmclassification{msc}{11Y65}
\pmclassification{msc}{11A55}

\endmetadata

\usepackage{amssymb}
\usepackage{amsmath}
\usepackage{amsfonts}
\usepackage{amsthm}

\newcommand{\Abs}[1]{\left\lvert #1 \right\rvert}
\newtheorem{thm}{Theorem}
\newtheorem{cor}[thm]{Corollary}
\newtheorem{lem}[thm]{Lemma}
\newtheorem{prop}[thm]{Proposition}
\newtheorem{ax}{Axiom}

\theoremstyle{definition}
\newtheorem{defn}{Definition}
\begin{document}
We know that periodic continued fractions represent quadratic irrationals; this article characterizes purely periodic continued fractions. We will use freely the results on convergents to a continued fraction.

\begin{thm} (Galois) A quadratic irrational $t$ is represented by a purely periodic simple continued fraction if and only if $t>1$ and its conjugate $s$ under the transformation $\sqrt{d}\mapsto -\sqrt{d}$ satisfies $-1<s<0$.
\end{thm}

\begin{proof} Suppose first that $t$ is represented by a purely periodic continued fraction
\[t=[\overline{a_0,a_1,\ldots,a_{r-1}}].\]
Note that $a_0\geq 1$ since it appears again in the continued fraction. Thus $t>1$. The $r^{\mathrm{th}}$ complete convergent is again $t$, so that we have
\[t = \frac{p_{r-2}+tp_{r-1}}{q_{r-2}+tq_{r-1}}\]
so that 
\[q_{r-1}t^2 + (q_{r-2}-p_{r-1})t - p_{r-2}=0\]
Consider the polynomial $f(x) = q_{r-1}x^2 + (q_{r-2}-p_{r-1})x - p_{r-2}$. $f(t)=0$, so the other root of $f(x)$ is the conjugate $s$ of $t$. But $f(-1) = (p_{r-1}-p_{r-2})+(q_{r-1}-q_{r-2})>0$ since the $p_i$ and the $q_i$ are both strictly increasing sequences, while $f(0) = -p_{r-2}<0$. Thus $s$ lies between $-1$ and $0$ and we are done.

Now suppose that $t>1$ and $-1<s<0$, and let the continued fraction for $t$ be $[a_0,a_1,\ldots]$. Let $t_n$ be the $n^{\mathrm{th}}$ complete convergent of $t$, and $s_n = \overline{t_n}$. Thus $s_0=s$. Then
\[t=t_0 = a_0 + \frac{1}{t_1}\]
so that
\[s_0 = \overline{t_0} = a_0+\frac{1}{\overline{t_1}} = a_0+\frac{1}{s_1}\]
and thus
\[\frac{1}{s_1} = -a_0 + s_0 < -a_0 \leq -1\]
so that $-1 < s_1 < 0$. Inductively, we have $-1 < s_n = \overline{t_n} < 0$ for all $n\geq 0$. Suppose now that the continued fraction for $t$ is not purely periodic, but rather has the form
\[t = [a_0,a_1,\ldots,a_{k-1},\overline{a_k,a_{k+1},\ldots,a_{k+j-1}}]\]
for $k\geq 1$. Then $t_k = t_{k+j}$ and so
\[t_{k-1}-t_{k+j-1} = \left(a_{k-1}+\frac{1}{t_k}\right) - \left(a_{k+j-1}+\frac{1}{t_{k+j}}\right) = a_{k-1}-a_{k+j-1}\]
But $a_{k-1}\neq a_{k+j-1}$, otherwise $a_{k-1}$ would have been the first element of the repeating period. Thus $t_{k-1}-t_{k+j-1}$ is a nonzero integer and thus $s_{k-1}-s_{k+j-1}$ is as well. But $-1<s_{k-1}-s_{k+j-1}<1$, which is a contradiction. Thus $k=0$ and the continued fraction is purely periodic.
\end{proof}

\begin{thebibliography}{10}
\bibitem{bib:rockett}
A.M.~Rockett~\& P.~Sz\"usz, \emph{Continued Fractions}, World Scientific Publishing, 1992.
\end{thebibliography}




%%%%%
%%%%%
\end{document}
