\documentclass[12pt]{article}
\usepackage{pmmeta}
\pmcanonicalname{PracticalNumber}
\pmcreated{2013-03-22 14:12:17}
\pmmodified{2013-03-22 14:12:17}
\pmowner{mathcam}{2727}
\pmmodifier{mathcam}{2727}
\pmtitle{practical number}
\pmrecord{6}{35637}
\pmprivacy{1}
\pmauthor{mathcam}{2727}
\pmtype{Definition}
\pmcomment{trigger rebuild}
\pmclassification{msc}{11A25}

% this is the default PlanetMath preamble.  as your knowledge
% of TeX increases, you will probably want to edit this, but
% it should be fine as is for beginners.

% almost certainly you want these
\usepackage{amssymb}
\usepackage{amsmath}
\usepackage{amsfonts}

% used for TeXing text within eps files
%\usepackage{psfrag}
% need this for including graphics (\includegraphics)
%\usepackage{graphicx}
% for neatly defining theorems and propositions
%\usepackage{amsthm}
% making logically defined graphics
%%%\usepackage{xypic}

% there are many more packages, add them here as you need them

% define commands here
\begin{document}
A positive integer $m$ is called a \emph{practical number} if every positive integer $n<m$ is a sum of distinct positive divisors of $m$. 

{\bf \PMlinkescapetext{Lemma}.}
An integer $\,m\ge 2,$
$\,m=p_1^{\alpha_1}p_2^{\alpha_2}\cdots p_\ell^{\alpha_\ell},$
with primes $p_1<p_2<\dots<p_\ell$ and integers $\alpha_i\ge 1,$
is practical\index{practical} if and only if $\,p_1=2\,$ and, for
$i=2,3,\dots,\ell,$
$$p_i\le\sigma\!\left(p_1^{\alpha_
1}p_2^{\alpha_2}\cdots p_{i-1}
^{\alpha_{i-1}}\right)+1,$$
where $\sigma(n)$ denotes the sum of the positive divisors of $n.$

Let $P(x)$ be the counting function of practical numbers. 
Saias~\cite{Sai}, using suitable sieve methods introduced by Tenenbaum
\cite{Te1,Te2}, proved
a good estimate in terms of
a Chebishev-type theorem: for suitable
constants $c_1$ and $c_2$,
$$c_1\frac x{\log x}<P(x)<c_2\frac x{\log x}.$$

In \cite{Me1} Melfi proved a Goldbach-type result showing
that every even positive integer is a sum of two practical numbers, and that there exist infinitely many triplets of practical numbers of the form $m-2,m,m+2$.


\begin{thebibliography}{9}


\bibitem{Me1} G.~Melfi, On two conjectures about practical numbers,
{\it J.~Number Theory} {\bf 56} (1996), 205--210.

\bibitem{Sai} E.~Saias, Entiers \`a
diviseurs denses 1, {\it J.~Number Theory} {\bf 62} (1997), 163--191. 

\bibitem{Te1} G.~Tenenbaum, Sur un probl\`eme de crible et ses applications, 
Ann. Sci.
\'Ec. Norm. Sup. (4) {\bf 19 } (1986), 1--30.

\bibitem{Te2} G.~Tenenbaum, Sur un probl\`eme de crible et ses 
applications, 2.
Corrigendum et \'etude du graphe divisoriel, Ann. Sci. \'Ec. Norm. Sup.
(4) {\bf 28 } (1995), 115--127.

\end{thebibliography}
%%%%%
%%%%%
\end{document}
