\documentclass[12pt]{article}
\usepackage{pmmeta}
\pmcanonicalname{ProofOfLagrangesFoursquareTheorem}
\pmcreated{2013-03-22 13:21:07}
\pmmodified{2013-03-22 13:21:07}
\pmowner{CWoo}{3771}
\pmmodifier{CWoo}{3771}
\pmtitle{proof of Lagrange's four-square theorem}
\pmrecord{13}{33871}
\pmprivacy{1}
\pmauthor{CWoo}{3771}
\pmtype{Proof}
\pmcomment{trigger rebuild}
\pmclassification{msc}{11P05}

% this is the default PlanetMath preamble.  as your knowledge
% of TeX increases, you will probably want to edit this, but
% it should be fine as is for beginners.

% almost certainly you want these
\usepackage{amssymb}
\usepackage{amsmath}
\usepackage{amsfonts}

% used for TeXing text within eps files
%\usepackage{psfrag}
% need this for including graphics (\includegraphics)
\usepackage{graphicx}
% for neatly defining theorems and propositions
\usepackage{amsthm}
% making logically defined graphics
%%%\usepackage{xypic}

% there are many more packages, add them here as you need them
\usepackage[usenames]{color}
\usepackage{latexsym}
% define commands here
\newtheorem{thm}{Theorem}
\newtheorem{lem}{Lemma}
\begin{document}
\PMlinkescapeword{complete}
\PMlinkescapeword{clearly}

The following proof is essentially Lagrange's original, from around 1770.  First, we need three lemmas.

\begin{lem} For any integers $a,b,c,d,w,x,y,z$, 
\begin{eqnarray*}
(a^2+b^2+c^2+d^2)(w^2+x^2+y^2+z^2) & = & (aw+bx+cy+dz)^2 \\
& + & (ax-bw-cz+dy)^2 \\
& + & (ay+bz-cw-dx)^2 \\
& + & (az-by+cx-dw)^2.
\end{eqnarray*}
\end{lem}
This is the Euler four-square identity, q.v., with different notation.

\begin{lem} If $2m$ is a sum of two squares, then so is $m$.\end{lem}

\begin{proof} Say $2m=x^2+y^2$. Then $x$ and $y$ are both even or both odd.
Therefore, in the identity $$m=\Big(\frac{x-y}{2}\Big)^2 +
\Big(\frac{x+y}{2}\Big)^2,$$
both fractions on the right side are integers. \end{proof}

\begin{lem} If $p$ is an odd prime,
then $a^2+b^2+1=kp$ for some integers $a,b,k$ with $0<k<p$. \end{lem}

\begin{proof} Let $p = 2n + 1$.  Consider the sets $$A:=\lbrace a^2 \mid a=0,1,...,n\rbrace\quad\mbox{ and }\quad B:=\lbrace -b^2-1 \mid b=0,1,...,n \rbrace.$$  We have the following facts:
\begin{enumerate}
\item No two elements in $A$ are congruent mod $p$, for if $a^2\equiv c^2 \pmod p$, then either $p\mid (a-c)$ or $p\mid (a+c)$ by unique factorization of primes.  Since $a-c, a+c \le 2n< p$, and $0\le a,c$, we must have $a=c$.  
\item Similarly, no two elements in $B$ are congruent mod $p$.  
\item Furthermore, $A\cap B=\varnothing$ since elements of $A$ are all non-negative, while elements of $B$ are all negative.
\item Therefore, $C:=A\cup B$ has $2n+2$, or $p+1$ elements.
\end{enumerate}
Therefore, by the pigeonhole principle, two elements in $C$ must be congruent mod $p$.  In addition, by the first two facts, the two elements must come from different sets.  As a result, we have the following equation: $$a^2+b^2+1=kp$$
for some $k$. Clearly $k$ is positive. Also, $p^2 = (2n + 1)^2 > 2n^2 + 1 \ge a^2+b^2+1 = kp$, so $p > k$.
\end{proof}

Basically, Lemma 3 says that for any prime $p$, some multiple $0<m<p$ of $p$ is a sum of four squares, since $a^2+b^2+1=a^2+b^2+1^2+0^2$.

\begin{proof}[Proof of Theorem]  By Lemma 1 we need only show that an arbitrary prime $p$ is a sum of
four squares. Since that is trivial for $p=2$, suppose $p$ is odd. By Lemma 3, we know $$mp=a^2+b^2+c^2+d^2$$
for some $m,a,b,c,d$ with $0<m<p$.  If $m=1$, then we are done.  To complete the proof, we will show that if $m>1$ then $np$ is a sum of four squares for some $n$ with $1\le n<m$.

If $m$ is even, then none, two, or all four of $a,b,c,d$ are even; in any of those cases, we may break up $a,b,c,d$ into two groups, each group containing elements of the same parity.  Then Lemma 2 allows us to take $n = m/2$.

Now assume $m$ is odd but $>1$. Write
\begin{eqnarray*}
       w & \equiv & a\pmod{m} \\
       x & \equiv & b\pmod{m} \\
       y & \equiv & c\pmod{m} \\
       z & \equiv & d\pmod{m}
\end{eqnarray*}
where $w,x,y,z$ are all in the interval $(-m/2, m/2)$. We have
       $$w^2+x^2+y^2+z^2 < 4\cdot \frac{m^2}{4} = m^2$$
       $$w^2+x^2+y^2+z^2\equiv 0\pmod{m}.$$
So $w^2+x^2+y^2+z^2 = nm$ for some integer non-negative $n$.  Since $w^2+x^2+y^2+z^2 < m^2$, $n<m$. In addition, if $n=0$, then $w=x=y=z=0$, so that $a\equiv b\equiv c\equiv d \equiv 0 \pmod m$, which implies $mp=a^2+b^2+c^2+d^2=m^2q$, or that $m|p$.  But $p$ is prime, forcing $m=p$, and contradicting $m<p$.  So $0<n<m$.  Look at the product $(a^2+b^2+c^2+d^2)(w^2+x^2+y^2+z^2)$ and examine Lemma 1. On the left is $nm^2p$. One the right, we have a sum of four squares.  Evidently three of them $$ax-bw-cz+dy=(ax-bw)+(dy-cz)$$ $$ay+bz-cw-dx=(ay-cw)+(bz-dx)$$ $$az-by+cx-dw=(az-dw)+(cx-by)$$
are multiples of $m$. The same is true of the other sum on the
right in Lemma 1:
       $$aw+bx+cy+dz \equiv w^2+x^2+y^2+z^2 \equiv 0\pmod{m}.$$
The equation in Lemma 1 can therefore be divided through by $m^2$.
The result is an expression for $np$ as a sum of four squares.
Since $0<n<m$, the proof is complete.
\end{proof}

\textbf{Remark:} Lemma 3 can be improved: it is enough for
$p$ to be an odd number, not necessarily prime.
But that stronger statement requires a longer proof.
%%%%%
%%%%%
\end{document}
