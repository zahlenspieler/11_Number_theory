\documentclass[12pt]{article}
\usepackage{pmmeta}
\pmcanonicalname{LinearCongruence}
\pmcreated{2013-03-22 14:18:15}
\pmmodified{2013-03-22 14:18:15}
\pmowner{Mathprof}{13753}
\pmmodifier{Mathprof}{13753}
\pmtitle{linear congruence}
\pmrecord{15}{35763}
\pmprivacy{1}
\pmauthor{Mathprof}{13753}
\pmtype{Definition}
\pmcomment{trigger rebuild}
\pmclassification{msc}{11A41}
\pmsynonym{first degree congruence}{LinearCongruence}
\pmrelated{QuadraticCongruence}
\pmrelated{SolvingLinearDiophantineEquation}
\pmrelated{GodelsBetaFunction}
\pmrelated{ConditionalCongruences}

% this is the default PlanetMath preamble.  as your knowledge
% of TeX increases, you will probably want to edit this, but
% it should be fine as is for beginners.

% almost certainly you want these
\usepackage{amssymb}
\usepackage{amsmath}
\usepackage{amsfonts}

% used for TeXing text within eps files
%\usepackage{psfrag}
% need this for including graphics (\includegraphics)
%\usepackage{graphicx}
% for neatly defining theorems and propositions
%\usepackage{amsthm}
% making logically defined graphics
%%%\usepackage{xypic}

% there are many more packages, add them here as you need them

% define commands here
\begin{document}
The {\em linear congruence}
                $$ax\equiv b \pmod{m},$$
where $a$, $b$ and $m$ are known integers and\, $\gcd{(a,\,m)} = 1$,\, has exactly one solution $x$ in $\mathbb{Z}$, when numbers congruent to each other are not regarded as different.\, The solution can be obtained as
                 $$x = a^{\varphi(m)-1}b,$$
where $\varphi$ means Euler's phi-function.

Solving the linear congruence also gives the solution of the {\em Diophantine equation}
                    $$ax\!-\!my = b,$$
and conversely.\, If\, $x = x_0$,\, $y = y_0$\, is a solution of this equation, then the general solution is
\[\begin{cases}
x = x_0\!+\!km,\\
y = y_0\!+\!ka,\\
\end{cases}\]
where\, $k = 0$, $\pm1$, $\pm2$, \ldots
%%%%%
%%%%%
\end{document}
