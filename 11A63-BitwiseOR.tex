\documentclass[12pt]{article}
\usepackage{pmmeta}
\pmcanonicalname{BitwiseOR}
\pmcreated{2013-03-22 17:02:48}
\pmmodified{2013-03-22 17:02:48}
\pmowner{PrimeFan}{13766}
\pmmodifier{PrimeFan}{13766}
\pmtitle{bitwise OR}
\pmrecord{4}{39336}
\pmprivacy{1}
\pmauthor{PrimeFan}{13766}
\pmtype{Definition}
\pmcomment{trigger rebuild}
\pmclassification{msc}{11A63}
\pmrelated{BitwiseAND}
\pmrelated{BitwiseXOR}
\pmrelated{BitwiseNOT}

% this is the default PlanetMath preamble.  as your knowledge
% of TeX increases, you will probably want to edit this, but
% it should be fine as is for beginners.

% almost certainly you want these
\usepackage{amssymb}
\usepackage{amsmath}
\usepackage{amsfonts}

% used for TeXing text within eps files
%\usepackage{psfrag}
% need this for including graphics (\includegraphics)
%\usepackage{graphicx}
% for neatly defining theorems and propositions
%\usepackage{amsthm}
% making logically defined graphics
%%%\usepackage{xypic}

% there are many more packages, add them here as you need them

% define commands here

\begin{document}
{\em Bitwise OR} is a bit-level operation on two binary values which indicates which bits are set in either value. For each position $i$, if the bit $d_i$ in either value is 1, then $d_i$ of the result is 1, otherwise it's 0. For example, given 50 and 163 in two unsigned bytes, a bitwise OR returns 179.

\begin{tabular}{|r|c|c|c|c|c|c|c|c|}
   & 0 & 0 & 1 & 1 & 0 & 0 & 1 & 0 \\
OR & 1 & 0 & 1 & 0 & 0 & 0 & 1 & 1 \\
 = & 1 & 0 & 1 & 1 & 0 & 0 & 1 & 1 \\
\end{tabular}

In most high-level programming languages that offer bitwise OR, the usual operator is a single pipe line (|), not to be confused with a double pipe line (||) which performs a Boolean OR, returning a True or False value.

The Windows Calculator offers bitwise OR in scientific calculator mode, while the Mac OS X Calculator offers it in programmer mode.
%%%%%
%%%%%
\end{document}
