\documentclass[12pt]{article}
\usepackage{pmmeta}
\pmcanonicalname{PollardsP1Algorithm}
\pmcreated{2013-03-22 16:43:47}
\pmmodified{2013-03-22 16:43:47}
\pmowner{PrimeFan}{13766}
\pmmodifier{PrimeFan}{13766}
\pmtitle{Pollard's $p - 1$ algorithm}
\pmrecord{8}{38951}
\pmprivacy{1}
\pmauthor{PrimeFan}{13766}
\pmtype{Algorithm}
\pmcomment{trigger rebuild}
\pmclassification{msc}{11A41}
\pmsynonym{Pollard $p - 1$ algorithm}{PollardsP1Algorithm}
\pmsynonym{Pollard's p minus 1 algorithm}{PollardsP1Algorithm}
\pmsynonym{Pollard p minus 1 algorithm}{PollardsP1Algorithm}
\pmsynonym{Pollard $p - 1$ method}{PollardsP1Algorithm}
\pmsynonym{Pollard's p minus 1 method}{PollardsP1Algorithm}
\pmsynonym{Pollard p minus 1 method}{PollardsP1Algorithm}
\pmsynonym{Pollard's $p - 1$ method}{PollardsP1Algorithm}

\endmetadata

% this is the default PlanetMath preamble.  as your knowledge
% of TeX increases, you will probably want to edit this, but
% it should be fine as is for beginners.

% almost certainly you want these
\usepackage{amssymb}
\usepackage{amsmath}
\usepackage{amsfonts}

% used for TeXing text within eps files
%\usepackage{psfrag}
% need this for including graphics (\includegraphics)
%\usepackage{graphicx}
% for neatly defining theorems and propositions
%\usepackage{amsthm}
% making logically defined graphics
%%%\usepackage{xypic}

% there are many more packages, add them here as you need them

% define commands here

\begin{document}
{\em Pollard's $p - 1$ method} is an integer factorization algorithm devised by John Pollard in 1974 to take advantage of Fermat's little theorem. Theoretically, trial division always returns a result (though of course in practice the computing engine's resources could be exhausted or the user might not to be around to care for the result). Pollard's $p - 1$ algorithm, on the other hand, was designed from the outset to cope with the possibility of failure to return a result.

Choose a test cap $B$ and call Pollard's $p - 1$ algorithm with a positive integer.

\begin{enumerate}
\item Use a simple method (such as the sieve of Eratosthenes) to find primes $p < B$, or even probable primes.
\item Choose an integer $a$ coprime to $n$. If $n$ is odd (which can be tested easily enough by looking at the least significant bit) then one possible choice is $a = 2$. For even $n$, we could choose $\lfloor \sqrt{n} \rfloor + 1$.
\item Find the exponent $e$ such that $p^e \le B$. Then for each $p < B$, compute $b = a^{p^e} \mod n$ and see if $1 < \gcd(b - 1, n) < n$. If that's the case, return those results of the greatest common divisor function, exit.
\item If the GCD function consistently returned 1s, one could try a higher test cap and try again from step 1.
\item If the GCD function consistently returned $n$ itself, this could indicate that $a$ is in fact not coprime to $n$, in which case one could try going back to step 2 to pick a different $a$.
\item Throw a failure exception.
\end{enumerate}

For example, $n = 221$. Of course it's overkill to use the Pollard $p - 1$ algorithm on such a small number, but it helps to keep the example simple. Since the running time of the algorithm is exponential, Crandall and Pomerance suggest picking small $B$. So for this example let's pick $B = 10$, the primes are then 2, 3, 5, 7, and the exponents are 3, 2, 1, 1. Since 221 is odd, we try $a = 2$. So we see that $2^8 \mod 221 = 35$, and $\gcd(34, 221) = 17$. We also see $3^9 \mod 221 = 14$ and $\gcd(13, 221) = 13$. Multiplication immediately verifies that $13 \times 17 = 221$.

As of version 5.2, Pollard's $p - 1$ algorithm is one of the methods used by Mathematica's \verb=FactorInteger= function after ferreting out small factors by trial division.

\begin{thebibliography}{1}
\bibitem{rc} R. Crandall \& C. Pomerance, {\it Prime Numbers: A Computational Perspective}, Springer, NY, 2001: 5.4.1
\end{thebibliography}
%%%%%
%%%%%
\end{document}
