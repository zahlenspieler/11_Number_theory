\documentclass[12pt]{article}
\usepackage{pmmeta}
\pmcanonicalname{FundamentalTheoremOfGaloisTheory}
\pmcreated{2013-03-22 12:08:31}
\pmmodified{2013-03-22 12:08:31}
\pmowner{yark}{2760}
\pmmodifier{yark}{2760}
\pmtitle{fundamental theorem of Galois theory}
\pmrecord{9}{31327}
\pmprivacy{1}
\pmauthor{yark}{2760}
\pmtype{Theorem}
\pmcomment{trigger rebuild}
\pmclassification{msc}{11S20}
\pmclassification{msc}{11R32}
\pmclassification{msc}{12F10}
\pmclassification{msc}{13B05}
\pmsynonym{Galois theory}{FundamentalTheoremOfGaloisTheory}
\pmsynonym{Galois correspondence}{FundamentalTheoremOfGaloisTheory}
\pmrelated{GaloisTheoreticDerivationOfTheCubicFormula}
\pmrelated{GaloisTheoreticDerivationOfTheQuarticFormula}
\pmrelated{InfiniteGaloisTheory}
\pmrelated{GaloisGroup}

\usepackage{amssymb}
\usepackage{amsmath}
\usepackage{amsfonts}

\begin{document}
Let $L/F$ be a Galois extension of finite degree,
with Galois group $G := \operatorname{Gal}(L/F)$.
There is a bijective, inclusion-reversing correspondence
between subgroups of $G$ and extensions of $F$ contained in $L$, given by
\begin{itemize}
\item $K \mapsto \operatorname{Gal}(L/K)$,
       for any field $K$ with $F \subseteq K \subseteq L$.
\item $H \mapsto L^H$ (the fixed field of $H$ in $L$),
       for any subgroup $H \leq G$.
\end{itemize}
The extension $L^H/F$ is normal if and only if $H$ is a normal subgroup of $G$,
and in this case the homomorphism $G \longrightarrow \operatorname{Gal}(L^H/F)$
given by $\sigma \mapsto \sigma|_{L^H}$
induces (via the first isomorphism theorem)
a natural identification $\operatorname{Gal}(L^H/F) = G/H$
between the Galois group of $L^H/F$ and the quotient group $G/H$.

For the case of Galois extensions of infinite degree,
see the entry on infinite Galois theory.
%%%%%
%%%%%
%%%%%
\end{document}
