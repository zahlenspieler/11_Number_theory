\documentclass[12pt]{article}
\usepackage{pmmeta}
\pmcanonicalname{HowToUseATable}
\pmcreated{2013-03-22 16:36:25}
\pmmodified{2013-03-22 16:36:25}
\pmowner{rspuzio}{6075}
\pmmodifier{rspuzio}{6075}
\pmtitle{how to use a table}
\pmrecord{5}{38803}
\pmprivacy{1}
\pmauthor{rspuzio}{6075}
\pmtype{Example}
\pmcomment{trigger rebuild}
\pmclassification{msc}{11B25}

% this is the default PlanetMath preamble.  as your knowledge
% of TeX increases, you will probably want to edit this, but
% it should be fine as is for beginners.

% almost certainly you want these
\usepackage{amssymb}
\usepackage{amsmath}
\usepackage{amsfonts}

% used for TeXing text within eps files
%\usepackage{psfrag}
% need this for including graphics (\includegraphics)
%\usepackage{graphicx}
% for neatly defining theorems and propositions
%\usepackage{amsthm}
% making logically defined graphics
%%%\usepackage{xypic}

% there are many more packages, add them here as you need them

% define commands here

\begin{document}
In this entry, how to use a table to multiply numbers will be explained and illustrated with examples.  The general rule is as follows:

Suppose you want to multiply two numbers.  Find the first number
in the very first column. Follow the row in which this number appears
until you land in the column which is labelled with the second number on
top. The number on which you land is your answer.

Let us illustrate with an example.  To keep things simple, we will
use a 5 by 5 multiplication table

\begin{tabular}
{|c|l|l|l|l|l|}
$\times$ & 1 & 2 & 3 & 4 & 5 \\
1 & 1 & 2 & 3 & 4 & 5 \\
2 & 2 & 4 & 6 & 8 & 10 \\
3 & 3 & 6 & 9 & 12 & 15 \\
4 & 4 & 8 & 12 & 16 & 20 \\
5 & 5 & 10 & 15 & 20 & 25 \\
\end{tabular}

Suppose we want to figure out $3 \times 4$.  We first find the
row which starts with ``3''.  That row is highlighted below:

\begin{tabular}
{|c|l|l|l|l|l|}
$\times$ & 1 & 2 & 3 & 4 & 5 \\
1 & 1 & 2 & 3 & 4 & 5 \\
2 & 2 & 4 & 6 & 8 & 10 \\
{\bf 3} & {\bf 3} & {\bf 6} & {\bf 9} & {\bf 12} & {\bf 15} \\
4 & 4 & 8 & 12 & 16 & 20 \\
5 & 5 & 10 & 15 & 20 & 25 \\
\end{tabular}

Next we follow this row until we find the column which has ``4''
on top.  This column is highlighted below:

\begin{tabular}
{|c|l|l|l|l|l|}
$\times$ & 1 & 2 & 3 & {\bf 4} & 5 \\
1 & 1 & 2 & 3 & {\bf 4} & 5 \\
2 & 2 & 4 & 6 & {\bf 8} & 10 \\
{\bf 3} & {\bf 3} & {\bf 6} & {\bf 9} & \underline{\bf 12} & {\bf 15} \\
4 & 4 & 8 & 12 & {\bf 16} & 20 \\
5 & 5 & 10 & 15 & {\bf 20} & 25 \\
\end{tabular}

Note that ``12'' appears where the highlighted row and the highlighted
column cross.  (For convenience, it has been underlined.) Therefore, 
we found that $3 \times 4 = 12$.
%%%%%
%%%%%
\end{document}
