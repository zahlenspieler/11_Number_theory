\documentclass[12pt]{article}
\usepackage{pmmeta}
\pmcanonicalname{BinomialTheorem}
\pmcreated{2013-03-22 11:46:47}
\pmmodified{2013-03-22 11:46:47}
\pmowner{KimJ}{5}
\pmmodifier{KimJ}{5}
\pmtitle{binomial theorem}
\pmrecord{17}{30247}
\pmprivacy{1}
\pmauthor{KimJ}{5}
\pmtype{Theorem}
\pmcomment{trigger rebuild}
\pmclassification{msc}{11B65}
\pmclassification{msc}{06F25}
\pmclassification{msc}{03E20}
%\pmkeywords{number theory combinatorics}
\pmrelated{BinomialFormula}
\pmrelated{BinomialCoefficient}
\pmrelated{BernoulliDistribution2}
\pmrelated{UsingThePrimitiveElementOfBiquadraticField}

\usepackage{amssymb}
\usepackage{amsmath}
\usepackage{amsfonts}
\usepackage{graphicx}
%%%%\usepackage{xypic}
\begin{document}
The binomial theorem is a formula for the expansion of $(a+b)^n$, for $n$ a positive integer and $a$ and $b$ any two real (or complex) numbers, into a sum of powers of $a$ and $b$. More precisely,
$$(a+b)^n  = a^n + \binom{n}{1} a^{n-1}b + \binom{n}{2} a^{n-2}b^2 + \cdots + b^n .
$$
For example, if $n$ is 3 or 4, we have:
\begin{eqnarray*}
(a+b)^3 &= a^3 + 3 a^2 b + 3 a b^2 + b^3 \\
(a+b)^4 &= a^4 + 4 a^3 b + 6 a^2 b^2 + 4 a b^3 + b^4 .
\end{eqnarray*}

This result actually holds more generally if $a$ and $b$ belong to a commutative \PMlinkname{rig}{Rig}.
%%%%%
%%%%%
%%%%%
%%%%%
\end{document}
