\documentclass[12pt]{article}
\usepackage{pmmeta}
\pmcanonicalname{PouletNumber}
\pmcreated{2013-03-22 18:11:12}
\pmmodified{2013-03-22 18:11:12}
\pmowner{CompositeFan}{12809}
\pmmodifier{CompositeFan}{12809}
\pmtitle{Poulet number}
\pmrecord{6}{40759}
\pmprivacy{1}
\pmauthor{CompositeFan}{12809}
\pmtype{Definition}
\pmcomment{trigger rebuild}
\pmclassification{msc}{11A51}
\pmsynonym{Sarrus number}{PouletNumber}

% this is the default PlanetMath preamble.  as your knowledge
% of TeX increases, you will probably want to edit this, but
% it should be fine as is for beginners.

% almost certainly you want these
\usepackage{amssymb}
\usepackage{amsmath}
\usepackage{amsfonts}

% used for TeXing text within eps files
%\usepackage{psfrag}
% need this for including graphics (\includegraphics)
%\usepackage{graphicx}
% for neatly defining theorems and propositions
%\usepackage{amsthm}
% making logically defined graphics
%%%\usepackage{xypic}

% there are many more packages, add them here as you need them

% define commands here

\begin{document}
A {\em Poulet number} or {\em Sarrus number} is a composite integer $n$ such that $2^n \equiv 2 \mod n$. In other words, a base 2 pseudoprime (thus a Poulet number that satisfies the congruence for other bases is a Carmichael number). The first few Poulet numbers are 341, 561, 645, 1105, 1387, 1729, 1905, listed in A001567 of Sloane's OEIS. 

For example, 561 is a Poulet number, since $2^{561} - 2$ is 75479248496430827044831091619765377
81833842440832880856752412600491248324784297704172253450355317535082936750061527
689799541169259849585265122868502865392087298790653950 and that's divisible by 561. The number 561 is not prime, it has the prime factors 3, 11, and 17.

Poulet numbers are counterexamples to the Chinese hypothesis.

\begin{thebibliography}{1}
\bibitem{dl} Derrick Henry Lehmer, ``Errata for Poulet's table,'' {\it Math. Comp.} {\bf 25} 25 (1971): 944 - 945.
\end{thebibliography}
%%%%%
%%%%%
\end{document}
