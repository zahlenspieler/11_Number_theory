\documentclass[12pt]{article}
\usepackage{pmmeta}
\pmcanonicalname{PrimitiveSemiperfectNumber}
\pmcreated{2013-03-22 16:19:00}
\pmmodified{2013-03-22 16:19:00}
\pmowner{Mravinci}{12996}
\pmmodifier{Mravinci}{12996}
\pmtitle{primitive semiperfect number}
\pmrecord{5}{38442}
\pmprivacy{1}
\pmauthor{Mravinci}{12996}
\pmtype{Definition}
\pmcomment{trigger rebuild}
\pmclassification{msc}{11D85}
\pmsynonym{primitive pseudoperfect number}{PrimitiveSemiperfectNumber}
\pmsynonym{irreducible semiperfect number}{PrimitiveSemiperfectNumber}

% this is the default PlanetMath preamble.  as your knowledge
% of TeX increases, you will probably want to edit this, but
% it should be fine as is for beginners.

% almost certainly you want these
\usepackage{amssymb}
\usepackage{amsmath}
\usepackage{amsfonts}

% used for TeXing text within eps files
%\usepackage{psfrag}
% need this for including graphics (\includegraphics)
%\usepackage{graphicx}
% for neatly defining theorems and propositions
%\usepackage{amsthm}
% making logically defined graphics
%%%\usepackage{xypic}

% there are many more packages, add them here as you need them

% define commands here

\begin{document}
Parse the sequence of semiperfect numbers in ascending order, striking out any numbers that are divisible by any smaller number in the sequence. A number that remains after this process is then called a \emph{primitive semiperfect number} or  \emph{primitive pseudoperfect number}. Put another way, none of the divisors of a primitive semiperfect number are themselves semiperfect. The first few primitive semiperfect numbers are 6, 20, 28, 88, 104, 272, 304, 350, listed in A6036 of Sloane's OEIS.

The list of the largest prime factors of each of the primitive semiperfect numbers gives the list of prime numbers almost in order but with many duplicates interspersed.

The smallest odd primitive semiperfect number is 945, which is also the smallest odd abundant number.

%%%%%
%%%%%
\end{document}
