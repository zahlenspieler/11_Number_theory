\documentclass[12pt]{article}
\usepackage{pmmeta}
\pmcanonicalname{ListOfAllImaginaryQuadraticExtensionsWhoseRingOfIntegersIsAPID}
\pmcreated{2013-03-22 16:56:42}
\pmmodified{2013-03-22 16:56:42}
\pmowner{rm50}{10146}
\pmmodifier{rm50}{10146}
\pmtitle{list of all imaginary quadratic extensions whose ring of integers is a PID}
\pmrecord{7}{39213}
\pmprivacy{1}
\pmauthor{rm50}{10146}
\pmtype{Result}
\pmcomment{trigger rebuild}
\pmclassification{msc}{11E12}
\pmclassification{msc}{11R29}
\pmclassification{msc}{11E16}
\pmrelated{EuclideanNumberField}
\pmrelated{ImaginaryQuadraticField}
\pmrelated{LemmaForImaginaryQuadraticFields}

\endmetadata

% this is the default PlanetMath preamble.  as your knowledge
% of TeX increases, you will probably want to edit this, but
% it should be fine as is for beginners.

% almost certainly you want these
\usepackage{amssymb}
\usepackage{amsmath}
\usepackage{amsfonts}

% used for TeXing text within eps files
%\usepackage{psfrag}
% need this for including graphics (\includegraphics)
%\usepackage{graphicx}
% for neatly defining theorems and propositions
\usepackage{amsthm}
% making logically defined graphics
%%%\usepackage{xypic}

% there are many more packages, add them here as you need them

% define commands here
\newcommand{\Nats}{\mathbb{N}}
\newcommand{\Ints}{\mathbb{Z}}
\newcommand{\Reals}{\mathbb{R}}
\newcommand{\Complex}{\mathbb{C}}
\newcommand{\Rats}{\mathbb{Q}}
\newcommand{\Gal}{\operatorname{Gal}}
\newcommand{\Cl}{\operatorname{Cl}}
\newcommand{\Alg}{\mathcal{O}}
\newcommand{\ol}{\overline}
\newcommand{\Leg}[2]{\left(\frac{#1}{#2}\right)}
\renewcommand{\frak}[1]{\mathfrak{#1}}
%
%% \theoremstyle{plain} %% This is the default
\newtheorem{thm}{Theorem}
\newtheorem{cor}[thm]{Corollary}
\newtheorem{lem}[thm]{Lemma}
\newtheorem{prop}[thm]{Proposition}
\newtheorem{ax}{Axiom}

\theoremstyle{definition}
\newtheorem{defn}{Definition}
\begin{document}
Gauss conjectured that for any $\Delta<0$, $\Delta\equiv 0,1\pmod 4$, then $\mathcal{C}_{\Delta}=1$ precisely when
\[
\Delta = -3,-4,-7,-8,-11,-12,-16,-19,-27,-28,-43,-67,-163
\]
In fact, he believed that as $\Delta \rightarrow -\infty, \Delta\equiv 0,1\pmod 4$, so does the number of classes of (primitive positive integral binary quadratic) forms with \PMlinkid{discriminant}{IntegralBinaryQuadraticForms} $\Delta$.

It is relatively easy to show that the only $\Delta\equiv 0\pmod 4$ with this property are the ones in this list; that proof is given in an addendum to this article.

However, proving the remainder of Gauss' hypotheses, regarding the odd values in the list, proved significantly harder. In the first half of the $20^{\mathrm{th}}$ century, Siegel showed that there was at most one such value beyond what Gauss had found. Heegner, Stark, and Baker showed, about $30$ years later, that there are in fact no more (\cite{bib:Heegner},\cite{bib:Stark},\cite{bib:Baker}). 

Thus given an imaginary quadratic extension $K$, it follows that the ring of integers of $K$, denoted $\Alg_K$, is a PID if and only if the class group of $K$ is trivial if and only if there is only one class of primitive quadratic forms of the appropriate \PMlinkid{discriminant}{DiscriminantOfANumberField} $d_K$ if and only if $d_K$ is in the set above. So in particular, there are a finite number of imaginary quadratic extensions of $\Rats$ whose ring of integers is a PID (and hence a UFD).

The values of $\Delta$ above that correspond to $\Alg_K$ for some $K$ are:
\begin{center}
\begin{tabular}{c|c|c}
$\Delta=d_K$ & $K$ & $\Alg_K$\\
	\hline
$-3$& $\Rats(\sqrt{-3})$ & $\Ints\left[\frac{1+\sqrt{-3}}{2}\right]$\\
$-4$ & $\Rats(\sqrt{-1})$ & $\Ints[\sqrt{-1}]$\\
$-7$ & $\Rats(\sqrt{-7})$ & $\Ints\left[\frac{1+\sqrt{-7}}{2}\right]$\\
$-8$ & $\Rats(\sqrt{-2})$ & $\Ints[\sqrt{-2}]$\\
$-11$ & $\Rats(\sqrt{-11})$ & $\Ints\left[\frac{1+\sqrt{-11}}{2}\right]$\\
$-19$ & $\Rats(\sqrt{-19})$ & $\Ints\left[\frac{1+\sqrt{-19}}{2}\right]$\\
$-43$ & $\Rats(\sqrt{-43})$ & $\Ints\left[\frac{1+\sqrt{-43}}{2}\right]$\\
$-67$ & $\Rats(\sqrt{-67})$ & $\Ints\left[\frac{1+\sqrt{-67}}{2}\right]$\\
$-163$ & $\Rats(\sqrt{-163})$ & $\Ints\left[\frac{1+\sqrt{-163}}{2}\right]$
\end{tabular}
\end{center}

We therefore get
\begin{thm} (Stark-Heegner)
\newline
If $d < 0$, then the class number of $\Rats(\sqrt{d})$ is equal to $1$ if and only if
\[d = -1, -2, -3, -7, -11, -19, -43, -67, \text{ or }-163\]
(where $d=-1,-2$ correspond to $\Delta=-4,-8$ and otherwise $d=\Delta$).
\end{thm}

How about the other four values $\Delta=-12,-16,-27,-28$? Each of these corresponds to a non-maximal \PMlinkid{order}{OrderInAnAlgebra} in a quadratic extension (i.e. a proper subring of the ring of algebraic integers). Specifically, we have
\begin{align*}
\Delta=-12\quad&\leftrightarrow\quad \Ints\left[2\frac{1+\sqrt{-3}}{2}\right]=\Ints[\sqrt{-3}]\subsetneq \Alg_K\text{ for }K=\Rats(\sqrt{-3})\\
\Delta=-16\quad&\leftrightarrow\quad \Ints[2\sqrt{-1}]\subsetneq \Alg_K\text{ for }K=\Rats(\sqrt{-1})\\
\Delta=-27\quad&\leftrightarrow\quad \Ints\left[3\frac{1+\sqrt{-3}}{2}\right]\subsetneq \Alg_K\text{ for }K=\Rats(\sqrt{-3})\\
\Delta=-28\quad&\leftrightarrow\quad \Ints\left[2\frac{1+\sqrt{-7}}{2}\right]=\Ints[\sqrt{-7}]\subsetneq \Alg_K\text{ for }K=\Rats(\sqrt{-7})
\end{align*}
Note that this does \emph{not} mean that these rings are PIDs, since the invertible ideals in an order that is not the entire ring of integers do not include all ideals.
\begin{thebibliography}{10}
\bibitem{bib:cox}
Cox,~D.A. \emph{Primes of the Form $x^2 + ny^2$: Fermat, Class Field Theory, and Complex Multiplication}, Wiley 1997.
\bibitem{bib:Heegner}
Heegner,~K., \emph{Diophantische Analysis und Modulfunktionen}, Math. Zeit., 56 (1952), pp. 227-253. 
\bibitem{bib:Stark}
Stark,~H.M., \emph{A complete determination of the complex quadratic fields with class number one}, Mich. Math. J., 14 (1967), pp. 1-27.
\bibitem{bib:Baker}
Baker,~A., \emph{Linear forms in the logarithms of algebraic numbers}, Mathematika, 13 (1966), pp. 204-216. 
\end{thebibliography}
%%%%%
%%%%%
\end{document}
