\documentclass[12pt]{article}
\usepackage{pmmeta}
\pmcanonicalname{ProofOfLucassTheorem}
\pmcreated{2013-03-22 13:22:56}
\pmmodified{2013-03-22 13:22:56}
\pmowner{mathcam}{2727}
\pmmodifier{mathcam}{2727}
\pmtitle{proof of Lucas's theorem}
\pmrecord{7}{33917}
\pmprivacy{1}
\pmauthor{mathcam}{2727}
\pmtype{Proof}
\pmcomment{trigger rebuild}
\pmclassification{msc}{11A07}
\pmrelated{BinomialCoefficient}

% this is the default PlanetMath preamble.  as your knowledge
% of TeX increases, you will probably want to edit this, but
% it should be fine as is for beginners.

% almost certainly you want these
\usepackage{amssymb}
\usepackage{amsmath}
\usepackage{amsfonts}

% used for TeXing text within eps files
%\usepackage{psfrag}
% need this for including graphics (\includegraphics)
%\usepackage{graphicx}
% for neatly defining theorems and propositions
%\usepackage{amsthm}
% making logically defined graphics
%%%\usepackage{xypic}

% there are many more packages, add them here as you need them

% define commands here
\newcommand{\pfac}[1]{\left(#1!\right)_p}
\begin{document}

Let $n \ge m \in \mathbb{N}$. Let $a_0, b_0$ be the least non-negative residues
of $n,m \pmod{p}$, respectively. (Additionally, we set $r=n-m$, and $r_0$ is the
least non-negative residue of $r$ modulo $p$.) Then the statement follows from
\begin{displaymath}
\binom{n}{m} \equiv \binom{a_0}{b_0}\binom{\left\lfloor
\frac{n}{p}\right\rfloor}{\left\lfloor\frac{m}{p}\right\rfloor} \pmod{p}.
\end{displaymath}

We define the 'carry indicators' $c_i$ for all $i \ge 0$ as
\begin{displaymath}
c_i =\left(
\begin{array}{lll}
1 & \mbox{if}
 & b_i +r_i \ge p \\
0 & \mbox{otherwise}
\end{array}\right.,
\end{displaymath}
and additionally $c_{-1} =0$.

The special case $s=1$ of Anton's congruence is:
\begin{equation}
\pfac{n} \equiv \left(- 1\right)^{\left\lfloor \frac{n}{p}\right\rfloor}\cdot
a_0! \pmod{p},
\end{equation}
where $a_0$ as defined above, and $\pfac{n}$ is the product of numbers $\le n$
not divisible by $p$.
So we have
\begin{displaymath}
\frac{n!}{\left\lfloor
\frac{n}{p}\right\rfloor!p^{\left\lfloor\frac{n}{p}\right\rfloor}} =\pfac{n}
\equiv (-1)^{\left\lfloor\frac{n}{p}\right\rfloor}\cdot a_0! \pmod{p}
\end{displaymath}
When dividing  by the left-hand terms of the congruences for $m$ and $r$, we see
that the power of $p$ is
\begin{displaymath}
\left\lfloor\frac{n}{p}\right\rfloor -\left\lfloor\frac{m}{p}\right\rfloor
-\left\lfloor\frac{r}{p}\right\rfloor =c_0
\end{displaymath}
So we get the congruence
\begin{displaymath}
\frac{\binom{n}{m}}{\binom{\left\lfloor\frac{n}{p}\right\rfloor}{\left\lfloor\frac{m}{p}\right\rfloor}\cdot
p^{c_0}} \equiv (-1)^{c_0}\cdot \binom{a_0}{b_0},
\end{displaymath}
or equivalently
\begin{equation}
\label{L1}
\left(\frac{-1}{p}\right)^{c_0}\cdot \binom{n}{m} \equiv
\binom{a_0}{b_0}\binom{\left\lfloor\frac{n}{p}\right\rfloor}{\left\lfloor\frac{m}{p}\right\rfloor} \pmod{p}.
\end{equation}

Now we consider $c_0 =1$. Since
\[a_0 =b_0 +r_0 -\underbrace{pc_0=p},\]
$b_0 +r_0 < b_0 \leftrightarrow c_0 =1 \leftrightarrow b_0 -(p -r_0) =a_0 <b_0
\leftrightarrow \binom{a_0}{b_0}=0 $. So both congruences--the one in the
statement and (\ref{L1})-- produce the same results.
%%%%%
%%%%%
\end{document}
