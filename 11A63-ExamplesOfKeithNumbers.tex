\documentclass[12pt]{article}
\usepackage{pmmeta}
\pmcanonicalname{ExamplesOfKeithNumbers}
\pmcreated{2013-03-22 16:01:35}
\pmmodified{2013-03-22 16:01:35}
\pmowner{PrimeFan}{13766}
\pmmodifier{PrimeFan}{13766}
\pmtitle{examples of Keith numbers}
\pmrecord{6}{38067}
\pmprivacy{1}
\pmauthor{PrimeFan}{13766}
\pmtype{Example}
\pmcomment{trigger rebuild}
\pmclassification{msc}{11A63}

% this is the default PlanetMath preamble.  as your knowledge
% of TeX increases, you will probably want to edit this, but
% it should be fine as is for beginners.

% almost certainly you want these
\usepackage{amssymb}
\usepackage{amsmath}
\usepackage{amsfonts}

% used for TeXing text within eps files
%\usepackage{psfrag}
% need this for including graphics (\includegraphics)
%\usepackage{graphicx}
% for neatly defining theorems and propositions
%\usepackage{amsthm}
% making logically defined graphics
%%%\usepackage{xypic}

% there are many more packages, add them here as you need them

% define commands here

\begin{document}
Take the number 47 as it is written in base 10, and start a Fibonacci-like sequence from its digits: 4, 7, 11, 18, 29, 47, ...

42, on the other hand, isn't a Keith number: 4, 2, 6, 8, 14, 22, 36, 58, etc.

For 3-digit numbers, the analogy is to the tribonacci sequence: 1, 9, 7, 17, 33, 57, 107, 197, ...

In binary, 2 is a Keith number: 1, 0, 1, 1, 10, ... Generalizing, $b$ is a Keith number in base $b$ only if it appears in the Fibonacci sequence, and $b^x$ if in the applicable $x$bonacci sequence.
%%%%%
%%%%%
\end{document}
