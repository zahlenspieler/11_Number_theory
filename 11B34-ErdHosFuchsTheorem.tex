\documentclass[12pt]{article}
\usepackage{pmmeta}
\pmcanonicalname{ErdHosFuchsTheorem}
\pmcreated{2013-03-22 13:20:59}
\pmmodified{2013-03-22 13:20:59}
\pmowner{bbukh}{348}
\pmmodifier{bbukh}{348}
\pmtitle{Erd\H{o}s-Fuchs theorem}
\pmrecord{12}{33868}
\pmprivacy{1}
\pmauthor{bbukh}{348}
\pmtype{Theorem}
\pmcomment{trigger rebuild}
\pmclassification{msc}{11B34}
%\pmkeywords{representation functions}

\endmetadata

\usepackage{amssymb}
\usepackage{amsmath}
\usepackage{amsfonts}

\makeatletter
\@ifundefined{bibname}{}{\renewcommand{\bibname}{References}}
\makeatother
\begin{document}
\PMlinkescapeword{states}
\PMlinkescapeword{represent}
\PMlinkescapeword{Ph}

Let $A$ be a set of natural numbers. Let $R_n(A)$ be the number of ways to represent $n$ as a sum of two elements in $A$, that is, 
\begin{equation*}
R_n(A)=\sum_{\substack{a_i+a_j=n\\a_i,a_j \in A}} 1.
\end{equation*}

Erd\H{o}s-Fuchs theorem \cite{cite:erdos_fuchs_lowbnd, cite:halberstam_sequences} states that if $c>0$, then
\begin{equation}\label{erdfcheq}
\sum_{n \leq N} R_n(A)=c N + o\left(N^{\frac{1}{4}}\log^{-\frac{1}{2}}N\right)
\end{equation}
cannot hold.

On the other hand, Ruzsa \cite{cite:ruzsa_upperdfch} constructed a set for which
\begin{equation*}
\sum_{n \leq N} R_n(A)=c N + O\left(N^{\frac{1}{4}}\log N\right).
\end{equation*}

Montgomery and Vaughan \cite{cite:montgomery_vaughan_fch} improved on the original Erd\H{o}s-Fuchs theorem by showing that for every $c>0$ 
\begin{equation*}
\max_{N\leq M}\left\lvert \sum_{n \leq N} R_n(A)-c n\right\rvert=
\Omega\left(M^{\frac{1}{4}}\log^{-\frac{1}{4}}M\right)
\end{equation*}
holds. In \cite{cite:montgomery_vaughan_fch} a result of Jurkat is cited
which appeared
in the Ph.~D. thesis of Hayashi \cite{cite:hayashi_fch} which 
improves $N^{\frac{1}{4}}\log^{-\frac{1}{2}}N$ in \eqref{erdfcheq} to $N^{\frac{1}{4}}$. 


\begin{thebibliography}{1}

\bibitem{cite:erdos_fuchs_lowbnd}
Paul Erd\H{o}s and Wolfgang~H.J. Fuchs.
\newblock On a problem of additive number theory.
\newblock {\em J. Lond. Math. Soc.}, 31:67--73, 1956.
\newblock \PMlinkexternal{Zbl 0070.04104}{http://www.emis.de/cgi-bin/zmen/ZMATH/en/quick.html?type=html&an=0070.04104}.

\bibitem{cite:halberstam_sequences}
Heini Halberstam and Klaus~Friedrich Roth.
\newblock {\em Sequences}.
\newblock Springer-Verlag, second edition, 1983.
\newblock \PMlinkexternal{Zbl 0498.10001}{http://www.emis.de/cgi-bin/zmen/ZMATH/en/quick.html?type=html&an=0498.10001}.

\bibitem{cite:hayashi_fch}
E.~K. Hayashi.
\newblock {\em Omega theorems for the iterated additive convolution of
  non-negative arithmetic function}.
\newblock PhD thesis, University of Illinois at Urbana-Champaign, 1973.

\bibitem{cite:montgomery_vaughan_fch}
H.~L. Montgomery and R.~C. Vaughan.
\newblock On the {E}rd{\H o}s-{F}uchs theorems.
\newblock In {\em A tribute to Paul Erd\H os}, pages 331--338. Cambridge Univ.
  Press, Cambridge, 1990.
\newblock \PMlinkexternal{Zbl 0715.11005}{http://www.emis.de/cgi-bin/zmen/ZMATH/en/quick.html?type=html&an=0715.11005}.

\bibitem{cite:ruzsa_upperdfch}
Imre Ruzsa.
\newblock A converse to a theorem of {E}rd{\H{o}}s and {F}uchs.
\newblock {\em J. Number Theory}, 62(2):397--402, 1997.
\newblock \PMlinkexternal{Zbl 0872.11014}{http://www.emis.de/cgi-bin/zmen/ZMATH/en/quick.html?type=html&an=0872.11014}.

\end{thebibliography}
%BibTeX bibliography
%@ARTICLE{cite:erdos_fuchs_lowbnd,
% author    = {Paul Erd{\H{o}}s and Fuchs, Wolfgang H.J.},
% title     = {On a problem of additive number theory},
% journal   = {J. Lond. Math. Soc.},
% volume    = 31,
% year      = 1956,
% pages     = {67--73},
% note      = {\PMlinkexternal{Zbl 0070.04104}{http://www.emis.de/cgi-bin/zmen/ZMATH/en/quick.html?type=html&an=0070.04104}}
%}
%
%@ARTICLE{cite:ruzsa_upperdfch,
% author    = {Imre Ruzsa},
% title     = {A converse to a theorem of {E}rd{\H{o}}s and {F}uchs},
% journal   = {J. Number Theory},
% volume    = 62,
% number    = 2,
% pages     = {397--402},
% year      = 1997,
% note      = {\PMlinkexternal{Zbl 0872.11014}{http://www.emis.de/cgi-bin/zmen/ZMATH/en/quick.html?type=html&an=0872.11014}}
%}
%
%@BOOK{cite:halberstam_sequences,
% author    = {Heini Halberstam and Klaus Friedrich Roth},
% title     = "Sequences",
% publisher = {Springer-Verlag},
% year      = {1983},
% edition   = {Second}
% note      = {\PMlinkexternal{Zbl 0498.10001}{http://www.emis.de/cgi-bin/zmen/ZMATH/en/quick.html?type=html&an=0498.10001}}
%}
%@phdthesis{cite:hayashi_fch,
% author    = {E. K. Hayashi},
% title     = {Omega theorems for the iterated additive convolution of non-negative arithmetic function},
% school    = {University of Illinois at Urbana-Champaign},
% year      = {1973}
%}
%%%%%
%%%%%
\end{document}
