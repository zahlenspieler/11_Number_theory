\documentclass[12pt]{article}
\usepackage{pmmeta}
\pmcanonicalname{LuckyPrime}
\pmcreated{2013-03-22 16:55:31}
\pmmodified{2013-03-22 16:55:31}
\pmowner{PrimeFan}{13766}
\pmmodifier{PrimeFan}{13766}
\pmtitle{lucky prime}
\pmrecord{4}{39188}
\pmprivacy{1}
\pmauthor{PrimeFan}{13766}
\pmtype{Definition}
\pmcomment{trigger rebuild}
\pmclassification{msc}{11A41}

% this is the default PlanetMath preamble.  as your knowledge
% of TeX increases, you will probably want to edit this, but
% it should be fine as is for beginners.

% almost certainly you want these
\usepackage{amssymb}
\usepackage{amsmath}
\usepackage{amsfonts}

% used for TeXing text within eps files
%\usepackage{psfrag}
% need this for including graphics (\includegraphics)
%\usepackage{graphicx}
% for neatly defining theorems and propositions
%\usepackage{amsthm}
% making logically defined graphics
%%%\usepackage{xypic}

% there are many more packages, add them here as you need them

% define commands here

\begin{document}
A {\em lucky prime} is a lucky number that is also a prime number. So, in a sense, such numbers are doubly ``lucky'' because they've survived two different sieves. The first few are 3, 7, 13, 31, 37, 43, 67, 73, 79, 127, 151, \PMlinkname{163}{OneHundredSixtyThree}, 193, etc., listed in A031157 of Sloane's OEIS. It's known that there are infinitely many primes and it's known that there are infinitely many lucky numbers, but it's not known if there are infinitely many lucky primes.
%%%%%
%%%%%
\end{document}
