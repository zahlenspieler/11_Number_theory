\documentclass[12pt]{article}
\usepackage{pmmeta}
\pmcanonicalname{FermatNumbersAreCoprime}
\pmcreated{2013-03-22 14:51:24}
\pmmodified{2013-03-22 14:51:24}
\pmowner{yark}{2760}
\pmmodifier{yark}{2760}
\pmtitle{Fermat numbers are coprime}
\pmrecord{5}{36529}
\pmprivacy{1}
\pmauthor{yark}{2760}
\pmtype{Theorem}
\pmcomment{trigger rebuild}
\pmclassification{msc}{11A51}

\endmetadata

\usepackage{amsthm}

\newtheorem*{thm*}{Theorem}
\begin{document}
\begin{thm*}
Any two Fermat numbers are coprime.
\end{thm*}

Proof.\\
Let $F_m$ and $F_n$ two Fermat numbers, and assume $m<n$.
Let $d$ a positive common divisor of $F_n$ and $F_m$, that is
\[
  d \mid F_m,\qquad d\mid F_n.
\]

If $d\mid F_m$ then $d\mid F_1F_2\cdots F_{n-1}$
since some factor must be $F_m$ itself.
But $F_n-F_1F_2\cdots F_{n-1}=2$ and so $d \mid 2$.
Since $d$ is odd, we must have $d=1$.

Therefore, the greatest common divisor of any two Fermat numbers must be $1$.

Q.E.D.
%%%%%
%%%%%
\end{document}
