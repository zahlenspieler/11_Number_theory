\documentclass[12pt]{article}
\usepackage{pmmeta}
\pmcanonicalname{OddNumber}
\pmcreated{2013-03-22 17:42:30}
\pmmodified{2013-03-22 17:42:30}
\pmowner{PrimeFan}{13766}
\pmmodifier{PrimeFan}{13766}
\pmtitle{odd number}
\pmrecord{7}{40152}
\pmprivacy{1}
\pmauthor{PrimeFan}{13766}
\pmtype{Definition}
\pmcomment{trigger rebuild}
\pmclassification{msc}{11A51}
\pmrelated{EvenNumber}
\pmrelated{SumOfOddNumbers}

% this is the default PlanetMath preamble.  as your knowledge
% of TeX increases, you will probably want to edit this, but
% it should be fine as is for beginners.

% almost certainly you want these
\usepackage{amssymb}
\usepackage{amsmath}
\usepackage{amsfonts}

% used for TeXing text within eps files
%\usepackage{psfrag}
% need this for including graphics (\includegraphics)
%\usepackage{graphicx}
% for neatly defining theorems and propositions
%\usepackage{amsthm}
% making logically defined graphics
%%%\usepackage{xypic}

% there are many more packages, add them here as you need them

% define commands here

\begin{document}
An {\em odd number} $n$ is an integer of the form $2m + 1$, and as such it is not divisible by 2. In terms of congruences, $n \equiv 1 \mod 2$, and in its binary representation the least significant bit is 1. With the exception of 2, all prime numbers are odd numbers.

The addition of two odd numbers gives an even number, or any even number of odd summands gives an even number as a result. For example, $11 + 13 + 17 + 19 = 60$.

But the multiplication of two odd numbers, or any even amount of odd multiplicands always gives an odd number. For example, $11 \times 13 \times 17 \times 19 = 46189$.

A negative number raised to the power of an odd number gives a negative number. For example, $(-1)^{11} = (-1)^{13} = (-1)^{17} = (-1)^{19} = -1$ (compare $(-1)^{10} = (-1)^{12} = (-1)^{16} = (-1)^{18} = 1$).

The sum of the first $n$ consecutive positive odd numbers is $n^2$.

The famous Gregory series which gives a quarter of $\pi$ is an alternating sum of the reciprocals of the odd numbers: $$\frac{\pi}{4} = \sum_{i = 0}^\infty (-1)^i \frac{1}{2i + 1} = 1 - \frac{1}{3} + \frac{1}{5} - \frac{1}{7} + \frac{1}{9} - \ldots$$ However, note that this series is not absolutely convergent.

%%%%%
%%%%%
\end{document}
