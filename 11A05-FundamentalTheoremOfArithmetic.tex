\documentclass[12pt]{article}
\usepackage{pmmeta}
\pmcanonicalname{FundamentalTheoremOfArithmetic}
\pmcreated{2013-03-22 11:46:03}
\pmmodified{2013-03-22 11:46:03}
\pmowner{CWoo}{3771}
\pmmodifier{CWoo}{3771}
\pmtitle{fundamental theorem of arithmetic}
\pmrecord{21}{30221}
\pmprivacy{1}
\pmauthor{CWoo}{3771}
\pmtype{Theorem}
\pmcomment{trigger rebuild}
\pmclassification{msc}{11A05}
\pmclassification{msc}{17B66}
\pmclassification{msc}{17B45}
%\pmkeywords{number theory}
\pmrelated{Divisibility}
\pmrelated{UFD}
\pmrelated{AnyNonzeroIntegerIsQuadraticResidue}
\pmrelated{NumberTheory}
\pmdefines{prime divisor}
\pmdefines{prime factor}

\endmetadata

\usepackage{amssymb}
\usepackage{amsmath}
\usepackage{amsfonts}
\usepackage{graphicx}
%%%%\usepackage{xypic}
\begin{document}
Each positive integer $n$ has a unique \PMlinkescapetext{decomposition} as a product
\[
n = \prod_{i=0}^l {p_i}^{a_i}
\]
of positive powers of its distinct positive {\it prime divisors} $p_i$.  The {\it prime divisor} of $n$ means a (rational) prime number \PMlinkname{dividing}{Divisibility} $n$.  A synonymous name is {\it prime factor}. 

The \PMlinkescapetext{decomposition is unique up to the order} of the prime divisors and for\, $n=1$\, is an empty product.

For some results it is useful to assume that 
$p_i < p_j$ whenever $i < j$.

The FTA was the last of the fundamental theorems proven by C.F. Gauss. Gauss wrote his proof in ``Discussions on Arithmetic'' ({\em Disquisitiones Arithmeticae}) in 1801 formalizing congruences. Euclid and Greeks used prime properties of the FTA without rigorously proving its existence. It appears that the fundamentals of the FTA were used centuries before, and after, the Greeks within Egyptian fraction arithmetic. Fibonacci, for example, wrote in Egyptian fraction arithmetic, used three notations to detail Euclidean and medieval factoring methods.   
%%%%%
%%%%%
%%%%%
%%%%%
\end{document}
