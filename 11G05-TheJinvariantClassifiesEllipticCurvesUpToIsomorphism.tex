\documentclass[12pt]{article}
\usepackage{pmmeta}
\pmcanonicalname{TheJinvariantClassifiesEllipticCurvesUpToIsomorphism}
\pmcreated{2013-03-22 15:06:25}
\pmmodified{2013-03-22 15:06:25}
\pmowner{alozano}{2414}
\pmmodifier{alozano}{2414}
\pmtitle{the $j$-invariant classifies elliptic curves up to isomorphism}
\pmrecord{5}{36839}
\pmprivacy{1}
\pmauthor{alozano}{2414}
\pmtype{Theorem}
\pmcomment{trigger rebuild}
\pmclassification{msc}{11G05}
\pmclassification{msc}{14H52}
\pmrelated{IsomorphismOfVarieties}
\pmrelated{ArithmeticOfEllipticCurves}

\endmetadata

% this is the default PlanetMath preamble.  as your knowledge
% of TeX increases, you will probably want to edit this, but
% it should be fine as is for beginners.

% almost certainly you want these
\usepackage{amssymb}
\usepackage{amsmath}
\usepackage{amsthm}
\usepackage{amsfonts}

% used for TeXing text within eps files
%\usepackage{psfrag}
% need this for including graphics (\includegraphics)
%\usepackage{graphicx}
% for neatly defining theorems and propositions
%\usepackage{amsthm}
% making logically defined graphics
%%%\usepackage{xypic}

% there are many more packages, add them here as you need them

% define commands here

\newtheorem{thm}{Theorem}
\newtheorem{defn}{Definition}
\newtheorem{prop}{Proposition}
\newtheorem{lemma}{Lemma}
\newtheorem{cor}{Corollary}

% Some sets
\newcommand{\Nats}{\mathbb{N}}
\newcommand{\Ints}{\mathbb{Z}}
\newcommand{\Reals}{\mathbb{R}}
\newcommand{\Complex}{\mathbb{C}}
\newcommand{\Rats}{\mathbb{Q}}
\begin{document}
In this entry, an isomorphism over $K$ should be understood in the sense of the entry isomorphism of varieties.

\begin{thm}
Let $K$ be a field, and let $\overline{K}$ be a fixed algebraic closure of $K$.
\begin{enumerate}
\item Two elliptic curves $E_1$ and $E_2$ are \PMlinkname{isomorphic}{IsomorphismOfVarieties} (over $\overline{K}$) if and only if they have the same $j$-invariant, i.e. $j(E_1)=j(E_2)$.\\
\item Let $j_0\in \overline{K}$ be fixed. There exists an elliptic curve $E$ defined over the field $K(j_0)$ such that $j(E)=j_0$.
\end{enumerate}
\end{thm}

\begin{proof}
For part $2$:
\begin{itemize}
\item For $j_0=0$, the curve $E_0\colon y^2+y=x^3$ satisfies $j(E)=0$;
\item For $j_0=1728$, the curve $E_{1728} \colon y^2=x^3+x$ satisfies $j(E_{1728})=1728$;
\item If $j_0\neq 0, 1728$ consider the elliptic curve:
$$E=E_{j_0} \colon y^2+xy=x^3-\frac{36}{j_0-1728}x-\frac{1}{j_0-1728}.$$
It satisfies $j(E)=j_0$ and it is defined over $K(j_0)$.
\end{itemize}
\end{proof}
%%%%%
%%%%%
\end{document}
