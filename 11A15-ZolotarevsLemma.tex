\documentclass[12pt]{article}
\usepackage{pmmeta}
\pmcanonicalname{ZolotarevsLemma}
\pmcreated{2013-03-22 13:28:25}
\pmmodified{2013-03-22 13:28:25}
\pmowner{mathcam}{2727}
\pmmodifier{mathcam}{2727}
\pmtitle{Zolotarev's lemma}
\pmrecord{15}{34043}
\pmprivacy{1}
\pmauthor{mathcam}{2727}
\pmtype{Theorem}
\pmcomment{trigger rebuild}
\pmclassification{msc}{11A15}
%\pmkeywords{reciprocity}
\pmrelated{GaussLemma}

\usepackage{amssymb}
\usepackage{amsmath}
\usepackage{amsfonts}
\usepackage{amsthm}
\newtheorem{Lemma}{Lemma}
\newcommand{\legsym}[2]{\left(\frac{#1}{#2}\right)}
\newcommand{\Zn}[1]{\mathbb{Z}_{#1}}
\newcommand{\Zpstar}{\mathbb{Z}_p^*}
\begin{document}
\PMlinkescapeword{size}
\PMlinkescapeword{evidently}
We will identify the ring $\mathbb{Z}_n$
of integers modulo $n$, with the set $\{0,1,\ldots n-1\}$.

\begin{Lemma}[Zolotarev] For any prime number $p$ and
any $m\in\Zpstar$, the Legendre symbol $\legsym{m}{p}$ is equal
to the \PMlinkname{signature}{SignatureOfAPermutation} of the permutation $\tau_m:x\mapsto mx$ of $\Zpstar$.
\end{Lemma}
\begin{proof}
We write $\epsilon(\sigma)$ for the signature of any permutation $\sigma$. If $\sigma$ is a circular permutation on a set of $k$ elements, then $\epsilon(\sigma)=(-1)^{k-1}$.  Let $i$ be the order of $m$ in $\Zpstar$. Then the permutation $\tau_m$ consists of $(p-1)/i$ orbits, each of size $i$, whence
$$\epsilon(\tau_m)=(-1)^{(i-1)(p-1)/i}$$
If $i$ is even, then
$$m^{(p-1)/2}=m^{\frac{i}{2}\frac{p-1}{i}}=(-1)^{\frac{p-1}{i}}=\epsilon(\tau_m)$$
And if $i$ is odd, then $2i$ divides $p-1$, so
$$m^{(p-1)/2}=m^{i\frac{p-1}{2i}}=1=\epsilon(\tau_m).$$
In both cases, the lemma follows from Euler's criterion.
\end{proof}

Lemma 1 extends easily from the Legendre symbol to the Jacobi symbol $\left(\frac{m}{n}\right)$ for odd $n$.  The following is Zolotarev's penetrating proof of the quadratic reciprocity law, using Lemma 1.

\begin{Lemma} Let $\lambda$ be the permutation of the set
$$A_{mn}=\{0,1,\ldots,m-1\}\times \{0,1,\ldots,n-1\}$$
which maps the $k$th element of the sequence
$$(0,0)(0,1)\ldots(0,n-1)(1,0)\ldots(1,n-1)(2,0)\ldots(m-1,n-1),$$
to the $k$th element of the sequence
$$(0,0)(1,0)\ldots(m-1,0)(0,1)\ldots(m-1,1)(0,2)\ldots(m-1,n-1),$$
for every $k$ from $1$ to $mn$. Then
$$\epsilon(\lambda)=(-1)^{m(m-1)n(n-1)/4}$$
and if $m$ and $n$ are both odd,
$$\epsilon(\lambda)=(-1)^{(m-1)(n-1)/4}.$$
\end{Lemma}
\begin{proof}
We will use the fact that the signature of a permutation of
a finite totally ordered set is determined by the number of
inversions of that permutation.
The sequence $(0,0),(0,1)\ldots$
defines on $A_{mn}$ a total order $\le$
in which the relation $(i,j)<(i',j')$ means
$$i<i'\text{ or }(i=i'\text{ and }j<j').$$
But $\lambda(i',j')<\lambda(i,j)$ means
$$j'<j\text{ or }(j'=j\text{ and }i'<i).$$
The only pairs $((i,j),(i',j'))$ that get inverted are, therefore,
the ones with $i<i'$ and $j>j'$.
There are indeed $\binom{m}{2}\binom{n}{2}$
such pairs, proving the first formula, and the second follows easily.
\end{proof}

And finally, we proceed to prove quadratic reciprocity.  Let $p$ and $q$ be distinct odd primes. Denote by $\pi$ the
canonical ring isomorphism $\Zn{pq}\to\Zn{p}\times\Zn{q}$.
Define two permutations $\alpha$ and $\beta$ of $\Zn{p}\times\Zn{q}$ by
$\alpha(x,y)=(qx+y,y)$ and $\beta(x,y)=(x,x+py).$  Finally, define a map $\lambda:\Zn{pq}\to\Zn{pq}$ by $\lambda(x+qy)=px+y$
for $x\in\{0,1,\ldots q-1\}$ and $y\in\{0,1,\ldots p-1\}$.
Evidently $\lambda$ is a permutation.

Note that we have $\pi(qx+y)=(qx+y,y)$ and $\pi(x+py)=(x,x+py)$, so
therefore
$$\pi\circ\lambda\circ\pi^{-1}\circ\alpha=\beta.$$

Let us compare the signatures of the two sides.  The permutation $m\mapsto qx+y$ is the composition of $m\mapsto qx$ and $m\mapsto m+y$. The latter has signature $1$, whence by Lemma 1, $$\epsilon(\alpha)=\legsym{q}{p}^q=\legsym{q}{p}$$
and similarly $$\epsilon(\beta)=\legsym{p}{q}^p=\legsym{p}{q}.$$

By Lemma 2,
$$\epsilon(\pi\circ\lambda\circ\pi^{-1})=(-1)^{(p-1)(q-1)/4}.$$
Thus $$(-1)^{(p-1)(q-1)/4}\legsym{q}{p}=\legsym{p}{q}$$
which is the quadratic reciprocity law.

\textbf{Reference}

G. Zolotarev, Nouvelle d\'emonstration de la loi de r\'eciprocit\'e de Legendre,
Nouv. Ann. Math (2), 11 (1872), 354-362
%%%%%
%%%%%
\end{document}
