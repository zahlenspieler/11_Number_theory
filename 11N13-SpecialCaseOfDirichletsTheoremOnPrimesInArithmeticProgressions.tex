\documentclass[12pt]{article}
\usepackage{pmmeta}
\pmcanonicalname{SpecialCaseOfDirichletsTheoremOnPrimesInArithmeticProgressions}
\pmcreated{2013-03-22 14:35:38}
\pmmodified{2013-03-22 14:35:38}
\pmowner{bbukh}{348}
\pmmodifier{bbukh}{348}
\pmtitle{special case of Dirichlet's theorem on primes in arithmetic progressions}
\pmrecord{9}{36158}
\pmprivacy{1}
\pmauthor{bbukh}{348}
\pmtype{Theorem}
\pmcomment{trigger rebuild}
\pmclassification{msc}{11N13}

\endmetadata

\usepackage{amssymb}
\usepackage{amsmath}
\usepackage{amsfonts}

\makeatletter
\@ifundefined{bibname}{}{\renewcommand{\bibname}{References}}
\makeatother
\begin{document}
The special case of Dirichlet's theorem for primes in arithmetic progressions for primes congruent to $1$ modulo $q$ where $q$ itself is a prime can be established by the following elegant modification of \PMlinkname{Euclid's proof}{ProofThatThereAreInfinitelyManyPrimes}.

Let $f(n)=\frac{n^q-1}{n-1}=1+n+n^2+\dotsb+n^{q-1}$. Let $n>1$ be an integer, and suppose $p\mid f(n)$. Then $n^q\equiv 1\pmod p$ which implies by Lagrange's theorem that either $q\mid p-1$ or $n\equiv 1\pmod p$. In other words, every prime divisor of $f(n)$ is congruent to $1$ modulo $q$ unless $n$ is congruent to $1$ modulo that divisor. 

Suppose there are only finitely many primes that are congruent to $1$ modulo $q$. Let $P$ be twice their product. Note that $P\equiv 2\pmod q$. Let $p$ be any prime divisor of $f(P)$. If $p\equiv 1\pmod q$, then $p\mid P$ which contradicts $f(P)\equiv 1\pmod P$. Therefore, by the above $P\equiv 1\pmod p$. Therefore $f(P)\equiv 1+P+P^2+\dotsb+P^{q-1}\equiv 1+1+1+\dotsb+1\equiv q\pmod p$. Since $q$ is prime, it follows that $p=q$. 
Then $P\equiv 1\pmod p$ implies $P\equiv 1\pmod q$. However, that is inconsistent with our deduction that $P\equiv 2\pmod q$ above. 
Therefore the original assumption that there are only finitely many primes congruent to $1$ modulo $q$ is false.

\begin{thebibliography}{1}

\bibitem{cite:iwaniec_kowalski_ant}
Henryk Iwaniec and Emmanuel Kowalski.
\newblock {\em Analytic Number Theory}, volume~53 of {\em AMS Colloquium
  Publications}.
\newblock AMS, 2004.

\end{thebibliography}
%%%%%
%%%%%
\end{document}
