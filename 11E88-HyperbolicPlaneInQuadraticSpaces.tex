\documentclass[12pt]{article}
\usepackage{pmmeta}
\pmcanonicalname{HyperbolicPlaneInQuadraticSpaces}
\pmcreated{2013-03-22 15:41:47}
\pmmodified{2013-03-22 15:41:47}
\pmowner{CWoo}{3771}
\pmmodifier{CWoo}{3771}
\pmtitle{hyperbolic plane in quadratic spaces}
\pmrecord{9}{37640}
\pmprivacy{1}
\pmauthor{CWoo}{3771}
\pmtype{Definition}
\pmcomment{trigger rebuild}
\pmclassification{msc}{11E88}
\pmclassification{msc}{15A63}
\pmdefines{hyperbolic plane}
\pmdefines{hyperbolic space}

\endmetadata

\usepackage{amssymb,amscd}
\usepackage{amsmath}
\usepackage{amsfonts}

% used for TeXing text within eps files
%\usepackage{psfrag}
% need this for including graphics (\includegraphics)
%\usepackage{graphicx}
% for neatly defining theorems and propositions
\usepackage{amsthm}
% making logically defined graphics
%%%\usepackage{xypic}

% define commands here
\begin{document}
A \PMlinkname{non-singular}{NonDegenerateQuadraticForm} isotropic quadratic space $\mathcal{H}$ of dimension
2 (over a field) is called a \emph{hyperbolic plane}.  In other
words, $\mathcal{H}$ is a 2-dimensional vector space over a field
equipped with a quadratic form $Q$ such that there exists a non-zero
vector $v$ with $Q(v)=0$.

\textbf{Examples}.  Fix the ground field to be $\mathbb{R}$, and
$\mathbb{R}^2$ be the two-dimensional vector space over $\mathbb{R}$
with the standard basis $(0,1)$ and $(1,0)$.
\begin{enumerate}
\item Let $Q_1(x,y)=xy$.  Then $Q_1(a,0)=Q_1(0,b)=0$ for all $a,b\in\mathbb{R}$.  $(\mathbb{R}^2,Q_1)$ is a hyperbolic
plane.  When $Q_1$ is written in matrix form, we have
\begin{center}$Q_1(x,y) =
\begin{pmatrix}x & y\end{pmatrix}
\begin{pmatrix}
0 & \frac{1}{2} \\
\frac{1}{2} & 0
\end{pmatrix}
\begin{pmatrix}x \\ y\end{pmatrix}=
\begin{pmatrix}x & y\end{pmatrix}
M(Q_1)
\begin{pmatrix}x \\ y\end{pmatrix}.$
\end{center}
\item Let $Q_2(r,s)=r^2-s^2$.  Then $Q_2(a,a)=0$ for all $a\in\mathbb{R}$.  $(\mathbb{R}^2,Q_2)$ is a hyperbolic
plane.  As above, $Q_2$ can be written in matrix form:
\begin{center}$Q_1(x,y) =
\begin{pmatrix}x & y\end{pmatrix}
\begin{pmatrix}
1 & 0 \\
0 & -1
\end{pmatrix}
\begin{pmatrix}x \\ y\end{pmatrix}=
\begin{pmatrix}x & y\end{pmatrix}
M(Q_2)
\begin{pmatrix}x \\ y\end{pmatrix}.$
\end{center}
\end{enumerate}

From the above examples, we see that the name ``hyperbolic plane''
comes from the fact that the associated quadratic form resembles the
equation of a hyperbola in a two-dimensional Euclidean plane.

It's not hard to see that the two examples above are equivalent
quadratic forms.  To transform from the first form to the second,
for instance, follow the linear substitutions $x=r-s$ and $y=r+s$,
or in matrix form:
\begin{center}$
\begin{pmatrix}
1 & 1 \\
-1 & 1
\end{pmatrix}
M(Q_1)
\begin{pmatrix}
1 & -1 \\
1 & 1
\end{pmatrix}
=
\begin{pmatrix}
1 & 1 \\
-1 & 1
\end{pmatrix}
\begin{pmatrix}
0 & \frac{1}{2} \\
\frac{1}{2} & 0
\end{pmatrix}
\begin{pmatrix}
1 & -1 \\
1 & 1
\end{pmatrix}
=\begin{pmatrix}
1 & 0 \\
0 & -1
\end{pmatrix}=
M(Q_2).$
\end{center}

In fact, we have the following

\textbf{Proposition}.  Any two hyperbolic planes over a field $k$ of
characteristic not 2 are isometric quadratic spaces.
\begin{proof}  From the first example above, we see that the quadratic space with the quadratic form $xy$ is a hyperbolic plane.  Conversely, if we can show that any hyperbolic plane $\mathcal{H}$ is isometric the example (with the ground field switched from $\mathbb{R}$ to $k$), we are done.

Pick a non-zero vector $u\in\mathcal{H}$ and suppose it is
isotropic: $Q(u)=0$.  Pick another vector $v\in\mathcal{H}$ so
$\lbrace u,v\rbrace$ forms a basis for $\mathcal{H}$.  Let $B$ be
the symmetric bilinear form associated with $Q$.  If $B(u,v)=0$,
then for any $w\in\mathcal{H}$ with $w=\alpha u+\beta v$, $B(u,w)=
\alpha B(u,u)+\beta B(u,v)=0$, contradicting the fact that
$\mathcal{H}$ is non-singular.  So $B(u,v)\neq 0$.  By dividing $v$
by $B(u,v)$, we may assume that $B(u,v)=1$.

Suppose $\alpha=B(v,v)$.  Then the matrix associated with the quadratic form $Q$ corresponding to the basis $\mathfrak{b}=\lbrace u,v\rbrace$ is  
\begin{center}$M_{\mathfrak{b}}(Q)=
\begin{pmatrix}
0 & 1 \\
1 & \alpha
\end{pmatrix}.$
\end{center}

If $\alpha=0$ then we are done, since $M_{\mathfrak{b}}(Q)$ is equivalent to $M_{\mathfrak{b}}(Q_1)$ via the isometry $T:\mathcal{H}\to\mathcal{H}$ given by 
\begin{center}$T=
\begin{pmatrix}
\frac{1}{2} & 0 \\
0 & 1
\end{pmatrix}\mbox{, so that }
T^t
\begin{pmatrix}
0 & 1 \\
1 & 0
\end{pmatrix}T=
\begin{pmatrix}
0 & \frac{1}{2} \\
\frac{1}{2} & 0
\end{pmatrix}.$
\end{center}

If $\alpha\neq 0$, then the trick is to replace $v$ with an isotropic vector $w$ so that the bottom right cell is also 0.  Let $w=-\frac{\alpha}{2}u+v$.  It's easy to verify that $Q(w)=0$.  As a result, the isometry $S$ required has the matrix form
\begin{center}$S=
\begin{pmatrix}
1 & -\frac{\alpha}{2} \\
0 & 1
\end{pmatrix}\mbox{, so that }
S^t
\begin{pmatrix}
0 & 1 \\
1 & \alpha
\end{pmatrix}S=
\begin{pmatrix}
0 & 1 \\
1 & 0
\end{pmatrix}.$
\end{center}
\end{proof}

Thus we may speak of the hyperbolic plane over a field without any ambiguity, and we may identify the hyperbolic plane with either of the two quadratic forms $xy$ or $x^2-y^2$.  Its notation, corresponding to the second of the forms, is $\langle 1\rangle \bot\langle-1\rangle$, or simply $\langle 1,-1\rangle$.

A \emph{hyperbolic space} is a finite dimensional orthogonal direct sum of hyperbolic planes.  It is always even dimensional and has the notation 
$\langle 1,-1,1,-1,\ldots,1,-1\rangle$ or simply $n\langle 1\rangle\bot n\langle -1\rangle$, where $2n$ is the dimensional of the hyperbolic space.

\textbf{Remarks.}  
\begin{itemize}
\item The notion of the hyperbolic plane encountered in the theory of quadratic forms is different from the ``hyperbolic plane'', a 2-dimensional space of constant negative curvature (Euclidean signature) that is commonly used in differential geometry, and in non-Euclidean geometry.
\item Instead of being associated with a quadratic form, a hyperbolic plane is sometimes defined in terms of an alternating form.  In any case, the two definitions of a hyperbolic plane coincide if the ground field has characteristic 2.
\end{itemize}
%%%%%
%%%%%
\end{document}
