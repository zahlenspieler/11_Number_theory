\documentclass[12pt]{article}
\usepackage{pmmeta}
\pmcanonicalname{FermatsTheoremProof}
\pmcreated{2013-03-22 11:46:10}
\pmmodified{2013-03-22 11:46:10}
\pmowner{drini}{3}
\pmmodifier{drini}{3}
\pmtitle{Fermat's theorem proof}
\pmrecord{11}{30223}
\pmprivacy{1}
\pmauthor{drini}{3}
\pmtype{Proof}
\pmcomment{trigger rebuild}
\pmclassification{msc}{11-00}
\pmclassification{msc}{37B55}
\pmrelated{EulerFermatTheorem}
\pmrelated{FermatsLittleTheorem}
\pmrelated{ProofOfEulerFermatTheoremUsingLagrangesTheorem}
\pmrelated{FermatsLittleTheoremProofInductive}

\usepackage{amssymb}
\usepackage{amsmath}
\usepackage{amsfonts}
\usepackage{graphicx}
%%%%\usepackage{xypic}
\begin{document}
Consider the sequence $a,\,2a,\,\ldots,\,(p-1)a$.

They are all different (modulo $p$), because if $ma=na$ with $1\le m<n\le p-1$ then
$0=a(m-n)$, and since\, $p\nmid a$, we get $p\mid(m-n)$,\, which is impossible.

Now, since all these numbers are different, the set \,$\{a,\,2a,\,3a,\,\ldots,\,(p-1)a\}$\, will have the $p-1$ possible congruence classes (although not necessarily in the same order) and therefore 
$$a\cdot2a\cdot3a\cdots (p-1)a\equiv (p-1)!a^{p-1}\equiv (p-1)!\pmod{p}$$
and using\, $\gcd((p-1)!,\;p)=1$\, we get
$$a^{p-1}\equiv 1\pmod{p}.$$
%%%%%
%%%%%
%%%%%
%%%%%
\end{document}
