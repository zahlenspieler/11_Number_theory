\documentclass[12pt]{article}
\usepackage{pmmeta}
\pmcanonicalname{PrimePyramid}
\pmcreated{2013-03-22 17:00:03}
\pmmodified{2013-03-22 17:00:03}
\pmowner{PrimeFan}{13766}
\pmmodifier{PrimeFan}{13766}
\pmtitle{prime pyramid}
\pmrecord{4}{39283}
\pmprivacy{1}
\pmauthor{PrimeFan}{13766}
\pmtype{Definition}
\pmcomment{trigger rebuild}
\pmclassification{msc}{11A41}

% this is the default PlanetMath preamble.  as your knowledge
% of TeX increases, you will probably want to edit this, but
% it should be fine as is for beginners.

% almost certainly you want these
\usepackage{amssymb}
\usepackage{amsmath}
\usepackage{amsfonts}

% used for TeXing text within eps files
%\usepackage{psfrag}
% need this for including graphics (\includegraphics)
%\usepackage{graphicx}
% for neatly defining theorems and propositions
%\usepackage{amsthm}
% making logically defined graphics
%%%\usepackage{xypic}

% there are many more packages, add them here as you need them

% define commands here

\begin{document}
A {\em prime pyramid} is a triangular arrangement of numbers in which each row $n$ has the integers from 1 to $n$ but in an order such that the sum of any two consecutive terms in a row is a prime number. The first number must be 1, and the last number is usually required to be $n$, all the numbers in between are in whatever order fulfills the requirement for prime sums. Unlike other triangular arrangements of numbers like Pascal's triangle or Losanitsch's triangle, the contents of a given row are not determined by those of the previous row. However, if it happens that one has calculated row $n - 1$ and that $2n - 1$ is a prime number, one could just copy the previous row and add $n$ at the end. Here is a prime pyramid reckoned that way:

\[
\begin{array}{cccccccccccccccccc}
& & & & & & & & & 1 & & & & & & & &\\
& & & & & & & & 1 & & 2 & & & & & & &\\
& & & & & & & 1 & & 2 & & 3 & & & & & &\\
& & & & & & 1 & & 2 & & 3 & & 4 & & & & &\\
& & & & & 1 & & 4 & & 3 & & 2 & & 5 & & & &\\
& & & & 1 & & 4 & & 3 & & 2 & & 5 & & 6 & & &\\
& & & 1 & & 4 & & 3 & & 2 & & 5 & & 6 & & 7 & &\\
& & 1 & & 4 & & 7 & & 6 & & 5 & & 2 & & 3 & & 8 &\\
& & & & &\vdots & & & & \vdots & & & & \vdots& & & & \\
\end{array}
\]

Often row 1 just contains an asterisk or some other non-numerical symbol, but since the idea of adding two numbers in row 1 is moot, here row 1 just contains a 1 per analogy to the following rows and to other triangular arrangements of numbers. % Certain other values of the pyramid given above coincide with Pascal's triangle, but this is most likely a coincidence than anything else

\begin{thebibliography}{2}
\bibitem{rg} R. K. Guy, {\it Unsolved Problems in Number Theory} New York: Springer-Verlag 2004: C1
\end{thebibliography}
%%%%%
%%%%%
\end{document}
