\documentclass[12pt]{article}
\usepackage{pmmeta}
\pmcanonicalname{PadicCyclotomicCharacter}
\pmcreated{2013-03-22 15:36:16}
\pmmodified{2013-03-22 15:36:16}
\pmowner{alozano}{2414}
\pmmodifier{alozano}{2414}
\pmtitle{$p$-adic cyclotomic character}
\pmrecord{6}{37521}
\pmprivacy{1}
\pmauthor{alozano}{2414}
\pmtype{Definition}
\pmcomment{trigger rebuild}
\pmclassification{msc}{11R34}
\pmclassification{msc}{11R32}
\pmclassification{msc}{11R04}
\pmsynonym{$p$-adic cyclotomic Galois representation}{PadicCyclotomicCharacter}
\pmsynonym{cyclotomic character}{PadicCyclotomicCharacter}

\endmetadata

% this is the default PlanetMath preamble.  as your knowledge
% of TeX increases, you will probably want to edit this, but
% it should be fine as is for beginners.

% almost certainly you want these
\usepackage{amssymb}
\usepackage{amsmath}
\usepackage{amsthm}
\usepackage{amsfonts}

% used for TeXing text within eps files
%\usepackage{psfrag}
% need this for including graphics (\includegraphics)
%\usepackage{graphicx}
% for neatly defining theorems and propositions
%\usepackage{amsthm}
% making logically defined graphics
%%%\usepackage{xypic}

% there are many more packages, add them here as you need them

% define commands here

\newtheorem{thm}{Theorem}
\newtheorem{defn}{Definition}
\newtheorem{prop}{Proposition}
\newtheorem{lemma}{Lemma}
\newtheorem{cor}{Corollary}

\theoremstyle{definition}
\newtheorem{exa}{Example}

% Some sets
\newcommand{\Nats}{\mathbb{N}}
\newcommand{\Ints}{\mathbb{Z}}
\newcommand{\Reals}{\mathbb{R}}
\newcommand{\Complex}{\mathbb{C}}
\newcommand{\Rats}{\mathbb{Q}}
\newcommand{\Gal}{\operatorname{Gal}}
\newcommand{\Cl}{\operatorname{Cl}}
\begin{document}
Let $G_{\Rats}=\Gal(\overline{\Rats}/\Rats)$ be the absolute Galois group of $\Rats$. The purpose of this entry is to define, for every prime $p$, a Galois representation:

$$\chi_p : G_{\Rats} \longrightarrow \Ints_p^\times$$

where $\Ints_p^\times$ is the group of units of $\Ints_p$, the $p$-adic integers. $\chi_p$ is a $\Ints_p^\times$ valued character, usually called the {\it cyclotomic character} of $G_{\Rats}$, or the $p$-adic cyclotomic Galois representation of $G_{\Rats}$. Here is the construction:

For each $n\geq 1$, let $\zeta_{p^n}$ be a primitive $p^n$-th root of unity and let $K_n=\Rats(\zeta_{p^n})$ be the corresponding cyclotomic extension of $\Rats$.  By the basic theory of cyclotomic extensions, we know that 

$$\Gal(K_n/\Rats)\cong (\Ints/p^n\Ints)^\times.$$

Moreover, the restriction map $\Gal(K_{n+1}/\Rats)\to \Gal(K_n/\Rats)$ is given by reduction modulo $p^n$ from $(\Ints/p^{n+1}\Ints)^\times$ to $(\Ints/p^n\Ints)^\times$.

Therefore, for each $n$ we can construct a representation:
$$\chi_{p,n} : G_{\Rats} \to \Gal(K_n/\Rats) \to (\Ints/p^n\Ints)^\times$$
where the first map is simply restriction to $K_n$ and the second map is an isomorphism. By the remarks above, the representations $\chi_{p,n}$ are coherent in a strong sense, i.e.

$$\chi_{p,n+1}(\sigma) \equiv \chi_{p,n}(\sigma) \mod p^n.$$

Therefore, one can construct a ``big'' Galois representation:
$$\chi_p : G_{\Rats} \longrightarrow \Ints_p^\times$$
by requiring $\chi(\sigma) \equiv \chi_{p,n}(\sigma) \mod p^n$, for every $n\geq 1$. 

One can rephrase the above definition as follows. Let $\sigma\in G_{\Rats}$. We need to define a group homomorphism $\chi_p:G_{\Rats} \to \Ints_p^\times$, so we need to first define $\chi_p(\sigma)$ and then check that it is a homomorphism. By the theory, $\sigma(\zeta_{p^n})$ is another primitive $p^n$-th root of unity, thus 
$$\sigma(\zeta_{p^n})=\zeta_{p^n}^{t_n}$$
for some integer $1\leq t_n \leq p^n-1$ with $\gcd(t_n,p)=1$ (so $t_n$ is a unit modulo $p^n$). Moreover, 
$$\sigma(\zeta_{p^{n-1}})=\sigma(\zeta_{p^{n}}^p)=\zeta_{p^n}^{pt_n}=\zeta_{p^{n-1}}^{t_n}$$
Therefore, $t_n \equiv t_{n-1}$ modulo $p^{n-1}$. Thus, we may define:
$$\chi_p(\sigma) = \varprojlim t_n \in \Ints_p$$
and as we have shown, $\chi_p(\sigma)$ is a unit of $\Ints_p$. Finally, the reader should check that $\chi_p$ is a group homomorphism.
%%%%%
%%%%%
\end{document}
