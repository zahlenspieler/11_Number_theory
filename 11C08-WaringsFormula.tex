\documentclass[12pt]{article}
\usepackage{pmmeta}
\pmcanonicalname{WaringsFormula}
\pmcreated{2013-03-22 15:34:26}
\pmmodified{2013-03-22 15:34:26}
\pmowner{alozano}{2414}
\pmmodifier{alozano}{2414}
\pmtitle{Waring's formula}
\pmrecord{9}{37480}
\pmprivacy{1}
\pmauthor{alozano}{2414}
\pmtype{Theorem}
\pmcomment{trigger rebuild}
\pmclassification{msc}{11C08}
\pmsynonym{Waring formula}{WaringsFormula}
%\pmkeywords{elementary symmetric polynomials}
\pmrelated{NewtonGirardFormulaSymmetricPolynomials}

% this is the default PlanetMath preamble.  as your knowledge
% of TeX increases, you will probably want to edit this, but
% it should be fine as is for beginners.

% almost certainly you want these
\usepackage{amssymb}
\usepackage{amsmath}
\usepackage{amsfonts}

% used for TeXing text within eps files
%\usepackage{psfrag}
% need this for including graphics (\includegraphics)
%\usepackage{graphicx}
% for neatly defining theorems and propositions
%\usepackage{amsthm}
% making logically defined graphics
%%%\usepackage{xypic}

% there are many more packages, add them here as you need them

% define commands here
\begin{document}
Let $x_1,\ldots, x_n$ be $n$ indeterminates. For $k\geq 1$, let
$\sigma_k$ be the $k$th elementary symmetric polynomials in $x_1,
\ldots, x_n$, and $S_k$ be the $k$th power sum defined as
\[
  S_k = \sum_{i=1}^n x_i^k.
\]

Like the Newton's formula, the Waring formula is a relation
between $\sigma_k$ and $S_k$:

\[S_k = \sum (-1)^{(i_2+i_4+i_6+\ldots)} \frac{(i_1+i_2+\ldots+i_n-1)!k}{i_1!i_2!\cdots i_n!}
\sigma_1^{i_1} \sigma_2^{i_2} \cdots \sigma_n^{i_n},
\]
where the summation is over all $n$-tuples $(i_1,\ldots, i_n)\in\mathbb{Z}^n$ with non-negative components, such that
\[
i_1+2i_2+\ldots+ni_n = k.
\]

In particular, when there are two indeterminates, i.e. $n=2$, the
Waring formula reads
\[
  x_1^k + x_2^k = \sum_{i=0}^{\lfloor k/2 \rfloor}
  (-1)^i\frac{k}{k-i}\binom{k-i}{i}(x_1+x_2)^{k-2i}(x_1x_2)^i.
\]
%%%%%
%%%%%
\end{document}
