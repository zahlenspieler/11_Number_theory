\documentclass[12pt]{article}
\usepackage{pmmeta}
\pmcanonicalname{DivisibilityOfNinenumbers}
\pmcreated{2013-03-22 19:04:43}
\pmmodified{2013-03-22 19:04:43}
\pmowner{pahio}{2872}
\pmmodifier{pahio}{2872}
\pmtitle{divisibility of nine-numbers}
\pmrecord{8}{41967}
\pmprivacy{1}
\pmauthor{pahio}{2872}
\pmtype{Theorem}
\pmcomment{trigger rebuild}
\pmclassification{msc}{11A63}
\pmclassification{msc}{11A05}
%\pmkeywords{nine-number}

\endmetadata

% this is the default PlanetMath preamble.  as your knowledge
% of TeX increases, you will probably want to edit this, but
% it should be fine as is for beginners.

% almost certainly you want these
\usepackage{amssymb}
\usepackage{amsmath}
\usepackage{amsfonts}

% used for TeXing text within eps files
%\usepackage{psfrag}
% need this for including graphics (\includegraphics)
%\usepackage{graphicx}
% for neatly defining theorems and propositions
 \usepackage{amsthm}
% making logically defined graphics
%%%\usepackage{xypic}

% there are many more packages, add them here as you need them

% define commands here

\theoremstyle{definition}
\newtheorem*{thmplain}{Theorem}

\begin{document}
We know that 9 is divisible by the prime number 3 and that 99 by another prime number 11.\, If we study the divisibility other ``nine-numbers'' by primes, we can see that 999 is divisible by a greater prime number 37 and 9999 by 101 which also is a prime, and so on.\, Such observations may be generalised to the following\\

\textbf{Proposition.}\, For every positive odd prime $p$ except 5, there is a nine-number $999...9$ divisible by $p$.\\

\emph{Proof.}\, Let $p$ be a positive odd prime $\neq 5$.\, Let's form the set of the integers
\begin{align}
9,\,99,\,999,\,\ldots,\, \underbrace{99...9}_{p\; \mathrm{nines}}.
\end{align}
We make the antithesis that no one of these numbers is divisible by $p$.\, Therefore, their least nonnegative remainders modulo $p$ are some of the $p\!-\!1$ numbers
\begin{align}
1,\,2,\,3,\,\ldots,\,p\!-\!1.
\end{align}
Thus there are at least two of the numbers (1), say $a$ and $b$ ($a< b$), having the same remainder.\, The difference 
$b\!-\!a$ then has the decadic \PMlinkescapetext{representation} of the form
$$b\!-\!a \;=\; 999...9000...0,$$
which comprises at least one 9 and one 0.\, Because of the equal remainders of $a$ and $b$, the difference is divisible by $p$.\, Since\, $b\!-\!a = 999...9\!\cdot\!1000...0$\, and 2 and 5 are the only prime factors of the latter \PMlinkname{factor}{Product}, $p$ must divide the former factor $999...9$ (cf. divisibility by prime).\, But this is one of the numbers (1), whence our antithesis is wrong.\, Consequently, at least one of (1) is divisible by $p$.\\

In other \PMlinkid{positional digital systems}{3313}, one can write propositions analogous to the above one concerning the decadic system, for example in the dyadic (a.k.a. \PMlinkescapetext{binary)} digital system:\\

\textbf{Proposition.}\, For every odd prime $p$, there is a number $111...1_{\mathrm{two}}$ divisible by $p$.

%%%%%
%%%%%
\end{document}
