\documentclass[12pt]{article}
\usepackage{pmmeta}
\pmcanonicalname{SolvingLinearDiophantineEquation}
\pmcreated{2013-03-22 17:45:55}
\pmmodified{2013-03-22 17:45:55}
\pmowner{pahio}{2872}
\pmmodifier{pahio}{2872}
\pmtitle{solving linear Diophantine equation}
\pmrecord{6}{40220}
\pmprivacy{1}
\pmauthor{pahio}{2872}
\pmtype{Topic}
\pmcomment{trigger rebuild}
\pmclassification{msc}{11D04}
\pmrelated{LinearCongruence}

% this is the default PlanetMath preamble.  as your knowledge
% of TeX increases, you will probably want to edit this, but
% it should be fine as is for beginners.

% almost certainly you want these
\usepackage{amssymb}
\usepackage{amsmath}
\usepackage{amsfonts}

% used for TeXing text within eps files
%\usepackage{psfrag}
% need this for including graphics (\includegraphics)
%\usepackage{graphicx}
% for neatly defining theorems and propositions
 \usepackage{amsthm}
% making logically defined graphics
%%%\usepackage{xypic}

% there are many more packages, add them here as you need them

% define commands here

\theoremstyle{definition}
\newtheorem*{thmplain}{Theorem}

\begin{document}
 \PMlinkescapeword{mean}
Here we \PMlinkescapetext{represent} an elementary but very comprehensible method for solving any linear Diophantine equation with two unknowns, i.e. for finding the general integer solution of an equation of the form
$$ax-my = b$$
where $a,\,b,\,m$ are known integers and $x,\,y$ the unknowns.

The method is illustrated via a numerical example:
\begin{align}
37x-107y = 25
\end{align}
We solve first (1) for $x$ (which has absolutely smaller coefficient than $y$):
\begin{align}
x = \frac{25+107y}{37}
\end{align}
The \PMlinkescapetext{terms} in the numerator may be split so that division yields a polynomial with integer coefficients and that the remainder has absolutely smaller coefficients (now $-12$ and $-4$) than the dividend in (2) had:
$$x = 1+3y-\frac{12+4y}{37}$$
Since $x$ and $1\!+\!3y$ mean integers, also
$$z := \frac{12+4y}{37}$$ 
must be an integer.\, Now solve this last equation for $y$ and split the new numerator similarly as above:
$$y = \frac{-12+37z}{4} = -3+9z+\frac{z}{4}$$
Since $y$ and $-3\!+\!9z$ mean integers, also 
$$t := \frac{z}{4}$$
must be an integer.\, It is apparent that we can give any integer value for $t$, which thus may be thought as a parameter determining the values of the other letters.\, We obtain successively
$$z = 4t, \quad y = -3+9\cdot4t+t = -3+37t, \quad x = 1+3(-3+37t)-4t = -8+107t.$$
Accordingly, we may write the \PMlinkescapetext{general solution} of (1) as
\begin{align*}
\begin{cases}
  x = -8+107t\\
  y = -3+37t
\end{cases}
\end{align*}
where\; $t = 0,\,\pm1,\,\pm2,\,\ldots$\\

This method and the use of a parameter for expressing the solution were well known in the ancient world, especially in solving astronomical cycles as noted by Brahmagupta (598--668). 

%%%%%
%%%%%
\end{document}
