\documentclass[12pt]{article}
\usepackage{pmmeta}
\pmcanonicalname{Idele}
\pmcreated{2013-03-22 12:39:28}
\pmmodified{2013-03-22 12:39:28}
\pmowner{djao}{24}
\pmmodifier{djao}{24}
\pmtitle{id\`{e}le}
\pmrecord{7}{32926}
\pmprivacy{1}
\pmauthor{djao}{24}
\pmtype{Definition}
\pmcomment{trigger rebuild}
\pmclassification{msc}{11R56}
\pmrelated{Adele}
\pmdefines{id\`{e}le group}
\pmdefines{group of id\`{e}les}

\endmetadata

% this is the default PlanetMath preamble.  as your knowledge
% of TeX increases, you will probably want to edit this, but
% it should be fine as is for beginners.

% almost certainly you want these
\usepackage{amssymb}
\usepackage{amsmath}
\usepackage{amsfonts}

% used for TeXing text within eps files
%\usepackage{psfrag}
% need this for including graphics (\includegraphics)
%\usepackage{graphicx}
% for neatly defining theorems and propositions
%\usepackage{amsthm}
% making logically defined graphics
%%%\usepackage{xypic} 

% there are many more packages, add them here as you need them

% define commands here
\newcommand{\p}{{\mathfrak{p}}}
\newcommand{\m}{{\mathfrak{m}}}
\newcommand{\M}{{\mathfrak{M}}}
\renewcommand{\P}{{\mathfrak{P}}}
\newcommand{\C}{\mathbb{C}}
\newcommand{\R}{\mathbb{R}}
\newcommand{\Z}{\mathbb{Z}}
\newcommand{\Q}{\mathbb{Q}}
\newcommand{\N}{\mathbb{N}}
\renewcommand{\H}{\mathcal{H}}
\newcommand{\A}{\mathbb{A}}
\newcommand{\I}{\mathbb{I}}
\renewcommand{\c}{\mathcal{C}}
\renewcommand{\O}{\mathcal{O}}
\renewcommand{\o}{\mathfrak{o}}
\newcommand{\D}{\mathcal{D}}
\newcommand{\lra}{\longrightarrow}
\renewcommand{\div}{\mid}
\newcommand{\res}{\operatorname{res}}
\newcommand{\Spec}{\operatorname{Spec}}
\newcommand{\Gal}{\operatorname{Gal}}
\newcommand{\id}{\operatorname{id}}
\newcommand{\diff}{\operatorname{diff}}
\newcommand{\incl}{\operatorname{incl}}
\newcommand{\Hom}{\operatorname{Hom}}
\renewcommand{\Re}{\operatorname{Re}}
\newcommand{\intersect}{\cap}
\newcommand{\union}{\cup}
\newcommand{\bigintersect}{\bigcap}
\newcommand{\bigunion}{\bigcup}
\newcommand{\ilim}{\,\underset{\longleftarrow}{\lim}\,}
\begin{document}
Let $K$ be a number field. For each finite prime $v$ of $K$, let $\o_v$ be the valuation ring of the completion $K_v$ of $K$ at $v$, and let $U_v$ be the group of units in $\o_v$. Then each group $U_v$ is a compact open subgroup of the group of units $K_v^*$ of $K_v$. The {\em id\`ele group} $\I_K$ of $K$ is defined to be the restricted direct product of the multiplicative groups $\{K_v^*\}$ with respect to the compact open subgroups $\{U_v\}$, taken over all finite primes and infinite primes $v$ of $K$.

The units $K^*$ in $K$ embed into $\I_K$ via the diagonal embedding
$$
x \mapsto \prod_v x_v,
$$
where $x_v$ is the image of $x$ under the embedding $K \hookrightarrow K_v$ of $K$ into its completion $K_v$. As in the case of ad\`eles, the group $K^*$ is a discrete subgroup of the group of id\`eles $\I_K$, but unlike the case of ad\`eles, the quotient group $\I_K/K^*$ is not a compact group. It is, however, possible to define a certain subgroup of the id\`eles (the subgroup of norm 1 elements) which does have compact quotient under $K^*$.

{\bf Warning:} The group $\I_K$ is a multiplicative subgroup of the ring of ad\`eles $\A_K$, but the topology on $\I_K$ is different from the subspace topology that $\I_K$ would have as a subset of $\A_K$.
%%%%%
%%%%%
\end{document}
