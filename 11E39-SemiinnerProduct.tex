\documentclass[12pt]{article}
\usepackage{pmmeta}
\pmcanonicalname{SemiinnerProduct}
\pmcreated{2013-03-22 17:47:10}
\pmmodified{2013-03-22 17:47:10}
\pmowner{asteroid}{17536}
\pmmodifier{asteroid}{17536}
\pmtitle{semi-inner product}
\pmrecord{7}{40246}
\pmprivacy{1}
\pmauthor{asteroid}{17536}
\pmtype{Definition}
\pmcomment{trigger rebuild}
\pmclassification{msc}{11E39}
\pmclassification{msc}{15A63}
\pmclassification{msc}{46C50}
\pmsynonym{positive semi-definite inner product}{SemiinnerProduct}
\pmsynonym{semi inner product}{SemiinnerProduct}
\pmdefines{semi-inner product space}
\pmdefines{Cauchy-Schwartz inequality for semi-inner products}

\endmetadata

% this is the default PlanetMath preamble.  as your knowledge
% of TeX increases, you will probably want to edit this, but
% it should be fine as is for beginners.

% almost certainly you want these
\usepackage{amssymb}
\usepackage{amsmath}
\usepackage{amsfonts}

% used for TeXing text within eps files
%\usepackage{psfrag}
% need this for including graphics (\includegraphics)
%\usepackage{graphicx}
% for neatly defining theorems and propositions
%\usepackage{amsthm}
% making logically defined graphics
%%%\usepackage{xypic}

% there are many more packages, add them here as you need them

% define commands here

\begin{document}
\subsubsection{Definition}

Let $V$ be a vector space over a field $\mathbb{K}$, where $\mathbb{K}$ is $\mathbb{R}$ or $\mathbb{C}$.

A {\bf semi-inner product} on $V$ is a function $\;\langle \cdot , \cdot \rangle : V \times V \longrightarrow \mathbb{K}\;$ that \PMlinkescapetext{satisfies} the following conditions:

\begin{enumerate}
\item $\langle \lambda_1 v_1 + \lambda_2 v_2 , w \rangle = \lambda_1 \langle v_1 , w \rangle + \lambda_2 \langle v_2 , w \rangle\;$ for every $v_1, v_2, w \in V$ and $\lambda_1, \lambda_2 \in \mathbb{K}$.
\item $\langle v ,w \rangle = \overline{\langle w ,v \rangle}\;$ for every $v, w \in V$, where the \PMlinkescapetext{line} above means complex conjugation. 
\item $\langle v ,v \rangle \geq 0$ (\PMlinkescapetext{positive} semi definite).
\end{enumerate}

Hence, a semi-inner product on a vector space is just like an inner product, but for which $\langle v ,v \rangle$ can be zero (\PMlinkescapetext{even} if $v \neq 0$).

A \emph{semi-inner product space} is just a vector space endowed with a semi-inner product.

\subsubsection{Topology}

Every semi-inner product space $V$ can be given a topology associated with the semi-inner product. In fact, a semi-norm $\| \cdot \|$ can be defined in $V$ by
\begin{displaymath}
\|v\| := \sqrt{\langle v ,v \rangle}
\end{displaymath}

\subsubsection{Cauchy-Schwarz inequality}

The Cauchy-Schwarz inequality is valid for semi-inner product spaces:
\begin{displaymath}
|\langle v , w \rangle| \leq \sqrt{\langle v,v \rangle}\sqrt{\langle w, w \rangle}
\end{displaymath}

\subsubsection{Properties}

Let $V$ be a semi-inner product space and $W:=\{v \in V : \langle v , v \rangle = 0\}$. It is not difficult to see, using the Cauchy-Schwarz inequality, that $W$ is a vector subspace.

The semi-inner product in $V$ induces a well defined semi-inner product in the \PMlinkname{quotient}{QuotientModule} $V/W$ which is, in fact, an inner product. Thus, the \PMlinkescapetext{quotient} $V/W$ is an inner product space.
%%%%%
%%%%%
\end{document}
