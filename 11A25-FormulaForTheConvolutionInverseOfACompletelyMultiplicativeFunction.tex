\documentclass[12pt]{article}
\usepackage{pmmeta}
\pmcanonicalname{FormulaForTheConvolutionInverseOfACompletelyMultiplicativeFunction}
\pmcreated{2013-03-22 16:55:09}
\pmmodified{2013-03-22 16:55:09}
\pmowner{Wkbj79}{1863}
\pmmodifier{Wkbj79}{1863}
\pmtitle{formula for the convolution inverse of a completely multiplicative function}
\pmrecord{5}{39181}
\pmprivacy{1}
\pmauthor{Wkbj79}{1863}
\pmtype{Corollary}
\pmcomment{trigger rebuild}
\pmclassification{msc}{11A25}
\pmrelated{CriterionForAMultiplicativeFunctionToBeCompletelyMultiplicative}

\usepackage{amssymb}
\usepackage{amsmath}
\usepackage{amsfonts}

\usepackage{psfrag}
\usepackage{graphicx}
\usepackage{amsthm}
%%\usepackage{xypic}

\newtheorem{cor*}{Corollary}

\begin{document}
\PMlinkescapeword{formula}

\begin{cor*}
If $f$ is a completely multiplicative function, then its convolution inverse is $f\mu$, where $\mu$ denotes the M\"{o}bius function.
\end{cor*}

\begin{proof}
Recall the M\"{o}bius inversion formula $1*\mu = \varepsilon$, where $\varepsilon$ denotes the convolution identity function.  Thus, $f(1*\mu) = f\varepsilon$.  Since \PMlinkname{pointwise multiplication of a completely multiplicative function distributes over convolution}{PropertyOfCompletelyMultiplicativeFunctions}, $(f \cdot 1)*(f\mu)=f\varepsilon$.  Note that, for all natural numbers $n$, $f(n)1(n)=f(n) \cdot 1=f(n)$ and $f(n)\varepsilon(n)=\varepsilon(n)$.  Thus, $f*(f \mu)=\varepsilon$.  It follows that $f\mu$ is the convolution inverse of $f$.
\end{proof}
%%%%%
%%%%%
\end{document}
