\documentclass[12pt]{article}
\usepackage{pmmeta}
\pmcanonicalname{RamanujanTauFunction}
\pmcreated{2013-03-22 17:51:24}
\pmmodified{2013-03-22 17:51:24}
\pmowner{Wkbj79}{1863}
\pmmodifier{Wkbj79}{1863}
\pmtitle{Ramanujan tau function}
\pmrecord{12}{40333}
\pmprivacy{1}
\pmauthor{Wkbj79}{1863}
\pmtype{Definition}
\pmcomment{trigger rebuild}
\pmclassification{msc}{11F11}
\pmclassification{msc}{11A25}
\pmsynonym{Ramanujan's tau function}{RamanujanTauFunction}
\pmrelated{ModularForms}
\pmrelated{ModularDiscriminant}
\pmrelated{Ramanujan}
\pmrelated{ApplicationsOfSecondOrderRecurrenceRelationFormula}
\pmdefines{Lehmer's conjecture}

\endmetadata

\usepackage{amssymb}
\usepackage{amsmath}
\usepackage{amsfonts}
\usepackage{pstricks}
\usepackage{psfrag}
\usepackage{graphicx}
\usepackage{amsthm}
%%\usepackage{xypic}

\begin{document}
The \emph{Ramanujan tau function} is the arithmetic function $\tau\colon\mathbb{N}\to\mathbb{Z}$ such that, for all $q\in\mathbb{C}$ with $|q|<1$,
\[
q\prod_{k=1}^{\infty}(1-q^k)^{24}=\sum_{n=1}^{\infty} \tau(n)q^n.
\]
Thus, the Ramanujan tau function is the generating function for the \PMlinkname{Weierstrass $\Delta$ function}{ModularForms}.

Determining values of the Ramanujan tau function directly can be somewhat involved.  For example, the values of $\tau(1)$, $\tau(2)$, and $\tau(3)$ will be determined:

To determine $\tau(1)$, $\tau(2)$, and $\tau(3)$, we need to find the coefficient of $q$, $q^2$, and $q^3$, respectively, of the expression
\[
q\prod_{k=1}^{\infty}(1-q^k)^{24}.
\]
Note that we only need to consider $k=1$ and $k=2$, since higher values of $k$ yield \PMlinkname{powers}{Power} of $q$ that are too large.  Thus:
\begin{align*}
q(1-q)^{24}(1-q^2)^{24} & =q(1-24q+276q^2-\dots)(1-24q^2+\dots) \\
& =q(1-24q+276q^2-\dots-24q^2+576q^3-\dots) \\
& =q(1-24q+252q^2-\dots) \\
& =q-24q^2+252q^3-\dots
\end{align*}
Hence, $\tau(1)=1$, $\tau(2)=-24$, and $\tau(3)=252$.

The sequence $\{\tau(n)\}$ appears in the OEIS as sequence \PMlinkexternal{A000594}{http://www.research.att.com/~njas/sequences/A000594}.

Although the values of $|\tau(n)|$ seem to increase rapidly as $n$ increases, the conjecture that $\tau(n)\neq 0$ for all $n\in\mathbb{N}$ has not yet been proven.  This conjecture is known as \emph{Lehmer's conjecture}.

The Ramanujan tau function has the following properties:
\begin{itemize}
\item It is a multiplicative function:  For $a,b\in\mathbb{N}$ with $\gcd(a,b)=1$, we have $\tau(ab)=\tau(a)\tau(b)$.
\item For any prime $p$ and any $n\in\mathbb{N}$,
\[
\tau(p^{n+1})=\tau(p)\tau(p^n)-p^{11}\tau(p^{n-1}).
\]
\item For any prime $p$,
\[
|\tau(p)|\le 2p^{\frac{11}{2}}.
\]
\end{itemize}

Ramanujan asserted that $\tau$ \PMlinkescapetext{satisfies} several congruences, all of which have been proven.  Some simpler examples of such congruences include:
\begin{itemize}
\item For any $n\in\mathbb{N}$,
\[
\tau(5n)\equiv 0\pmod 5.
\]
\item For any $n\in\mathbb{N}$ and for any nonnegative integer $r<7$ which is a quadratic residue modulo $7$,
\[
\tau(7n-r)\equiv 0\pmod 7.
\]
\item For any $n\in\mathbb{N}$ and for any nonnegative integer $r<23$ which is a quadratic residue modulo $23$,
\[
\tau(23n-r)\equiv 0\pmod{23}.
\]
\end{itemize}

\begin{thebibliography}{9}
\bibitem{berndt} Berndt, Bruce C. \emph{Number Theory in the Spirit of Ramanujan}. Providence, RI: American Mathematical Society, 2006.
\end{thebibliography}
%%%%%
%%%%%
\end{document}
