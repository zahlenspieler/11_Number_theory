\documentclass[12pt]{article}
\usepackage{pmmeta}
\pmcanonicalname{QuadraticFieldsThatAreNotIsomorphic}
\pmcreated{2013-03-22 16:19:44}
\pmmodified{2013-03-22 16:19:44}
\pmowner{Wkbj79}{1863}
\pmmodifier{Wkbj79}{1863}
\pmtitle{quadratic fields that are not isomorphic}
\pmrecord{9}{38458}
\pmprivacy{1}
\pmauthor{Wkbj79}{1863}
\pmtype{Theorem}
\pmcomment{trigger rebuild}
\pmclassification{msc}{11R11}

\endmetadata

\usepackage{amssymb}
\usepackage{amsmath}
\usepackage{amsfonts}

\usepackage{psfrag}
\usepackage{graphicx}
\usepackage{amsthm}
%%\usepackage{xypic}

\newtheorem*{thm*}{Theorem}
\newtheorem*{cor*}{Corollary}
\begin{document}
\PMlinkescapeword{isomorphic}
\PMlinkescapeword{isomorphism}

Within this entry, $S$ denotes the set of all squarefree integers not equal to $1$.

\begin{thm*}
Let $m,n \in S$ with $m \neq n$.  Then $\mathbb{Q}(\sqrt{m})$ and $\mathbb{Q}(\sqrt{n})$ are not \PMlinkname{isomorphic}{FieldIsomorphism}.
\end{thm*}

\begin{proof}
Suppose that $\mathbb{Q}(\sqrt{m})$ and $\mathbb{Q}(\sqrt{n})$ are isomorphic.  Let $\varphi \colon \mathbb{Q}(\sqrt{m}) \to \mathbb{Q}(\sqrt{n})$ be a field isomorphism.  Recall that field homomorphisms fix prime subfields.  Thus, for every $x \in \mathbb{Q}$, $\varphi(x)=x$.

Let $a,b \in \mathbb{Q}$ with $\varphi(\sqrt{m})=a+b\sqrt{n}$.  Since $\varphi(a)=a$ and $\varphi$ is injective, $b \neq 0$.  Also, $m=\varphi(m)=\varphi((\sqrt{m})^2)=(\varphi(\sqrt{m}))^2=(a+b\sqrt{n})^2=a^2+2ab\sqrt{n}+b^2n$.  If $a \neq 0$, then $\displaystyle \sqrt{n}=\frac{m-a^2-b^2n}{2ab} \in \mathbb{Q}$, a contradiction.  Thus, $a=0$.  Therefore, $m=b^2n$.  Since $m$ is squarefree, $b^2=1$.  Hence, $m=n$, a contradiction.  It follows that $K$ and $L$ are not isomorphic.
\end{proof}

This yields an obvious corollary:

\begin{cor*}
There are infinitely many distinct quadratic fields.
\end{cor*}

\begin{proof}
Note that there are infinitely many elements of $S$.  Moreover, if $m$ and $n$ are distinct elements of $S$, then $\mathbb{Q}(\sqrt{m})$ and $\mathbb{Q}(\sqrt{n})$ are not isomorphic and thus cannot be equal.
\end{proof}

Note that the above corollary could have also been obtained by using the result regarding \PMlinkname{Galois groups of finite abelian extensions of $\mathbb{Q}$}{GaloisGroupsOfFiniteAbelianExtensionsOfMathbbQ}.  On the other hand, using this result to prove the above corollary can be likened to ``using a sledgehammer to kill a housefly''.
%%%%%
%%%%%
\end{document}
