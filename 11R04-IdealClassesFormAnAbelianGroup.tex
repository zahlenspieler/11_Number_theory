\documentclass[12pt]{article}
\usepackage{pmmeta}
\pmcanonicalname{IdealClassesFormAnAbelianGroup}
\pmcreated{2013-03-22 12:49:40}
\pmmodified{2013-03-22 12:49:40}
\pmowner{mathcam}{2727}
\pmmodifier{mathcam}{2727}
\pmtitle{ideal classes form an abelian group}
\pmrecord{13}{33151}
\pmprivacy{1}
\pmauthor{mathcam}{2727}
\pmtype{Theorem}
\pmcomment{trigger rebuild}
\pmclassification{msc}{11R04}
\pmclassification{msc}{11R29}
\pmrelated{NumberField}
\pmrelated{ClassNumbersAndDiscriminantsTopicsOnClassGroups}
\pmrelated{FractionalIdealOfCommutativeRing}
\pmdefines{ideal class group}
\pmdefines{class number}

% this is the default PlanetMath preamble.  as your knowledge
% of TeX increases, you will probably want to edit this, but
% it should be fine as is for beginners.

% almost certainly you want these
\usepackage{amssymb}
\usepackage{amsmath}
\usepackage{amsfonts}

% used for TeXing text within eps files
%\usepackage{psfrag}
% need this for including graphics (\includegraphics)
%\usepackage{graphicx}
% for neatly defining theorems and propositions
%\usepackage{amsthm}
% making logically defined graphics
%%%\usepackage{xypic}

% there are many more packages, add them here as you need them

% define commands here
\begin{document}
Let $K$ be a number field, and let $\cal C$ be the set of ideal classes of $K$, with multiplication $\cdot$ defined by \[ [\mathfrak{a}] \cdot [\mathfrak{b}]=[\mathfrak{a} \mathfrak{b}] \]
where $\mathfrak{a}, \mathfrak{b}$ are ideals of ${\cal{O}}_K$.

We shall check the group properties:

\begin{enumerate}
\item Associativity: $[\mathfrak{a}] \cdot ([\mathfrak{b}] \cdot [\mathfrak{c}])=
                      [\mathfrak{a}] \cdot [\mathfrak{b}\mathfrak{c}]=
                      [\mathfrak{a}(\mathfrak{b}\mathfrak{c})]=
                      [\mathfrak{a}\mathfrak{b}\mathfrak{c}]=
                      [(\mathfrak{a}\mathfrak{b})\mathfrak{c}]=
                      [\mathfrak{a}\mathfrak{b}] \cdot [\mathfrak{c}]=
                      ([\mathfrak{a}] \cdot [\mathfrak{b}]) \cdot [\mathfrak{c}]$
\item Identity element: $[ {\cal{O}}_K ] \cdot [\mathfrak{b}]=
[\mathfrak{b}]=[\mathfrak{b}] \cdot [ {\cal{O}}_K ]$.
\item Inverses: Consider $[\mathfrak{b}]$.  Let $b$ be an integer in $\mathfrak{b}$.  Then
         $\mathfrak{b} \supseteq (b)$, so there exists $\mathfrak{c}$
         such that $\mathfrak{b}\mathfrak{c}=(b)$.\\
         Then the ideal class $[\mathfrak{b}] \cdot [\mathfrak{c}] = [(b)]=[ {\cal{O}}_K ]$.
\end{enumerate}
Then ${\cal C}$ is a group under the operation $\cdot$.

It is abelian since $[\mathfrak{a}][\mathfrak{b}]=[\mathfrak{a}\mathfrak{b}]=
[\mathfrak{b}\mathfrak{a}]=[\mathfrak{b}][\mathfrak{a}]$.

This is group is called the \emph{ideal class group} of $K$.  The ideal class group is one of the principal objects of algebraic number theory.  In particular, for an arbitrary number field $K$, very little is known about the size of this group, called the \emph{class number} of $K$. See the analytic class number formula.
%%%%%
%%%%%
\end{document}
