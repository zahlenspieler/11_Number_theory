\documentclass[12pt]{article}
\usepackage{pmmeta}
\pmcanonicalname{ProofOfMinkowskisBound}
\pmcreated{2013-03-22 18:33:41}
\pmmodified{2013-03-22 18:33:41}
\pmowner{gel}{22282}
\pmmodifier{gel}{22282}
\pmtitle{proof of Minkowski's bound}
\pmrecord{5}{41284}
\pmprivacy{1}
\pmauthor{gel}{22282}
\pmtype{Proof}
\pmcomment{trigger rebuild}
\pmclassification{msc}{11R29}
\pmclassification{msc}{11H06}
%\pmkeywords{ideal class group}
%\pmkeywords{lattice}
\pmrelated{MinkowskisTheorem}
\pmrelated{MinkowskisConstant}
\pmrelated{IdealClass}

% this is the default PlanetMath preamble.  as your knowledge
% of TeX increases, you will probably want to edit this, but
% it should be fine as is for beginners.

% almost certainly you want these
\usepackage{amssymb}
\usepackage{amsmath}
\usepackage{amsfonts}

% used for TeXing text within eps files
%\usepackage{psfrag}
% need this for including graphics (\includegraphics)
%\usepackage{graphicx}
% for neatly defining theorems and propositions
\usepackage{amsthm}
% making logically defined graphics
%%%\usepackage{xypic}

% there are many more packages, add them here as you need them

% define commands here
\newtheorem*{theorem*}{Theorem}
\newtheorem{lemma}{Lemma}
\begin{document}
The proof of Minkowski's bound will rely on \PMlinkname{Minkowski's lattice point theorem}{MinkowskisTheorem}, but we first need to establish some lemmas.

\begin{lemma}\label{lem:1}
Let $M$ be a real number and suppose that for every non-zero ideal $\mathfrak{a}$ of the ring of integers $\mathcal{O}_K$ there exists a non-zero $x\in\mathfrak{a}$ with norm $\operatorname{N}(x)\le M \operatorname{N}(\mathfrak{a})$.

Then, every ideal class of $\mathcal{O}_K$ has a representative $\mathfrak{a}$ satisfying $\operatorname{N}(\mathfrak{a})\le M$.
\end{lemma}
\begin{proof}
Let $[\mathfrak{b}]$ be an ideal class represented by the ideal $\mathfrak{b}$. Choosing a non-zero $x\in\mathfrak{b}$ then $x\mathfrak{b}^{-1}$ is an ideal of $\mathcal{O}_K$ and, by the condition of the lemma, contains a non-zero $y$ satisfying $\operatorname{N}(y)\le M \operatorname{N}(x\mathfrak{b}^{-1})$.
Then, $\mathfrak{a}\equiv x^{-1}y\mathfrak{b}$ is an ideal representing $[\mathfrak{b}]$ and $\operatorname{N}(\mathfrak{a})=\operatorname{N}(y)/\operatorname{N}(x\mathfrak{b}^{-1})\le M$.
\end{proof}

If the real embeddings of $K$ are denoted by $\sigma_k\colon K\rightarrow\mathbb{R}$ ($k=1,\ldots,r_1$) and the complex embeddings are $\tau_k\colon K\rightarrow\mathbb{C}$ together with their complex conjugates $\bar\tau_k$ ($k=1,\ldots,r_2$), then we define
\begin{align*}
&j\colon K\rightarrow\mathbb{R}^{r_1}\times\mathbb{C}^{r_2},\\
&j(x)=(\sigma_1(x),\ldots,\sigma_{r_1}(x),\tau_1(x),\ldots,\tau_{r_2}(x)).
\end{align*}
Also note that $\mathbb{R}^{r_1}\times\mathbb{C}^{r_2}$ is isomorphic as a real vector space to $\mathbb{R}^{r_1+2r_2}=\mathbb{R}^{n}$ given by the isomorphism
\begin{gather*}
f\colon\mathbb{R}^{r_1}\times\mathbb{C}^{r_2}\rightarrow\mathbb{R}^n,\\
f(x_1,\dots,x_{r_1},y_{1},\ldots,y_{r_2})=(x_1,\ldots,x_{r_1},\Re(y_{1}),\ldots,\Re(y_{r_2}),\Im(y_{1}),\ldots,\Im(y_{r_2})).
\end{gather*}
As $f$ and $j$ are linear maps (with respect to the field of rationals $\mathbb{Q}$), the combination $f\circ j$ gives a $\mathbb{Q}$-linear map from $K$ to $\mathbb{R}^n$. The image will be a lattice, and we can compute its volume.

\begin{lemma}\label{lem:2}
If $\mathfrak{a}$ is a non-zero ideal of $\mathcal{O}_K$, then $\Gamma=f\circ j(\mathfrak{a})$ is a \PMlinkname{lattice in $\mathbb{R}^n$}{LatticeInMathbbRn}. Its fundamental mesh has volume
\begin{equation*}
\operatorname{vol}(\Gamma)=2^{-r_2}\sqrt{|D_K|}\operatorname{N}(\mathfrak{a}).
\end{equation*}
\end{lemma}
\begin{proof}
The proof of this is to be added.
\end{proof}

\begin{lemma}\label{lem:3}
For any $L>0$, let $S$ be the set in $\mathbb{R}^{r_1}\times\mathbb{C}^{r_2}$ consisting of points $(x_1,\ldots,x_{r_1},y_1,y_{r_2})$ satisfying
\begin{equation*}
\sum_{k=1}^{r_1}|x_k|+2\sum_{k=1}^{r_2}|y_k|\le L.
\end{equation*}
Then, $f(S)$ has volume $(2^{r_1-r_2}\pi^{r_2}/n!) L^n$.
\end{lemma}
\begin{proof}
The proof of this is to be added.
\end{proof}

\noindent{\bf Proof of Minkowski's bound}

For an ideal $\mathfrak{a}$ and any constant $b>1$, let $L>0$ be given by
\begin{equation*}
\frac{2^{r_1-r_2}\pi^{r_2}}{n!} L^n=2^nb2^{-r_2}\sqrt{|D_K|}\operatorname{N}(\mathfrak{a}).
\end{equation*}
Letting $S$ be the set given in Lemma \ref{lem:3} and $\Gamma=f\circ j(\mathfrak{a})$, Lemmas \ref{lem:2} and \ref{lem:3} give $\operatorname{vol}(S)>2^n\operatorname{vol}(\Gamma)$. As $S$ is convex and symmetric about the origin, Minkowski's theorem tells us that there is a non-zero $x\in\mathfrak{a}$ with $f\circ j(x)\in S$.

As the geometric mean is always bounded above by the arithmetic mean, we get the inequality
\begin{equation*}\begin{split}
\operatorname{N}(x)&=\prod_{k=1}^{r_1}|\sigma_k(x)|\prod_{k=1}^{r_2}|\tau_k(x)|^2\\
&\le n^{-n}\left(\sum_{k=1}^{r_1}|\sigma_k(x)|+2\sum_{k=1}^{r_2}|\tau_k(x)|\right)^n\\
&\le n^{-n}L^n=b M_K\sqrt{|D_K|}\operatorname{N}(\mathfrak{a})
\end{split}\end{equation*}
where $M_K=(n!/n^n)(4/\pi)^{r_2}$. If we choose $b$ such that $bM_K\sqrt{|D_K|}\operatorname{N}(\mathfrak{a})$ is less than the smallest integer greater than $M_K\sqrt{|D_K|}\operatorname{N}(\mathfrak{a})$, then this gives $\operatorname{N}(x)\le M_K\sqrt{|D_K|}\operatorname{N}(\mathfrak{a})$ and Minkowski's bound follows from Lemma \ref{lem:1}.

%%%%%
%%%%%
\end{document}
