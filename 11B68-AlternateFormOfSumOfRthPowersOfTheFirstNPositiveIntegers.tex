\documentclass[12pt]{article}
\usepackage{pmmeta}
\pmcanonicalname{AlternateFormOfSumOfRthPowersOfTheFirstNPositiveIntegers}
\pmcreated{2013-03-22 17:46:10}
\pmmodified{2013-03-22 17:46:10}
\pmowner{rm50}{10146}
\pmmodifier{rm50}{10146}
\pmtitle{alternate form of sum of $r$th powers of the first $n$ positive integers}
\pmrecord{4}{40225}
\pmprivacy{1}
\pmauthor{rm50}{10146}
\pmtype{Proof}
\pmcomment{trigger rebuild}
\pmclassification{msc}{11B68}
\pmclassification{msc}{05A15}

% this is the default PlanetMath preamble.  as your knowledge
% of TeX increases, you will probably want to edit this, but
% it should be fine as is for beginners.

% almost certainly you want these
\usepackage{amssymb}
\usepackage{amsmath}
\usepackage{amsfonts}

% used for TeXing text within eps files
%\usepackage{psfrag}
% need this for including graphics (\includegraphics)
%\usepackage{graphicx}
% for neatly defining theorems and propositions
%\usepackage{amsthm}
% making logically defined graphics
%%%\usepackage{xypic}

% there are many more packages, add them here as you need them

% define commands here

\begin{document}
\PMlinkescapeword{basic}
We will show that
\[\sum_{k=0}^n k^r = \int_1^{n+1} b_r(x)dx\]

We need two basic facts. First, a property of the Bernoulli polynomials is that $b_r'(x)=rb_{r-1}(x)$. Second, the Bernoulli polynomials can be written as
\[b_r(x) = \sum_{k=1}^r \binom{r}{k}B_{r-k}x^k + B_r\]

We then have
\begin{align*}
\int_1^{n+1} b_r(x)&=\frac{1}{r+1}(b_{r+1}(n+1)-b_{r+1}(1)) = \frac{1}{r+1}\sum_{k=0}^{r+1}\binom{r+1}{k}B_{r+1-k}((n+1)^k-1) \\
&= \frac{1}{r+1}\sum_{k=1}^{r+1}\binom{r+1}{k}B_{r+1-k}(n+1)^k
\end{align*}
Now reverse the order of summation (i.e. replace $k$ by $r+1-k$) to get
\[
\int_1^{n+1} b_r(x)=\frac{1}{r+1}\sum_{k=0}^r\binom{r+1}{r+1-k}B_k(n+1)^{r+1-k}=\frac{1}{r+1}\sum_{k=0}^r\binom{r+1}{k}B_r (n+1)^{r+1-k}\]
which is equal to $\sum_{k=0}^n k^r$ (see the \PMlinkname{parent}{SumOfKthPowersOfTheFirstNPositiveIntegers} article).
%%%%%
%%%%%
\end{document}
