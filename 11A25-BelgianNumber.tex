\documentclass[12pt]{article}
\usepackage{pmmeta}
\pmcanonicalname{BelgianNumber}
\pmcreated{2013-03-22 19:20:45}
\pmmodified{2013-03-22 19:20:45}
\pmowner{Kausthub}{26471}
\pmmodifier{Kausthub}{26471}
\pmtitle{Belgian number}
\pmrecord{4}{42295}
\pmprivacy{1}
\pmauthor{Kausthub}{26471}
\pmtype{Definition}
\pmcomment{trigger rebuild}
\pmclassification{msc}{11A25}
\pmsynonym{Eric number}{BelgianNumber}

\endmetadata

% this is the default PlanetMath preamble.  as your knowledge
% of TeX increases, you will probably want to edit this, but
% it should be fine as is for beginners.

% almost certainly you want these
\usepackage{amssymb}
\usepackage{amsmath}
\usepackage{amsfonts}

% used for TeXing text within eps files
%\usepackage{psfrag}
% need this for including graphics (\includegraphics)
%\usepackage{graphicx}
% for neatly defining theorems and propositions
%\usepackage{amsthm}
% making logically defined graphics
%%%\usepackage{xypic}

% there are many more packages, add them here as you need them

% define commands here

\begin{document}
108 is a Belgian-0 number because if we start a sequence with 0 with recurring differences of 1, 0 and 8 (digits of the number), 108 is in that sequence.The sequence is 1, 1, 9, 10, 10, 18, 19, 19, 27, 28, 28, 36, 37, 37, 45, 46, 46, 54, 55, 55, 63, 64, 64, 72, 73, 73, 81, 82, 82, 90, 91, 91, 99, 100, 100, 108.Similarly a number of the form abcd for example is called as a Belgian-n number if it belongs to the sequence starting with n and having reccuring differences of a, b, c, d ...
The first few Belgian-0 numbers are 0, 1, 2, 3, 4, 5, 6, 7, 8, 9, 10, 11, 12, 13, 17, 18, 20, 21, 22, 24 ... (Sequence A106039 of OEIS). The first few Belgian-1 numbers are 1, 10, 11, 13, 16, 17, 21, 23, 41, 43, 56, 58 ... (Sequence A106439 of OEIS). The first few Belgian-2 numbers are 4, 10, 11, 13, 14, 20, 21, 22, 24, 25, 31, 32, 37, 40, 47 ... (Sequence A106331 of OEIS). The first few Belgian-3 numbers are 3, 10, 11, 12, 14, 15, 21, 23, 30 ... (Sequence A106596 of OEIS). The first few Belgian-4 numbers are 4, 10, 11, 13, 14, 20, 21, 22, 25 ... (Sequence A106331 of OEIS).
All Harshad numbers are Belgian-0 numbers. For all k, the number of k-Belgian numbers are infinite. They have a positive density. Every positive integer is a Belgian-k number for some k.

Self-Belgian numbers are of two types. Self-Belgian number of type 1 is a number whose initial digit is 'a' and is an a-Belgian number. For example 68 begins with 6 and is a 6-Belgian number. The self-Belgian numbers of type 1 are 0, 1, 2, 3, 4, 5, 6, 7, 8, 9, 10, 11, 13, 16, 17, 20, 22, 25, 26, 30, 31, 33, 34 ... (Sequence A107062 of OEIS). There are infinitely meny Self-Belgian numbers of type 1.
Self Belgian numbers of type two are the Self-Belgian numbers of type 1 which fully show all their digits at the starting of the sequence. For example, 61 is a Self-Belgian number of type 2 as its sequence is 6, 12, 13, 19, 20, 26, 27, 33, 34, 40, 41, 47, 48, 54, 55 and 61, which begins with digits 6 and 1. The Self-Belgian numbers of type 2 are 1, 2, 3, 4, 5, 6, 7, 8, 9, 61, 71, 918, 3612, 5101, 8161, 12481, 51011 ... (Sequence A107070 of OEIS).
%%%%%
%%%%%
\end{document}
