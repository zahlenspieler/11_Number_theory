\documentclass[12pt]{article}
\usepackage{pmmeta}
\pmcanonicalname{ProbablePrime}
\pmcreated{2013-03-22 15:53:46}
\pmmodified{2013-03-22 15:53:46}
\pmowner{PrimeFan}{13766}
\pmmodifier{PrimeFan}{13766}
\pmtitle{probable prime}
\pmrecord{5}{37898}
\pmprivacy{1}
\pmauthor{PrimeFan}{13766}
\pmtype{Definition}
\pmcomment{trigger rebuild}
\pmclassification{msc}{11A41}

\endmetadata

% this is the default PlanetMath preamble.  as your knowledge
% of TeX increases, you will probably want to edit this, but
% it should be fine as is for beginners.

% almost certainly you want these
\usepackage{amssymb}
\usepackage{amsmath}
\usepackage{amsfonts}

% used for TeXing text within eps files
%\usepackage{psfrag}
% need this for including graphics (\includegraphics)
%\usepackage{graphicx}
% for neatly defining theorems and propositions
%\usepackage{amsthm}
% making logically defined graphics
%%%\usepackage{xypic}

% there are many more packages, add them here as you need them

% define commands here

\begin{document}
A sufficiently large odd integer $q$ believed to be a prime number because it has passed some preliminary primality test relative to a given base, or a pattern suggests it might be prime, but it has not yet been subjected to a conclusive primality test.

For primes with no specific form, it is required to test every potential prime factor $p < \sqrt{q}$ to be absolutely sure that $q$ is in fact a prime. For Mersenne probable primes, the Lucas-Lehmer test is accepted as a conclusive primality test.

Once a probable prime is conclusively shown to be a prime, it of course loses the label "probable." It also loses it if conclusively shown to be composite, but in that case it might then be called a \PMlinkname{pseudoprime}{PseudoprimeP} relative to base $a$.


%%%%%
%%%%%
\end{document}
