\documentclass[12pt]{article}
\usepackage{pmmeta}
\pmcanonicalname{WilsonQuotient}
\pmcreated{2013-03-22 17:57:47}
\pmmodified{2013-03-22 17:57:47}
\pmowner{PrimeFan}{13766}
\pmmodifier{PrimeFan}{13766}
\pmtitle{Wilson quotient}
\pmrecord{6}{40468}
\pmprivacy{1}
\pmauthor{PrimeFan}{13766}
\pmtype{Definition}
\pmcomment{trigger rebuild}
\pmclassification{msc}{11A51}
\pmclassification{msc}{11A41}

% this is the default PlanetMath preamble.  as your knowledge
% of TeX increases, you will probably want to edit this, but
% it should be fine as is for beginners.

% almost certainly you want these
\usepackage{amssymb}
\usepackage{amsmath}
\usepackage{amsfonts}

% used for TeXing text within eps files
%\usepackage{psfrag}
% need this for including graphics (\includegraphics)
%\usepackage{graphicx}
% for neatly defining theorems and propositions
%\usepackage{amsthm}
% making logically defined graphics
%%%\usepackage{xypic}

% there are many more packages, add them here as you need them

% define commands here

\begin{document}
The {\em Wilson quotient} $W_n$ for a given positive integer $n$ is the rational number $\displaystyle \frac{\Gamma(n) + 1}{n}$, where $\Gamma(x)$ is Euler's Gamma function (since we're dealing with integer inputs here, in effect this is merely a quicker way to write $(n - 1)!$).

From Wilson's theorem it follows that the Wilson quotient is an integer only if $n$ is not composite. When $n$ is composite, the numerator of the Wilson quotient is $(n - 1)! + 1$ and the denominator is $n$. For example, if $n = 7$ we have numerator 721 with denominator 7, and since these have 7 as their greatest common divisor, in lowest terms the Wilson quotient  of 7 is 103 (with 1 as tacit numerator). But for $n = 8$ we have $$W_8 = \frac{5041}{8}.$$

\begin{thebibliography}{1}
\bibitem{rc} R. Crandall \& C. Pomerance, {\it Prime Numbers: A Computational Perspective}. New York: Springer (2001): 29.
\end{thebibliography}
%%%%%
%%%%%
\end{document}
