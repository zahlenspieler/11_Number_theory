\documentclass[12pt]{article}
\usepackage{pmmeta}
\pmcanonicalname{GaloisGroupOfABiquadraticExtension}
\pmcreated{2013-03-22 17:44:06}
\pmmodified{2013-03-22 17:44:06}
\pmowner{rm50}{10146}
\pmmodifier{rm50}{10146}
\pmtitle{Galois group of a biquadratic extension}
\pmrecord{5}{40184}
\pmprivacy{1}
\pmauthor{rm50}{10146}
\pmtype{Theorem}
\pmcomment{trigger rebuild}
\pmclassification{msc}{11R16}

% this is the default PlanetMath preamble.  as your knowledge
% of TeX increases, you will probably want to edit this, but
% it should be fine as is for beginners.

% almost certainly you want these
\usepackage{amssymb}
\usepackage{amsmath}
\usepackage{amsfonts}

% used for TeXing text within eps files
%\usepackage{psfrag}
% need this for including graphics (\includegraphics)
%\usepackage{graphicx}
% for neatly defining theorems and propositions
\usepackage{amsthm}
% making logically defined graphics
%%%\usepackage{xypic}

% there are many more packages, add them here as you need them

% define commands here
\newtheorem{thm}{Theorem}
\DeclareMathOperator{\Gal}{Gal}

\begin{document}
This article proves that biquadratic extensions correspond precisely to Galois extensions with Galois group isomorphic to the Klein $4$-group $V_4$ (at least if the characteristic of the base field is not $2$). More precisely,

\begin{thm} Let $F$ be a field of characteristic $\neq 2$ and $K$ a finite extension of $F$. Then the following are equivalent:
\begin{enumerate}
\item $K=F(\sqrt{D_1},\sqrt{D_2})$ for some $D_1,D_2\in F$ such that none of $D_1, D_2$, or $D_1D_2$ is a square in $F$.
\item $K$ is a Galois extension of $F$ with $\Gal(K/F)\cong V_4$;
\end{enumerate}
\end{thm}

\textbf{Proof. } Suppose first that condition (1) holds. Then $[F(\sqrt{D_1}):F]=[F(\sqrt{D_2}):F]=2$ since neither $D_1$ nor $D_2$ is a square in $F$. Now obviously \[[K:F]=[F(\sqrt{D_1},\sqrt{D_2}):F(\sqrt{D_1})][F(\sqrt{D_1}):F]\leq 4\]
and so $[K:F(\sqrt{D_1})]\leq 2$. If $K=F(\sqrt{D_1})$, then $\sqrt{D_2}\in F(\sqrt{D_1})$, so $\sqrt{D_2}=a+b\sqrt{D_1}$ and $D_2=a^2+b^2D_1+2ab\sqrt{D_1}$. Thus $a=0$ or $b=0$. If $b=0$, then $D_2$ is a square. If $a=0$, then $D_1D_2=b^2D_1^2$ is a square. In any case, this is a contradiction. Thus $K$ is a quadratic extension of $F(\sqrt{D_1})$. So $[K:F]=4$.  But $K$ is the splitting field for $(x^2-D_1)(x^2-D_2)$, since the splitting field must contain both square roots, and the polynomial obviously splits in $K$, so $G=\Gal(K/F)$ has four elements
\begin{center}
\begin{tabular}{llll}
$id$ & 
$\sigma:\begin{cases}\sqrt{D_1}\mapsto -\sqrt{D_1}\\\sqrt{D_2}\mapsto \sqrt{D_2}\end{cases}$ &
$\tau:\begin{cases}\sqrt{D_1}\mapsto \sqrt{D_1}\\\sqrt{D_2}\mapsto -\sqrt{D_2}\end{cases}$ &
$\sigma\tau:\begin{cases}\sqrt{D_1}\mapsto -\sqrt{D_1}\\\sqrt{D_2}\mapsto -\sqrt{D_2}\end{cases}$
\end{tabular}
\end{center}
and is thus isomorphic to $V_4$.

Now assume that condition (2) holds. Since $\Gal(K/F)\cong V_4$, there must be three intermediate subfields $E_1, E_2, E_3$ between $F$ and $K$ of degree $2$ over $F$ corresponding to the three subgroups of $V_4$ of order $2$. Thus each of these is a quadratic extension. Suppose $E_1=F(\sqrt{D_1}), E_2=F(\sqrt{D_2})$ where neither $D_1$ nor $D_2$ is a square in $F$. The fact that $E_1\neq E_2$ implies as above that $D_1D_2$ is also not a square in $F$ (in fact $E_3=F(\sqrt{D_1D_2})$. Thus $E_1E_2\supsetneq E_1, E_2$, and is of degree $4$ over $F$, so $K=E_1E_2 =F(\sqrt{D_1},\sqrt{D_2})$.

%%%%%
%%%%%
\end{document}
