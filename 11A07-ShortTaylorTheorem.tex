\documentclass[12pt]{article}
\usepackage{pmmeta}
\pmcanonicalname{ShortTaylorTheorem}
\pmcreated{2013-04-01 13:19:12}
\pmmodified{2013-04-01 13:19:12}
\pmowner{pahio}{2872}
\pmmodifier{pahio}{2872}
\pmtitle{short Taylor theorem}
\pmrecord{1}{}
\pmprivacy{1}
\pmauthor{pahio}{2872}
\pmtype{Definition}
\pmclassification{msc}{11A07}

% this is the default PlanetMath preamble.  as your knowledge
% of TeX increases, you will probably want to edit this, but
% it should be fine as is for beginners.

% almost certainly you want these
\usepackage{amssymb}
\usepackage{amsmath}
\usepackage{amsfonts}

% need this for including graphics (\includegraphics)
\usepackage{graphicx}
% for neatly defining theorems and propositions
\usepackage{amsthm}

% making logically defined graphics
%\usepackage{xypic}
% used for TeXing text within eps files
%\usepackage{psfrag}

% there are many more packages, add them here as you need them

% define commands here

\begin{document}
If $f(x)$ is a polynomial with integer coefficients and $x_0$ and $h$ integers, then the congruence
\begin{align}
f(x_0\!+\!h) \;\equiv\; f(x_0)+f'(x_0)h \;\; \pmod{h^2}
\end{align}
is in force.


{\it Proof.}\; Because of the linear properties of (1) we can confine us to the monomials\, $f(x) := x^n$.\, Then\, $f'(x) = nx^{n-1}$.\,  By the binomial theorem we have
\begin{align}
(x_0\!+\!h)^n \;=\; x_0^n\!+\!nx_0^{n-1}h+h^2P(x_0)
\end{align}
where $P(x_0)$ is a polynomial in $x_0$ with integer coefficients.\, The equality (2) may be written as the asserted congruence (1).
\end{document}
