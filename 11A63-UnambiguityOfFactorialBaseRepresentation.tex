\documentclass[12pt]{article}
\usepackage{pmmeta}
\pmcanonicalname{UnambiguityOfFactorialBaseRepresentation}
\pmcreated{2013-03-22 16:38:35}
\pmmodified{2013-03-22 16:38:35}
\pmowner{rspuzio}{6075}
\pmmodifier{rspuzio}{6075}
\pmtitle{unambiguity of factorial base representation}
\pmrecord{14}{38846}
\pmprivacy{1}
\pmauthor{rspuzio}{6075}
\pmtype{Definition}
\pmcomment{trigger rebuild}
\pmclassification{msc}{11A63}
\pmrelated{UniquenessOfDigitalRepresentation}

\endmetadata

% this is the default PlanetMath preamble.  as your knowledge
% of TeX increases, you will probably want to edit this, but
% it should be fine as is for beginners.

% almost certainly you want these
\usepackage{amssymb}
\usepackage{amsmath}
\usepackage{amsfonts}

% used for TeXing text within eps files
%\usepackage{psfrag}
% need this for including graphics (\includegraphics)
%\usepackage{graphicx}
% for neatly defining theorems and propositions
\usepackage{amsthm}
% making logically defined graphics
%%%\usepackage{xypic}

% there are many more packages, add them here as you need them

% define commands here
\newtheorem{theorem}{Theorem}
\begin{document}
\begin{theorem}
For all positive integers $n$, it is the case that 
 \[n! = 1 + \sum_{k = 1}^{n - 1} k \cdot k!\]
\end{theorem}

\begin{proof}
We first consider two degenerate cases.  When $n=0$
or $n=1$, the sum has no terms because the lower limit
is bigger than the upper limit.  By the convention that
a sum with no terms equals zero, the equation reduces 
to $0! = 1$ or $1! = 1$, which is correct.

Next, assume that $n > 1$.  Note that 
 \[k \cdot k! = (k + 1) k! - k! = (k + 1)! - k!,\]
hence, we have a telescoping sum:
 \[\sum_{k = 1}^{n - 1} k \cdot k! = \sum_{k = 1}^{n - 1}
\left( (k + 1)! - k! \right) = n! - 1\]
Adding 1, we conclude that
 \[n! = 1 + \sum_{k = 1}^{n - 1} k \cdot k!\]
\end {proof}

\begin{theorem}
If $d_m \ldots d_2 d_1$ is the factoradic representation of
a positive integer $n$, then $(d_m + 1) \, m! > n \ge d_m 
\cdot m!$
\end{theorem}

\begin{proof}
By definition,
\[ n = d_m \cdot m! + \sum_{k=1}^{m-1} d_k \cdot k! .\]
Since each term of the sum is nonnegative, the sum is positive,
hence $n \ge d_m \cdot m!$.  Using the fact that $0 \le d_k < k$ 
to bound each term in the sum, we have
\[ n \le d_m \cdot m! + \sum_{k=1}^{m-1} k \cdot k!.\]
By theorem 1, the right-hand side equals $m! - 1$, so we have
$n \le d_m \cdot m! + m! - 1$, which is the same as saying 
$(d_m + 1) \, m! > n$.
\end{proof}

\begin{theorem}
If $d_m \ldots d_2 d_1$ and ${d'}_{m'} \ldots {d'}_2 {d'}_1$ are
distinct factoradic representations with $d_m \neq 0$ and ${d'}_{m'}
\neq 0$  (i.e. not padded with leading zeros), then they denote 
distinct integers.
\end{theorem}

\begin{proof}
Let $n = \sum_{k=1}^m d_k \cdot k!$ and let $n' = \sum_{k=1}^{m'} 
{d'}_k \cdot k!$.

Suppose that $m$ and $m'$ are distinct.  Without loss of generality,
we may assume that $m < m'$.  Then, $(m+1)! \le {m'}!$.  By theorem 2, 
we have $n < (d_m + 1) \, m!$ and $d_{m'} \cdot {m'}! \le n'$.  Because
$d_m \le m$, we have $(d_m + 1) \,m! \le (m + 1) \, m! = (m + 1)!$.
Because $d_{m'} \neq 0$, we have $n' > {m'}!$.  Hence, $n < n'$, so $n$ 
does not equal $n'$.

To complete the proof, we use the method of infinite descent.  Assume
that factoradic representations are not unique.  Then there must exist
a least integer $n$ such that $n$ has two distinct factoradic 
representations $d_m \ldots d_2 d_1$ and ${d'}_{m'} \ldots {d'}_2 {d'}_1$
with $d_m \neq 0$ and ${d'}_{m'} \neq 0$.  By the conclusion of the
previous paragraph, we must have $m = m'$.  By theorem 2, we must have
$d_m = {d'}_m$. Let $m_1$ be the chosen such
that $d_{m_1} \neq 0$ but $d_{m_1 + 1} = \cdots = d_{m-1} = 0$ and let
$m_2$ be the chosen such that ${d'}_{m_2} \neq 0$ but ${d'}_{m_2 + 1} = 
\cdots = {d'}_{m-1} = 0$.  Then $d_{m_1} \ldots d_2 d_1$ and ${d'}_{m_2}
\ldots {d'}_2 {d'}_1$ would be distinct factoradic representations of
$n - d_m \cdot m!$ whose leading digits are not zero but this would 
contradict the assumption that $n$ is the smallest number with this
property. 
\end{proof}
%%%%%
%%%%%
\end{document}
