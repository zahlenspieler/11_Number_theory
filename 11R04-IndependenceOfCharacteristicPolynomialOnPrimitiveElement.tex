\documentclass[12pt]{article}
\usepackage{pmmeta}
\pmcanonicalname{IndependenceOfCharacteristicPolynomialOnPrimitiveElement}
\pmcreated{2014-02-04 8:07:18}
\pmmodified{2014-02-04 8:07:18}
\pmowner{pahio}{2872}
\pmmodifier{pahio}{2872}
\pmtitle{independence of characteristic polynomial on primitive element}
\pmrecord{10}{42054}
\pmprivacy{1}
\pmauthor{pahio}{2872}
\pmtype{Topic}
\pmcomment{trigger rebuild}
\pmclassification{msc}{11R04}
\pmclassification{msc}{12F05}
\pmclassification{msc}{11C08}
\pmclassification{msc}{12E05}
\pmsynonym{norm and trace functions in number field}{IndependenceOfCharacteristicPolynomialOnPrimitiveElement}
%\pmkeywords{norm}
%\pmkeywords{trace.}
\pmrelated{Norm}
\pmrelated{NormAndTraceOfAlgebraicNumber}
\pmrelated{PropertiesOfMathbbQvarthetaConjugates}
\pmrelated{DiscriminantInAlgebraicNumberField}
\pmdefines{norm in number field}
\pmdefines{trace in number field}
\pmdefines{norm}
\pmdefines{trace}

\endmetadata

% this is the default PlanetMath preamble.  as your knowledge
% of TeX increases, you will probably want to edit this, but
% it should be fine as is for beginners.

% almost certainly you want these
\usepackage{amssymb}
\usepackage{amsmath}
\usepackage{amsfonts}

% used for TeXing text within eps files
%\usepackage{psfrag}
% need this for including graphics (\includegraphics)
%\usepackage{graphicx}
% for neatly defining theorems and propositions
 \usepackage{amsthm}
% making logically defined graphics
%%%\usepackage{xypic}

% there are many more packages, add them here as you need them

% define commands here

\theoremstyle{definition}
\newtheorem*{thmplain}{Theorem}

\begin{document}
\PMlinkescapeword{degree}

The simple field extension $\mathbb{Q}(\vartheta)/\mathbb{Q}$ where $\vartheta$ is an algebraic number of \PMlinkname{degree}{DegreeOfAnAlgebraicNumber} $n$ may be determined also by using another primitive element $\eta$.\, Then we have
$$\eta \in \mathbb{Q}(\vartheta),$$
whence, by the entry degree of algebraic number, the degree of $\eta$ divides the degree of $\vartheta$.\, But also
$$\vartheta \in \mathbb{Q}(\eta),$$
whence the degree of $\vartheta$ divides the degree of $\eta$.\, Therefore any possible primitive element of the field extension has the same degree $n$.\, This number is the \PMlinkname{degree of the number field}{NumberField}, i.e. the degree of the field extension, as comes clear from the entry canonical form of element of number field.

Although the characteristic polynomial
$$g(x) \;:=\; \prod_{i=1}^n[x-r(\vartheta_i)] \;=\; \prod_{i=1}^n(x-\alpha^{(i)})$$
of an element $\alpha$ of the algebraic number field $\mathbb{Q}(\vartheta)$ is based on the primitive element 
$\vartheta$, the equation 
\begin{align}
g(x) \;=\; (x-\alpha_1)^m(x-\alpha_2)^m\cdots(x-\alpha_k)^m
\end{align}
in the entry \PMlinkid{degree of algebraic number}{12050} shows that the polynomial is fully determined by the algebraic conjugates of $\alpha$ itself and the number $m$ which equals the degree $n$ divided by the degree $k$ of 
$\alpha$. \\

The above stated makes meaningful to define the norm and the trace functions in an algebraic number field as follows.

\textbf{Definition.}\, If $\alpha$ is an element of the number field $\mathbb{Q}(\vartheta)$, then the \emph{norm} 
$\mbox{N}(\alpha)$ and the \emph{trace} $\mbox{S}(\alpha)$ of $\alpha$ are the product and the sum, respectively, of all \PMlinkid{$\mathbb{Q}(\vartheta)$-conjugates}{12046} $\alpha^{(i)}$ of $\alpha$.\\


Since the coefficients of the characteristic equation of $\alpha$ are rational, one has
$$\mbox{N}\!:\,\mathbb{Q}(\vartheta) \to \mathbb{Q} \quad\mbox{and}\quad 
  \mbox{S}\!:\,\mathbb{Q}(\vartheta) \to \mathbb{Q}.$$
In fact, one can infer from (1) that
\begin{align}
\mbox{N}(\alpha) \;=\; a_k^m, \qquad \mbox{S}(\alpha) \;=\; -ma_1,
\end{align}
where $x^k\!+\!a_1x^{k-1}\!+\ldots+\!a_k$ is the minimal polynomial of $\alpha$.



%%%%%
%%%%%
\end{document}
