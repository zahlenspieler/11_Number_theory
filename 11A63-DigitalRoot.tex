\documentclass[12pt]{article}
\usepackage{pmmeta}
\pmcanonicalname{DigitalRoot}
\pmcreated{2013-03-22 15:59:34}
\pmmodified{2013-03-22 15:59:34}
\pmowner{PrimeFan}{13766}
\pmmodifier{PrimeFan}{13766}
\pmtitle{digital root}
\pmrecord{13}{38017}
\pmprivacy{1}
\pmauthor{PrimeFan}{13766}
\pmtype{Definition}
\pmcomment{trigger rebuild}
\pmclassification{msc}{11A63}
\pmsynonym{repeated digit sum}{DigitalRoot}
\pmsynonym{repeated digital sum}{DigitalRoot}
\pmdefines{additive persistence}

\endmetadata

% this is the default PlanetMath preamble.  as your knowledge
% of TeX increases, you will probably want to edit this, but
% it should be fine as is for beginners.

% almost certainly you want these
\usepackage{amssymb}
\usepackage{amsmath}
\usepackage{amsfonts}

% used for TeXing text within eps files
%\usepackage{psfrag}
% need this for including graphics (\includegraphics)
%\usepackage{graphicx}
% for neatly defining theorems and propositions
%\usepackage{amsthm}
% making logically defined graphics
%%%\usepackage{xypic}

% there are many more packages, add them here as you need them

% define commands here

\begin{document}
Given an integer $m$ consisting of $k$ digits $d_1, \dots, d_k$ in base $b$, let $$j = \sum_{i = 1}^{k} d_i,$$ then repeat this operation on the digits of $j$ until $j < b$. This stores in $j$ the {\em digital root} of $m$. The number of iterations of the sum operation is called the {\em additive persistence} of $m$.

The digital root of $b^x$ is always 1 for any natural $x$, while the digital root of $yb^n$ (where $y$ is another natural number) is the same as the digital root of $y$. This should not be taken to imply that the digital root is necessarily a multiplicative function.

The digital root of an integer of the form $n(b - 1)$ is always $b - 1$. 

Another way to calculate the digital root for $m > b$ is with the formula $m - (b - 1)\lfloor {{m - 1} \over {b - 1}} \rfloor$.
%%%%%
%%%%%
\end{document}
