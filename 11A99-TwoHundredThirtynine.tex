\documentclass[12pt]{article}
\usepackage{pmmeta}
\pmcanonicalname{TwoHundredThirtynine}
\pmcreated{2013-03-22 16:57:56}
\pmmodified{2013-03-22 16:57:56}
\pmowner{PrimeFan}{13766}
\pmmodifier{PrimeFan}{13766}
\pmtitle{two hundred thirty-nine}
\pmrecord{5}{39239}
\pmprivacy{1}
\pmauthor{PrimeFan}{13766}
\pmtype{Feature}
\pmcomment{trigger rebuild}
\pmclassification{msc}{11A99}
\pmsynonym{two hundred and thirty-nine}{TwoHundredThirtynine}

\endmetadata

% this is the default PlanetMath preamble.  as your knowledge
% of TeX increases, you will probably want to edit this, but
% it should be fine as is for beginners.

% almost certainly you want these
\usepackage{amssymb}
\usepackage{amsmath}
\usepackage{amsfonts}

% used for TeXing text within eps files
%\usepackage{psfrag}
% need this for including graphics (\includegraphics)
%\usepackage{graphicx}
% for neatly defining theorems and propositions
%\usepackage{amsthm}
% making logically defined graphics
%%%\usepackage{xypic}

% there are many more packages, add them here as you need them

% define commands here

\begin{document}
The MIT Artificial Intelligence Laboratory Memo 239 (entitled {\it HAKMEM}) of February 1972 listed various properties of the integer 239, some more interesting than others, such as: that 239 needs the maximum number of powers in Waring's problem for squares, cubes and fourth powers; that it appears in Machin's formula for $\pi$: $$\frac{1}{4}\pi = 4\cot^{-1}5 - \cot^{-1}239;$$ that since $239^2 = 2 \times 13^4 - 1$, the fraction $\frac{239}{13^2}$ is a convergent for the continued fraction of $\sqrt{2}$; that since $\pi(1500) = 239$ (with $\pi(x)$ being the prime counting function; etc. At the time, the memo couldn't have mentioned that 239 pounds is Homer Simpson's weight (as established by at least two episodes of {\it The Simpsons}).

The prime 239, like many in its vicinity, is a Chen prime. As a Sophie Germain prime, it begins a Cunningham chain of just length 2 (which ends with 479). On the plane of Eisenstein integers, 239 is an Eisenstein prime (its real part being of the form $3n - 1$ and it not having an imaginary part), and it is also a Gaussian prime (its real part being of the form $4n - 1$); these two properties it has in common with all real primes $p \equiv 11 \mod 12$. Much more rare is that it is the third Newman-Shanks-Williams prime (the ninth is more than $10^{98}$).
%%%%%
%%%%%
\end{document}
