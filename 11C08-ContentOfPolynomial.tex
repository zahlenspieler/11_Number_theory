\documentclass[12pt]{article}
\usepackage{pmmeta}
\pmcanonicalname{ContentOfPolynomial}
\pmcreated{2013-11-19 18:51:57}
\pmmodified{2013-11-19 18:51:57}
\pmowner{pahio}{2872}
\pmmodifier{pahio}{2872}
\pmtitle{content of polynomial}
\pmrecord{10}{42124}
\pmprivacy{1}
\pmauthor{pahio}{2872}
\pmtype{Definition}
\pmcomment{trigger rebuild}
\pmclassification{msc}{11C08}
\pmrelated{CoefficientModule}
\pmrelated{PruferRing}
\pmrelated{GaussianPolynomials}
\pmdefines{content of polynomial}
\pmdefines{Gaussian ideal}
\pmdefines{Gaussian polynomial}
\pmdefines{Gaussian ring}

\endmetadata

% this is the default PlanetMath preamble.  as your knowledge
% of TeX increases, you will probably want to edit this, but
% it should be fine as is for beginners.

% almost certainly you want these
\usepackage{amssymb}
\usepackage{amsmath}
\usepackage{amsfonts}

% used for TeXing text within eps files
%\usepackage{psfrag}
% need this for including graphics (\includegraphics)
%\usepackage{graphicx}
% for neatly defining theorems and propositions
 \usepackage{amsthm}
% making logically defined graphics
%%%\usepackage{xypic}

% there are many more packages, add them here as you need them

% define commands here

\theoremstyle{definition}
\newtheorem*{thmplain}{Theorem}

\begin{document}
The \emph{content of a polynomial} $f$ may be defined in any polynomial ring $R[x]$ over a commutative ring $R$ as the ideal of $R$ generated by the coefficients of the polynomial.\, It is denoted by $\operatorname{cont}(f)$ or $c(f)$.\, Coefficient module is a little more general concept.\\

If $R$ is a \PMlinkname{unique factorisation domain}{UFD} and\, $f,\,g \in R[x]$,\, the Gauss lemma I 
implies \footnote{In a UFD, one can use as contents of $f$ and $g$ the \PMlinkid{greatest common divisors}{5800} $a$ and $b$ of the coefficients of these polynomials, when one has\, $f(x) = af_1(x)$,\; $g(x) = bg_1(x)$\, with $f_1(x)$ and $g_1(x)$ primitive polynomials.\, Then\, $f(x)g(x) = abf_1(x)g_1(x)$,\, and since also $f_1g_1$ is a primitive polynomial, we see that\, $c(fg) = ab = c(f)c(g)$.} that
\begin{align}
c(fg) \;=\; c(f)c(g).
\end{align}

For an arbitrary commutative ring $R$, there is only the containment
\begin{align}
c(fg) \;\subseteq\; c(f)c(g)
\end{align}
(cf. product of finitely generated ideals).\, The ideal $c(fg)$ is called the \emph{Gaussian ideal of} the polynomials
$f$ and $g$.\, The polynomial $f$ in $R[x]$ is a \PMlinkescapetext{\emph{Gaussian polynomial}}, if (2) becomes the equality (1) for all polynomials $g$ in the ring $R[x]$.\, The ring $R$ is a \emph{Gaussian ring}, if all polynomials in 
$R[x]$ are \PMlinkescapetext{Gaussian polynomials}.\\

It's quite interessant, that the equation (1) multiplied by the power $[c(f)]^n$, where $n$ is the degree of the other polynomial $g$, however is true in any commutative ring $R$, thus replacing the containment (2):
\begin{align}
[c(f)]^nc(fg) \;=\; [c(f)]^{n+1}c(g).
\end{align}
This result is called the 
\emph{Hilfssatz von Dedekind--Mertens}, i.e. the 
Dedekind--Mertens lemma.\, A generalised form of it is in the 
entry 
\PMlinkname{product of finitely generated ideals}{ProductOfFinitelyGeneratedIdeals}.

\begin{thebibliography}{8}
\bibitem{CG}{\sc Alberto Corso \& Sarah Glaz}: ``Gaussian ideals and the Dedekind--Mertens lemma'' in J\"urgen Herzog \& Gaetana Restuccia (eds.): {\em Geometric and combinatorial aspects of commutative algebra}.\, Marcel Dekker Inc., New York (2001).
\end{thebibliography}

%%%%%
%%%%%
\end{document}
