\documentclass[12pt]{article}
\usepackage{pmmeta}
\pmcanonicalname{sigmaFunction}
\pmcreated{2013-03-22 16:07:12}
\pmmodified{2013-03-22 16:07:12}
\pmowner{CompositeFan}{12809}
\pmmodifier{CompositeFan}{12809}
\pmtitle{$\sigma$ function}
\pmrecord{8}{38188}
\pmprivacy{1}
\pmauthor{CompositeFan}{12809}
\pmtype{Definition}
\pmcomment{trigger rebuild}
\pmclassification{msc}{11A25}
\pmsynonym{divisor sigma}{sigmaFunction}
\pmsynonym{sum of divisors function}{sigmaFunction}
\pmsynonym{$\sigma_1$ function}{sigmaFunction}

\endmetadata

% this is the default PlanetMath preamble.  as your knowledge
% of TeX increases, you will probably want to edit this, but
% it should be fine as is for beginners.

% almost certainly you want these
\usepackage{amssymb}
\usepackage{amsmath}
\usepackage{amsfonts}

% used for TeXing text within eps files
%\usepackage{psfrag}
% need this for including graphics (\includegraphics)
%\usepackage{graphicx}
% for neatly defining theorems and propositions
%\usepackage{amsthm}
% making logically defined graphics
%%%\usepackage{xypic}

% there are many more packages, add them here as you need them

% define commands here

\begin{document}
Given a positive integer $n$, the sum of the integers $0 < d \le n$ such that $d|n$ is the value of the {\em sum of divisors function} for $n$, often symbolized by a  Greek lowercase $\sigma$. Thus, $$\sigma(n) = \sum_{d|n} d.$$ Sometimes this function is referred to as $\sigma_1(n)$, highlighting its relation to the divisor function.

Given coprime integers $m$ and $n$ (that is, $\gcd(m, n) = 1$) then $\sigma(mn) = \sigma(m)\sigma(n)$, meaning that the sum of divisors function is a multiplicative function.

%%%%%
%%%%%
\end{document}
