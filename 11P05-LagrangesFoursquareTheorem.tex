\documentclass[12pt]{article}
\usepackage{pmmeta}
\pmcanonicalname{LagrangesFoursquareTheorem}
\pmcreated{2013-03-22 12:35:17}
\pmmodified{2013-03-22 12:35:17}
\pmowner{bbukh}{348}
\pmmodifier{bbukh}{348}
\pmtitle{Lagrange's four-square theorem}
\pmrecord{10}{32838}
\pmprivacy{1}
\pmauthor{bbukh}{348}
\pmtype{Theorem}
\pmcomment{trigger rebuild}
\pmclassification{msc}{11P05}
\pmrelated{WaringsProblem}
\pmrelated{EulerFourSquareIdentity}

\usepackage{amssymb}
\usepackage{amsmath}
\usepackage{amsfonts}
\begin{document}
\PMlinkescapeword{states}

Lagrange's four-square theorem states that every non-negative integer may be expressed as the sum of at most four squares. By the Euler four-square identity, it is enough to show that every prime is expressible by at most four squares. It was later proved that only the numbers of the form $4^n(8m + 7)$ require four squares.

This shows that $g(2) = G(2) = 4$, where $g$ and $G$ are the \PMlinkname{Waring functions}{WaringsProblem}.
%%%%%
%%%%%
\end{document}
