\documentclass[12pt]{article}
\usepackage{pmmeta}
\pmcanonicalname{SidonSet}
\pmcreated{2013-03-22 13:59:40}
\pmmodified{2013-03-22 13:59:40}
\pmowner{bbukh}{348}
\pmmodifier{bbukh}{348}
\pmtitle{Sidon set}
\pmrecord{12}{34771}
\pmprivacy{1}
\pmauthor{bbukh}{348}
\pmtype{Definition}
\pmcomment{trigger rebuild}
\pmclassification{msc}{11B05}
\pmclassification{msc}{11B34}
\pmsynonym{Sidon sequence}{SidonSet}
%\pmkeywords{difference set}
\pmrelated{ErdHosTuranConjecture}
\pmdefines{$B_h[g]$ set}

\endmetadata

\usepackage{amssymb}
\usepackage{amsmath}
\usepackage{amsfonts}

% used for TeXing text within eps files
%\usepackage{psfrag}
% need this for including graphics (\includegraphics)
%\usepackage{graphicx}
% for neatly defining theorems and propositions
%\usepackage{amsthm}
% making logically defined graphics
%%%\usepackage{xypic}

\makeatletter
\@ifundefined{bibname}{}{\renewcommand{\bibname}{References}}
\makeatother

\newcommand*{\abs}[1]{\left\lvert#1\right\rvert}
\begin{document}
A set of natural numbers is called a \emph{Sidon set} if all
pairwise sums of its elements are distinct. Equivalently, the equation
$a+b=c+d$ has only the trivial solution $\{a,b\}=\{c,d\}$ in
elements of the set.

Sidon sets are a special case of so-called $B_h[g]$ sets. A set $A$
is called a $B_h[g]$ set if for every $n\in \mathbb{N}$ the equation
$n=a_1+\dotsc+a_h$ has at most $g$ different solutions with
$a_1\leq \dotsb\leq a_h$ being elements of $A$. The Sidon sets are
$B_2[1]$ sets.

Define $F_h(n,g)$ as the size of the largest $B_h[g]$ set
contained in the interval $[1,n]$. Whereas it is known that
$F_2(n,1)=n^{1/2}+O(n^{5/16})$\cite[p.~85]{cite:halberstam_sequences},
no asymptotical results are known for $g>1$ or $h>2$
\cite{cite:green_bhg}.

Every finite set has a rather large $B_h[1]$ subset. Koml\'os,
Sulyok, and
Szemer\'edi\cite{cite:komlos_szem_sulyok_linearsubsets} proved
that in every $n$-element set there is a $B_h[1]$ subset of size at
least $cF_h(n,1)$ for $c=\tfrac{1}{8}(2h)^{-6}$. For Sidon sets
Abbott\cite{cite:abbott_sidonsubsets} improved that to $c=2/25$.
It is not known whether one can take $c=1$.

The infinite $B_h[g]$ sets are understood even worse. Erd\H{o}s
\cite[p.~89]{cite:halberstam_sequences} proved that for every
infinite Sidon set $A$ we have $\liminf_{n\to\infty}(n/\log
n)^{-1/2} \abs{A\cap [1,n]}<C$ for some constant $C$. On the other
hand, for a long time no example of a set for which $\abs{A\cap
[1,n]}>n^{1/3+\epsilon}$ for some $\epsilon>0$ was known. Only
recently Ruzsa\cite{cite:ruzsa_densesidon} used an extremely
clever construction to prove the existence of a set $A$ for which
$\abs{A\cap [1,n]}>n^{\sqrt{2}-1-\epsilon}$ for every 
$\epsilon>0$ and for all sufficiently large $n$.

For an excellent survey of Sidon sets see
\cite{cite:obryant_sidon}.

\begin{thebibliography}{1}

\bibitem{cite:abbott_sidonsubsets}
Harvey~Leslie Abbott.
\newblock Sidon sets.
\newblock {\em Canad. Math. Bull.}, 33(3):335--341, 1990.
\newblock \PMlinkexternal{Zbl 0715.11004}{http://www.emis.de/cgi-bin/zmen/ZMATH/en/quick.html?type=html&an=0715.11004}.

\bibitem{cite:green_bhg}
Ben Green.
\newblock {$B_h[g]$} sets: The current state of affairs.
\newblock \PMlinkexternal{http://www.dpmms.cam.ac.uk/~bjg23/papers/bhgbounds.dvi}{http://www.dpmms.cam.ac.uk/~bjg23/papers/bhgbounds.dvi}, 2000.

\bibitem{cite:halberstam_sequences}
Heini Halberstam and Klaus~Friedrich Roth.
\newblock {\em Sequences}.
\newblock Springer-Verlag, second edition, 1983.
\newblock \PMlinkexternal{Zbl 0498.10001}{http://www.emis.de/cgi-bin/zmen/ZMATH/en/quick.html?type=html&an=0498.10001}.

\bibitem{cite:komlos_szem_sulyok_linearsubsets}
J{\'a}nos Koml{\'o}s, Miklos Sulyok, and Endre Szemer{\'e}di.
\newblock Linear problems in combinatorial number theory.
\newblock {\em Acta Math. Acad. Sci. Hung.}, 26:113--121, 1975.
\newblock \PMlinkexternal{Zbl 0303.10058}{http://www.emis.de/cgi-bin/zmen/ZMATH/en/quick.html?type=html&an=0303.10058}.

\bibitem{cite:obryant_sidon}
Kevin O'Bryant.
\newblock A complete annotated bibliography of work related to {Sidon}
  sequences.
\newblock {\em Electronic Journal of Combinatorics}.
\newblock
\PMlinkexternal{http://arxiv.org/abs/math.NT/0407117}{http://arxiv.org/abs/math.NT/0407117}.

\bibitem{cite:ruzsa_densesidon}
Imre Ruzsa.
\newblock An infinite {Sidon} sequence.
\newblock {\em J.\ Number Theory}, 68(1):63--71, 1998.
\newblock Available at \PMlinkexternal{http://www.math-inst.hu/~ruzsa/cikkek.html}{http://www.math-inst.hu/~ruzsa/cikkek.html}.

\end{thebibliography}
%%%%%
%%%%%
\end{document}
