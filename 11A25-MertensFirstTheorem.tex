\documentclass[12pt]{article}
\usepackage{pmmeta}
\pmcanonicalname{MertensFirstTheorem}
\pmcreated{2013-03-22 11:46:06}
\pmmodified{2013-03-22 11:46:06}
\pmowner{KimJ}{5}
\pmmodifier{KimJ}{5}
\pmtitle{Mertens' first theorem}
\pmrecord{8}{30222}
\pmprivacy{1}
\pmauthor{KimJ}{5}
\pmtype{Theorem}
\pmcomment{trigger rebuild}
\pmclassification{msc}{11A25}
\pmclassification{msc}{17B45}
\pmclassification{msc}{17B66}
%\pmkeywords{number theory}

\endmetadata

\usepackage{amssymb}
\usepackage{amsmath}
\usepackage{amsfonts}
\usepackage{graphicx}
%%%%\usepackage{xypic}
\begin{document}
For any real number $x \geq 2$ we have
\[ \sum_{p \leq x} \frac{\ln p}{p} = \ln x + O(1) \] for all prime integers $p$.

Moreover, the term $O(1)$ arising in this formula lies in the open interval $(-1-\ln 4, \ln 4)$.
%%%%%
%%%%%
%%%%%
%%%%%
\end{document}
