\documentclass[12pt]{article}
\usepackage{pmmeta}
\pmcanonicalname{ProofOfFermatsLittleTheoremUsingLagrangesTheorem}
\pmcreated{2013-03-22 14:23:53}
\pmmodified{2013-03-22 14:23:53}
\pmowner{alozano}{2414}
\pmmodifier{alozano}{2414}
\pmtitle{proof of Fermat's little theorem using Lagrange's theorem}
\pmrecord{4}{35895}
\pmprivacy{1}
\pmauthor{alozano}{2414}
\pmtype{Proof}
\pmcomment{trigger rebuild}
\pmclassification{msc}{11-00}
\pmrelated{LagrangesTheorem}

% this is the default PlanetMath preamble.  as your knowledge
% of TeX increases, you will probably want to edit this, but
% it should be fine as is for beginners.

% almost certainly you want these
\usepackage{amssymb}
\usepackage{amsmath}
\usepackage{amsthm}
\usepackage{amsfonts}

% used for TeXing text within eps files
%\usepackage{psfrag}
% need this for including graphics (\includegraphics)
%\usepackage{graphicx}
% for neatly defining theorems and propositions
%\usepackage{amsthm}
% making logically defined graphics
%%%\usepackage{xypic}

% there are many more packages, add them here as you need them

% define commands here

\newtheorem{thm}{Theorem}
\newtheorem*{thm2}{Theorem}
\newtheorem{defn}{Definition}
\newtheorem{prop}{Proposition}
\newtheorem{lemma}{Lemma}
\newtheorem{cor}{Corollary}

% Some sets
\newcommand{\Nats}{\mathbb{N}}
\newcommand{\Ints}{\mathbb{Z}}
\newcommand{\Reals}{\mathbb{R}}
\newcommand{\Complex}{\mathbb{C}}
\newcommand{\Rats}{\mathbb{Q}}
\begin{document}
\begin{thm2}
If $a, p \in \mathbb{Z}$ with $p$ a prime and $p \nmid a$, then $a^{p-1} \equiv 1 \pmod{p}$.
\end{thm2}
\begin{proof}
We will make use of Lagrange's Theorem: Let $G$ be a finite group and let $H$ be a subgroup of $G$. Then the order of $H$ divides the order of $G$. 

Let $G=(\Ints/p\Ints)^\times$ and let $H$ be the multiplicative subgroup of $G$ generated by $a$ (so $H=\{ 1, a ,a^2,\ldots \}$). Notice that the order of $H$, $h=|H|$ is also the order of $a$, i.e. the smallest natural number $n>1$ such that $a^n$ is the identity in $G$, i.e. $a^h\equiv 1 \mod p$. 

By Lagrange's theorem $h \mid |G|=p-1$, so $p-1=h\cdot m$ for some $m$. Thus:
$$a^{p-1}=(a^h)^m\equiv 1^m \equiv 1 \mod p$$
as claimed.
\end{proof}
%%%%%
%%%%%
\end{document}
