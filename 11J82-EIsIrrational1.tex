\documentclass[12pt]{article}
\usepackage{pmmeta}
\pmcanonicalname{EIsIrrational1}
\pmcreated{2013-03-22 17:02:32}
\pmmodified{2013-03-22 17:02:32}
\pmowner{rspuzio}{6075}
\pmmodifier{rspuzio}{6075}
\pmtitle{e is irrational}
\pmrecord{8}{39330}
\pmprivacy{1}
\pmauthor{rspuzio}{6075}
\pmtype{Proof}
\pmcomment{trigger rebuild}
\pmclassification{msc}{11J82}
\pmclassification{msc}{11J72}
\pmrelated{LeibnizEstimateForAlternatingSeries}

\endmetadata

% this is the default PlanetMath preamble.  as your knowledge
% of TeX increases, you will probably want to edit this, but
% it should be fine as is for beginners.

% almost certainly you want these
\usepackage{amssymb}
\usepackage{amsmath}
\usepackage{amsfonts}

% used for TeXing text within eps files
%\usepackage{psfrag}
% need this for including graphics (\includegraphics)
%\usepackage{graphicx}
% for neatly defining theorems and propositions
%\usepackage{amsthm}
% making logically defined graphics
%%%\usepackage{xypic}

% there are many more packages, add them here as you need them

% define commands here

\begin{document}
We have the series
\[
e^{-1} = \sum_{k=0}^\infty {(-1)^k \over k!}
\]
Note that this is an alternating series and that the magnitudes of the
terms decrease.  Hence, for every integer $n > 0$, we have the bound
\[
0 <
\left| \sum_{k=0}^{n} {(-1)^k \over k!} - e^{-1} \right| <
{1 \over (n+1)!},
\]
by the \PMlinkname{Leibniz' estimate for alternating 
series}{LeibnizEstimateForAlternatingSeries}.\, Assume 
that $e = n/m$, where $m$ and $n$ are integers and $n > 0$.\, 
Then we would have
\[
0 <
\left| \sum_{k=0}^{n} {(-1)^k \over k!} - 
{m \over n} \right| <
{1 \over (n+1)!} .
\]
Multiplying both sides by $n!$, this would imply
\[
0 <
\left| \sum_{k=0}^{n} {(-1)^k n! \over k!} - 
m (n-1)! \right| <
{1 \over n+1} ,
\]
which is a contradiction because every term in the sum is an integer,
but there are no integers between $0$ and $1/(n+1)$.
%%%%%
%%%%%
\end{document}
