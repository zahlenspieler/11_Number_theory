\documentclass[12pt]{article}
\usepackage{pmmeta}
\pmcanonicalname{MangoldtSummatoryFunctionIsOx}
\pmcreated{2013-03-22 17:42:59}
\pmmodified{2013-03-22 17:42:59}
\pmowner{rm50}{10146}
\pmmodifier{rm50}{10146}
\pmtitle{Mangoldt summatory function is $O(x)$}
\pmrecord{5}{40161}
\pmprivacy{1}
\pmauthor{rm50}{10146}
\pmtype{Theorem}
\pmcomment{trigger rebuild}
\pmclassification{msc}{11A41}

\endmetadata

% this is the default PlanetMath preamble.  as your knowledge
% of TeX increases, you will probably want to edit this, but
% it should be fine as is for beginners.

% almost certainly you want these
\usepackage{amssymb}
\usepackage{amsmath}
\usepackage{amsfonts}

% used for TeXing text within eps files
%\usepackage{psfrag}
% need this for including graphics (\includegraphics)
%\usepackage{graphicx}
% for neatly defining theorems and propositions
%\usepackage{amsthm}
% making logically defined graphics
%%%\usepackage{xypic}

% there are many more packages, add them here as you need them

% define commands here
\newtheorem{thm}{Theorem}

\begin{document}
\begin{thm} $\psi(x)=O(x)$, in other \PMlinkescapetext{words}, $\frac{\psi(x)}{x}$ is bounded.
\end{thm}
\textbf{Proof. }
\[\psi(x)=\sum_1^x \Lambda(n)=
\sum_{\substack{p\text{ prime}\\p\leq x}}\lfloor \log_p x\rfloor \ln p=
\sum_{\substack{p\text{ prime}\\p\leq x}}\left\lfloor\frac{\ln x}{\ln p}\right\rfloor\ln p=
\sum_{\substack{p\text{ prime}\\p\leq \sqrt{x}}}\left\lfloor\frac{\ln x}{\ln p}\right\rfloor\ln p +
   \sum_{\substack{p\text{ prime}\\\sqrt{x}<p\leq x}}\ln p\]
since $1\leq \frac{\ln x}{\ln p}<2$ if $p>\sqrt{x}$.
Continuing, we have
\[\sum_{\substack{p\text{ prime}\\p\leq \sqrt{x}}}\left\lfloor\frac{\ln x}{\ln p}\right\rfloor\ln p +
   \sum_{\substack{p\text{ prime}\\\sqrt{x}<p\leq x}}\ln p \leq
\sqrt{x}\ln x+\pi(x)\ln x\leq\sqrt{x}\ln x+8x\ln 2=O(x)
\]
Note that $\pi(x)\ln x\leq 8x\ln 2$ by Chebyshev's \PMlinkname{bounds on $\pi(x)$}{BoundsOnPin}.
%%%%%
%%%%%
\end{document}
