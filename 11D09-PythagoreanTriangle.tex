\documentclass[12pt]{article}
\usepackage{pmmeta}
\pmcanonicalname{PythagoreanTriangle}
\pmcreated{2013-11-23 11:53:13}
\pmmodified{2013-11-23 11:53:13}
\pmowner{pahio}{2872}
\pmmodifier{pahio}{2872}
\pmtitle{Pythagorean triangle}
\pmrecord{15}{87608}
\pmprivacy{1}
\pmauthor{pahio}{2872}
\pmtype{Result}
\pmclassification{msc}{11D09}
\pmclassification{msc}{51M05}

% this is the default PlanetMath preamble.  as your knowledge
% of TeX increases, you will probably want to edit this, but
% it should be fine as is for beginners.

% almost certainly you want these
\usepackage{amssymb}
\usepackage{amsmath}
\usepackage{amsfonts}

% need this for including graphics (\includegraphics)
\usepackage{graphicx}
% for neatly defining theorems and propositions
\usepackage{amsthm}

% making logically defined graphics
%\usepackage{xypic}
% used for TeXing text within eps files
%\usepackage{psfrag}

% there are many more packages, add them here as you need them

% define commands here

\begin{document}
The side lengths of any right triangle satisfy the equation of the Pythagorean theorem, 
but if they are integers then the triangle is a {\it Pythagorean triangle}.

The side lengths are said to form a Pythagorean triple.\, They are always different 
integers, the smallest of them being at least 3.\\

Any Pythagorean triangle has the property that the hypotenuse is 
the contraharmonic mean 
\begin{align}
c \;=\; \frac{u^2\!+\!v^2}{u\!+\!v}
\end{align}
and one cathetus is the harmonic mean 
\begin{align}
h \;=\; \frac{2uv}{u\!+\!v}
\end{align}
of a certain pair of distinct positive integers $u$, $v$; the
other cathetus is simply $|u\!-\!v|$.\\

If there is given the value of $c$ as the length of the 
hypotenuse and a compatible value\, $h$ as the length of one 
cathetus, the pair of equations (1) and (2) does not determine 
the numbers $u$ and $v$ uniquely (cf. the Proposition 4 in the 
entry integer contraharmonic means).\, For example, if\, 
$c = 61$\, and\, $h = 11$, then the equations give for\, 
$(u, v)$\, either\, $(6,\,66)$\, or\, $(55,\,66)$.\\


As for the hypotenuse and (1), the proof is found in [1] and also 
in the PlanetMath article contraharmonic means and Pythagorean 
hypotenuses.\, The contraharmonic and the harmonic mean of two 
integers are simultaneously integers (see 
\PMlinkname{this article}{IntegerHarmonicMeans}).\, The above 
claim concerning the catheti of the Pythagorean triangle is 
evident from the identity
$$\left(\frac{2uv}{u\!+\!v}\right)^2\!+\!\left|u\!-\!v\right|^2 
\;=\; \left(\frac{u^2\!+\!v^2}{u\!+\!v}\right)^2.$$\\


If the catheti of a Pythagorean triangle are $a$ and $b$,
then the values of the parameters $u$ and $v$ determined by 
the equations (1) and (2) are
\begin{align}
\frac{c\!+\!b\!\pm\!a}{2}
\end{align}
as one instantly sees by substituting them into the equations.


\begin{thebibliography}{8}
\bibitem{K}{\sc J. Pahikkala}: ``On contraharmonic mean and Pythagorean triples''.\, -- \emph{Elemente der Mathematik} \textbf{65}:2 (2010).
\end{thebibliography}




\end{document}
