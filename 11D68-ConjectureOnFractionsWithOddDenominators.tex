\documentclass[12pt]{article}
\usepackage{pmmeta}
\pmcanonicalname{ConjectureOnFractionsWithOddDenominators}
\pmcreated{2013-03-22 12:48:34}
\pmmodified{2013-03-22 12:48:34}
\pmowner{drini}{3}
\pmmodifier{drini}{3}
\pmtitle{conjecture on fractions with odd denominators}
\pmrecord{9}{33128}
\pmprivacy{1}
\pmauthor{drini}{3}
\pmtype{Conjecture}
\pmcomment{trigger rebuild}
\pmclassification{msc}{11D68}
\pmclassification{msc}{11A67}
\pmrelated{SierpinskiErdosEgyptianFractionConjecture}

% this is the default PlanetMath preamble.  as your knowledge
% of TeX increases, you will probably want to edit this, but
% it should be fine as is for beginners.

% almost certainly you want these
\usepackage{amssymb}
\usepackage{amsmath}
\usepackage{amsfonts}

% used for TeXing text within eps files
%\usepackage{psfrag}
% need this for including graphics (\includegraphics)
%\usepackage{graphicx}
% for neatly defining theorems and propositions
\usepackage{amsthm}
% making logically defined graphics
%%%\usepackage{xypic}

% there are many more packages, add them here as you need them

% define commands here

\newcommand{\Prob}[2]{\mathbb{P}_{#1}\left\{#2\right\}}
\newcommand{\Expect}{\mathbb{E}}
\newcommand{\norm}[1]{\left\|#1\right\|}

% Some sets
\newcommand{\Nats}{\mathbb{N}}
\newcommand{\Ints}{\mathbb{Z}}
\newcommand{\Reals}{\mathbb{R}}
\newcommand{\Complex}{\mathbb{C}}


%%%%%% END OF SAVED PREAMBLE %%%%%%

\newtheorem{conjecture}{Conjecture}
\begin{document}
Egyptian fractions raise many open problems; this is one of the most famous of them.

Suppose we wish to write fractions as sums of distinct unit fractions \emph{with odd denominators}.  Obviously, every such sum will have a reduced representation with an odd denominator.

For instance, the greedy algorithm applied to $\frac{2}{7}$ gives $\frac{1}{4}+\frac{1}{28}$, but we may also write $\frac{2}{7}$ as $\frac{1}{7}+\frac{1}{9}+\frac{1}{35}+\frac{1}{315}$.

It is known that we can we represent \emph{every} rational number with odd denominator as a sum of distinct unit fractions with odd denominators.

However it is not known whether the \PMlinkname{greedy algorithm}{AnyRationalNumberIsASumOfUnitFractions} works when limited to odd denominators.  

\begin{conjecture}
For any fraction $0\le \frac{a}{2k+1} < 1$ with odd denominator, if we repeatedly subtract the largest unit fraction with odd denominator that is smaller than our fraction, we will eventually reach 0.
\end{conjecture}
%%%%%
%%%%%
\end{document}
