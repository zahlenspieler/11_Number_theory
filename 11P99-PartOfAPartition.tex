\documentclass[12pt]{article}
\usepackage{pmmeta}
\pmcanonicalname{PartOfAPartition}
\pmcreated{2013-03-22 15:01:34}
\pmmodified{2013-03-22 15:01:34}
\pmowner{drini}{3}
\pmmodifier{drini}{3}
\pmtitle{part of  a partition}
\pmrecord{5}{36735}
\pmprivacy{1}
\pmauthor{drini}{3}
\pmtype{Definition}
\pmcomment{trigger rebuild}
\pmclassification{msc}{11P99}
\pmclassification{msc}{05A17}
\pmrelated{IntegerPartition}
\pmdefines{length}

\endmetadata

\usepackage{graphicx}
%%%\usepackage{xypic} 
\usepackage{bbm}
\newcommand{\Z}{\mathbbmss{Z}}
\newcommand{\C}{\mathbbmss{C}}
\newcommand{\R}{\mathbbmss{R}}
\newcommand{\Q}{\mathbbmss{Q}}
\newcommand{\mathbb}[1]{\mathbbmss{#1}}
\newcommand{\figura}[1]{\begin{center}\includegraphics{#1}\end{center}}
\newcommand{\figuraex}[2]{\begin{center}\includegraphics[#2]{#1}\end{center}}
\newtheorem{dfn}{Definition}
\begin{document}
If $\lambda=(\lambda_1,\lambda_2,\ldots,\lambda_k)$ is an integer partition, then each $\lambda_j$ is a \emph{part} of $\lambda$. The \emph{length} of $\lambda$ is defined as the number of its parts.
If $m_j$ is the number of parts equal to $j$, then the partition $\lambda$ is also written as $\lambda = (1^{m_1},2^{m_2},3^{m_3},\ldots)$.

For example, if $\lambda=(5,4,4,4,3,3,3,3,3,1,1)$ then we also write $\lambda=(1^2,3^5,4^3,5^1)$.
%%%%%
%%%%%
\end{document}
