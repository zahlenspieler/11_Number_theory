\documentclass[12pt]{article}
\usepackage{pmmeta}
\pmcanonicalname{ProthNumber}
\pmcreated{2013-03-22 17:21:09}
\pmmodified{2013-03-22 17:21:09}
\pmowner{PrimeFan}{13766}
\pmmodifier{PrimeFan}{13766}
\pmtitle{Proth number}
\pmrecord{4}{39710}
\pmprivacy{1}
\pmauthor{PrimeFan}{13766}
\pmtype{Definition}
\pmcomment{trigger rebuild}
\pmclassification{msc}{11A51}

\endmetadata

% this is the default PlanetMath preamble.  as your knowledge
% of TeX increases, you will probably want to edit this, but
% it should be fine as is for beginners.

% almost certainly you want these
\usepackage{amssymb}
\usepackage{amsmath}
\usepackage{amsfonts}

% used for TeXing text within eps files
%\usepackage{psfrag}
% need this for including graphics (\includegraphics)
%\usepackage{graphicx}
% for neatly defining theorems and propositions
%\usepackage{amsthm}
% making logically defined graphics
%%%\usepackage{xypic}

% there are many more packages, add them here as you need them

% define commands here

\begin{document}
A {\em Proth number} $p$ is a number of the form $k2^m + 1$, where $k$ is odd (e.g., $k = 2n + 1$ with an integer $n > (-1)$), $m$ is also an integer and $2^m > k$. The first few Proth numbers are 3, 5, 9, 13, 17, 25, 33, 41, 49, 57, 65, 81, 97, 113, 129, 145, 161, 177, 193, etc. (listed in A080075 of Sloane's OEIS). The Fermat numbers are a subset of the Proth numbers, namely with $k = 1$.
%%%%%
%%%%%
\end{document}
