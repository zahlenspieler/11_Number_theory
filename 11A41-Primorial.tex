\documentclass[12pt]{article}
\usepackage{pmmeta}
\pmcanonicalname{Primorial}
\pmcreated{2013-03-22 16:00:23}
\pmmodified{2013-03-22 16:00:23}
\pmowner{PrimeFan}{13766}
\pmmodifier{PrimeFan}{13766}
\pmtitle{primorial}
\pmrecord{7}{38036}
\pmprivacy{1}
\pmauthor{PrimeFan}{13766}
\pmtype{Definition}
\pmcomment{trigger rebuild}
\pmclassification{msc}{11A41}

% this is the default PlanetMath preamble.  as your knowledge
% of TeX increases, you will probably want to edit this, but
% it should be fine as is for beginners.

% almost certainly you want these
\usepackage{amssymb}
\usepackage{amsmath}
\usepackage{amsfonts}

% used for TeXing text within eps files
%\usepackage{psfrag}
% need this for including graphics (\includegraphics)
%\usepackage{graphicx}
% for neatly defining theorems and propositions
%\usepackage{amsthm}
% making logically defined graphics
%%%\usepackage{xypic}

% there are many more packages, add them here as you need them

% define commands here

\begin{document}
The {\em primorial} of $n$, or $n\#$, is the product of the first $n$ consecutive primes, thus:

$$\prod_{i = 1}^n p_i$$

($p_i$ is the $i$th prime number).

The first few primorials are 2, 6, 30, 210, 2310, 30030, 510510, 9699690, 223092870, 6469693230, 200560490130; these are listed in A002110 of Sloane's OEIS. Sometimes the notation $n\#$ is used to refer to the product of all primes $p < \pi(n)$, where $\pi$ is the prime counting function (so then $4\# = 6$ rather than 210).

Primorials are used in the classic proof that there are infinitely many primes: assuming that there are exactly $n$ primes and no more, $n\# + 1$ is a number that is not divisible by any of the existing primes, but if that is a prime then it contradicts the initial assumption.

If, in reckoning the sieve of Eratosthenes, one strikes out again numbers that have already been struck off, the sequence of the smallest number struck off $n$ times is precisely the sequence of the primorials.

Any highly composite number (with the exception of 1) can be expressed as a product of primorials in at least one way.
%%%%%
%%%%%
\end{document}
