\documentclass[12pt]{article}
\usepackage{pmmeta}
\pmcanonicalname{PerfectPower}
\pmcreated{2013-03-22 19:15:51}
\pmmodified{2013-03-22 19:15:51}
\pmowner{pahio}{2872}
\pmmodifier{pahio}{2872}
\pmtitle{perfect power}
\pmrecord{11}{42194}
\pmprivacy{1}
\pmauthor{pahio}{2872}
\pmtype{Definition}
\pmcomment{trigger rebuild}
\pmclassification{msc}{11D61}
\pmclassification{msc}{11D45}
\pmclassification{msc}{11B83}
\pmsynonym{Pillai's conjecture}{PerfectPower}
\pmrelated{PerfectSquare}
\pmrelated{LimitInferior}
\pmrelated{DoubleSeries}
\pmrelated{SolutionsOfXyYx}
\pmrelated{MarshallHallsConjecture}

% this is the default PlanetMath preamble.  as your knowledge
% of TeX increases, you will probably want to edit this, but
% it should be fine as is for beginners.

% almost certainly you want these
\usepackage{amssymb}
\usepackage{amsmath}
\usepackage{amsfonts}

% used for TeXing text within eps files
%\usepackage{psfrag}
% need this for including graphics (\includegraphics)
%\usepackage{graphicx}
% for neatly defining theorems and propositions
 \usepackage{amsthm}
% making logically defined graphics
%%%\usepackage{xypic}

% there are many more packages, add them here as you need them

% define commands here

\theoremstyle{definition}
\newtheorem*{thmplain}{Theorem}

\begin{document}
\PMlinkescapeword{conjecture}

The power $m^n$ is called a \emph{perfect power} if $m$ and $n$ are integers both greater than 1.  The perfect powers form the ascending order sequence (cf. Sloane's \PMlinkexternal{A001597}{http://oeis.org/classic/A001597})
\begin{align}
4,\,8,\,9,\,16,\,25,\,27,\,32,\,36,\,49,\,64,\,81,\,100,\,121,\,125,\,\ldots,
\end{align}
i.e.
$$2^2,\;2^3,\,3^2,\;2^4 = 4^2,\;5^2,\;3^3,\;2^5,\,6^2,\;7^2,\;2^6 = 4^3 = 8^2,\;3^4 = 9^2,\;10^2,\;11^2,\;5^3,\;\ldots$$
S. S. Pillai has conjectured in 1945, that if the $i^{\mathrm{th}}$ member of the sequence (1) is denoted by $a_i$, then 
\begin{align}
\liminf_{i\to\infty}(a_{i+1}\!-\!a_i) \;=\; \infty.
\end{align}
This does not necessarily \PMlinkescapetext{mean} that one had\, $\lim_{i\to\infty}(a_{i+1}\!-\!a_i) = \infty$,\, since there may always exist little differences $a_{i+1}\!-\!a_i$ arbitrarily far from the begin of the sequence (1).\\

The equation (2) is \PMlinkname{equivalent}{Equivalent3} to the

\textbf{Pillai's conjecture.}\, For any \PMlinkescapetext{fixed} positive integer $k$, the Diophantine equation
                                        $$x^m\!-\!y^n \;=\; k$$
has only a finite number of  solutions \,$(x,\,y,\,m,\,n)$\, where the integers $x,\,y,\,m,\,n$ all are greater than 1. \\

Pillai's conjecture generalises the Catalan's conjecture ($k = 1$) in which the number of solutions is 1.\\


The series formed by the inverse numbers of the perfect powers converges absolutely, and its sum may be calculated easily:
\begin{align*}
\sum_{m,n=2}^\infty\frac{1}{m^n} &\;=\; \sum_{m=2}^\infty\sum_{n=2}^\infty\frac{1}{m^n}\\
&\;=\; \sum_{m=2}^\infty\sum_{n=2}^\infty\frac{1}{m^2}\!\cdot\!\frac{1}{m^{n-2}}\\
&\;=\; \sum_{m=2}^\infty\frac{1}{m^2}\sum_{n=2}^\infty\frac{1}{m^{n-2}}\\
&\;=\; \sum_{m=2}^\infty\frac{1}{m^2}\sum_{n=0}^\infty\left(\frac{1}{m}\right)^{\!n}\\
&\;=\; \sum_{m=2}^\infty\frac{1}{m^2}\!\cdot\!\frac{1}{1\!-\!\frac{1}{m}}\\
&\;=\; \sum_{m=2}^\infty\frac{1}{m(m\!-\!1)}\\
&\;=\; \sum_{m=2}^\infty\left(\frac{1}{m\!-\!1}-\frac{1}{m}\right)
\end{align*}
The sum of this \PMlinkname{telescoping series}{TelescopingSum} is equal to 1.



\begin{thebibliography}{8}
\bibitem{P}{\sc S. S. Pillai}:\, On $a^x\!-\!b^y = c$.\; -- \emph{J. Indian math. Soc.} \textbf{2} (1936).
\end{thebibliography}












%%%%%
%%%%%
\end{document}
