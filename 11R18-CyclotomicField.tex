\documentclass[12pt]{article}
\usepackage{pmmeta}
\pmcanonicalname{CyclotomicField}
\pmcreated{2013-03-22 17:10:25}
\pmmodified{2013-03-22 17:10:25}
\pmowner{Wkbj79}{1863}
\pmmodifier{Wkbj79}{1863}
\pmtitle{cyclotomic field}
\pmrecord{9}{39486}
\pmprivacy{1}
\pmauthor{Wkbj79}{1863}
\pmtype{Definition}
\pmcomment{trigger rebuild}
\pmclassification{msc}{11R18}
\pmclassification{msc}{11-00}
\pmsynonym{cyclotomic number field}{CyclotomicField}
\pmrelated{CyclotomicExtension}
\pmrelated{CyclotomicPolynomial}

\usepackage{amssymb}
\usepackage{amsmath}
\usepackage{amsfonts}
\usepackage{pstricks}
\usepackage{psfrag}
\usepackage{graphicx}
\usepackage{amsthm}
%%\usepackage{xypic}

\begin{document}
A \emph{cyclotomic field} (or \emph{cyclotomic number field}) is a cyclotomic extension of $\mathbb{Q}$.  These are all of the form $\mathbb{Q}(\omega_n)$, where $\omega_n$ is a \PMlinkname{primitive $n$th root of unity}{PrimitiveNthRootOfUnity}.

The ring of integers of a cyclotomic field always has a \PMlinkname{power basis over $\mathbb{Z}$}{PowerBasisOverMathbbZ}.  Specifically, the ring of integers of $\mathbb{Q}(\omega_n)$ is $\mathbb{Z}[\omega_n]$.

Given a \PMlinkescapetext{primitive $n$th root of unity} $\omega_n$, its minimal polynomial over $\mathbb{Q}$ is the cyclotomic polynomial $\Phi_n(x)$.  Thus, $[\mathbb{Q}(\omega_n)\!:\!\mathbb{Q}]=\varphi(n)$, where $\varphi$ denotes the Euler phi function.

If $n$ is odd, then $\mathbb{Q}(\omega_{2n})=\mathbb{Q}(\omega_n)$.  There are many ways to prove this, but the following is a relatively short proof:  Since $\omega_n={\omega_{2n}}^2\in \mathbb{Q}(\omega_{2n})$, we have that $\mathbb{Q}(\omega_n)\subseteq\mathbb{Q}(\omega_{2n})$.  We also have that $[\mathbb{Q}(\omega_{2n})\!:\!\mathbb{Q}]=\varphi(2n)=\varphi(2)\varphi(n)=\varphi(n)=[\mathbb{Q}(\omega_n)\!:\!\mathbb{Q}]$.  Thus, $[\mathbb{Q}(\omega_{2n})\!:\!\mathbb{Q}(\omega_n)]=1$.  It follows that $\mathbb{Q}(\omega_{2n})=\mathbb{Q}(\omega_n)$.

\textbf{Note.}\, If $n$ is a positive integer and $m$ is an integer such that $\gcd(m,n)=1$, then\, $\omega_n$\, and\, ${\omega_n}^m$\, are \PMlinkescapetext{primitive $n$th roots of unity and generate} the same cyclotomic field.
%%%%%
%%%%%
\end{document}
