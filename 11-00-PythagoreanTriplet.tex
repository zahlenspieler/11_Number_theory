\documentclass[12pt]{article}
\usepackage{pmmeta}
\pmcanonicalname{PythagoreanTriplet}
\pmcreated{2013-03-22 11:43:48}
\pmmodified{2013-03-22 11:43:48}
\pmowner{drini}{3}
\pmmodifier{drini}{3}
\pmtitle{Pythagorean triplet}
\pmrecord{23}{30138}
\pmprivacy{1}
\pmauthor{drini}{3}
\pmtype{Definition}
\pmcomment{trigger rebuild}
\pmclassification{msc}{11-00}
\pmclassification{msc}{01A20}
\pmclassification{msc}{11A05}
\pmclassification{msc}{11-03}
\pmclassification{msc}{11-01}
\pmclassification{msc}{51M05}
\pmclassification{msc}{51M04}
\pmclassification{msc}{51-03}
\pmclassification{msc}{51-01}
\pmclassification{msc}{01-01}
\pmclassification{msc}{55-00}
\pmclassification{msc}{55-01}
\pmsynonym{Pythagorean triple}{PythagoreanTriplet}
%\pmkeywords{Triangle}
%\pmkeywords{Pythagoras}
%\pmkeywords{Geometry}
\pmrelated{PythagorasTheorem}
\pmrelated{IncircleRadiusDeterminedByPythagoreanTriple}
\pmrelated{ContraharmonicMeansAndPythagoreanHypotenuses}
\pmrelated{PythagoreanHypotenusesAsContraharmonicMeans}
\pmdefines{seed number}
\pmdefines{primitive Pythagorean triple}
\pmdefines{primitive Pythagorean triplet}

\usepackage{amssymb}
\usepackage{amsmath}
\usepackage{amsfonts}
\usepackage{graphicx}
%%%%%%%\usepackage{xypic}

\begin{document}
A \emph{Pythagorean triplet} is a set $\{a, b, c\}$ of three positive 
integers such that
\[ 
   a^2 + b^2 = c^2.  
\]

That is, $\{a, b, c\}$ is a Pythagorean triplet if there exists a
right triangle whose sides have lengths $a$, $b$, and $c$,
respectively.  For example, $\{3, 4, 5\}$ is a Pythagorean triplet.
Given one Pythagorean triplet $\{a, b, c\}$, we can produce another by
multiplying $a$, $b$, and $c$ by the same factor $k$.  It follows that
there are countably many Pythagorean triplets.

\subsubsection*{Primitive Pythagorean triplets}

A Pythagorean triplet is \emph{primitive} if its elements are
coprimes.  All primitive Pythagorean triplets are given by
\begin{align}
\begin{cases}
a = 2mn,\\
b = m^2\!-\!n^2,\\
c = m^2\!+\!n^2,
\end{cases}
\end{align}

where the \emph{seed numbers} $m$ and $n$ are any two coprime positive
integers, one odd and one even, such tht $m > n$.\, If one presumes of the positive integers $m$ and $n$ only that\, $m > n$, one obtains also many non-primitive triplets, but not e.g. $(6,\,8,\,10)$.\, For getting all, one needs to multiply the right hand sides of (1) by an additional integer parametre $q$.

\textbf{Note 1.}\, Among the primitive Pythagorean triples, the odd cathetus $a$ may attain all odd values except 1 (set e.g.\, $m := n\!+\!1$) and the even cathetus $b$ all values divisible by 4 (set\, $n := 1$).\\

\textbf{Note 2.}\, In the primitive triples, the hypothenuses $c$ are always odd.\, All possible Pythagorean hypotenuses are contraharmonic means of two different integers (and conversely).\\

\textbf{Note 3.}\, N.B. that any triplet (1) is obtained from the square of a Gaussian integer $(m\!+\!in)^2$ as its real part, imaginary part and absolute value.\\

\textbf{Note 4.}\, The equations (1) imply that the sum of a cathetus and the hypotenuse is always a perfect square or a double perfect square.\\

\textbf{Note 5.}\, One can form the sequence (cf. Sloane's \PMlinkexternal{A100686}{http://www.research.att.com/~njas/sequences/?q=A100686&language=english&go=Search})
\[
  1,\,2,\,3,\,4,\,7,\,24,\,527,\,336,\,164833,\,354144,\,...
\]
taking first the seed numbers 1 and 2 which give the legs 3 and 4,
taking these as new seed numbers which give the legs 7 and 24, and
so on.

% related: PythagoreanTriplesAndRationalPointsOnAUnitHyperbola

%%%%%
%%%%%
%%%%%
%%%%%
%%%%%
%%%%%
%%%%%
\end{document}
