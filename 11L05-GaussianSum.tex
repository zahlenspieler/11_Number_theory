\documentclass[12pt]{article}
\usepackage{pmmeta}
\pmcanonicalname{GaussianSum}
\pmcreated{2013-03-22 19:00:48}
\pmmodified{2013-03-22 19:00:48}
\pmowner{pahio}{2872}
\pmmodifier{pahio}{2872}
\pmtitle{Gaussian sum}
\pmrecord{12}{41883}
\pmprivacy{1}
\pmauthor{pahio}{2872}
\pmtype{Definition}
\pmcomment{trigger rebuild}
\pmclassification{msc}{11L05}
\pmrelated{GaussSum}
\pmrelated{RepresentantsOfQuadraticResidues}

\endmetadata

% this is the default PlanetMath preamble.  as your knowledge
% of TeX increases, you will probably want to edit this, but
% it should be fine as is for beginners.

% almost certainly you want these
\usepackage{amssymb}
\usepackage{amsmath}
\usepackage{amsfonts}

% used for TeXing text within eps files
%\usepackage{psfrag}
% need this for including graphics (\includegraphics)
%\usepackage{graphicx}
% for neatly defining theorems and propositions
 \usepackage{amsthm}
% making logically defined graphics
%%%\usepackage{xypic}
\usepackage{pstricks}
\usepackage{pst-plot}

% there are many more packages, add them here as you need them

% define commands here

\theoremstyle{definition}
\newtheorem*{thmplain}{Theorem}

\begin{document}
A \emph{Gaussian sum} is a sum of the form
\begin{align}
S \;:=\; \sum_{j=0}^{n-1}e^{\frac{2\pi i}{n}j^2}
\end{align}
where $n$ is a positive integer.\\

The explicit value of $S$ may be found by using the Cauchy residue theorem for determining the contour integral
$$I \;:=\; \oint_\gamma\frac{e^{\frac{2\pi i}{n}z^2}}{1\!-\!e^{2\pi iz}}\,dz$$
where $\gamma$ goes once anticlockwise round the set of the \PMlinkname{poles}{Pole} $1,\,2,\,\ldots,\,\lfloor\frac{n-1}{2}\rfloor$ of the integrand ($\gamma$ is formed by a rectangle where two segments are replaced by arcs of half-circle).

\begin{center}
\begin{pspicture}(-1,-4)(13,4)
\psaxes[Dx=13,Dy=13]{->}(0,0)(-0.5,-3.7)(12.5,3.7)
\psline[linewidth=0.05,linecolor=blue](0,-0.6)(0,-2.5)(10.5,-2.5)(10.5,-0.6)
\psline[linewidth=0.05,linecolor=blue](0,+0.6)(0,+2.5)(10.5,+2.5)(10.5,+0.6)
\psarc[linewidth=0.05,linecolor=blue](0,0){0.6}{-90}{90}
\psarc[linewidth=0.05,linecolor=blue](10.5,0){0.6}{90}{270}
\psline(10.5,-0.06)(10.5,0.06)
\psdot(0,0)
\psdots[linecolor=red](1.3,0)(2.6,0)(3.9,0)(5.2,0)(6.5,0)
\rput(1.3,-0.3){1}
\rput(2.6,-0.3){2}
\rput(-0.3,0.6){$ri$}
\rput(-0.3,2.5){$ai$}
\rput(-0.4,-0.6){$-ri$}
\rput(-0.4,-2.5){$-ai$}
\rput(11.1,0.6){$\frac{n}{2}\!+\!ri$}
\rput(11.1,2.5){$\frac{n}{2}\!+\!ai$}
\rput(11.1,-0.6){$\frac{n}{2}\!-\!ri$}
\rput(11.1,-2.5){$\frac{n}{2}\!-\!ai$}
\rput(5,2.7){$\gamma$}
\psline{<-}(4.5,2.2)(5.5,2.2)
\rput(5,-3){($0 < r < \frac{1}{2},\;\; a > r$)}
\end{pspicture}
\end{center}
The \PMlinkescapetext{connection} between $S$ and $I$ is
\begin{align*}
S \;=\; -2I+\!
\begin{cases}
1 \qquad \mbox{when}\;\; 2 \nmid n, \\
1\!+\!i^{\,n} \quad \mbox{when}\;\; 2 \mid n.
\end{cases}
\end{align*}
Letting\, $a \to \infty$\, and\, $r \to 0+$,\, one can reduce $S$ to the form
$$S \;=\; 2i[1\!+\!(-i)^n]\!\int_0^\infty\!e^{-\frac{2\pi i}{n}t^2}\,dt$$
and using the \PMlinkid{substitution}{11373} \,$t = u\sqrt{n}$ evaluate this integral.\, The result may be written
\begin{align*}
S \;=\; \frac{1\!+\!i}{2}[1\!+\!(-i)^n]\sqrt{n} \;=\;
\begin{cases}
(1\!+\!i)\sqrt{n} \quad \mbox{when}\;\; n \equiv 0 \pmod{4}, \\
\sqrt{n} \quad \mbox{when}\;\; n \equiv 1 \pmod{4}, \\
0 \quad \mbox{when}\;\; n \equiv 2 \pmod{4}, \\
i\sqrt{n} \quad \mbox{when}\;\; n \equiv 3 \pmod{4}.
\end{cases}
\end{align*}

%%%%%
%%%%%
\end{document}
