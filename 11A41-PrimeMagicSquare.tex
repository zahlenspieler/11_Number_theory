\documentclass[12pt]{article}
\usepackage{pmmeta}
\pmcanonicalname{PrimeMagicSquare}
\pmcreated{2013-03-22 16:43:28}
\pmmodified{2013-03-22 16:43:28}
\pmowner{PrimeFan}{13766}
\pmmodifier{PrimeFan}{13766}
\pmtitle{prime magic square}
\pmrecord{5}{38945}
\pmprivacy{1}
\pmauthor{PrimeFan}{13766}
\pmtype{Definition}
\pmcomment{trigger rebuild}
\pmclassification{msc}{11A41}

% this is the default PlanetMath preamble.  as your knowledge
% of TeX increases, you will probably want to edit this, but
% it should be fine as is for beginners.

% almost certainly you want these
\usepackage{amssymb}
\usepackage{amsmath}
\usepackage{amsfonts}

% used for TeXing text within eps files
%\usepackage{psfrag}
% need this for including graphics (\includegraphics)
%\usepackage{graphicx}
% for neatly defining theorems and propositions
%\usepackage{amsthm}
% making logically defined graphics
%%%\usepackage{xypic}

% there are many more packages, add them here as you need them

% define commands here

\begin{document}
A {\em prime magic square} is a magic square consisting only of prime numbers (the magic constant may be a composite number, especially if the sides are of even length). The primes don't have to be consecutive, though it is sometimes convenient to consider 1 a prime number for the purpose of constructing these squares.

The smallest prime magic square with the smallest possible magic constant (111) is 

$$\begin{bmatrix}
67 & 1 & 43 \\
13 & 37 & 61 \\
31 & 73 & 7 \\
\end{bmatrix}$$

first published by Henry Ernest Dudeney in 1917.

Rudolf Ondrejka constructed this magic square using only Chen primes:

$$\begin{bmatrix}
17 & 89 & 71 \\
113 & 59 & 5 \\
47 & 29 & 101 \\
\end{bmatrix}$$

The magic constant is 177.

\begin{thebibliography}{1}
\bibitem{ed} Dudeney, E. {\it Amusements in Mathematics} New York: Dover (1970): Problem 408
\end{thebibliography}
%%%%%
%%%%%
\end{document}
