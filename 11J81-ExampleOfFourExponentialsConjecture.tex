\documentclass[12pt]{article}
\usepackage{pmmeta}
\pmcanonicalname{ExampleOfFourExponentialsConjecture}
\pmcreated{2013-03-22 14:09:09}
\pmmodified{2013-03-22 14:09:09}
\pmowner{archibal}{4430}
\pmmodifier{archibal}{4430}
\pmtitle{example of four exponentials conjecture}
\pmrecord{6}{35570}
\pmprivacy{1}
\pmauthor{archibal}{4430}
\pmtype{Example}
\pmcomment{trigger rebuild}
\pmclassification{msc}{11J81}

\endmetadata

% this is the default PlanetMath preamble.  as your knowledge
% of TeX increases, you will probably want to edit this, but
% it should be fine as is for beginners.

% almost certainly you want these
\usepackage{amssymb}
\usepackage{amsmath}
\usepackage{amsfonts}

% used for TeXing text within eps files
%\usepackage{psfrag}
% need this for including graphics (\includegraphics)
%\usepackage{graphicx}
% for neatly defining theorems and propositions
%\usepackage{amsthm}
% making logically defined graphics
%%%\usepackage{xypic}

% there are many more packages, add them here as you need them

% define commands here
\begin{document}
\PMlinkescapeword{states}
Taking $x_1=i\pi$, $x_2=i\pi\sqrt{2}$, $y_1=1$, $y_2=\sqrt{2}$, we see that this conjecture implies that one of $e^{i\pi}$, $e^{i\pi\sqrt{2}}$, or $e^{i2\pi}$ is transcendental.  Since the first is $-1$ and the last is $1$, the conjecture states that second must be transcendental, that is, $e^{i\pi\sqrt{2}}$ is (conjecturally) transcendental.

In this particular case, the result is known already, so the conjecture is verified.  Using Gelfond's theorem, take $\alpha=e^{i\pi}$ and $\beta=\sqrt{2}$ and it follows that $\alpha^\beta$ is transcendental.
%%%%%
%%%%%
\end{document}
