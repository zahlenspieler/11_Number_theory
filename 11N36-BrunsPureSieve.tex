\documentclass[12pt]{article}
\usepackage{pmmeta}
\pmcanonicalname{BrunsPureSieve}
\pmcreated{2013-03-22 14:10:54}
\pmmodified{2013-03-22 14:10:54}
\pmowner{bbukh}{348}
\pmmodifier{bbukh}{348}
\pmtitle{Brun's pure sieve}
\pmrecord{11}{35608}
\pmprivacy{1}
\pmauthor{bbukh}{348}
\pmtype{Topic}
\pmcomment{trigger rebuild}
\pmclassification{msc}{11N36}
\pmclassification{msc}{11N35}
%\pmkeywords{inclusion-exclusion}
%\pmkeywords{twin primes}
%\pmkeywords{Bonferroni inequalities}
\pmrelated{PrincipleOfInclusionExclusion}
\pmrelated{SieveOfEratosthenes2}
\pmrelated{BrunsConstant}
\pmrelated{BonferroniInequalities}
\pmdefines{combinatorial sieve}

\usepackage{amssymb}
\usepackage{amsmath}
\usepackage{amsfonts}
\usepackage{amsthm}

\newcommand*{\abs}[1]{\left\lvert #1\right\rvert}
\newtheorem*{theorem}{Theorem}

\makeatletter
\@ifundefined{bibname}{}{\renewcommand{\bibname}{References}}
\makeatother
\begin{document}
\PMlinkescapeword{information}
\PMlinkescapeword{order}
\PMlinkescapeword{name}
\PMlinkescapeword{similar}
\PMlinkescapeword{bounded}
\PMlinkescapeword{observation}

In the first quarter of the twentieth century Viggo Brun developed
an extension of the sieve of Eratosthenes that yielded good
estimates on the number of elements of a set $\mathcal{A}$ that
are not \PMlinkname{divisible}{Divisibility} by any of the primes $p_1,\dotsc,p_k$ provided
only that $\mathcal{A}$ is ``sufficiently regularly distributed''
modulo these primes. That allowed him to prove that the sum of
reciprocals of twin primes converges (the limit is now known as
Brun's constant), and that every sufficiently large even number is
a sum of two numbers each having at most $9$ prime factors. In
what follows we describe the simplest form of Brun's sieve, known
as Brun's pure sieve.

The \PMlinkname{sieve of Eratosthenes}{SieveOfEratosthenes2} is
based on the principle of inclusion-exclusion in the form
\begin{align}\notag
\sum_{n\in\mathcal{A}}\sum_{\substack{p_1\nmid n\\\dots\\p_k\nmid
n}}1&=\sum_{n\in\mathcal{A}}\sum_{(n,p_1\dotsb p_k)=1}
1=\sum_{n\in\mathcal{A}} \sum_{d\mid (n,p_1\dotsb p_k)}\mu(d)\\
\notag&=\sum_{d\mid p_1\dotsb
p_k}\mu(d)\sum_{\substack{n\mid\mathcal{A}\\d\mid n}}1=\sum_{d\mid
p_1\dotsb p_k}\mu(d)A_d\\\label{eq:incl_ex} &=A_1-\sum_{p\mid
p_1\dotsb p_k} A_p+\sum_{pq\mid p_1\dotsb p_k} A_{pq}+\dotsb.
\end{align}
Brun's sieve is based on the observation that the partial sums 
of~\eqref{eq:incl_ex} alternatively overcount and
undercount the number of elements of $\mathcal{A}$ counted by the
left side, that is,
\begin{equation}\label{eq:bonf_ineq}
\sum_{\substack{d\mid p_1\dotsb p_k\\\nu(d)\leq
2h-1}}\mu(d)A_d\leq \sum_{n\in\mathcal{A}}\sum_{\substack{p_1\nmid
n\\\dots\\p_k\nmid n}}1\leq \sum_{\substack{d\mid p_1\dotsb
p_k\\\nu(d)\leq 2h}} \mu(d)A_d
\end{equation}
where $\nu(d)$ denotes the number of \PMlinkname{prime}{Prime} factors of $d$. In
applications $A_d$ can usually be approximated by a multiple of a
multiplicative function, that is,
\begin{equation*}
A_d=X w(d)/d+R_d
\end{equation*}
where $w(d)$ is some multiplicative function of $d$, and $R_d$ is
small compared to $X w(d)/d$. Then the estimate
in~\eqref{eq:bonf_ineq} takes the form
\begin{align}\notag
\sum_{\substack{d\mid p_1\dotsb p_k\\\nu(d)\leq t}}
\mu(d)A_d&=\sum_{\substack{d\mid p_1\dotsb p_k\\\nu(d)\leq t}}
\mu(d)(X w(d)/d+R_d)
\\\label{eq:intmd}
&=X\sum_{\substack{d\mid p_1\dotsb p_k\\\nu(d)\leq t}} \mu(d)
\frac{w(d)}{d}+O\left(\sum_{\substack{d\mid p_1\dotsb
p_k\\\nu(d)\leq t}} R_d
\right)
\end{align}
If the truncated sum is a good approximation to the full sum, then
this formula is an improvement over the 
\PMlinkname{sieve of Eratosthenes}{SieveOfEratosthenes2} because the sum over error
terms $R_d$ is shorter.

Since $u^{\nu(d)-t}$ is greater than $1$ for every $u>1$ and every
$d$ such that $\nu(d)>t$, we have that
\begin{align*}
\sum_{\substack{d\mid p_1\dotsb p_k\\\nu(d)\leq t}} \mu(d)
\frac{w(d)}{d}&=\sum_{d\mid p_1\dotsb p_k} \mu(d)
\frac{w(d)}{d}+O\left(\sum_{\substack{d\mid p_1\dotsb p_k\\\nu(d)>
t}} \frac{w(d)}{d}\right)\\
&=\sum_{d\mid p_1\dotsb p_k} \mu(d)
\frac{w(d)}{d}+O\left(\sum_{d\mid p_1\dotsb p_k}
\frac{w(d)}{d}u^{\nu(d)-t}\right)\\
&=\sum_{d\mid p_1\dotsb p_k} \mu(d)
\frac{w(d)}{d}+O\left(u^{-t}\sum_{d\mid p_1\dotsb p_k}
\frac{w(d)}{d}u^{\nu(d)}\right)\\
\intertext{and since the sum of a multiplicative function can be
written as an Euler product, we get} &=\prod_{p\in \mathcal{P}}
\left(1-\frac{w(p)}{p}\right)+O\left(u^{-t}\prod_{p\in
\mathcal{P}} \left(1+u\frac{w(p)}{p}\right) \right)\\
&=\prod_{p\in \mathcal{P}}
\left(1-\frac{w(p)}{p}\right)+O\left(u^{-t}\prod_{p\in
\mathcal{P}} \left(1+\frac{w(p)}{p}\right)^u \right)\\
\intertext{to minimize the error term we choose
$u=t/\sum_{p\in\mathcal{P}} \log(1+\frac{w(p)}{p})$, and get}
&=\prod_{p\in \mathcal{P}}
\left(1-\frac{w(p)}{p}\right)+O\left(\frac{1}{t}\sum_{p\in\mathcal{P}}
\frac{w(p)}{p} \right)^t
\end{align*}
Combining this with~\eqref{eq:intmd} and~\eqref{eq:bonf_ineq} we
obtain
\begin{equation}
\sum_{n\in\mathcal{A}}\sum_{\substack{p_1\nmid n\\\dots\\p_k\nmid
n}}1=X\prod_{p\in \mathcal{P}}
\left(1-\frac{w(p)}{p}\right)+X\cdot
O\left(\frac{1}{t}\sum_{p\in\mathcal{P}} \frac{w(p)}{p}
\right)^t+O\left(\sum_{\substack{d\mid p_1\dotsb p_k\\\nu(d)\leq
t}} R_d\right)\label{eq:brunpure}
\end{equation}
provided that $u=t/\sum_{p\in\mathcal{P}}
\log(1+\frac{w(p)}{p})\geq 1$.

\emph{Example. (Upper bound on the number of primes.)} If we set
$\mathcal{P}$ to be all the primes less $y$, and
$\mathcal{A}=\{1,2,\dotsc,x\}$, then the right hand side
of~\eqref{eq:brunpure} provides an upper bound for the number of
primes between $y$ and $x$.

We have that $w(d)=1$ and $R_d\leq 1$. The second error term
in~\eqref{eq:brunpure} has at most $y^t$ summands, and the first
error term is bounded by $(\frac{c}{t}\log\log y)^t$ by Merten's
theorem. Thus, the estimate~\eqref{eq:brunpure} becomes
\begin{equation*}
\pi(x)-\pi(y)\leq x\frac{e^{-\gamma}+o(1)}{\log y}+x\cdot
O\left(\frac{1}{t}\log\log y\right)^t+O(y^t)
\end{equation*}
In order to squeeze out the best upper bound on the number of primes
not exceeding $x$ we have to minimize the right side of the
inequality above. The nearly optimal choice is $t=c\log \log x$,
and $y=x^{1/2c\log\log x}$ for a sufficiently large constant $c$.
With this choice we obtain
\begin{equation*}
\pi(x)\leq O\left(\frac{x\log\log x}{\log x}\right).
\end{equation*}
This inequality is stronger than the one obtained by an
application of the \PMlinkname{sieve of Eratosthenes}{SieveOfEratosthenes2}.

\emph{Example. (Upper bound on the number of twin primes.)} If
$p<n$ and $p\mid n(n+2)$, then the integers $n$ and $n+2$ cannot
both be primes. Let $\mathcal{P}$ be as above, and set
$\mathcal{A}=\{ n(n+1) : n\in [y..x]\}$. Then $w(2)=1$, and
$w(p)=2$ if $p>2$. The remainder $R_d$ does not exceed
$2^{\nu(d)}$. Like above we obtain
\begin{equation*}
\pi_2(x)-\pi_2(y)\leq x\frac{c}{(\log y)^2}+x\cdot
O\left(\frac{1}{t}\log\log y\right)^t+O((2y)^t)
\end{equation*}
where $\pi_2(x)$ denotes the number of twin primes not exceeding
$x$. Upon setting $t=c\log \log x$, and $y=x^{1/2c\log\log x}$ we
obtain
\begin{equation*}
\pi_2(x)\leq O\left(\frac{x(\log\log x)^2}{(\log x)^2}\right).
\end{equation*}
This is the original result for which Viggo Brun developed the
sieve now bearing his name. This result can be put in following
striking form:
\begin{theorem}The sum
\begin{equation*}
\sum_{p,p+2\text{ primes}} \left(\frac{1}{p}+\frac{1}{p+2}\right)
\end{equation*}
converges.
\end{theorem}

\section*{General combinatorial sieve}
The inequality~\eqref{eq:bonf_ineq} was based on the observation
that
\begin{equation}\label{eq:gen_ineq}
\sum_{d\mid n}\mu(d)\chi_-(d)\leq \sum_{d\mid n} \mu(d)\leq
\sum_{d\mid n}\mu(d)\chi_+(d)
\end{equation}
where $\chi_-$ is the characteristic function of the set of
integers with no more than $2h-1$ prime factors, and $\chi_+$ is
the characteristic function of the set of integers with no more
than $2h$ prime factors. It is possible to choose different
functions $\chi_-$ and $\chi_+$ that satisfy the
inequality~\eqref{eq:gen_ineq} to obtain bounds similar
to~\eqref{eq:brunpure}. The problem of optimal choice of these
functions in general is very hard. For more detailed information
on Brun's sieve and sieves in general one should consult the
monographs
\cite{cite:greaves_sieves,cite:halberstam_richert_sieves,cite:hooley_appsievemethods}.

\begin{thebibliography}{1}

\bibitem{cite:greaves_sieves}
George Greaves.
\newblock {\em Sieves in number theory}.
\newblock Springer, 2001.
\newblock \PMlinkexternal{Zbl 1003.11044}{http://www.emis.de/cgi-bin/zmen/ZMATH/en/quick.html?type=html&an=1003.11044}.

\bibitem{cite:halberstam_richert_sieves}
H.~Halberstam and H.-E. Richert.
\newblock {\em Sieve methods}, volume~4 of {\em London Mathematical Society
  Monographs}.
\newblock 1974.
\newblock \PMlinkexternal{Zbl 0298.10026}{http://www.emis.de/cgi-bin/zmen/ZMATH/en/quick.html?type=html&an=0298.10026}.

\bibitem{cite:hooley_appsievemethods}
C.~Hooley.
\newblock {\em Application of sieve methods to the theory of numbers},
  volume~70 of {\em Cambridge Tracts in Mathematics}.
\newblock Cambridge Univ. Press, 1976.
\newblock \PMlinkexternal{Zbl 0327.10044}{http://www.emis.de/cgi-bin/zmen/ZMATH/en/quick.html?type=html&an=0327.10044}.

\bibitem{cite:tenenbaum_analytic_prob_numth}
G{\'e}rald Tenenbaum.
\newblock {\em Introduction to Analytic and Probabilistic Number Theory},
  volume~46 of {\em Cambridge Studies in Advanced Mathematics}.
\newblock Cambridge Univ. Press.
\newblock \PMlinkexternal{Zbl 0831.11001}{http://www.emis.de/cgi-bin/zmen/ZMATH/en/quick.html?type=html&an=0831.11001}.

\end{thebibliography}
%%%%%
%%%%%
\end{document}
