\documentclass[12pt]{article}
\usepackage{pmmeta}
\pmcanonicalname{BernoulliPolynomialsAndNumbers}
\pmcreated{2013-03-22 17:58:43}
\pmmodified{2013-03-22 17:58:43}
\pmowner{pahio}{2872}
\pmmodifier{pahio}{2872}
\pmtitle{Bernoulli polynomials and numbers}
\pmrecord{13}{40487}
\pmprivacy{1}
\pmauthor{pahio}{2872}
\pmtype{Definition}
\pmcomment{trigger rebuild}
\pmclassification{msc}{11B68}
\pmsynonym{Bernoulli numbers and polynomials}{BernoulliPolynomialsAndNumbers}
\pmrelated{BernoulliNumber}
\pmrelated{CoefficientsOfBernoulliPolynomials}
\pmrelated{TaylorSeriesViaDivision}
\pmrelated{ReferenceRelatedToBernoulliPolynomialsAndNumbers}
\pmrelated{EulerPolynomial}
\pmdefines{Bernoulli polynomial}
\pmdefines{Bernoulli number}

\endmetadata

% this is the default PlanetMath preamble.  as your knowledge
% of TeX increases, you will probably want to edit this, but
% it should be fine as is for beginners.

% almost certainly you want these
\usepackage{amssymb}
\usepackage{amsmath}
\usepackage{amsfonts}

% used for TeXing text within eps files
%\usepackage{psfrag}
% need this for including graphics (\includegraphics)
%\usepackage{graphicx}
% for neatly defining theorems and propositions
% \usepackage[utf8]{inputenc}
 \usepackage{amsthm}
 \usepackage[T2A]{fontenc}
 \usepackage[russian, english]{babel}

% making logically defined graphics
%%%\usepackage{xypic}

% there are many more packages, add them here as you need them

% define commands here

\theoremstyle{definition}
\newtheorem*{thmplain}{Theorem}
\begin{document}
For\, $n = 0,\,1,\,2,\,\ldots$,\, the {\em Bernoulli polynomial} may be defined as the uniquely determined polynomial $b_n(x)$ satisfying
\begin{align}
\int_x^{x+1}\!b_n(t)\,dt \;=\; x^n.
\end{align}

The constant term of $b_n(x)$ is the $n^{\mathrm{th}}$ {\em Bernoulli number} $B_n$.

The Bernoulli polynomial is often denoted also $B_n(x)$.\\


The uniqueness of the solution $b_n(x)$ in (1) is justificated by the 

\textbf{Lemma.}\, For any polynomial $f(x)$, there exists a unique polynomial $g(x)$ with the same degree satisfying
\begin{align}
\int_x^{x+1}\!g(t)\,dt \;=\; f(x).
\end{align}

{\em Proof.}\, For every\, $n = 0,\,1,\,2,\,\ldots$,\, the polynomial
$$g_n(x) \;=:\; \int_x^{x+1}\!t^n\,dt \;=\; \frac{(x\!+\!1)^{n+1}-x^{n+1}}{n\!+\!1}$$
is monic and its degree is $n$.\, If the coefficient of $x^n$ in $f(x)$ is $a_0$, then the difference $f(x)\!-\!a_0g_n(x)$ is a polynomial of degree $n\!-\!1$.\, Correspondingly we obtain $f(x)-a_0g_n(x)-a_1g_{n-1}(x)$ having the degree $n\!-\!2$ and so on.\, Finally we see that 
$$f(x)-a_0g_n(x)-a_1g_{n-1}(x)-\ldots-a_ng_0(x)$$
must be the zero polynomial.\, Therefore
\begin{align*}
f(x) & \;=\; a_0g_n(x)+a_1g_{n-1}(x)+\ldots+a_ng_0(x)\\
     & \;=\; \sum_{i=0}^na_ig_{n-i}(x)\\
     & \;=\; \sum_{i=0}^na_i\int_x^{x+1}t^{n-i}\,dt\\
     & \;=\; \int_x^{x+1}\sum_{i=0}^na_it^{n-i}\,dt
\end{align*}
whence we have\, $\displaystyle g(x) = \sum_{i=0}^na_ix^{n-i}$.\\

The proof implies also that the coefficients of $g(x)$ are rational, if the coefficients of $f(x)$ are such.\, So we know that all Bernoulli polynomials have only rational coefficients.\\


The relation (1) implies easily, that the Bernoulli polynomials form an Appell sequence.


\begin{thebibliography}{7}
\bibitem{MMP} \CYRM. \CYRM. \CYRP\cyro\cyrs\cyrt\cyrn\cyri\cyrk\cyro\cyrv: 
{\em \CYRV\cyrv\cyre\cyrd\cyre\cyrn\cyri\cyre\, \cyrv\, \cyrt\cyre\cyro\cyrr\cyri\cyryu\, \cyra\cyrl\cyrg\cyre\cyrb\cyrr\cyra\cyri\cyrch\cyre\cyrs\cyrk\cyri\cyrh \,
\cyrch\cyri\cyrs\cyre\cyrl}. \,\CYRI\cyrz\cyrd\cyra\cyrt\cyre\cyrl\cyrsftsn\cyrs\cyrt\cyrv\cyro \,
``\CYRN\cyra\cyru\cyrk\cyra''. \CYRM\cyro\cyrs\cyrk\cyrv\cyra \,(1982).
\end{thebibliography}

English translation:

M. M. Postnikov: \emph{Introduction to algebraic number theory}. Science Publs (``Nauka'').
Moscow (1982).
%%%%%
%%%%%
\end{document}
