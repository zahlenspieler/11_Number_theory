\documentclass[12pt]{article}
\usepackage{pmmeta}
\pmcanonicalname{LemmaForImaginaryQuadraticFields}
\pmcreated{2013-03-22 18:31:23}
\pmmodified{2013-03-22 18:31:23}
\pmowner{pahio}{2872}
\pmmodifier{pahio}{2872}
\pmtitle{lemma for imaginary quadratic fields}
\pmrecord{7}{41216}
\pmprivacy{1}
\pmauthor{pahio}{2872}
\pmtype{Theorem}
\pmcomment{trigger rebuild}
\pmclassification{msc}{11R11}
\pmclassification{msc}{11R04}
\pmrelated{ListOfAllImaginaryQuadraticPIDs}
\pmrelated{ClassNumbersOfImaginaryQuadraticFields}

\endmetadata

% this is the default PlanetMath preamble.  as your knowledge
% of TeX increases, you will probably want to edit this, but
% it should be fine as is for beginners.

% almost certainly you want these
\usepackage{amssymb}
\usepackage{amsmath}
\usepackage{amsfonts}

% used for TeXing text within eps files
%\usepackage{psfrag}
% need this for including graphics (\includegraphics)
%\usepackage{graphicx}
% for neatly defining theorems and propositions
 \usepackage{amsthm}
% making logically defined graphics
%%%\usepackage{xypic}

% there are many more packages, add them here as you need them

% define commands here

\theoremstyle{definition}
\newtheorem*{thmplain}{Theorem}

\begin{document}
\PMlinkescapeword{irreducible}
For determining the imaginary quadratic fields whose ring of integers has unique factorization, one can use the following

\textbf{Lemma.}\, Let $d$ be a negative integer with\, $d \equiv 1 \pmod{4}$,\, $p$ the greatest odd \PMlinkname{irreducible}{Irreducible} integer with\, $p \leqq \sqrt{\frac{1}{3}|d|}$\, and\, $q = \frac{1}{4}(1\!-\!d)$.\, In the imaginary quadratic field 
$\mathbb{Q}(\sqrt{d})$, the factorization of integers is \PMlinkname{unique}{Ufd} if and only if the integers
\begin{align}
t^2\!-\!t\!+\!q \quad\; \left(t = 1,\,2,\,\ldots,\, \frac{p\!+\!1}{2}\right)
\end{align}
are \PMlinkname{irreducible}{Irreducible} in the field of the rational numbers.\\


The lemma yields the below table:
\begin{center}
\begin{tabular}{||c|c|c|c|c||}
\hline\hline
$q$ & $d = 1-4q$ & $p$ & $\frac{1}{2}(p\!+\!1)$ & the numbers (1)\\
\hline\hline
$1$ & $-3$ & $1$ & $1$ & 1\\
\hline
$2$ & $-7$ & $1$ & $1$ & 2\\
\hline
$3$ & $-11$ & $1$ & $1$ & 3\\
\hline
$5$ & $-19$ & $1$ & $1$ & 5\\
\hline
$11$ & $-43$ & $3$ & $2$ & 11, 13\\
\hline
$17$ & $-67$ & $3$ & $2$ & 17, 19\\
\hline
$41$ & $-163$ & $7$ & $4$ & 41, 43, 47, 53\\
\hline
\end{tabular}
\end{center}

\begin{thebibliography}{9}
\bibitem{K.V.} {\sc K. V\"ais\"al\"a}: {\em Lukuteorian ja korkeamman algebran alkeet}.\, Tiedekirjasto No. 17.\quad  Kustannusosakeyhti\"o Otava, Helsinki (1950).
\end{thebibliography}
%%%%%
%%%%%
\end{document}
