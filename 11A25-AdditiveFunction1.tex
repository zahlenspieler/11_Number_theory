\documentclass[12pt]{article}
\usepackage{pmmeta}
\pmcanonicalname{AdditiveFunction1}
\pmcreated{2013-03-22 16:07:03}
\pmmodified{2013-03-22 16:07:03}
\pmowner{Wkbj79}{1863}
\pmmodifier{Wkbj79}{1863}
\pmtitle{additive function}
\pmrecord{12}{38184}
\pmprivacy{1}
\pmauthor{Wkbj79}{1863}
\pmtype{Definition}
\pmcomment{trigger rebuild}
\pmclassification{msc}{11A25}
%\pmkeywords{number theory}
%\pmkeywords{arithmetic function}
\pmrelated{MultiplicativeFunction}
\pmdefines{additive}
\pmdefines{completely additive}
\pmdefines{completely additive function}

\endmetadata

% this is the default PlanetMath preamble.  as your knowledge
% of TeX increases, you will probably want to edit this, but
% it should be fine as is for beginners.

% almost certainly you want these
\usepackage{amssymb}
\usepackage{amsmath}
\usepackage{amsfonts}

% used for TeXing text within eps files
%\usepackage{psfrag}
% need this for including graphics (\includegraphics)
%\usepackage{graphicx}
% for neatly defining theorems and propositions
%\usepackage{amsthm}
% making logically defined graphics
%%%\usepackage{xypic}

% there are many more packages, add them here as you need them

% define commands here

\begin{document}
In number theory, an {\sl additive function\/} is an arithmetic function $f \colon \mathbb{N} \to \mathbb{C}$ with the property that $f(1)=0$ and, for all $a,b \in \mathbb{N}$ with $\gcd(a,b)=1$, $f(ab)=f(a)+f(b)$.

An arithmetic function $f$ is said to be {\sl completely additive\/} if $f(1)=0$ and $f(ab)=f(a)+f(b)$ holds for {\sl all\/} positive integers $a$ and $b$, \PMlinkescapetext{even} when they are not relatively prime. In this case, the function is a homomorphism of monoids and, because of the fundamental theorem of arithmetic, is completely determined by its restriction to prime numbers. Every completely additive function is additive.

Outside of number theory, the \PMlinkescapetext{term} additive is usually used for all functions with the property $f(a+b) = f(a)+f(b)$ for all arguments $a$ and $b$. (For instance, see the other entry titled \PMlinkname{additive function}{AdditiveFunction2}.) This entry discusses number theoretic additive functions.

Additive functions cannot have convolution inverses since an arithmetic function $f$ has a convolution inverse if and only if $f(1) \neq 0$.  A proof of this equivalence is supplied \PMlinkname{here}{ConvolutionInversesForArithmeticFunctions}.

The most common \PMlinkescapetext{type} of additive function in all of mathematics is the logarithm.  Other additive functions that are useful in number theory are:

\begin{itemize}
\item $\omega(n)$, the number of distinct prime factors function
\item $\Omega(n)$, the \PMlinkname{number of (nondistinct) prime factors function}{NumberOfNondistinctPrimeFactorsFunction}
\end{itemize}

By exponentiating an additive function, a multiplicative function is obtained.  For example, the function $\displaystyle 2^{\omega(n)}$ is multiplicative.  Similarly, by exponentiating a completely additive function, a completely multiplicative function is obtained.  For example, the function $\displaystyle 2^{\Omega(n)}$ is completely multiplicative.
%%%%%
%%%%%
\end{document}
