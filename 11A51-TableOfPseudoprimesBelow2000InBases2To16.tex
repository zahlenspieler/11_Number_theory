\documentclass[12pt]{article}
\usepackage{pmmeta}
\pmcanonicalname{TableOfPseudoprimesBelow2000InBases2To16}
\pmcreated{2013-03-22 16:50:15}
\pmmodified{2013-03-22 16:50:15}
\pmowner{PrimeFan}{13766}
\pmmodifier{PrimeFan}{13766}
\pmtitle{table of pseudoprimes below 2000 in bases 2 to 16}
\pmrecord{5}{39081}
\pmprivacy{1}
\pmauthor{PrimeFan}{13766}
\pmtype{Example}
\pmcomment{trigger rebuild}
\pmclassification{msc}{11A51}

\endmetadata

% this is the default PlanetMath preamble.  as your knowledge
% of TeX increases, you will probably want to edit this, but
% it should be fine as is for beginners.

% almost certainly you want these
\usepackage{amssymb}
\usepackage{amsmath}
\usepackage{amsfonts}

% used for TeXing text within eps files
%\usepackage{psfrag}
% need this for including graphics (\includegraphics)
%\usepackage{graphicx}
% for neatly defining theorems and propositions
%\usepackage{amsthm}
% making logically defined graphics
%%%\usepackage{xypic}

% there are many more packages, add them here as you need them

% define commands here

\begin{document}
This table lists \PMlinkname{pseudoprimes}{PseudoprimeP} $p$ of the form $b^{p - 1} \equiv 1 \mod b$ below 2000.

\begin{tabular}{|r|l|l|}
Base $b$ & \PMlinkescapetext{Pseudoprimes} to base $b$ & OEIS reference \\
2 & 341, 561, 645, 1105, 1387, 1729, 1905 & A001567 \\
3 & 91, 121, 286, 671, 703, 949, 1105, 1541, 1729, 1891 & A005935 \\
4 & 15, 85, 91, 341, 435, 451, 561, 645, 703, 1105, 1247, 1271, 1387, 1581, 1695, 1729, 1891, 1905 & A020136 \\
5 & 4, 124, 217, 561, 781, 1541, 1729, 1891 & A005936 \\
6 & 35, 185, 217, 301, 481, 1105, 1111, 1261, 1333, 1729 & A005937 \\
7 & 6, 25, 325, 561, 703, 817, 1105, 1825 & A005938 \\
8 & 9, 21, 45, 63, 65, 105, 117, 133, 153, 231, 273, 341, 481, 511, 561, 585, 645, 651, 861, 949, 1001, 1105, 1281, 1365, 1387, 1417, 1541, 1649, 1661, 1729, 1785, 1905 & A020137 \\
9 & 4, 8, 28, 52, 91, 121, 205, 286, 364, 511, 532, 616, 671, 697, 703, 946, 949, 1036, 1105, 1288, 1387, 1541, 1729, 1891 & A020138 \\
10 & 9, 33, 91, 99, 259, 451, 481, 561, 657, 703, 909, 1233, 1729 & A005939 \\
11 & 10, 15, 70, 133, 190, 259, 305, 481, 645, 703, 793, 1105, 1330, 1729 & A020139 \\
12 & 65, 91, 133, 143, 145, 247, 377, 385, 703, 1045, 1099, 1105, 1649, 1729, 1885, 1891 & A020140 \\
13 & 4, 6, 12, 21, 85, 105, 231, 244, 276, 357, 427, 561, 1099, 1785, 1891 & A020141 \\
14 & 15, 39, 65, 195, 481, 561, 781, 793, 841, 985, 1105, 1111, 1541, 1891 & A020142 \\
15 & 14, 341, 742, 946, 1477, 1541, 1687, 1729, 1891, 1921 & A020143 \\
16 & 15, 51, 85, 91, 255, 341, 435, 451, 561, 595, 645, 703, 1105, 1247, 1261, 1271, 1285, 1387, 1581, 1687, 1695, 1729, 1891, 1905 & A020144 \\
\end{tabular}

As you may have noticed, 1729 appears in all of these except for bases 7, 13 and 14, which share factors 7, 13 and 7, respectively, with 1729.
%%%%%
%%%%%
\end{document}
