\documentclass[12pt]{article}
\usepackage{pmmeta}
\pmcanonicalname{NumberTheory}
\pmcreated{2013-03-22 14:07:53}
\pmmodified{2013-03-22 14:07:53}
\pmowner{olivierfouquetx}{2421}
\pmmodifier{olivierfouquetx}{2421}
\pmtitle{number theory}
\pmrecord{37}{35541}
\pmprivacy{1}
\pmauthor{olivierfouquetx}{2421}
\pmtype{Topic}
\pmcomment{trigger rebuild}
\pmclassification{msc}{11-01}
%\pmkeywords{number}
%\pmkeywords{integer}
%\pmkeywords{Gauss}
%\pmkeywords{Legendre}
%\pmkeywords{Euler}
%\pmkeywords{Dedekind}
%\pmkeywords{Wiles}
%\pmkeywords{Weil}
%\pmkeywords{Grothendieck}
%\pmkeywords{Deligne}
%\pmkeywords{Faltings}
%\pmkeywords{Serre}
%\pmkeywords{prime}
\pmrelated{BibliographyForNumberTheory}
\pmrelated{Divisibility}
\pmrelated{Congruences}
\pmrelated{FundamentalTheoremOfArithmetic}
\pmrelated{NumberField}

% this is the default PlanetMath preamble.  as your knowledge
% of TeX increases, you will probably want to edit this, but
% it should be fine as is for beginners.

% almost certainly you want these
\usepackage{amssymb}
\usepackage{amsmath}
\usepackage{amsfonts}

% used for TeXing text within eps files
%\usepackage{psfrag}
% need this for including graphics (\includegraphics)
%\usepackage{graphicx}
% for neatly defining theorems and propositions
%\usepackage{amsthm}
% making logically defined graphics
%%%\usepackage{xypic}

% there are many more packages, add them here as you need them

% define commands here
\begin{document}
\PMlinkescapeword{branch}
\PMlinkescapeword{branches}
\PMlinkescapeword{diamond}
\PMlinkescapeword{part}
\PMlinkescapeword{parts}
\PMlinkescapeword{states}
\PMlinkescapeword{objects}
\PMlinkescapeword{structures}
\PMlinkescapeword{contains}

Number theory is the branch of math concerned with the study of the integers, and of the objects and structures that naturally arise from their study.   

It is one of the oldest parts of mathematics, alongside geometry, and has been studied at least since the ancient Mesopotamians and Egyptians. Perhaps because of its purely mathematical nature (at least until the development of    \PMlinkexternal{cryptography and cryptanalysis}{http://en.wikipedia.org/wiki/Cryptanalysis} in the twentieth century, number theory was thought to be devoid of practical applications), number theory has often been considered as a central and particularly beautiful part of mathematics. Carl Friedrich Gauss, arguably the greatest number theorist of all time, has called mathematics the ``Queen of science'' and he referred to Number Theory as the ``Queen of Mathematics''. Number theory has attracted many of the most outstanding mathematicians in history: Euclid, Diophantus, Fermat, Legendre, Euler, Gauss, Dedekind, Jacobi, Eisenstein and Hilbert all made immense contribution to its development. Great twentieth century number theorists include Artin, Hardy, Ramanujan, Andr\'{e} Weil, Alexandre Grothendieck,  Jean-Pierre Serre, Pierre Deligne, Gerd Faltings, John Tate and Andrew Wiles.

Number theory is also remarkable because small, easy-to-understand conjectures abound alongside far-reaching problems. 
One can mention the twin-prime problem, the Goldbach conjecture, and the odd perfect number problem. Recent progress includes the resolution of Catalan's conjecture (Mihailescu, 2002) and Fermat's last theorem (Taylor and Wiles, 1994) .

The most important unsolved problem in number theory is probably the Riemann Hypothesis, which states that all non-trivial zeros of the Riemann zeta function have real part equal to one-half. Though formulated in terms of complex analysis, this problem is central to understanding the distribution of prime numbers, the distribution of power residues modulo an integer and the size of the class group of a number field (to give just a few examples). 

The Greeks, notably Euclid and Pythagoras, were the first to elucidate the basic theory of irrational numbers. Greeks improved a 1,500 year older Egyptian rational number system. Greeks wrote 1/p as p' (in ciphered letters) and generally converted rational numbers to exact unit fraction series by selecting optimizing aliquot parts (of denominators). A proof of the irrationality of $\sqrt{2}$. Euclid also understood and proved some basic properties of prime and composite numbers. Euler, Liebniz, Liouville, and Lindemann, among many others, defined and refined the basic theory of algebraic and transcendental numbers. These four ideas recur over and over in number theory.

Four of the greatest achievements of number theorists in the twentieth century were abelian class field theory (the theory of abelian extensions of number fields, which extends the law of quadratic reciprocity) by Hilbert, Takagi, Artin, Tate and others; the proof of the Weil conjectures (which include the Riemann hypothesis for function fields and are intimately linked to geometry over finite fields) by Dwork, Grothendieck and Deligne using methods of $\ell$-adic cohomology; the proof of the Mordell conjecture (which asserts that a curve of genus greater than one has only finitely many rational points) by Faltings; and the proof of the Taniyama-Shimura-Weil conjecture (that states that every rational elliptic curve is modular) by Wiles, Taylor, Diamond, Conrad and Breuil.  

Number theory, long regarded as the purest of the pure sciences, has recently begun to find \PMlinkname{applications in cryptography}{CryptographyAndNumberTheory}.  The recent invention of public-key cryptosystems, which are usually based on the difficulty of a particular number-theoretic computation, has encouraged research in number theory which is essentially applied. 

Problems in number theory are often solved using sophisticated techniques from different branches of mathematics. Number theory itself can be loosely divided (not partitioned!) as follows
\begin{itemize}
\item Analytic number theory uses the machinery of analysis to tackle questions related to integers and transcendence. One of its most famous achievements is the proof of the prime number theorem. 
\item Algebraic number theory can either be defined as the study of algebraic numbers or as an algebraic study of number theory (depending on how you \PMlinkname{associate}{Associative} in English). By the definition of algebraic numbers, these two are equivalent. In the theory of abelian extensions of number fields, which extends the reciprocity laws of Gauss, Legendre, Hilbert et al.\ of the first half of the past century, class field theory constituted the bulk of algebraic number theory research. In the latter half, algebraic number theory has been subsumed under arithmetic geometry (see below) largely due to the efforts of Grothendieck, Serre, Deligne, et al.
\item Arithmetic geometry seeks to bring methods that have been spectacularly successful in ``classical'' geometric theories (such as complex manifold theory) to have bearing on problems in number theory. An illustrative example is the identification of the absolute Galois group of a field with the fundamental group of the associated affine scheme. Arithmetic geometry is often referred to as arithmetic algebraic geometry or Diophantine geometry.
\item Finally, computational number theory is the study of computations with numbers, developing algorithms to calculate things such as factorizations, discrete logarithms, numbers of points on curves, class groups and cohomology groups.
\end{itemize}

The attached bibliography for number theory contains many additional references for these topics.
%%%%%
%%%%%
\end{document}
