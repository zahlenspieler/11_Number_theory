\documentclass[12pt]{article}
\usepackage{pmmeta}
\pmcanonicalname{CongruenceOfClausenAndVonStaudt}
\pmcreated{2013-03-22 15:11:58}
\pmmodified{2013-03-22 15:11:58}
\pmowner{alozano}{2414}
\pmmodifier{alozano}{2414}
\pmtitle{congruence of Clausen and von Staudt}
\pmrecord{4}{36957}
\pmprivacy{1}
\pmauthor{alozano}{2414}
\pmtype{Theorem}
\pmcomment{trigger rebuild}
\pmclassification{msc}{11B68}
\pmsynonym{Staudt-Clausen theorem}{CongruenceOfClausenAndVonStaudt}
\pmsynonym{von Staudt-Clausen theorem}{CongruenceOfClausenAndVonStaudt}
%\pmkeywords{Bernoulli number}
%\pmkeywords{congruence}
\pmrelated{KummersCongruence}
\pmrelated{OddBernoulliNumbersAreZero}

\endmetadata

% this is the default PlanetMath preamble.  as your knowledge
% of TeX increases, you will probably want to edit this, but
% it should be fine as is for beginners.

% almost certainly you want these
\usepackage{amssymb}
\usepackage{amsmath}
\usepackage{amsthm}
\usepackage{amsfonts}

% used for TeXing text within eps files
%\usepackage{psfrag}
% need this for including graphics (\includegraphics)
%\usepackage{graphicx}
% for neatly defining theorems and propositions
%\usepackage{amsthm}
% making logically defined graphics
%%%\usepackage{xypic}

% there are many more packages, add them here as you need them

% define commands here

\newtheorem*{thm}{Theorem}
\newtheorem{defn}{Definition}
\newtheorem{prop}{Proposition}
\newtheorem{lemma}{Lemma}
\newtheorem*{cor}{Corollary}

\theoremstyle{definition}
\newtheorem{exa}{Example}

% Some sets
\newcommand{\Nats}{\mathbb{N}}
\newcommand{\Ints}{\mathbb{Z}}
\newcommand{\Reals}{\mathbb{R}}
\newcommand{\Complex}{\mathbb{C}}
\newcommand{\Rats}{\mathbb{Q}}
\newcommand{\Gal}{\operatorname{Gal}}
\newcommand{\Cl}{\operatorname{Cl}}
\begin{document}
Let $B_k$ denote the $k$th Bernoulli number:
$$B_0=1,\quad B_1=-\frac{1}{2},\quad B_2=\frac{1}{6},\quad B_3=0,\quad B_4=-\frac{1}{30},\ldots$$
In fact, $B_k=0$ for all odd $k\geq 3$, so we will only consider $B_k$ for even $k$. The following is a well-known congruence, due to Thomas Clausen and Karl von Staudt.

\begin{thm}[Congruence of Clausen and von Staudt]

For an even integer $k\geq 2$,

$$B_k \equiv -\sum_{p \text{ prime},\ (p-1)|k} \frac{1}{p} \mod \Ints$$
where the sum is over all primes $p$ such that $(p-1)$ divides $k$. In other words, there exists an integer $n_k$ such that 
$$B_k=n_k -\sum_{p \text{ prime},\ (p-1)|k} \frac{1}{p}.$$
\end{thm}

For example:
$$B_2=\frac{1}{6}=1-\frac{1}{2}-\frac{1}{3}, \quad B_4=-\frac{1}{30}=1-\frac{1}{2}-\frac{1}{3}-\frac{1}{5}.$$
Sometimes the theorem is stated in this alternative form:

\begin{cor}
For an even integer $k\geq 2$ and any prime $p$ the product $pB_k$ is $p$-integral, that is, $pB_k$ is a rational number $t/s$ (in lowest terms) such that $p$ does not divide $s$. Moreover:
$$pB_k \equiv \begin{cases}
-1 \mod p, \text{ if $(p-1)$ divides $k$;}\\
0 \mod p, \text{ if $(p-1)$ does not divide $k$}.
\end{cases}$$ 
\end{cor}
%%%%%
%%%%%
\end{document}
