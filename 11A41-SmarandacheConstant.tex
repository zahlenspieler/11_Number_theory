\documentclass[12pt]{article}
\usepackage{pmmeta}
\pmcanonicalname{SmarandacheConstant}
\pmcreated{2013-03-22 17:17:49}
\pmmodified{2013-03-22 17:17:49}
\pmowner{dankomed}{17058}
\pmmodifier{dankomed}{17058}
\pmtitle{Smarandache constant}
\pmrecord{17}{39642}
\pmprivacy{1}
\pmauthor{dankomed}{17058}
\pmtype{Conjecture}
\pmcomment{trigger rebuild}
\pmclassification{msc}{11A41}
\pmrelated{FlorentinSmarandache}

% this is the default PlanetMath preamble.  as your knowledge
% of TeX increases, you will probably want to edit this, but
% it should be fine as is for beginners.

% almost certainly you want these
\usepackage{amssymb}
\usepackage{amsmath}
\usepackage{amsfonts}

% used for TeXing text within eps files
%\usepackage{psfrag}
% need this for including graphics (\includegraphics)
%\usepackage{graphicx}
% for neatly defining theorems and propositions
%\usepackage{amsthm}
% making logically defined graphics
%%%\usepackage{xypic}

% there are many more packages, add them here as you need them

% define commands here

\begin{document}
The \emph{Smarandache constant} is the minimal solution $x_{\min}$ of the generalized Andrica equation:
$p_{n+1}^{x}-p_{n}^{x}=1$
where $p_n$ is the n$^{\text{th}}$ prime number, and $x \in \mathbb{R}$. The exact value of the Smarandache constant and whether it exists is currently unknown, however according to the generalized Andrica conjecture proposed by Florentin Smarandache the value of the Smarandache constant is $x_{\min}\approx0.5671481302\ldots$

The Smarandache constant should not be confused with a list of sixteen Smarandache constants $s_{1}-s_{16}$ defined as limits of different convergent series involving the Smarandache function.
%%%%%
%%%%%
\end{document}
