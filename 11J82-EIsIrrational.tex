\documentclass[12pt]{article}
\usepackage{pmmeta}
\pmcanonicalname{EIsIrrational}
\pmcreated{2013-03-22 12:33:02}
\pmmodified{2013-03-22 12:33:02}
\pmowner{mathwizard}{128}
\pmmodifier{mathwizard}{128}
\pmtitle{e is irrational}
\pmrecord{13}{32795}
\pmprivacy{1}
\pmauthor{mathwizard}{128}
\pmtype{Theorem}
\pmcomment{trigger rebuild}
\pmclassification{msc}{11J82}
\pmclassification{msc}{11J72}
\pmrelated{ErIsIrrationalForRinmathbbQsetminus0}
\pmrelated{NaturalLogBase}

% this is the default PlanetMath preamble.  as your knowledge
% of TeX increases, you will probably want to edit this, but
% it should be fine as is for beginners.

% almost certainly you want these
\usepackage{amssymb}
\usepackage{amsmath}
\usepackage{amsfonts}

% used for TeXing text within eps files
%\usepackage{psfrag}
% need this for including graphics (\includegraphics)
%\usepackage{graphicx}
% for neatly defining theorems and propositions
%\usepackage{amsthm}
% making logically defined graphics
%%%\usepackage{xypic}

% there are many more packages, add them here as you need them

% define commands here
\begin{document}
From the Taylor series for $e^x$ we know the following equation:
\begin{equation}
e=\sum_{k=0}^{\infty}\frac{1}{k!}.
\end{equation}
Now let us assume that $e$ is rational. This would \PMlinkescapetext{mean} there are two natural numbers $a$ and $b$, such that:
$$e=\frac{a}{b}.$$
This yields:
$$b!e\in\mathbb{N}.$$
Now we can write $e$ using (1):
$$b!e=b!\sum_{k=0}^{\infty}\frac{1}{k!}.$$
This can also be written:
$$b!e=\sum_{k=0}^{b}\frac{b!}{k!}+\sum_{k=b+1}^{\infty}\frac{b!}{k!}.$$
The first sum is obviously a natural number, and thus
$$\sum_{k=b+1}^{\infty}\frac{b!}{k!}$$
must also be \PMlinkescapetext{natural}. Now we see:
$$\sum_{k=b+1}^{\infty}\frac{b!}{k!}=\frac{1}{b+1}+\frac{1}{(b+1)(b+2)}+...<
\sum_{k=1}^{\infty}\left(\frac{1}{b+1}\right)^k=\frac{1}{b}.$$
Since $\frac{1}{b}\leq 1$ we conclude:
$$0<\sum_{k=b+1}^{\infty}\frac{b!}{k!}<1.$$
We have also seen that this is an integer, but there is no integer between 0 and 1. So there cannot exist two natural numbers $a$ and $b$ such that $e=\frac{a}{b}$, so $e$ is irrational.
%%%%%
%%%%%
\end{document}
