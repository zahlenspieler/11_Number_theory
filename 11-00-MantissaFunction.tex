\documentclass[12pt]{article}
\usepackage{pmmeta}
\pmcanonicalname{MantissaFunction}
\pmcreated{2013-03-22 18:32:32}
\pmmodified{2013-03-22 18:32:32}
\pmowner{pahio}{2872}
\pmmodifier{pahio}{2872}
\pmtitle{mantissa function}
\pmrecord{7}{41261}
\pmprivacy{1}
\pmauthor{pahio}{2872}
\pmtype{Definition}
\pmcomment{trigger rebuild}
\pmclassification{msc}{11-00}
\pmclassification{msc}{26A09}
\pmrelated{Floor}
\pmdefines{mantissa}
\pmdefines{mantissa of real number}

\endmetadata

% this is the default PlanetMath preamble.  as your knowledge
% of TeX increases, you will probably want to edit this, but
% it should be fine as is for beginners.

% almost certainly you want these
\usepackage{amssymb}
\usepackage{amsmath}
\usepackage{amsfonts}

% used for TeXing text within eps files
%\usepackage{psfrag}
% need this for including graphics (\includegraphics)
%\usepackage{graphicx}
% for neatly defining theorems and propositions
 \usepackage{amsthm}
% making logically defined graphics
%%%\usepackage{xypic}
\usepackage{pstricks}
\usepackage{pst-plot}

% there are many more packages, add them here as you need them

% define commands here

\theoremstyle{definition}
\newtheorem*{thmplain}{Theorem}

\begin{document}
If we subtract from a real number $x$ the greatest integer not exceeding $x$, we obtain a number $y$ between 0 and 1, which can equal 0 if $x$ is an integer.\, In other \PMlinkescapetext{words},
                     $$y \;=\; x\!-\!\lfloor{x}\rfloor,$$
where $\lfloor{x}\rfloor$ is the floor of $x$.
Such a number $y$ is called the {\em mantissa} of $x$.\, So we have for example\\

$2.7-2 \;=\; 0.7$,\\
$1.7-1 \;=\; 0.7$,\\
$0.7-0 \;=\; 0.7$,\\
$-0.3\!-\!(-1) = 0.7$,\\
$-1.3\!-\!(-2) = 0.7,$\\

i.e. these numbers 2.7, 1.7, 0.7, $-0.3$, $-1.3$ at mutual distances an integer have the same mantissa (0.7).\, This is apparently always true --- thus the {\em mantissa function}
$$x \mapsto x\!-\!\lfloor{x}\rfloor$$
is \PMlinkname{periodic}{QuasiPeriodicFunction}:\, its \PMlinkname{least period}{PeriodicFunctions} is 1.\\

The mantissa is identic with the mantissa used in the Briggsian logarithm calculations.\\

When $x$ increases from an integer $n$ towards the next integer $n\!+\!1$, its mantissa $x\!-\!\lfloor{x}\rfloor$ increases with the same speed from 0 tending to 1, but at $n\!+\!1$ it falls back to 0.

\begin{center}
\begin{pspicture}(-5.5,-2.5)(5.5,3.5)
\psaxes[Dx=1,Dy=1]{->}(0,0)(-4.5,-1.9)(4.5,3)
\rput(0.3,3.1){$y$}
\rput(4.6,0.2){$x$}
\psdots[linecolor=blue](-4,0)(-3,0)(-2,0)(-1,0)(1,0)(2,0)(3,0)(4,0)
\psdot[linecolor=blue,linewidth=0.03](0,0)
\psline[linecolor=blue](-4,0)(-3,1)
\psline[linecolor=blue](-3,0)(-2,1)
\psline[linecolor=blue](-2,0)(-1,1)
\psline[linecolor=blue](-1,0)(0,1)
\psline[linecolor=blue](0,0)(1,1)
\psline[linecolor=blue](1,0)(2,1)
\psline[linecolor=blue](2,0)(3,1)
\psline[linecolor=blue](3,0)(4,1)
\rput(2.5,2.5){$\mbox{Graph\; } y = x\!-\!\lfloor{x}\rfloor$}
\end{pspicture}
\end{center}

Being a periodic function, the \PMlinkname{Fourier expansion}{DeterminationOfFourierCoefficients} of the function is easy to form:
$$x\!-\!\lfloor{x}\rfloor \;=\; \frac{1}{2}-\sum_{n=1}^\infty\frac{\sin 2n\pi{x}}{n\pi}$$
This is valid for\, $x \not\in \mathbb{Z}$,\, since the series gives in the jump discontinuity points the arithmetic means ($= \frac{1}{2}$) of \PMlinkname{left and right limits}{OneSidedLimit}.
%%%%%
%%%%%
\end{document}
