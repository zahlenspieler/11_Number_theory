\documentclass[12pt]{article}
\usepackage{pmmeta}
\pmcanonicalname{DivisorFunctionIsMultiplicativeThe}
\pmcreated{2013-03-22 15:03:47}
\pmmodified{2013-03-22 15:03:47}
\pmowner{yark}{2760}
\pmmodifier{yark}{2760}
\pmtitle{divisor function is multiplicative, the}
\pmrecord{9}{36783}
\pmprivacy{1}
\pmauthor{yark}{2760}
\pmtype{Theorem}
\pmcomment{trigger rebuild}
\pmclassification{msc}{11A25}

\endmetadata

\usepackage{amsthm}

\newtheorem*{thm*}{Theorem}

\begin{document}
\begin{thm*}
The \PMlinkname{divisor function}{TauFunction} is multiplicative.
\end{thm*}

{\bf Proof.}
Let $t=mn$ with $m,n$ coprime.
Applying the fundamental theorem of arithmetic, we can write
\[
  m=p_1^{a_1}p_2^{a_2}\cdots p_{r}^{a_r},
  \qquad n=q_1^{b_1}q_2^{b_2}\cdots q_s^{b_s},
\]
where each $p_j$ and $q_i$ are prime.
Moreover, since $m$ and $n$ are coprime, we conclude that
\[
t=p_1^{a_1}p_2^{a_2}\cdots p_{r}^{a_r}q_1^{b_1}q_2^{b_2}\cdots q_s^{b_s}.
\]

Now, each divisor of $t$ is of the form
\[
t=p_1^{k_1}p_2^{k_2}\cdots p_{r}^{k_r}q_1^{h_1}q_2^{h_2}\cdots q_s^{h_s}.
\]
with $0\leq k_j\leq a_j$ and $0\leq h_i\leq b_i$,
and for each such divisor we get a divisor of $m$ and a divisor of $n$,
given respectively by
\[
  u=p_1^{k_1}p_2^{k_2}\cdots p_{r}^{k_r},
  \qquad v=q_1^{h_1}q_2^{h_2}\cdots q_s^{h_s}.
\]

Now, each respective divisor of $m$, $n$ is of the form above,
and for each such pair their product is also a divisor of $t$.
Therefore we get a bijection between the set of positive divisors of $t$
and the set of pairs of divisors of $m$, $n$ respectively.
Such bijection implies that the cardinalities of both sets are the same,
and thus
\[
  d(mn)=d(m)d(n).
\]
%%%%%
%%%%%
\end{document}
