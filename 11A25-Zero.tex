\documentclass[12pt]{article}
\usepackage{pmmeta}
\pmcanonicalname{Zero}
\pmcreated{2013-03-22 18:04:09}
\pmmodified{2013-03-22 18:04:09}
\pmowner{Mravinci}{12996}
\pmmodifier{Mravinci}{12996}
\pmtitle{zero}
\pmrecord{7}{40602}
\pmprivacy{1}
\pmauthor{Mravinci}{12996}
\pmtype{Definition}
\pmcomment{trigger rebuild}
\pmclassification{msc}{11A25}
\pmsynonym{null}{Zero}
\pmsynonym{nought}{Zero}
\pmrelated{IsomorphismSwappingZeroAndUnity}

\endmetadata

% this is the default PlanetMath preamble.  as your knowledge
% of TeX increases, you will probably want to edit this, but
% it should be fine as is for beginners.

% almost certainly you want these
\usepackage{amssymb}
\usepackage{amsmath}
\usepackage{amsfonts}

% used for TeXing text within eps files
%\usepackage{psfrag}
% need this for including graphics (\includegraphics)
%\usepackage{graphicx}
% for neatly defining theorems and propositions
%\usepackage{amsthm}
% making logically defined graphics
%%%\usepackage{xypic}

% there are many more packages, add them here as you need them

% define commands here

\begin{document}
\emph{Zero} (or \emph{null} or \emph{nought}) is the integer between $-1$ and 1, usually represented as 0 (the glyph had a diagonal slash in early computer fonts). In the complex plane, zero is at the intersection of the line of numbers with no imaginary part and the line of numbers with no real part, and as such can be represented as $0 + 0i$.

Zero is an identity element for addition, meaning that adding 0 to another number just gives that other number: $x + 0 = x$. But under multiplication, 0 is a fixed element: $0x = 0$. Also, $\frac{0}{x} = 0$, but division by zero is not allowed for most applications. Whether $x$ is positive or negative, $x^0 = 1$. But $0^0$ is a whole other can of worms.

As a digit in a positional base, 0 indicates the absence of a particular power of the base in the partition representing a given number. For example, in the base 10 self-descriptive number 6210001000, the 0s indicate the absence of ones, tens, hundreds, ten thousands, hundred thousands, and millions in this particular representation of this number. For computers and their binary representation of numbers, a bit set to 0 is a bit that is off, having no electrical current at the moment.

Though not as late as for the negative numbers, the concept of zero met with some resistance at first. But by the 20th Century, 0 had become indispensable for all branches of mathematics. In elementary algebra textbooks, 0 is often the goal of solving a problem, e.g., the quadratic equation $4x^2 + 3x + 1 = 0$, in which the student is asked to find the value of $x$ that makes the left-side of the equation equal to zero. Even Euler's identity is sometimes stated in this way: $e^{i \pi} + 1 = 0$.

Given a function $f$, the values $x$ such that $f(x) = 0$ are called the ``zeroes'' of that function and are often of special interest to mathematicians and amateurs alike; see: \PMlinkname{zero of a function}{ZeroOfAFunction}. (For example, the zeroes of the Riemann zeta function).

For more on zero through the ages, see: history of zero.
%%%%%
%%%%%
\end{document}
