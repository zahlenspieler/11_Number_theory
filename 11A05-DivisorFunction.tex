\documentclass[12pt]{article}
\usepackage{pmmeta}
\pmcanonicalname{DivisorFunction}
\pmcreated{2013-11-27 18:13:38}
\pmmodified{2013-11-27 18:13:38}
\pmowner{pahio}{2872}
\pmmodifier{pahio}{2872}
\pmtitle{divisor function}
\pmrecord{8}{87979}
\pmprivacy{1}
\pmauthor{pahio}{2872}
\pmtype{Theorem}
\pmclassification{msc}{11A05}
\pmclassification{msc}{11A25}

\endmetadata

% this is the default PlanetMath preamble.  as your knowledge
% of TeX increases, you will probably want to edit this, but
% it should be fine as is for beginners.

% almost certainly you want these
\usepackage{amssymb}
\usepackage{amsmath}
\usepackage{amsfonts}

% need this for including graphics (\includegraphics)
\usepackage{graphicx}
% for neatly defining theorems and propositions
\usepackage{amsthm}

% making logically defined graphics
%\usepackage{xypic}
% used for TeXing text within eps files
%\usepackage{psfrag}

% there are many more packages, add them here as you need them

% define commands here

\begin{document}
In the parent article there has been proved the formula
$$
\sigma_1(n) \;=\; \sum_{0 < d \mid n}\!d \;=\;
\prod_{i=1}^k\frac{p_i^{m_i+1}-1}{p_i-1}
$$
giving the sum of all positive divisors of an integer $n$; 
there the $p_i$'s are the distinct positive prime factors of $n$ and $m_i$'s their multiplicities.\\

It follows that the sum of the $z$'th powers of those divisors is given by
\begin{align}
\sigma_z(n) \;=\; \sum_{0 < d \mid n}\!d^z \;=\;
\prod_{i=1}^k\frac{p_i^{(m_i+1)z}-1}{p_i^z-1}.
\end{align}
This complex function of $z$ is called 
{\it \PMlinkname{divisor function}{DivisorFunction}}.\, The 
equation (1) may be written in the form
\begin{align}
\sigma_z(n)\;=\; 
\prod_{i=1}^k(1+p_i^z+p_i^{2z}+\ldots+p_i^{m_iz})
\end{align}
usable also for\, $z = 0$.\, For the special case of one prime 
power the function consists of the single 
\PMlinkname{geometric sum}{GeometricSeries}
$$\sigma_z(p^m) \;=\; 1+p^z+p^{2z}+\ldots+p^{mz},$$
which particularly gives $m\!+\!1$ when $p^z = 1$, i.e. when $z$ 
is a multiple of $2i\pi/\ln{p}$.


A special case of the function (1) is the 
\PMlinkname{$\tau$ function}{TauFunction} of $n$:
$$
\sigma_0(n) \;=\; \sum_{0 < d \mid n}\!1 \;=\;
\prod_{i=1}^k(m_i+1) \;=\; \tau(n)
$$\\








\textbf{Some inequalities}

$$\sigma_m(n)\,\ge\;n^{\frac{m}{2}}\sigma_0(n)
\quad\mbox{for}\quad m = 0,\,1,\,2,\,\ldots$$

$$\sigma_1(mn) > \sigma_1(m)+\sigma_1(n) 
\quad \forall\, m, n \in \mathbb{Z}$$

$$\sigma_1(n)\,\le\;\frac{n\!+\!1}{2}\sigma_0(n)$$
$$n\!+\!\sqrt{n} \,< \sigma_1(n) < 
\frac{6}{\pi^2}\!\cdot\!n\sqrt{n}$$


\end{document}
