\documentclass[12pt]{article}
\usepackage{pmmeta}
\pmcanonicalname{PepinsTheorem}
\pmcreated{2013-03-22 18:53:09}
\pmmodified{2013-03-22 18:53:09}
\pmowner{PrimeFan}{13766}
\pmmodifier{PrimeFan}{13766}
\pmtitle{P\'epin's theorem}
\pmrecord{4}{41734}
\pmprivacy{1}
\pmauthor{PrimeFan}{13766}
\pmtype{Theorem}
\pmcomment{trigger rebuild}
\pmclassification{msc}{11A51}
\pmsynonym{Pepin's theorem}{PepinsTheorem}

% this is the default PlanetMath preamble.  as your knowledge
% of TeX increases, you will probably want to edit this, but
% it should be fine as is for beginners.

% almost certainly you want these
\usepackage{amssymb}
\usepackage{amsmath}
\usepackage{amsfonts}

% used for TeXing text within eps files
%\usepackage{psfrag}
% need this for including graphics (\includegraphics)
%\usepackage{graphicx}
% for neatly defining theorems and propositions
%\usepackage{amsthm}
% making logically defined graphics
%%%\usepackage{xypic}

% there are many more packages, add them here as you need them

% define commands here

\begin{document}
{\bf Theorem} (P\'epin). A Fermat number $F_n = 2^{2^n} + 1$ is prime only if $$3^{\frac{F_n - 1}{2}} \equiv -1 \mod F_n.$$ In other words, if 3 raised to the largest power of two not greater than the Fermat number leaves as a remainder the next higher power of two when divided by that Fermat number (since $F_n - 1 = 2^{2^n}$), then that Fermat number is a Fermat prime.

For example, $17 = 2^{2^2} + 1$ is a Fermat prime, and we can see that $3^8 = 6561$, which leaves a remainder of 16 when divided by 17. The smallest Fermat number not to be a prime is 4294967297, as it is the product of 641 and 6700417, and $3^{2147483648}$ divided by 4294967297 leaves a remainder of 10324303 rather than 4294967296.
%%%%%
%%%%%
\end{document}
