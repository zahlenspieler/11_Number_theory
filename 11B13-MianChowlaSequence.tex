\documentclass[12pt]{article}
\usepackage{pmmeta}
\pmcanonicalname{MianChowlaSequence}
\pmcreated{2013-03-22 16:27:49}
\pmmodified{2013-03-22 16:27:49}
\pmowner{PrimeFan}{13766}
\pmmodifier{PrimeFan}{13766}
\pmtitle{Mian-Chowla sequence}
\pmrecord{6}{38622}
\pmprivacy{1}
\pmauthor{PrimeFan}{13766}
\pmtype{Definition}
\pmcomment{trigger rebuild}
\pmclassification{msc}{11B13}

\endmetadata

% this is the default PlanetMath preamble.  as your knowledge
% of TeX increases, you will probably want to edit this, but
% it should be fine as is for beginners.

% almost certainly you want these
\usepackage{amssymb}
\usepackage{amsmath}
\usepackage{amsfonts}

% used for TeXing text within eps files
%\usepackage{psfrag}
% need this for including graphics (\includegraphics)
%\usepackage{graphicx}
% for neatly defining theorems and propositions
%\usepackage{amsthm}
% making logically defined graphics
%%%\usepackage{xypic}

% there are many more packages, add them here as you need them

% define commands here

\begin{document}
\PMlinkescapeword{length}

The {\em Mian-Chowla sequence} is a $B_2$ sequence with $a_1 = 1$ and $a_n$ for $n > 2$ being the smallest integer such that each pairwise sum $a_i + a_j$ is distinct, where $0 < i < (n + 1)$ and likewise for $j$, that is, $1 \le i \le j \le n$. The case $i = j$ is always considered.

At the beginning, with $a_1$, there is only one pairwise sum, 2. $a_2$ can be 2 since the pairwise sums then are 2, 3 and 4. $a_3$ can't be 3 because then there would be the pairwise sums 1 + 3 = 2 + 2 = 4. Thus $a_3 = 4$. The sequence, listed in A005282 of Sloane's OEIS, continues 8, 13, 21, 31, 45, 66, 81, 97, 123, 148, 182, 204, 252, 290, 361, 401, 475, ... If we define $a_1 = 0$, the resulting sequence is the same except each term is one less.

Rachel Lewis noticed that $$\sum_{i = 1}^\infty \frac{1}{a_i} \equiv 2.1585$$, a constant listed in Finch's book.

One way to calculate the Mian-Chowla sequence in Mathematica is thus:

\begin{verbatim}
a = Table[1, {40}];
n = 2;
test = 1;
While[n < 41,
      mcFlag = False;
      While[Not[mcFlag], 
            test++;
            a[[n]] = test;
            pairSums = Flatten[Table[a[[i]] + a[[j]], {i, n}, {j, i, n}]];
            mcFlag = TrueQ[Length[pairSums] == Length[Union[pairSums]]]
      ];
      n++
];
a
\end{verbatim}

\begin{thebibliography}{5}
\bibitem{sf} S. R. Finch, {\it Mathematical Constants}, Cambridge (2003): Section 2.20.2
\bibitem{rg} R. K. Guy {\it Unsolved Problems in Number Theory}, New York: Springer (2003)
\end{thebibliography}
%%%%%
%%%%%
\end{document}
