\documentclass[12pt]{article}
\usepackage{pmmeta}
\pmcanonicalname{DerivationOfPlasticNumber}
\pmcreated{2013-03-22 19:09:41}
\pmmodified{2013-03-22 19:09:41}
\pmowner{pahio}{2872}
\pmmodifier{pahio}{2872}
\pmtitle{derivation of plastic number}
\pmrecord{13}{42067}
\pmprivacy{1}
\pmauthor{pahio}{2872}
\pmtype{Derivation}
\pmcomment{trigger rebuild}
\pmclassification{msc}{11B39}
%\pmkeywords{plastic constant}
%\pmkeywords{silver number}
\pmrelated{LimitRulesOfSequences}
\pmrelated{GoldenRatio}
\pmrelated{LimitRulesOfFunctions}

\endmetadata

% this is the default PlanetMath preamble.  as your knowledge
% of TeX increases, you will probably want to edit this, but
% it should be fine as is for beginners.

% almost certainly you want these
\usepackage{amssymb}
\usepackage{amsmath}
\usepackage{amsfonts}

% used for TeXing text within eps files
%\usepackage{psfrag}
% need this for including graphics (\includegraphics)
%\usepackage{graphicx}
% for neatly defining theorems and propositions
 \usepackage{amsthm}
% making logically defined graphics
%%%\usepackage{xypic}

% there are many more packages, add them here as you need them

% define commands here

\theoremstyle{definition}
\newtheorem*{thmplain}{Theorem}

\begin{document}
The \emph{plastic number} may be defined to be the limit of the ratio of two successive \PMlinkname{members}{Sequence} of the Padovan sequence or the Perrin sequence, both of which obey the recurrence relation
\begin{align}
a_n \;=\; a_{n-3}+a_{n-2}.
\end{align}
Supposing that such a limit 
\begin{align}
P \;:=\; \lim_{n\to\infty}\frac{a_{n+1}}{a_n}
\end{align}
exists (and is $\neq 0$), we first write (1) as
\begin{align}
\frac{a_n}{a_{n-1}}\cdot\frac{a_{n-1}}{a_{n-2}} \;=\; \frac{a_{n-3}}{a_{n-2}}+1
\end{align}
and then let $n \to \infty$.\, It follows the limit equation
$$P\!\cdot\!P \;=\; \frac{1}{P}\!+\!1,$$
which is same as
\begin{align}
P^3 \;=\; P\!+\!1.
\end{align}
Thinking the graphs of the equations \,$y = x^3$\, and\, $y = x\!+\!1$, it is clear that the cubic equation
\begin{align}
x^3\!-\!x\!-\!1 \;=\; 0
\end{align}
has only one real \PMlinkname{root}{Equation}, which is $P$.  \\

For solving the plastic number from the cubic, substitute by \PMlinkname{Cardano}{CardanosFormulae} into (5) the sum 
\,$x := u\!+\!v$\, of two auxiliary unknowns, when the equation may be written
$$(u^3\!+\!v^3\!-\!1)+(3uv\!-1)(u\!+\!v) \;=\; 0.$$
Then, as in the example of solving a cubic equation, $u$ and $v$ are determined such that the first two parentheses vanish:
\begin{align*}
\begin{cases}
u^2\!+\!v^3 \;=\; 1,\\
uv \;=\; \frac{1}{3}, \quad \mbox{or} \quad u^3v^3 \;=\; \frac{1}{27}
\end{cases}
\end{align*}
Thus $u^3$ and $v^3$ are the roots of the resolvent equation
$$z^2\!-\!z\!+\!\frac{1}{27} \;=\; 0,$$
i.e.
$$z \;=\;\frac{9\!\pm\!\sqrt{69}}{18}$$
and accordingly
$$u \;=\; \sqrt[3]{\frac{9\!+\!\sqrt{69}}{18}}, \quad v \;=\; \sqrt[3]{\frac{9\!-\!\sqrt{69}}{18}}.$$
Fixing that these \PMlinkescapetext{mean} the real values of the cube roots, we obtain the value of the plastic number in the form
$$x \;=\; \sqrt[3]{\frac{9\!+\!\sqrt{69}}{18}}+\sqrt[3]{\frac{9\!-\!\sqrt{69}}{18}},$$
or
\begin{align}
P \;=\; \frac{ \sqrt[3]{12(9\!+\!\sqrt{69})}+\sqrt[3]{12(9\!-\!\sqrt{69})} } {6}.
\end{align}
By (5), $P$ is an algebraic integer of \PMlinkname{degree}{DegreeOfAnAlgebraicNumber} 3 (and a unit of the ring of integers of the number field $\mathbb{Q}(P)$).\, For computing an approximate value of $P$, see e.g. nth root by Newton's method.






%%%%%
%%%%%
\end{document}
