\documentclass[12pt]{article}
\usepackage{pmmeta}
\pmcanonicalname{ExampleOfGcd}
\pmcreated{2014-10-25 7:09:30}
\pmmodified{2014-10-25 7:09:30}
\pmowner{pahio}{2872}
\pmmodifier{pahio}{2872}
\pmtitle{example of gcd}
\pmrecord{19}{37307}
\pmprivacy{1}
\pmauthor{pahio}{2872}
\pmtype{Example}
\pmcomment{trigger rebuild}
\pmclassification{msc}{11A51}
\pmrelated{Divisibility}
\pmrelated{GreatestCommonDivisor}
\pmrelated{Congruences}
\pmrelated{PolynomialCongruence}
\pmrelated{GroupingMethodForFactorizingPolynomials}

% this is the default PlanetMath preamble.  as your knowledge
% of TeX increases, you will probably want to edit this, but
% it should be fine as is for beginners.

% almost certainly you want these
\usepackage{amssymb}
\usepackage{amsmath}
\usepackage{amsfonts}

% used for TeXing text within eps files
%\usepackage{psfrag}
% need this for including graphics (\includegraphics)
%\usepackage{graphicx}
% for neatly defining theorems and propositions
 \usepackage{amsthm}
% making logically defined graphics
%%%\usepackage{xypic}

% there are many more packages, add them here as you need them

% define commands here

\theoremstyle{definition}
\newtheorem*{thmplain}{Theorem}
\begin{document}
If one calculates values of the polynomial
     $$f(x) \;:=\; x^4\!+\!x^2\!+\!1$$
on the successive positive integers $n$, one may observe an interesting thing when factoring the values:

$f(1) \;=\; 3$\\
$f(2) \;=\; 21 \;=\; 3\cdot 7$\\
$f(3) \;=\; 91 \;=\; 7\cdot 13$\\
$f(4) \;=\; 273 \;=\; 3\cdot 7\cdot 13$\\
$f(5) \;=\; 651 \;=\; 3\cdot 7\cdot31$\\
$f(6) \;=\; 1333 \;=\; 31\cdot 43$\\
$f(7) \;=\; 2451 \;=\; 3\cdot 19\cdot 43$\\
$f(8) \;=\; 4161 \;=\; 3\cdot 19\cdot 73$\\
$f(9) \;=\; 6643 \;=\; 7\cdot 13\cdot 73$\\
$f(10) \;=\; 10101 \;=\; 3\cdot 7\cdot 13\cdot 37$\\
$f(11) \;=\; 14763 \;=\; 3\cdot 7\cdot 19\cdot 37$\\
$\mbox{ . . . }$

It seems as if two consecutive values always have at least one common odd prime factor, i.e. they have the greatest common divisor $> 2$.

It is indeed true.\, The reason of this fact is not particularly 
deep.\, It is easily understood if we can 
\PMlinkname{factorize this polynomial}{GroupingMethodForFactoringPolynomials}:
    $$f(x) \;=\; (x^2\!-\!x\!+\!1)(x^2\!+\!x\!+\!1)$$
Then
\begin{align}
 f(n) \;=\; (n^2\!-\!n\!+\!1)(n^2\!+\!n\!+\!1),
\end{align}
and the next value is
\begin{align}
f(n\!+\!1) \;=\; [(n\!+\!1)^2\!-\!(n\!+\!1)\!+\!1][(n\!+\!1)^2\!+\!(n\!+\!1)\!+\!1] \;=\; (n^2\!+\!3n\!+\!3)(n^2\!+\!n\!+\!1).
\end{align}
Thus $f(n)$ and $f(n\!+\!1)$ have as their common factor at least the number $n^2\!+\!n\!+\!1$, which is $\geqq 3$.

Moreover, we may show that the greatest common divisor of $f(n)$ and $f(n\!+\!1)$ is $n^2\!+\!n\!+\!1$ except in the case\, $n \equiv 3 \pmod{7}$\, where it is $7(n^2\!+\!n\!+\!1)$.

Let $d$ be a common divisor, greater than 1, of the first factors $n^2\!-\!n\!+\!1$ and $n^2\!+\!3n\!+\!3$ of (1) and (2).\, It's clear that $d$ is \PMlinkname{odd}{OddNumber}.\, Then $d$ must divide the difference\, $(n^2\!+\!3n\!+\!3)-(n^2\!-\!n\!+\!1) = 2(2n\!+\!1)$\, and the sum\, $(n^2\!+\!3n\!+\!3)+3(n^2\!-\!n\!+\!1)= 4n^2\!+\!6$\, and hence also the difference\, $2(2n\!+\!1)+(4n^2\!+\!6)-(2n\!+\!1)^2 = 7$.\, It means that\, $d = 7$.\, Thus the only possible common prime factor of $n^2\!-\!n\!+\!1$ and $n^2\!+\!3n\!+\!3$ is 7.\, If we denote\, $n = 7k+r$\, where\, $r\in\{0,\,1,\,\ldots,\,6\}$,\, we see that
$$n^2\!-\!n\!+\!1 \;=\; 49k^2\!+\!14kr\!-\!7k\!+\!r^2\!-\!r\!+\!1 \;\;\overline{\equiv}\;\; r^2\!-\!r\!+\!1 \pmod{7},$$
$$n^2\!+\!3n\!+\!3 \;=\; 49k^2\!+\!14kr\!+\!21k\!+\!r^2\!+\!3r\!+\!3 \;\;\overline{\equiv}\;\; r^2\!+\!3r\!+\!3 \pmod{7}.$$
It's easy to check that the right hand \PMlinkescapetext{sides} of these polynomial congruences are divisible by 7 only if\, $r = 3$,\, i.e. if\, $n \equiv 3 \pmod{7}$.\\

\textbf{Note.}\, The sequence formed by the successive values 
of\, $\gcd(f(n), f(n\!+\!1))$ is number A111002 in Sloane's 
register (\PMlinkexternal{http://www.research.att.com/~njas/sequences/}{https://oeis.org/A111002}).
%%%%%
%%%%%
\end{document}
