\documentclass[12pt]{article}
\usepackage{pmmeta}
\pmcanonicalname{WeierstrassEquationOfAnEllipticCurve}
\pmcreated{2013-03-22 15:48:00}
\pmmodified{2013-03-22 15:48:00}
\pmowner{alozano}{2414}
\pmmodifier{alozano}{2414}
\pmtitle{Weierstrass equation of an elliptic curve}
\pmrecord{6}{37762}
\pmprivacy{1}
\pmauthor{alozano}{2414}
\pmtype{Definition}
\pmcomment{trigger rebuild}
\pmclassification{msc}{11G05}
\pmclassification{msc}{14H52}
\pmclassification{msc}{11G07}
\pmsynonym{Weierstrass model}{WeierstrassEquationOfAnEllipticCurve}
\pmdefines{Weierstrass equation}

\endmetadata

% this is the default PlanetMath preamble.  as your knowledge
% of TeX increases, you will probably want to edit this, but
% it should be fine as is for beginners.

% almost certainly you want these
\usepackage{amssymb}
\usepackage{amsmath}
\usepackage{amsthm}
\usepackage{amsfonts}

% used for TeXing text within eps files
%\usepackage{psfrag}
% need this for including graphics (\includegraphics)
%\usepackage{graphicx}
% for neatly defining theorems and propositions
%\usepackage{amsthm}
% making logically defined graphics
%%%\usepackage{xypic}

% there are many more packages, add them here as you need them

% define commands here

\newtheorem*{thm}{Theorem}
\newtheorem*{defn}{Definition}
\newtheorem{prop}{Proposition}
\newtheorem*{lemma}{Lemma}
\newtheorem*{cor}{Corollary}

\theoremstyle{definition}
\newtheorem{exa}{Example}

% Some sets
\newcommand{\Nats}{\mathbb{N}}
\newcommand{\Ints}{\mathbb{Z}}
\newcommand{\Reals}{\mathbb{R}}
\newcommand{\Complex}{\mathbb{C}}
\newcommand{\Rats}{\mathbb{Q}}
\newcommand{\Gal}{\operatorname{Gal}}
\newcommand{\Cl}{\operatorname{Cl}}
\begin{document}
Recall that an {\it elliptic curve} over a field $K$ is a projective nonsingular curve $E$ defined over $K$ of genus $1$ together with a point $O\in E$ defined over $K$. 

\begin{defn}
Let $K$ be an arbitrary field. A Weierstrass equation for an elliptic curve $E/K$ is an equation of the form:
$$y^2+a_1xy+a_3y=x^3+a_2x^2+a_4x+a_6$$
where $a_1, a_2, a_3,a_4,a_6$ are constants in $K$.
\end{defn}

All elliptic curves have a Weierstrass model in $\mathbb{P}^2(K)$, the projective plane over $K$. This is a simple application of the \PMlinkid{Riemann Roch theorem for curves}{RiemannRochTheorem}:

\begin{thm}
Let $E$ be an elliptic curve defined over a field $K$. Then there exists rational functions $x,y\in K(E)$ such that the map $\psi:E\to \mathbb{P}^2(K)$ sending $P$ to $[x(P),y(P),1]$ is an isomorphism of $E/K$ to the projective curve given by
$$ y^2+a_1xy+a_3y=x^3+a_2x^2+a_4x+a_6$$
where $a_1, a_2, a_3,a_4,a_6$ are constants in $K$.
\end{thm}

Moreover, the following proposition specifies any possible change of variables.
\begin{prop}
\label{change}
Let $E/K$ be an elliptic curve given by a Weierstrass model of the form:
$$ y^2+a_1xy+a_3y=x^3+a_2x^2+a_4x+a_6$$
with $a_i\in K$. Then:
\begin{enumerate}
\item The only change of variables $(x,y)\mapsto (x',y')$ preserving the projective point $[0,1,0]$ and  which also result in a Weierstrass equation, are of the form:
$$x=u^2x'+r,\quad y=u^3y'+su^2x'+t$$ 
with $u,r,s,t\in K$ and $u\neq 0$.

\item Any two Weierstrass equations for $E/K$ differ by a change of variables of the form given in $(1)$.
\end{enumerate}
\end{prop}

Once we have one Weierstrass model for a given elliptic curve $E/K$, and as long as the characteristic of $K$ is not $2$ or $3$, there exists a change of variables (of the form given in the previous proposition) which simplifies the model considerably.

\begin{cor}
Let $K$ be a field of characteristic different from $2$ or $3$. Let $E$ be an elliptic curve defined over $K$. Then there exists a Weierstrass model for $E$ of the form:
$$y^2=x^3+Ax+B$$
where $A,B$ are elements of $K$.
\end{cor}

Finally, remember that the $j$-invariant of an elliptic curve is invariant under isomorphism, but the discriminant depends on the model chosen.

\begin{prop}
Let $E/K$ be an elliptic curve and let
$$ E_1: y^2+a_1xy+a_3y=x^3+a_2x^2+a_4x+a_6, \quad E_2: y'^2+a_1x'y'+a_3y'=x'^3+a_2x'^2+a_4x'+a_6$$
be two distinct Weierstrass models for $E/K$. Then (by Prop. \ref{change}) there exists a change of variables $(x,y)\mapsto (x',y')$ of the form:
$$x=u^2x'+r,\quad y=u^3y'+su^2x'+t$$ 
with $u,r,s,t\in K$ and $u\neq 0$. Moreover, $j(E_1)=j(E_2)$, i.e. the $j$ invariants are equal ($j(E)$ is defined in \PMlinkid{this entry}{JInvariant}) and $\Delta(E_1)=u^{12}\Delta(E_2)$, where $\Delta(E_i)$ is the discriminant (as defined in \PMlinkid{here}{JInvariant}).
\end{prop}
%%%%%
%%%%%
\end{document}
