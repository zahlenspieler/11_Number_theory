\documentclass[12pt]{article}
\usepackage{pmmeta}
\pmcanonicalname{HermitianDotProductfiniteFields}
\pmcreated{2013-03-22 15:13:26}
\pmmodified{2013-03-22 15:13:26}
\pmowner{GrafZahl}{9234}
\pmmodifier{GrafZahl}{9234}
\pmtitle{Hermitian dot product (finite fields)}
\pmrecord{7}{36988}
\pmprivacy{1}
\pmauthor{GrafZahl}{9234}
\pmtype{Definition}
\pmcomment{trigger rebuild}
\pmclassification{msc}{11E39}
\pmclassification{msc}{12E20}
\pmsynonym{Hermitian dot product}{HermitianDotProductfiniteFields}
%\pmkeywords{inner product}
\pmrelated{FiniteField}
\pmdefines{conjugate (finite fields)}
\pmdefines{conjugation (finite fields)}

\endmetadata

% this is the default PlanetMath preamble.  as your knowledge
% of TeX increases, you will probably want to edit this, but
% it should be fine as is for beginners.

% almost certainly you want these
\usepackage{amssymb}
\usepackage{amsmath}
\usepackage{amsfonts}

% used for TeXing text within eps files
%\usepackage{psfrag}
% need this for including graphics (\includegraphics)
%\usepackage{graphicx}
% for neatly defining theorems and propositions
%\usepackage{amsthm}
% making logically defined graphics
%%%\usepackage{xypic}

% there are many more packages, add them here as you need them

% define commands here
\newcommand{\Prod}{\prod\limits}
\newcommand{\Sum}{\sum\limits}
\newcommand{\mbb}{\mathbb}
\newcommand{\mbf}{\mathbf}
\newcommand{\mc}{\mathcal}
\newcommand{\ol}{\overline}

% Math Operators/functions
\DeclareMathOperator{\Frob}{Frob}
\DeclareMathOperator{\cwe}{cwe}
\DeclareMathOperator{\we}{we}
\DeclareMathOperator{\wt}{wt}
\begin{document}
Let $q$ be an \PMlinkid{even}{4703} \PMlinkid{prime}{438} power (in particular, $q$ is a square) and
$\mbb{F}_q$ the finite field with $q$ elements. Then
$\mbb{F}_{\sqrt{q}}$ is a subfield of $\mbb{F}_q$. The
\emph{\PMlinkescapetext{conjugate}} $\ol{k}$ of an element $k\in\mbb{F}_q$ is defined by
the
$\sqrt{q}$-th power Frobenius map
\begin{equation*}
\ol{k}:=\Frob_{\sqrt{q}}(k)=k^{\sqrt{q}}.
\end{equation*}

The \PMlinkescapetext{conjugate} has properties similar to the complex conjugate. Let
$k_1,k_2\in\mbb{F}_q$, then
\begin{enumerate}
\item $\ol{k_1+k_2}=\ol{k_1}+\ol{k_2}$,
\item $\ol{k_1k_2}=\ol{k_1}\,\ol{k_2}$,
\item $\ol{\ol{k_1}}=k_1$.
\end{enumerate}
Properties 1 and 2 hold because the Frobenius map is a
\PMlinkid{homomorphism}{1011}.
Property 3 holds because of the identity $k^q=k$ which
holds for any $k$ in any finite field with $q$ elements.
See also \emph{\PMlinkid{finite field}{2893}}.

Now let $\mbb{F}_q^n$ be the $n$-dimensional vector space over
$\mbb{F}_q$, then the \emph{Hermitian dot product} of two vectors
$(u_1,\ldots,u_n),(v_1,\ldots,v_n)\in\mbb{F}_q^n$ is
\begin{equation*}
(u_1,\ldots,u_n)\cdot(v_1,\ldots,v_n):=\Sum_{i=1}^nu_i\ol{v_i}.
\end{equation*}

Again, this kind of Hermitian dot product has properties similar to
Hermitian inner products on complex vector spaces. Let
$k_1,k_2\in\mbb{F}_q$ and $v_1,v_2,v,,w\in\mbb{F}_q^n$, then
\begin{enumerate}
\item $(k_1v_1+k_2v_2)\cdot w=k_1(v_1\cdot w)+k_2(v_2\cdot w)$
  (linearity)
\item $v\cdot w=\ol{w\cdot v}$
\item $v\cdot v\in\mbb{F}_{\sqrt{q}}$
\end{enumerate}
Property 3 follows since $\sqrt{q}+1$ divides $q-1$ (see \emph{\PMlinkid{finite field}{2893}}).
%%%%%
%%%%%
\end{document}
