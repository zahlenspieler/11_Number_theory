\documentclass[12pt]{article}
\usepackage{pmmeta}
\pmcanonicalname{BibliographyForKtheoryAndVectorBundles}
\pmcreated{2013-03-22 19:31:18}
\pmmodified{2013-03-22 19:31:18}
\pmowner{bci1}{20947}
\pmmodifier{bci1}{20947}
\pmtitle{bibliography for K-theory and vector bundles}
\pmrecord{12}{42497}
\pmprivacy{1}
\pmauthor{bci1}{20947}
\pmtype{Bibliography}
\pmcomment{trigger rebuild}
\pmclassification{msc}{11S70}
\pmclassification{msc}{11G45}
\pmclassification{msc}{19E08}
\pmclassification{msc}{18G60}
\pmclassification{msc}{18-00}
\pmclassification{msc}{19D55}
\pmclassification{msc}{57R67}
\pmclassification{msc}{57R65}
\pmclassification{msc}{46L80}
\pmclassification{msc}{14C35}
\pmclassification{msc}{46L80}
\pmclassification{msc}{16E20}
\pmclassification{msc}{18F25}
\pmclassification{msc}{11R70}

% this is the default PlanetMath preamble. as your knowledge
% of TeX increases, you will probably want to edit this, but
\usepackage{amsmath, amssymb, amsfonts, amsthm, amscd, latexsym}
%%\usepackage{xypic}
\usepackage[mathscr]{eucal}
% define commands here
\theoremstyle{plain}
\newtheorem{lemma}{Lemma}[section]
\newtheorem{proposition}{Proposition}[section]
\newtheorem{theorem}{Theorem}[section]
\newtheorem{corollary}{Corollary}[section]
\theoremstyle{definition}
\newtheorem{definition}{Definition}[section]
\newtheorem{example}{Example}[section]
%\theoremstyle{remark}
\newtheorem{remark}{Remark}[section]
\newtheorem*{notation}{Notation}
\newtheorem*{claim}{Claim}
\renewcommand{\thefootnote}{\ensuremath{\fnsymbol{footnote%%@
}}}
\numberwithin{equation}{section}
\newcommand{\Ad}{{\rm Ad}}
\newcommand{\Aut}{{\rm Aut}}
\newcommand{\Cl}{{\rm Cl}}
\newcommand{\Co}{{\rm Co}}
\newcommand{\DES}{{\rm DES}}
\newcommand{\Diff}{{\rm Diff}}
\newcommand{\Dom}{{\rm Dom}}
\newcommand{\Hol}{{\rm Hol}}
\newcommand{\Mon}{{\rm Mon}}
\newcommand{\Hom}{{\rm Hom}}
\newcommand{\Ker}{{\rm Ker}}
\newcommand{\Ind}{{\rm Ind}}
\newcommand{\IM}{{\rm Im}}
\newcommand{\Is}{{\rm Is}}
\newcommand{\ID}{{\rm id}}
\newcommand{\GL}{{\rm GL}}
\newcommand{\Iso}{{\rm Iso}}
\newcommand{\Sem}{{\rm Sem}}
\newcommand{\St}{{\rm St}}
\newcommand{\Sym}{{\rm Sym}}
\newcommand{\SU}{{\rm SU}}
\newcommand{\Tor}{{\rm Tor}}
\newcommand{\U}{{\rm U}}
\newcommand{\A}{\mathcal A}
\newcommand{\Ce}{\mathcal C}
\newcommand{\D}{\mathcal D}
\newcommand{\E}{\mathcal E}
\newcommand{\F}{\mathcal F}
\newcommand{\G}{\mathcal G}
\newcommand{\Q}{\mathcal Q}
\newcommand{\R}{\mathcal R}
\newcommand{\cS}{\mathcal S}
\newcommand{\cU}{\mathcal U}
\newcommand{\W}{\mathcal W}
\newcommand{\bA}{\mathbb{A}}
\newcommand{\bB}{\mathbb{B}}
\newcommand{\bC}{\mathbb{C}}
\newcommand{\bD}{\mathbb{D}}
\newcommand{\bE}{\mathbb{E}}
\newcommand{\bF}{\mathbb{F}}
\newcommand{\bG}{\mathbb{G}}
\newcommand{\bK}{\mathbb{K}}
\newcommand{\bM}{\mathbb{M}}
\newcommand{\bN}{\mathbb{N}}
\newcommand{\bO}{\mathbb{O}}
\newcommand{\bP}{\mathbb{P}}
\newcommand{\bR}{\mathbb{R}}
\newcommand{\bV}{\mathbb{V}}
\newcommand{\bZ}{\mathbb{Z}}
\newcommand{\bfE}{\mathbf{E}}
\newcommand{\bfX}{\mathbf{X}}
\newcommand{\bfY}{\mathbf{Y}}
\newcommand{\bfZ}{\mathbf{Z}}
\renewcommand{\O}{\Omega}
\renewcommand{\o}{\omega}
\newcommand{\vp}{\varphi}
\newcommand{\vep}{\varepsilon}
\newcommand{\diag}{{\rm diag}}
\newcommand{\grp}{{\mathbb G}}
\newcommand{\dgrp}{{\mathbb D}}
\newcommand{\desp}{{\mathbb D^{\rm{es}}}}
\newcommand{\Geod}{{\rm Geod}}
\newcommand{\geod}{{\rm geod}}
\newcommand{\hgr}{{\mathbb H}}
\newcommand{\mgr}{{\mathbb M}}
\newcommand{\ob}{{\rm Ob}}
\newcommand{\obg}{{\rm Ob(\mathbb G)}}
\newcommand{\obgp}{{\rm Ob(\mathbb G')}}
\newcommand{\obh}{{\rm Ob(\mathbb H)}}
\newcommand{\Osmooth}{{\Omega^{\infty}(X,*)}}
\newcommand{\ghomotop}{{\rho_2^{\square}}}
\newcommand{\gcalp}{{\mathbb G(\mathcal P)}}
\newcommand{\rf}{{R_{\mathcal F}}}
\newcommand{\glob}{{\rm glob}}
\newcommand{\loc}{{\rm loc}}
\newcommand{\TOP}{{\rm TOP}}
\newcommand{\wti}{\widetilde}
\newcommand{\what}{\widehat}
\renewcommand{\a}{\alpha}
\newcommand{\be}{\beta}
\newcommand{\ga}{\gamma}
\newcommand{\Ga}{\Gamma}
\newcommand{\de}{\delta}
\newcommand{\del}{\partial}
\newcommand{\ka}{\kappa}
\newcommand{\si}{\sigma}
\newcommand{\ta}{\tau}
\newcommand{\lra}{{\longrightarrow}}
\newcommand{\ra}{{\rightarrow}}
\newcommand{\rat}{{\rightarrowtail}}
\newcommand{\oset}[1]{\overset {#1}{\ra}}
\newcommand{\osetl}[1]{\overset {#1}{\lra}}
\newcommand{\hr}{{\hookrightarrow}}

\begin{document}
\section{Bibliography for K-Theory and vector bundles}

\subsection{Books}

M. F. Atiyah, K--Theory, W. A. Benjamin, 1967.

R. Bott, Lectures on $K(X)$, W. A. Benjamin, 1969.

J. $Dieudonn\'{e}$, A History of Algebraic and Differential Topology 1900-1960, 
$Birkh\"{a}user$, 1989.

W. Fulton, Young Tableaux, Cambridge Univ. Press, 1997.

H. Hiller, Geometry of Coxeter Groups, Pitman, 1982.

D. Husemoller, Fibre Bundles, McGraw--Hill, 1966 (later editions by Springer-Verlag).

M. Karoubi, K--Theory: An Introduction, Springer--Verlag, 1978.

S. Lang, Algebra, Addison--Wesley, 1965.

J. W. Milnor and J. D. Stasheff, Characteristic Classes, Princeton Univ. Press, 1974.

N. Steenrod, Topology of Fiber Bundles, Princeton Univ. Press, 1951.

R. Stong, Notes on Cobordism Theory, Princeton Univ. Press, 1968.

B. L. van der Waerden, Modern Algebra, Ungar, 1949.

\subsection{Papers}

J. F. Adams, On the non--existence of elements of Hopf invariant one, Ann. of Math. 72 (1960),20--104.

J. F. Adams, Vector fields on spheres, Ann. of Math. 75 (1962), 603--632.

J. F. Adams, On the groups $J(X)$, IV, Topology 5 (1966), 21--71.

J. F. Adams and M. F. Atiyah, K--theory and the Hopf invariant, Quart. J. Math. 17 (1966), 31--38.

M. Atiyah and F. Hirzebruch, Vector Bundles and Homogeneous Spaces, Proc. Sym. Pure Math. III (1961), 7--38.

M. Atiyah, Vector bundles and the K\"unneth formula, Topology 1 (1962), 245--248.

M. Atiyah, Power Operations in K--Theory, Quart. J. Math. Oxford. 17 (1966), 
165--193.

M. Atiyah, K--Theory and reality, Quart. J. Math. Oxford. 17 (1966), 367--386.

M. Atiyah and R. Bott, On the periodicity theorem for complex vector bundles, Acta Math.112 (1964), 229--247.

M. Atiyah, R. Bott and A. Shapiro, Clifford Modules, Topology 3 (1964), 3--38.

R. Bott and J. Milnor, On the parallelizability of spheres, Bull. A.M.S. 64 (1958), 87--89.

R. Bott, The stable homotopy of the classical groups, Ann. of Math. 70 (1959), 313--337.

A. Hatcher. Vector Bundles and K-Theory. Web-book, 110 pages, May 2009. $http://aux.pp-dev.org:8080/files/books/413/VBKtheoryHatcher.pdf$

L. Hodgkin, On the K--Theory of Lie groups, Topology. 6 (1967), 1--36.

M. Mahowald, The order of the image of the $J$--homomorphism, Bull. A.M.S. 76 (1970), 1310--1313.


J. Milnor, Some consequences of a theorem of Bott, Ann. of Math. 68 (1958), 
444--449.

D. Quillen, The Adams conjecture, Topology 10 (1971), 67--80.

D. Sullivan, Genetics of homotopy theory and the Adams conjecture, Ann. of Math. 100 (1974),1--79.

L. M. Woodward, Vector fields on spheres and a generalization, Quart. J. Math. Oxford. 24 (1973), 357--366.

\subsection{K--Theory MSC Classifications}

18--XX                   Category theory; homological algebra For commutative rings see 13Dxx, for associative rings \\ 
16Exx, for groups 20Jxx, for topological groups and related structures 57Txx; see also 55Nxx and 55Uxx for algebraic  topology \\
18Axx           General theory of categories and functors\\
18Fxx           Categories and geometry\\
18F25   Algebraic $K$--theory and $L$--theory [See also 11Exx, 11R70, 11S70, 12-XX, 13D15, 14Cxx, 16E20, 19-XX, 46L80, 57R65, 57R67]\\
19--XX                   $K$--theory [See also 16E20, 18F25]\\

19Dxx           Higher algebraic $K$--theory \\
19D10   Algebraic $K$-theory of spaces

19D25   Karoubi-Villamayor-Gersten $K$--theory

19D35   Negative $K$-theory, NK and Nil

19D45   Higher symbols, Milnor $K$--theory

19D50   Computations of higher $K$--theory of rings [See also 13D15, 16E20]

19D55   $K$-theory and homology; cyclic homology and cohomology [See also 18G60]

19Exx           $K$--theory in geometry

19E08   $K$--theory of schemes [See also 14C35]

19Fxx           $K$-theory in number theory [See also 11R70, 11S70]

19F05   Generalized class field theory [See also 11G45]

19Gxx           $K$-theory of forms [See also 11Exx]

19G24   $L$-theory of group rings [See also 11E81]

19G38   Hermitian $K$-theory, relations with $K$-theory of rings

19Kxx           $K$-theory and operator algebras [See mainly 46L80, and also 46M20]

19K35   Kasparov theory ($KK$--theory) [See also 58J22]

19K56   Index theory [See also 58J20, 58J22]

19Lxx           Topological $K$--theory [See also 55N15, 55R50, 55S25]

19L41   Connective $K$--theory, cobordism [See also 55N22]

19L47   Equivariant $K$--theory [See also 55N91, 55P91, 55Q91, 55R91, 55S91]

19L50   Twisted $K$-theory; differential $K$--theory

19Mxx           Miscellaneous applications of $K$-theory

19M05   Miscellaneous applications of $K$-theory


$http://www.ams.org/mathscinet/msc/msc2010.html?t=&s=K-theory&btn=Search&ls=s$

\PMlinkexternal{AMS MSC Classification Search engine}{http://www.ams.org/mathscinet/msc/msc2010.html}

$11Exx, 11R70, 11S70, 12-XX, 13D15, 14Cxx, 16E20, 19-XX, 46L80, 57R65, 57R67$

%%%%%
%%%%%
\end{document}
