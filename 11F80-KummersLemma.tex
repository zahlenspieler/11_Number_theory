\documentclass[12pt]{article}
\usepackage{pmmeta}
\pmcanonicalname{KummersLemma}
\pmcreated{2013-03-22 15:55:21}
\pmmodified{2013-03-22 15:55:21}
\pmowner{alozano}{2414}
\pmmodifier{alozano}{2414}
\pmtitle{Kummer's lemma}
\pmrecord{6}{37929}
\pmprivacy{1}
\pmauthor{alozano}{2414}
\pmtype{Theorem}
\pmcomment{trigger rebuild}
\pmclassification{msc}{11F80}
\pmclassification{msc}{14H52}
\pmclassification{msc}{11D41}

% this is the default PlanetMath preamble.  as your knowledge
% of TeX increases, you will probably want to edit this, but
% it should be fine as is for beginners.

% almost certainly you want these
\usepackage{amssymb}
\usepackage{amsmath}
\usepackage{amsthm}
\usepackage{amsfonts}

% used for TeXing text within eps files
%\usepackage{psfrag}
% need this for including graphics (\includegraphics)
%\usepackage{graphicx}
% for neatly defining theorems and propositions
%\usepackage{amsthm}
% making logically defined graphics
%%%\usepackage{xypic}

% there are many more packages, add them here as you need them

% define commands here

\newtheorem*{thm}{Theorem}
\newtheorem{defn}{Definition}
\newtheorem{prop}{Proposition}
\newtheorem{lemma}{Lemma}
\newtheorem{cor}{Corollary}

\theoremstyle{definition}
\newtheorem{exa}{Example}

% Some sets
\newcommand{\Nats}{\mathbb{N}}
\newcommand{\Ints}{\mathbb{Z}}
\newcommand{\Reals}{\mathbb{R}}
\newcommand{\Complex}{\mathbb{C}}
\newcommand{\Rats}{\mathbb{Q}}
\newcommand{\Gal}{\operatorname{Gal}}
\newcommand{\Cl}{\operatorname{Cl}}
\begin{document}
The following result is a key ingredient in the proof of Fermat's last theorem for regular primes. More concretely, the lemma is needed to show the so-called second case of Fermat, i.e. $x^p+y^p=z^p$ does not have any non-trivial solutions in $\Ints$ with $p>2$ a regular prime and $p|xyz$. It is due to Ernst Kummer, thus the name.

\begin{thm}[Kummer's Lemma]
Let $p>2$ be a prime, let $\zeta_p$ be a primitive $p$th root of unity and let $K=\Rats(\zeta_p)$ be the corresponding cyclotomic field. Let $E$ be the group of algebraic units of the ring of integers $\mathcal{O}_K$. Suppose that $p$ is a regular prime. If a unit $\epsilon\in E$ is congruent modulo $p$ to a rational integer, then $\epsilon$ is the $p$th power of another unit also $E$.
\end{thm}

For a proof, see \cite{wash}, Theorem 5.36. The reader may also be interested in generalizations due to \cite{wash2} and \cite{ozaki}.

\begin{thebibliography}{Washington 1992}
\bibitem[Ozaki 1997]{ozaki} Ozaki, M., {\em Kummer's lemma for $\Ints_p$-extensions over totally real number fields},  Acta Arith.  81  (1997),  no. 1, 37--44.

\bibitem[Washington]{wash} Washington L. C., {\em Introduction to Cyclotomic
Fields}, Second Edition, Springer-Verlag, New York.

\bibitem[Washington 1992]{wash2} Washington, L. C., {\em Kummer's lemma for prime power cyclotomic fields},  J. Number Theory  40  (1992),  no. 2, 165--173.
\end{thebibliography}
%%%%%
%%%%%
\end{document}
