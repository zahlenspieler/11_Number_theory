\documentclass[12pt]{article}
\usepackage{pmmeta}
\pmcanonicalname{OppositePolynomial}
\pmcreated{2013-03-22 14:47:41}
\pmmodified{2013-03-22 14:47:41}
\pmowner{pahio}{2872}
\pmmodifier{pahio}{2872}
\pmtitle{opposite polynomial}
\pmrecord{9}{36447}
\pmprivacy{1}
\pmauthor{pahio}{2872}
\pmtype{Definition}
\pmcomment{trigger rebuild}
\pmclassification{msc}{11C08}
\pmclassification{msc}{12E05}
\pmclassification{msc}{13P05}
%\pmkeywords{coefficient}
\pmrelated{OppositeNumber}
\pmrelated{Unity}
\pmrelated{BasicPolynomial}
\pmrelated{MinimalPolynomialEndomorphism}

\endmetadata

% this is the default PlanetMath preamble.  as your knowledge
% of TeX increases, you will probably want to edit this, but
% it should be fine as is for beginners.

% almost certainly you want these
\usepackage{amssymb}
\usepackage{amsmath}
\usepackage{amsfonts}

% used for TeXing text within eps files
%\usepackage{psfrag}
% need this for including graphics (\includegraphics)
%\usepackage{graphicx}
% for neatly defining theorems and propositions
%\usepackage{amsthm}
% making logically defined graphics
%%%\usepackage{xypic}

% there are many more packages, add them here as you need them

% define commands here
\begin{document}
The {\em opposite polynomial} of a polynomial $P$ in a polynomial ring $R[X]$ is a polynomial \,$-P$\, such that
                     $$P\!+\!(-P) \;=\; \textbf{0},$$
where $\textbf{0}$ denotes the zero polynomial. \,It is clear that \,$-P$\, is obtained by changing the signs of all of the coefficients of $P$, \PMlinkname{i.e.}{Ie}
  $$-\sum_{\nu = 0}^n a_\nu X^\nu \;=\; \sum_{\nu = 0}^n (-a_\nu)X^\nu.$$

The opposite polynomial may be used to define subtraction of polynomials:
                         $$P\!-\!Q \;=:\; P\!+\!(-Q)$$

Forming the opposite polynomial is a linear mapping \,\,$R[X]\to R[X]$.
%%%%%
%%%%%
\end{document}
