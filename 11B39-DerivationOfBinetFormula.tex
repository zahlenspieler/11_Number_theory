\documentclass[12pt]{article}
\usepackage{pmmeta}
\pmcanonicalname{DerivationOfBinetFormula}
\pmcreated{2013-03-22 15:03:50}
\pmmodified{2013-03-22 15:03:50}
\pmowner{drini}{3}
\pmmodifier{drini}{3}
\pmtitle{derivation of Binet formula}
\pmrecord{4}{36784}
\pmprivacy{1}
\pmauthor{drini}{3}
\pmtype{Derivation}
\pmcomment{trigger rebuild}
\pmclassification{msc}{11B39}

\usepackage{graphicx}
%%%\usepackage{xypic} 
\usepackage{bbm}
\newcommand{\Z}{\mathbbmss{Z}}
\newcommand{\C}{\mathbbmss{C}}
\newcommand{\R}{\mathbbmss{R}}
\newcommand{\Q}{\mathbbmss{Q}}
\newcommand{\mathbb}[1]{\mathbbmss{#1}}
\newcommand{\figura}[1]{\begin{center}\includegraphics{#1}\end{center}}
\newcommand{\figuraex}[2]{\begin{center}\includegraphics[#2]{#1}\end{center}}
\newtheorem{dfn}{Definition}
\begin{document}
The characteristic polynomial for the Fibonacci recurrence $f_n = f_{n-1}+f_{n-2}$ is
\[
x^2 = x +1.
\]

The solutions of the characteristic equation $x^2-x-1=0$ are 
\[
\phi=\frac{1+\sqrt5}2,\qquad \psi=\frac{1-\sqrt5}2
\]
so the closed formula for the Fibonacci sequence must be of the form
\[
f_n = u\phi^n +v\psi^n
\]
for some real numbers $u,v$. Now we use the boundary conditions of the recurrence, that is, $f_0=0, f_1=1$, which means we have to solve the system
\[
0=u \phi^0 +v\psi^0, \qquad 1=u\phi^1 + v\psi^1
\]
The first equation simplifies to $u=-v$ and substituting into the second one gives:
\[
1=u\left(\frac{1+\sqrt5}2\right) - u\left(\frac{1-\sqrt5}2\right) = u\left(\frac{2\sqrt{5}}2\right)=u\sqrt{5}.
\]

Therefore
\[
u=\frac{1}{\sqrt5},\qquad v=\frac{-1}{\sqrt5}
\]
and so
\[
f_n = \frac{\phi^n}{\sqrt5}- \frac{\psi^n}{\sqrt5}=\frac{\phi^n-\psi^n}{\sqrt5}.
\]
%%%%%
%%%%%
\end{document}
