\documentclass[12pt]{article}
\usepackage{pmmeta}
\pmcanonicalname{BaselProblemDerivation}
\pmcreated{2013-03-22 18:40:50}
\pmmodified{2013-03-22 18:40:50}
\pmowner{curious}{18562}
\pmmodifier{curious}{18562}
\pmtitle{Basel problem derivation}
\pmrecord{5}{41431}
\pmprivacy{1}
\pmauthor{curious}{18562}
\pmtype{Derivation}
\pmcomment{trigger rebuild}
\pmclassification{msc}{11A25}

% this is the default PlanetMath preamble.  as your knowledge
% of TeX increases, you will probably want to edit this, but
% it should be fine as is for beginners.

% almost certainly you want these
\usepackage{amssymb}
\usepackage{amsmath}
\usepackage{amsfonts}

% used for TeXing text within eps files
%\usepackage{psfrag}
% need this for including graphics (\includegraphics)
%\usepackage{graphicx}
% for neatly defining theorems and propositions
%\usepackage{amsthm}
% making logically defined graphics
%%%\usepackage{xypic}

% there are many more packages, add them here as you need them

% define commands here

\begin{document}
The basis for this derivation is the assumption that the properties of finite polynomials hold true for infinite series. First, consider the Taylor series expansion of the sine function:

$\displaystyle \sin x = \sum_{n = 0}^{\infty} \frac{(-1)^n}{(2n + 1)!} x^{2n + 1} = x - \frac{x^3}{3!} + \frac{x^5}{5!} - \frac{x^7}{7!} + \cdots$

Dividing through by x, we get:

$\displaystyle \frac{\sin x}{x} = 1 - \frac{x^2}{3!} + \frac{x^4}{5!} - \frac{x^6}{7!} + \cdots$

The roots of the function $\sin x/x$ occur at $x = n\pi$, where $n = \pm 1, \pm 2, \pm 3, \ldots$.
Let us assume that we can express this infinite series as a product of linear factors given by its roots, just as we do for finite polynomials:

\begin{eqnarray*}
\frac{\sin x}{x} &= \left( 1 - \frac{x}{\pi} \right)\left( 1 + \frac{x}{\pi} \right)\left( 1 - \frac{x}{2\pi} \right)\left( 1 + \frac{x}{2\pi} \right)\left( 1 - \frac{x}{3\pi} \right)\left( 1 + \frac{x}{3\pi} \right)\cdots\\
&= \left( 1 - \frac{x^2}{\pi^2} \right)\left( 1 - \frac{x^2}{4\pi^2} \right)\left( 1 - \frac{x^2}{9\pi^2} \right)\cdots.
\end{eqnarray*}

If we multiply out this product and collect all the $x^2$ terms, we get:

$$\displaystyle -\left(\frac{1}{\pi^2} + \frac{1}{4\pi^2} + \frac{1}{9\pi^2} + \cdots \right) = -\frac{1}{\pi^2} \sum_{n=1}^{\infty}\frac{1}{n^2}$$

From the original infinite series of $\sin x/x$ the coefficient of $x^2$ is $-1/3!$. Equating both expressions gives:

$$\displaystyle -\frac{1}{6} = -\frac{1}{\pi^2}\sum_{n=1}^{\infty}\frac{1}{n^2}$$

$$\displaystyle \sum_{n=1}^{\infty}\frac{1}{n^2} = \frac{\pi^2}{6}$$

This was the method Euler used to solve the Basel problem. There are more rigorous and modern derivations, but this one is a simple, straight forward, and easily understood one.

%%%%%
%%%%%
\end{document}
