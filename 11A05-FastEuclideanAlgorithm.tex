\documentclass[12pt]{article}
\usepackage{pmmeta}
\pmcanonicalname{FastEuclideanAlgorithm}
\pmcreated{2013-03-22 14:28:52}
\pmmodified{2013-03-22 14:28:52}
\pmowner{mathcam}{2727}
\pmmodifier{mathcam}{2727}
\pmtitle{fast Euclidean algorithm}
\pmrecord{7}{36014}
\pmprivacy{1}
\pmauthor{mathcam}{2727}
\pmtype{Algorithm}
\pmcomment{trigger rebuild}
\pmclassification{msc}{11A05}
\pmsynonym{Half-GCD Algorithm}{FastEuclideanAlgorithm}
\pmsynonym{Lehmer's Algorithm}{FastEuclideanAlgorithm}
%\pmkeywords{Lehmer}
%\pmkeywords{Euclidean Algorithm}
%\pmkeywords{GCD}
%\pmkeywords{Donald Knuth}
\pmrelated{EuclidsAlgorithm}

% this is the default PlanetMath preamble.  as your knowledge
% of TeX increases, you will probably want to edit this, but
% it should be fine as is for beginners.

% almost certainly you want these
\usepackage{amssymb}
\usepackage{amsmath}
\usepackage{amsfonts}

% used for TeXing text within eps files
%\usepackage{psfrag}
% need this for including graphics (\includegraphics)
%\usepackage{graphicx}
% for neatly defining theorems and propositions
%\usepackage{amsthm}
% making logically defined graphics
%%%\usepackage{xypic}

% there are many more packages, add them here as you need them

% define commands here
\begin{document}
Given two polynomials of degree $n$ with coefficients from a field $K$, the straightforward Eucliean Algorithm uses $O(n^2)$ field operations to compute their greatest common divisor. The Fast Euclidean Algorithm computes the same GCD in $O(\mathsf{M}(n) \log(n))$ field operations, where $\mathsf{M}(n)$ is the time to multiply two $n$-degree polynomials; with FFT multiplication the GCD can thus be computed in time $O(n \log^2(n) \log(\log(n)))$. The algorithm can also be used to compute any particular pair of coefficients from the Extended Euclidean Algorithm, although computing every pair of coefficients would involve $O(n^2)$ outputs and so the efficiency is not as helpful when all are needed.

The algorithm can be made to work over $\mathbb{Z}$ but it is very tricky. A newer version that is easier to understand was published by Damien Stehl\'e and Paul Zimmerman, ``A Binary Recursive Gcd Algorithm.''

Here we sketch the algorithm over $K[x]$. The basic idea is that the quotients $q_i$ computed by the Euclidean Algorithm can usually be computed by looking at only the first few coefficients of the polynomial; for example, if
$$A(x) = a_nx^n + a_{n-1}x^{n-1} + \ldots +a_0, \quad B(x) = b_{n-1}x^{n-1} + \ldots +b_0$$
then
$$quo(A(x), B(x)) =  \frac{a_n}{b_{n-1}}x+\frac{b_{n-1}a_{n-1}-a_nb_{n-2}}{b_{n-1}^2}$$

With more detailed analysis, we can show that in fact a divide-and-conquer approach can be used to calculate the GCD. First, we remove the terms whose degree is $n/2$ or less from both polynomials $A$ and $B$. Then, we recursively compute their GCD and Euclidean coefficients. We then apply the Euclidean coefficients to $A$ and $B$, and recursively complete the Euclidean Algorithm. 

The full algorithm, and a comprehensive runtime analysis is given in ``Modern Computer Algebra'' by von zur Gathen and Gerhard.
%%%%%
%%%%%
\end{document}
