\documentclass[12pt]{article}
\usepackage{pmmeta}
\pmcanonicalname{SummedNumeratorAndSummedDenominator}
\pmcreated{2013-10-11 15:35:42}
\pmmodified{2013-10-11 15:35:42}
\pmowner{pahio}{2872}
\pmmodifier{pahio}{2872}
\pmtitle{summed numerator and summed denominator}
\pmrecord{11}{37203}
\pmprivacy{1}
\pmauthor{pahio}{2872}
\pmtype{Theorem}
\pmcomment{Remark added}
\pmclassification{msc}{11A99}
\pmrelated{InequalityForRealNumbers}

\endmetadata

% this is the default PlanetMath preamble.  as your knowledge
% of TeX increases, you will probably want to edit this, but
% it should be fine as is for beginners.

% almost certainly you want these
\usepackage{amssymb}
\usepackage{amsmath}
\usepackage{amsfonts}

% used for TeXing text within eps files
%\usepackage{psfrag}
% need this for including graphics (\includegraphics)
%\usepackage{graphicx}
% for neatly defining theorems and propositions
 \usepackage{amsthm}
% making logically defined graphics
%%%\usepackage{xypic}

% there are many more packages, add them here as you need them

% define commands here

\theoremstyle{definition}
\newtheorem*{thmplain}{Theorem}

\begin{document}
If\, $\displaystyle\frac{a_1}{b_1},\,\ldots,\,\frac{a_n}{b_n}$\, are any  real fractions with positive denominators and
 $$m \;:=\; \min\left\{\frac{a_1}{b_1},\,\ldots,\,\frac{a_n}{b_n}\right\}, \quad 
   M \;:=\; \max\left\{\frac{a_1}{b_1},\,\ldots,\,\frac{a_n}{b_n}\right\}$$
are \PMlinkname{the least and the greatest}{MinimalAndMaximalNumber} of the fractions, then
\begin{align}
m \;\leqq\; \frac{a_1\!+\!\ldots\!+\!a_n}{b_1\!+\!\ldots\!+\!b_n} \;\leqq\; M.
\end{align}
The equality signs are valid if and only if all fractions are equal; in this case one has 
 $$\frac{a_1}{b_1} \;=\; \ldots \;=\; \frac{a_n}{b_n} \;=\; \frac{a_1\!+\!\ldots+\!a_n}{b_1\!+\!\ldots\!+\!b_n}.$$
{\em Proof.}\, Set\; $\displaystyle q_1 := \frac{a_1}{b_1}$,\, \ldots,\; $\displaystyle q_n := \frac{a_n}{b_n}$.\, Then we have\, 
$a_1\!+\!\ldots\!+\!a_n = b_1q_1\!+\!\ldots\!+\!b_nq_n$,\, which apparently has the lower bound\, $(b_1\!+\cdots+\!b_n)m$\, and the upper bound\, $(b_1\!+\ldots+\!b_n)M$.\, Dividing the three last expressions by the sum\,  $b_1\!+\ldots+\!b_n$\, yields the asserted double inequality (1).


\textbf{Remark.}\, Cf. also the mediant.
%%%%%
%%%%%
\end{document}
