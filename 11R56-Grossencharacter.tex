\documentclass[12pt]{article}
\usepackage{pmmeta}
\pmcanonicalname{Grossencharacter}
\pmcreated{2013-03-22 15:45:19}
\pmmodified{2013-03-22 15:45:19}
\pmowner{alozano}{2414}
\pmmodifier{alozano}{2414}
\pmtitle{gr\"ossencharacter}
\pmrecord{5}{37709}
\pmprivacy{1}
\pmauthor{alozano}{2414}
\pmtype{Definition}
\pmcomment{trigger rebuild}
\pmclassification{msc}{11R56}
\pmrelated{GrossencharacterAssociatedToACMEllipticCurve}
\pmdefines{grossencharacter}

\endmetadata

% this is the default PlanetMath preamble.  as your knowledge
% of TeX increases, you will probably want to edit this, but
% it should be fine as is for beginners.

% almost certainly you want these
\usepackage{amssymb}
\usepackage{amsmath}
\usepackage{amsthm}
\usepackage{amsfonts}

% used for TeXing text within eps files
%\usepackage{psfrag}
% need this for including graphics (\includegraphics)
%\usepackage{graphicx}
% for neatly defining theorems and propositions
%\usepackage{amsthm}
% making logically defined graphics
%%%\usepackage{xypic}

% there are many more packages, add them here as you need them

% define commands here

\newtheorem{thm}{Theorem}
\newtheorem{defn}{Definition}
\newtheorem{prop}{Proposition}
\newtheorem{lemma}{Lemma}
\newtheorem{cor}{Corollary}

\theoremstyle{definition}
\newtheorem{exa}{Example}

% Some sets
\newcommand{\Nats}{\mathbb{N}}
\newcommand{\Ints}{\mathbb{Z}}
\newcommand{\Reals}{\mathbb{R}}
\newcommand{\Complex}{\mathbb{C}}
\newcommand{\Rats}{\mathbb{Q}}
\newcommand{\Gal}{\operatorname{Gal}}
\newcommand{\Cl}{\operatorname{Cl}}
\begin{document}
Let $K$ be a number field and let $A_K$ be idele group of $K$, i.e.

$$A_K={\prod_\nu}' K_\nu^\ast$$
where the product is a restricted direct product running over all places (infinite and finite) of $K$ (see entry on \PMlinkid{ideles}{Idele}). Recall that $K^\ast$ embeds into $A_K$ diagonally:
$$x\in K^\ast \mapsto (x_\nu)_\nu$$
where $x_\nu$ is the image of $x$ under the embedding of $K$ into its completion at the place $\nu$, $K_\nu$.

\begin{defn}
A Gr\"ossencharacter $\psi$ on $K$ is a continuous homomorphism:
$$\psi:A_K \longrightarrow \Complex^\ast$$
which is trivial on $K^\ast$, i.e. if $x\in K^\ast$ then $\psi((x_\nu)_\nu)=1$. We say that $\psi$ is unramified at a prime $\wp$ of $K$ if $\psi(\mathcal{O}_\wp^\ast)=1$, where $\mathcal{O}_\wp$ is the ring of integers inside $K_\wp$. Otherwise we say that $\psi$ is ramified at $\wp$.
\end{defn}

Let $\mathcal{O}_K$ be the ring of integers in $K$. We may define a homomorphism on the (multiplicative) group of non-zero fractional ideals of $K$ as follows. Let $\wp$ be a prime of $K$, let $\pi$ be a uniformizer of $K_\wp$ and let $\alpha_\wp\in A_K$ be the element which is $\pi$ at the place $\wp$ and $1$ at all other places. We define:
$$\psi(\wp)=\begin{cases}
0, \text{ if } \psi \text{ is ramified at }\wp;\\
\psi(\alpha_\wp), \text{ otherwise}.
\end{cases}$$

\begin{defn}
The Hecke L-series attached to a Gr\"ossencharacter $\psi$ of $K$ is given by the Euler product over all primes of $K$:
$$L(\psi,s)=\prod_\wp\left(1-\frac{\psi(\wp)}{(N_{\Rats}^K(\wp))^s}\right)^{-1}.$$
\end{defn}

Hecke L-series of this form have an analytic continuation and satisfy a certain functional equation. This fact was first proved by Hecke himself but later was vastly generalized by Tate using Fourier analysis on the ring $A_K$ (what is usually called Tate's thesis).
%%%%%
%%%%%
\end{document}
