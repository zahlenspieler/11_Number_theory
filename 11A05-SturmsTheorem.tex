\documentclass[12pt]{article}
\usepackage{pmmeta}
\pmcanonicalname{SturmsTheorem}
\pmcreated{2013-03-22 14:28:49}
\pmmodified{2013-03-22 14:28:49}
\pmowner{rspuzio}{6075}
\pmmodifier{rspuzio}{6075}
\pmtitle{Sturm's theorem}
\pmrecord{14}{36013}
\pmprivacy{1}
\pmauthor{rspuzio}{6075}
\pmtype{Theorem}
\pmcomment{trigger rebuild}
\pmclassification{msc}{11A05}
\pmclassification{msc}{26A06}
%\pmkeywords{Root-counting}
%\pmkeywords{Real roots}
\pmrelated{EuclidsAlgorithm}
\pmdefines{Sturm Sequence}

% this is the default PlanetMath preamble.  as your knowledge
% of TeX increases, you will probably want to edit this, but
% it should be fine as is for beginners.

% almost certainly you want these
\usepackage{amssymb}
\usepackage{amsmath}
\usepackage{amsfonts}

% used for TeXing text within eps files
%\usepackage{psfrag}
% need this for including graphics (\includegraphics)
%\usepackage{graphicx}
% for neatly defining theorems and propositions
\usepackage{amsthm}
% making logically defined graphics
%%%\usepackage{xypic}

% there are many more packages, add them here as you need them

% define commands here
\newtheorem{theorem}{Theorem}
\newtheorem{definition}{Definition}
\begin{document}
This root-counting theorem was produced by the French mathematician Jacques Sturm in 1829.

\begin{definition}
Let $P(x)$ be a real polynomial in $x$, and define the
Sturm sequence of polynomials $\big(P_0(x), P_1(x), \ldots \big)$ by
\begin{eqnarray*}
P_0(x) &=& P(x)\\
P_1(x) &=& P'(x)\\
P_n(x) &=& -\mathrm{rem}(P_{n-2}, P_{n-1}), n \geq 2
\end{eqnarray*}
Here $\mathrm{rem}(P_{n-2}, P_{n-1})$ denotes the remainder of the
polynomial $P_{n-2}$ upon division by the polynomial $P_{n-1}$. The
sequence terminates once one of the $P_i$ is zero. 
\end{definition}

\begin{definition}
For any number
$t$, let $var_P(t)$ denote the number of sign changes in the
sequence $P_0(t), P_1(t), \ldots$. 
\end{definition}

\begin{theorem} For real numbers $a$ and $b$
that are both not roots of $P(x)$,
$$\#\{\textrm{distinct real roots of }P \textrm{ in } (a, b)\} = var_P(a)-var_P(b)$$
\end{theorem}

In particular, we can count the
total number of distinct real roots by looking at the limits as
$a\rightarrow-\infty$ and $b\rightarrow+\infty$. The total number of
distinct real roots will depend only on the leading terms of the
Sturm sequence polynomials.

Note that deg $P_n <$ deg $P_{n-1}$, and so the longest possible
Sturm sequence has deg $P+1$ terms.

Also, note that this sequence is very closely related to the sequence of remainders generated by the Euclidean Algorithm; in fact, the term $P_i$ is the exact same except with a sign changed when $i \equiv 2$ or $3 \pmod 4$. Thus, the Half-GCD Algorithm may be used to compute this sequence. Be aware that some computer algebra systems may normalize remainders from the Euclidean Algorithm which messes up the sign.

For a proof, see Wolpert, N., ``
\PMlinkexternal{Proof of Sturm's Theorem}{http://web.archive.org/web/20050412175929/http://www.mpi-sb.mpg.de/~nicola/Vorlesung/sturm.ps}''

%%%%%
%%%%%
\end{document}
