\documentclass[12pt]{article}
\usepackage{pmmeta}
\pmcanonicalname{ExamplesOfPrimeIdealDecompositionInNumberFields}
\pmcreated{2013-03-22 13:53:05}
\pmmodified{2013-03-22 13:53:05}
\pmowner{alozano}{2414}
\pmmodifier{alozano}{2414}
\pmtitle{examples of prime ideal decomposition in number fields}
\pmrecord{12}{34628}
\pmprivacy{1}
\pmauthor{alozano}{2414}
\pmtype{Example}
\pmcomment{trigger rebuild}
\pmclassification{msc}{11S15}
%\pmkeywords{decomposition}
%\pmkeywords{inertia}
%\pmkeywords{prime ideal decomposition}
\pmrelated{DecompositionGroup}
\pmrelated{Discriminant}
\pmrelated{NumberField}
\pmrelated{PrimeIdealDecompositionInQuadraticExtensionsOfMathbbQ}
\pmrelated{PrimeIdealDecompositionInCyclotomicExtensionsOfMathbbQ}
\pmrelated{ExamplesOfRamificationOfArchimedeanPlaces}

% this is the default PlanetMath preamble.  as your knowledge
% of TeX increases, you will probably want to edit this, but
% it should be fine as is for beginners.

% almost certainly you want these
\usepackage{amssymb}
\usepackage{amsmath}
\usepackage{amsthm}
\usepackage{amsfonts}

% used for TeXing text within eps files
%\usepackage{psfrag}
% need this for including graphics (\includegraphics)
%\usepackage{graphicx}
% for neatly defining theorems and propositions
%\usepackage{amsthm}
% making logically defined graphics
%%\usepackage{xypic}

% there are many more packages, add them here as you need them

% define commands here

\newtheorem{thm}{Theorem}
\newtheorem{defn}{Definition}
\newtheorem{prop}{Proposition}
\newtheorem{lemma}{Lemma}
\newtheorem{cor}{Corollary}

% Some sets
\newcommand{\Nats}{\mathbb{N}}
\newcommand{\Ints}{\mathbb{Z}}
\newcommand{\Reals}{\mathbb{R}}
\newcommand{\Complex}{\mathbb{C}}
\newcommand{\Rats}{\mathbb{Q}}
\begin{document}
Here we follow the notation of the entry on the decomposition
group. See also \PMlinkexternal{this entry}{http://planetmath.org/encyclopedia/PrimeIdealDecompositionInQuadraticExtensionsOfMathbbQ.html}.

{\bf Example 1}

Let $K=\Rats(\sqrt{-7})$; then
$\operatorname{Gal}(K/\Rats)=\{\operatorname{Id},\sigma\} \cong
\Ints/2\Ints$, where $\sigma$ is the complex conjugation map. Let
$\mathcal{O}_K$ be the ring of integers of $K$. In this case:$$
\mathcal{O}_K=\Ints\left[\frac{1+\sqrt{-7}}{2}\right]$$ The
discriminant of this field is $D_{K/\Rats}=-7$. We look at the
decomposition in prime ideals of some prime ideals in $\Ints$:
\begin{enumerate}
\item The only prime ideal in $\Ints$ that ramifies is $(7)$:
$$(7)\mathcal{O}_K=(\sqrt{-7})^2$$ and we have $e=2,f=g=1$. Next
we compute the decomposition and inertia groups from the
definitions. Notice that both $\operatorname{Id}, \sigma$ fix the
ideal $(\sqrt{-7})$. Thus:
$$D((\sqrt{-7})/(7))=\operatorname{Gal}(K/\Rats)$$
For the inertia group, notice that $\sigma\equiv
\operatorname{Id}\ \operatorname{mod}\ (\sqrt{-7})$. Hence:
$$T((\sqrt{-7})/(7))=\operatorname{Gal}(K/\Rats)$$
Also note that this is trivial if we use the properties of the
fixed field of $D((\sqrt{-7})/(7))$ and $T((\sqrt{-7})/(7))$ (see
the section on ``decomposition of extensions'' in the entry on
decomposition group), and the fact that $e\cdot f\cdot g=n$, where
$n$ is the degree of the extension ($n=2$ in our case).

\item The primes $(5),(13)$ are inert, i.e. they are prime ideals
in $\mathcal{O}_K$. Thus $e=1=g,f=2$. Obviously the conjugation
map $\sigma$ fixes the ideals $(5),(13)$, so
$$D(5\mathcal{O}_K/(5))=\operatorname{Gal}(K/\Rats)=D(13\mathcal{O}_K/(13))$$
On the other hand $\sigma(\sqrt{-7})\equiv-\sqrt{-7}\
\operatorname{mod} (5),(13)$, so $\sigma\neq \operatorname{Id}\
\operatorname{mod} (5),(13)$ and
$$T(5\mathcal{O}_K/(5))=\{\operatorname{Id}\}=T(13\mathcal{O}_K/(13))$$

\item The primes $(2),(29)$ are split:
$$2\mathcal{O}_K={\left( 2,\frac{1+\sqrt{-7}}{2}\right)}{\left(
2,\frac{1-\sqrt{-7}}{2}\right)}=\mathcal{P}\cdot\mathcal{P'}$$
$$29\mathcal{O}_K=\left(29,14+\sqrt{-7}\right)\left(29,14-\sqrt{-7}\right)=\mathcal{R}\cdot\mathcal{R'}$$
so $e=f=1,g=2$ and
$$D(\mathcal{P}/(2))=T(\mathcal{P}/(2))=\{\operatorname{Id}\}=D(\mathcal{R}/(29))=T(\mathcal{R}/(29))$$

\end{enumerate}

{\bf Example 2}

Let $\zeta_7=e^{\frac{2\pi i}{7}}$, i.e. a $7^{th}$-root of unity,
and let $L=\Rats(\zeta_7)$. This is a cyclotomic extension of
$\Rats$ with Galois group
$$\operatorname{Gal}(L/\Rats)\cong
\left(\Ints/7\Ints\right)^{\times}\cong \Ints/6\Ints$$
Moreover
$$\operatorname{Gal}(L/\Rats)=\{\sigma_a\colon L\to L \mid
\sigma_a(\zeta_7)=\zeta_7^a,\quad a\in
\left(\Ints/7\Ints\right)^{\times} \}$$

Galois theory gives us the subfields of $L$:
$\xymatrix@dr@C=1pc{
 L=\Rats(\zeta_7) \ar@{-}[r] \ar@{-}[d] & \Rats(\zeta_7+\zeta_7^6) \ar@{-}[d] \\
 \Rats(\sqrt{-7}) \ar@{-}[r]       & \Rats }$

The discriminant of the extension ${L/\Rats}$ is
$D_{L/\Rats}=-7^5$. Let $\mathcal{O}_L$ denote the ring of
integers of $L$, thus $\mathcal{O}_L=\Ints[\zeta_7]$. We use the results of \PMlinkexternal{this entry}{http://planetmath.org/encyclopedia/PrimeIdealDecompositionInCyclotomicExtensionsOfMathbbQ.html} to find the decomposition of the primes $2,5,7,13,29$:

$\xymatrix{
{L=\Rats(\zeta_7)} \ar@{-}[d]^3 & {(1-\zeta_7)^6} \ar@{-}[d] & {\mathfrak{P}\cdot\mathfrak{P}'} \ar@{-}[d] & {(5)} \ar@{-}[d] & {\mathfrak{Q}_1\cdot\mathfrak{Q}_2\cdot\mathfrak{Q}_3} \ar@{-}[d] \\
{K=\Rats(\sqrt{-7})}\ar@{-}[d]^2 & {(\sqrt{-7})^2} \ar@{-}[d] & {\left( 2,\frac{1+\sqrt{-7}}{2}\right)}{\left( 2,\frac{1-\sqrt{-7}}{2}\right)}\ar@{-}[d] & (5) \ar@{-}[d]& (13) \ar@{-}[d]\\
\Rats & (7) & (2) & (5) & (13) }$

\begin{enumerate}
\item The prime ideal $7\Ints$ is totally ramified in $L$, and the
only prime ideal that ramifies:
$$7\mathcal{O}_L=(1-\zeta_7)^6=\mathfrak{T}^6$$
Thus
$$e(\mathfrak{T}/(7))=6,\quad f(\mathfrak{T}/(7))=g(\mathfrak{T}/(7))=1$$
Note that, by the properties of the fixed fields of decomposition
and inertia groups, we must have
$L^{T(\mathfrak{T}/(7))}=\Rats=L^{D(\mathfrak{T}/(7))}$, thus, by
Galois theory,
$$D(\mathfrak{T}/(7))=T(\mathfrak{T}/(7))=\operatorname{Gal}(L/\Rats)$$

\item The ideal $2\Ints$ factors in $K$ as above,
$2\mathcal{O}_K=\mathcal{P}\cdot\mathcal{P'}$, and each of the
prime ideals $\mathcal{P},\mathcal{P'}$ remains inert from $K$ to
$L$, i.e. $\mathcal{P}\mathcal{O}_L=\mathfrak{P}$, a prime ideal
of $L$. Note also that the order of $2\ \operatorname{mod}\ 7$ is $3$, and since $g$ is at least $2$, $2\cdot3=6$, so $e$ must equal $1$ (recall that $efg=n$):
$$e(\mathfrak{P}/(2))=1,\quad f(\mathfrak{P}/(2))=3,\quad g(\mathfrak{P}/(2))=2$$
Since $e=1$, $L^{T(\mathfrak{P}/(2))}=L$, and $[L\colon
L^{D(\mathfrak{P}/(2))}]=3$, so
$$D(\mathfrak{P}/(2))=<\sigma_2>\cong \Ints/3\Ints,\quad
T(\mathfrak{P}/(2))=\{\operatorname{Id}\}$$

\item The ideal $(5)$ is inert, $5\mathcal{O}_L=\mathfrak{S}$ is
prime and the order of $5$ modulo $7$ is $6$. Thus:
$$e(\mathfrak{S}/(5))=1,\quad f(\mathfrak{S}/(5))=6,\quad
g(\mathfrak{S}/(5))=1$$
$$D(\mathfrak{S}/(5))=\operatorname{Gal}(L/\Rats),\quad
T(\mathfrak{S}/(5))=\{\operatorname{Id}\}$$

\item The prime ideal $13\Ints$ is inert in $K$ but it splits in
$L$,
$13\mathcal{O}_L=\mathfrak{Q}_1\cdot\mathfrak{Q}_2\cdot\mathfrak{Q}_3$, and $13\equiv 6\equiv -1\ \operatorname{mod}\ 7$, so the order of $13$ is $2$:
$$e(\mathfrak{Q}_i/(13))=1,\quad f(\mathfrak{Q}_i/(13))=2,\quad
g(\mathfrak{Q}_i/(13))=3$$
$$D(\mathfrak{Q}_i/(13))=<\sigma_6>\cong\Ints/2\Ints,\quad
T(\mathfrak{Q}_i/(13))=\{\operatorname{Id}\}$$

\item The prime ideal $29\Ints$ is splits completely in $L$,
$$29\mathcal{O}_L=\mathfrak{R}_1\cdot\mathfrak{R}_2\cdot\mathfrak{R}_3\cdot\mathfrak{R'}_1\cdot\mathfrak{R'}_2\cdot\mathfrak{R'}_3$$
Also $29\equiv 1\ \operatorname{mod}\ 7$, so $f=1$,
$$e(\mathfrak{R}_i/(29))=1,\quad f(\mathfrak{R}_i/(29)=1,\quad
g(\mathfrak{R}_i/(29))=6$$
$$D(\mathfrak{R}_i/(29))=T(\mathfrak{R}_i/(29))=\{\operatorname{Id}\}$$
\end{enumerate}
%%%%%
%%%%%
\end{document}
