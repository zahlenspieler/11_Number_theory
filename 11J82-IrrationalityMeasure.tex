\documentclass[12pt]{article}
\usepackage{pmmeta}
\pmcanonicalname{IrrationalityMeasure}
\pmcreated{2013-03-22 14:12:22}
\pmmodified{2013-03-22 14:12:22}
\pmowner{mathcam}{2727}
\pmmodifier{mathcam}{2727}
\pmtitle{irrationality measure}
\pmrecord{8}{35639}
\pmprivacy{1}
\pmauthor{mathcam}{2727}
\pmtype{Definition}
\pmcomment{trigger rebuild}
\pmclassification{msc}{11J82}
\pmrelated{DirichletsApproximationTheorem}

\endmetadata

% this is the default PlanetMath preamble.  as your knowledge
% of TeX increases, you will probably want to edit this, but
% it should be fine as is for beginners.

% almost certainly you want these
\usepackage{amssymb}
\usepackage{amsmath}
\usepackage{amsfonts}

% used for TeXing text within eps files
%\usepackage{psfrag}
% need this for including graphics (\includegraphics)
%\usepackage{graphicx}
% for neatly defining theorems and propositions
%\usepackage{amsthm}
% making logically defined graphics
%%%\usepackage{xypic}

% there are many more packages, add them here as you need them

% define commands here
\begin{document}
Let $\alpha\not\in\mathbb Q$. 
Let $$M(\alpha)=\{\mu>0 \mid \exists q_0=q_0(\alpha,
\mu)>0
\mbox{ such that }
\left|\alpha-\frac pq\right|>\frac1{q^\mu} \quad
\forall p,q\in\mathbb Z, q>q_0\}.$$
The irrationality measure of $\alpha$,
denoted by $\mu(\alpha)$, is defined by
$$\mu(\alpha)=\inf M(\alpha).$$

If $M(\alpha)=\emptyset$, we set $\mu(\alpha)=\infty$.

This definition is (loosely) a measure of the extent to which $\alpha$ can be approximated by rational numbers.  Of course, by the fact that $\mathbb{Q}$ is dense in $\mathbb{R}$, we can make arbitrarily good approximations to real numbers by rationals.  Thus this definition was made to represent a stronger statement:  it is the ability of rational numbers to approximate $\alpha$ given a fixed growth bound on the denominators of those rational numbers.

By the Dirichlet's Lemma, $\mu(\alpha)\ge 2$.
Roth \cite{Roth955,Roth955b} 
proved in 1955 that
$\mu(\alpha)=2$ for every algebraic real number. 
It is well known also 
that $\mu(e)=2$. For almost all real numbers
the irrationality measure is 2.
However, for special constants, only some upper bounds
are known:

\begin{center}
\begin{tabular}{|c|c|c|}\hline
Constant & Upper bound & Reference \\ \hline 
$\pi$ & 8.0161 & Hata (1993) \cite{hata}\\ 
$\pi/\sqrt3$ & 4.6016 & Hata (1993) \cite{hata}\\
$e$ & 2 &  Davis (1978) \cite{davis978} \\
$\pi^2$ & 5.4413 & Rhin and Viola (1996) \cite{RhinViol996}\\ 
$\log2$ & 3.8914 & Rukhadze (1987) \cite{rukh}, Hata (1990) \cite{hata990}\\ 
$\zeta(3)$ & 5.5139 &  Rhin and Viola (2001) \cite{RhinViol001}\\  \hline
\end{tabular}
\end{center}

It is worth noting that the last column of the above table is simply a list of references, not a collection of discoverers.  For example that fact that the irrationality measure of $e$ is 2 was known to Euler.

\begin{thebibliography}{9}

\bibitem{davis978}
Davis, C.S., \textit{`Rational approximations to $e$'},
J. Austral. Math. Soc. Ser. A \textbf{25} (1978), 497--502.

\bibitem{hata990}
Hata, M. \textit{`Legendre Type Polynomials and Irrationality Measures'}, J. reine angew. Math. \textbf{407}, 99--125, 1990.

\bibitem{hata}
Hata, M.,
\textit{`Rational approximations to $\pi$ and some other numbers'},
Acta Arith. \textbf{63}, 335--349 (1993).

\bibitem{RhinViol996}
Rhin, G. and Viola, C. {\em `On a permutation group related to  
zeta(2)}', 
Acta Arith. {\bf 77} (1996),
23--56.

\bibitem{RhinViol001}
Rhin, G. and Viola, C. \textit{`The group structure for $\zeta(3)$'}, 
Acta Arith. {\bf 97} (2001), 269--293.

\bibitem{Roth955}
Roth, K.F., \textit{`Rational Approximations to Algebraic Numbers'},
Mathematika {\bf 2} (1955), 1--20. 

\bibitem{Roth955b}
Roth, K.F. \textit{`Corrigendum to 
'Rational Approximations to Algebraic Numbers''}
Mathematika {\bf 2} (1955), 168. 

\bibitem{rukh}
Rukhadze, E.A. \textit{`A Lower Bound for the Rational Approximation of  by Rational Numbers'} Vestnik Moskov Univ. Ser. I Math. Mekh., \textbf{6} (1987), 25-29 and 97. 


\end{thebibliography}
%%%%%
%%%%%
\end{document}
