\documentclass[12pt]{article}
\usepackage{pmmeta}
\pmcanonicalname{NormalOrder}
\pmcreated{2013-03-22 12:36:23}
\pmmodified{2013-03-22 12:36:23}
\pmowner{mathcam}{2727}
\pmmodifier{mathcam}{2727}
\pmtitle{normal order}
\pmrecord{5}{32862}
\pmprivacy{1}
\pmauthor{mathcam}{2727}
\pmtype{Definition}
\pmcomment{trigger rebuild}
\pmclassification{msc}{11B05}
\pmdefines{average order}

\endmetadata

% this is the default PlanetMath preamble.  as your knowledge
% of TeX increases, you will probably want to edit this, but
% it should be fine as is for beginners.

% almost certainly you want these
\usepackage{amssymb}
\usepackage{amsmath}
\usepackage{amsfonts}

% used for TeXing text within eps files
%\usepackage{psfrag}
% need this for including graphics (\includegraphics)
%\usepackage{graphicx}
% for neatly defining theorems and propositions
%\usepackage{amsthm}
% making logically defined graphics
%%%\usepackage{xypic}

% there are many more packages, add them here as you need them

% define commands here
\begin{document}
Let $f(n)$ and $F(n)$ be functions from $\mathbb{Z}^{+} \rightarrow \mathbb{R}$.  We say that $f(n)$ has {\em normal order} $F(n)$ if for each $\epsilon>0$ the
set
\[ A(\epsilon)=\{n \in \mathbb{Z}^{+} :
     (1-\epsilon)F(n)<f(n)<(1+\epsilon)F(n) \} \]
has the property that $\underline{d}(A(\epsilon))=1$.
Equivalently, if $B(\epsilon)=\mathbb{Z}^{+} \backslash A(\epsilon)$, then
 $\underline{d}(B(\epsilon))=0$.  (Note that $\underline{d}(X)$ denotes the lower asymptotic density of $X$).

We say that $f$ has {\em average order} $F$ if
\[ \sum_{j=1}^n f(j) \sim \sum_{j=1}^n F(j) \]
%%%%%
%%%%%
\end{document}
