\documentclass[12pt]{article}
\usepackage{pmmeta}
\pmcanonicalname{CubanPrime}
\pmcreated{2013-03-22 16:19:27}
\pmmodified{2013-03-22 16:19:27}
\pmowner{PrimeFan}{13766}
\pmmodifier{PrimeFan}{13766}
\pmtitle{cuban prime}
\pmrecord{4}{38452}
\pmprivacy{1}
\pmauthor{PrimeFan}{13766}
\pmtype{Definition}
\pmcomment{trigger rebuild}
\pmclassification{msc}{11N05}
\pmrelated{CubeOfANumber}

% this is the default PlanetMath preamble.  as your knowledge
% of TeX increases, you will probably want to edit this, but
% it should be fine as is for beginners.

% almost certainly you want these
\usepackage{amssymb}
\usepackage{amsmath}
\usepackage{amsfonts}

% used for TeXing text within eps files
%\usepackage{psfrag}
% need this for including graphics (\includegraphics)
%\usepackage{graphicx}
% for neatly defining theorems and propositions
%\usepackage{amsthm}
% making logically defined graphics
%%%\usepackage{xypic}

% there are many more packages, add them here as you need them

% define commands here

\begin{document}
A {\em cuban prime} is a prime number that is a solution to one of two different specific equations involving third powers of $x$ and $y$.

The first of these equations is $$p = \frac{x^3 - y^3}{x - y},$$ with $x = y + 1$ and $y > 0$. The first few cuban primes from this equation are: 7, 19, 37, 61, 127, 271, 331, 397, 547, 631, 919.

The general cuban prime of this kind can be rewritten as $$\frac{(y + 1)^3 - y^3}{y + 1 - y},$$ which simplifies to $3y^2 + 3y + 1$. This is exactly the general form of a centered hexagonal number; that is, all of these cuban primes are centered hexagonal numbers.  

This kind of cuban primes has been researched by A. J. C. Cunningham, in a paper entitled ''On quasi-Mersennian numbers''.

As of January 2006 the largest known cuban prime has 65537 digits with $y = 100000845^{4096}$, discovered by Jens Kruse Andersen, according to the Prime Pages of the University of Tennessee at Martin.

The second of these equations is $$p = \frac{x^3 - y^3}{x - y},$$ with $x = y + 2$. It simplifies to $3y^2 + 6y + 4$. The first few cuban primes on this form are: 13, 109, 193, 433, 769.

This kind of cuban primes have also been researched by Cunningham, in his book {\it Binomial Factorisations}.

The name "cuban prime" has to do with the r\^ole cubes (third powers) play in the equations, and has nothing to do with the prime minister of Cuba.
%%%%%
%%%%%
\end{document}
