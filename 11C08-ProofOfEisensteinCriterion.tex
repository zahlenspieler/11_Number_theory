\documentclass[12pt]{article}
\usepackage{pmmeta}
\pmcanonicalname{ProofOfEisensteinCriterion}
\pmcreated{2013-03-22 12:42:11}
\pmmodified{2013-03-22 12:42:11}
\pmowner{rspuzio}{6075}
\pmmodifier{rspuzio}{6075}
\pmtitle{proof of Eisenstein criterion}
\pmrecord{11}{32986}
\pmprivacy{1}
\pmauthor{rspuzio}{6075}
\pmtype{Proof}
\pmcomment{trigger rebuild}
\pmclassification{msc}{11C08}
\pmclassification{msc}{13F15}
\pmrelated{GausssLemmaII}

% this is the default PlanetMath preamble.  as your knowledge
% of TeX increases, you will probably want to edit this, but
% it should be fine as is for beginners.

% almost certainly you want these
\usepackage{amssymb}
\usepackage{amsmath}
\usepackage{amsfonts}

% used for TeXing text within eps files
%\usepackage{psfrag}
% need this for including graphics (\includegraphics)
%\usepackage{graphicx}
% for neatly defining theorems and propositions
%\usepackage{amsthm}
% making logically defined graphics
%%%\usepackage{xypic}

% there are many more packages, add them here as you need them

% define commands here
\begin{document}
Let $f(x) \in R[x]$ be a polynomial satisfying Eisenstein's Criterion with prime $p$.

Suppose that $f(x)=g(x)h(x)$ with $g(x), h(x) \in F[x]$, where $F$ is the field of fractions of $R$.  Gauss' Lemma II \PMlinkescapetext{states that} there exist $g'(x), h'(x) \in R[x]$ such that $f(x)=g'(x)h'(x)$, i.e. any factorization can be converted to a factorization in $R[x]$.

Let $f(x) = \sum_{i=0}^n a_i x^i$, $g'(x)= \sum_{j=0}^\ell b_j x^j$,
$h'(x)= \sum_{k=0}^m c_k x^k$ be the expansions of $f(x), g'(x)$, and $h'(x)$ respectively.

Let $\varphi : R[x] \rightarrow R/pR[x]$ be the natural homomorphism from $R[x]$ to $R/pR[x]$.  Note that since $p \mid a_i$ for $i<n$ and $p \nmid a_n$, we have $\varphi(a_i)=0$ for $i < n$ and $\varphi(a_i)=\alpha \neq 0$
\[ \varphi(f(x)) = \varphi \left(\sum_{i=0}^n a_i x^i \right)
     = \sum_{i=0}^n \varphi(a_i) x^i = \varphi(a_n) x^n = \alpha x^n \]

Therefore we have $\alpha x^n = \varphi(f(x))=\varphi(g'(x)h'(x))=\varphi(g'(x))\varphi(h'(x))$ so we must have $\varphi(g'(x))=\beta x^{\ell'}$ and $\varphi(h'(x)=\gamma x^{m'}$ for some $\beta, \gamma \in R/pR$ and some integers $\ell', m'$.

Clearly $\ell' \leq \deg(g'(x))=\ell$ and $m' \leq \deg(h'(x))=m$, and therefore since $\ell' m' = n = \ell m$, we must have $\ell'=\ell$ and $m'=m$.
Thus $\varphi(g'(x))=\beta x^\ell$ and $\varphi(h'(x))=\gamma x^m$.

If $\ell>0$, then $\varphi(b_i)=0$ for $i<\ell$.  In particular, $\varphi(b_0)=0$, hence $p \mid b_0$.  Similarly if $m>0$, then $p \mid c_0$.

Since $f(x)=g'(x)h'(x)$, by equating coefficients we see that $a_0 = b_0 c_0$.

If $\ell>0$ and $m>0$, then $p \mid b_0$ and $p \mid c_0$, which implies that $p^2 \mid a_0$.  But this contradicts our assumptions on $f(x)$, and therefore we must have $\ell=0$ or $m=0$, that is, we must have a trivial factorization.  Therefore $f(x)$ is irreducible.
%%%%%
%%%%%
\end{document}
