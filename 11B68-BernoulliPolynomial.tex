\documentclass[12pt]{article}
\usepackage{pmmeta}
\pmcanonicalname{BernoulliPolynomial}
\pmcreated{2013-03-22 11:45:51}
\pmmodified{2013-03-22 11:45:51}
\pmowner{KimJ}{5}
\pmmodifier{KimJ}{5}
\pmtitle{Bernoulli polynomial}
\pmrecord{12}{30215}
\pmprivacy{1}
\pmauthor{KimJ}{5}
\pmtype{Definition}
\pmcomment{trigger rebuild}
\pmclassification{msc}{11B68}
\pmclassification{msc}{65-01}
%\pmkeywords{number theory}
\pmrelated{BernoulliNumber}

\usepackage{amssymb}
\usepackage{amsmath}
\usepackage{amsfonts}
\usepackage{graphicx}
%%%%\usepackage{xypic}
\begin{document}
The \emph{Bernoulli polynomials} are the sequence $\{ b_r(x) \} _{r=0}^{\infty}$ of polynomials defined on $[0,1]$ by the conditions:
\begin{eqnarray*}
b_0(x) & = & 1, \\
b'_r(x) & = & r b_{r-1}(x), r \geq 1, \\
\int_0^1 b_r(x)dx & = & 0, r \geq 1
\end{eqnarray*}

These assumptions imply the identity
\[ \sum_{r=0}^{\infty} b_r(x) \frac{y^r}{r!} = \frac{ye^{xy}}{e^y-1} \]
allowing us to calculate the $b_r$. We have

\begin{eqnarray*}
b_0(x) & = & 1 \\
b_1(x) & = & x-\frac{1}{2} \\
b_2(x) & = & x^2 - x + \frac{1}{6} \\
b_3(x) & = & x^3 - \frac{3}{2}x^2 + \frac{1}{2}x \\
b_4(x) & = & x^4 - 2x^3 + x^2 - \frac{1}{30} \\
\vdots & &
\end{eqnarray*}
%%%%%
%%%%%
%%%%%
%%%%%
\end{document}
