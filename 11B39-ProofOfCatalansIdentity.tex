\documentclass[12pt]{article}
\usepackage{pmmeta}
\pmcanonicalname{ProofOfCatalansIdentity}
\pmcreated{2013-03-22 14:45:01}
\pmmodified{2013-03-22 14:45:01}
\pmowner{PrimeFan}{13766}
\pmmodifier{PrimeFan}{13766}
\pmtitle{proof of Catalan's Identity}
\pmrecord{20}{36389}
\pmprivacy{1}
\pmauthor{PrimeFan}{13766}
\pmtype{Proof}
\pmcomment{trigger rebuild}
\pmclassification{msc}{11B39}
%\pmkeywords{Fibonacci Numbers}

\endmetadata

% this is the default PlanetMath preamble.  as your knowledge
% of TeX increases, you will probably want to edit this, but
% it should be fine as is for beginners.

% almost certainly you want these
\usepackage{amssymb}
\usepackage{amsmath}
\usepackage{amsfonts}

% used for TeXing text within eps files
%\usepackage{psfrag}
% need this for including graphics (\includegraphics)
\usepackage{graphicx}
% for neatly defining theorems and propositions
\usepackage{amsthm}
% making logically defined graphics
%%%\usepackage{xypic}

% there are many more packages, add them here as you need them

% define commands here
\newtheorem{thm}{Theorem}[section]
\newtheorem{cor}[thm]{Corollary}
\newtheorem{lem}[thm]{Lemma}
\newtheorem{prop}[thm]{Proposition}
\theoremstyle{definition}
\newtheorem{defn}[thm]{Definition}
\theoremstyle{remark}
\newtheorem{rem}[thm]{Remark}
\numberwithin{equation}{section}
\begin{document}
For all positive integers $i$, let $F_{i}$ denote the $i^{th}$ Fibonacci number, with $F_{1}$ = $F_{2}$ = 1. We will show that for all positive integers $n$ and $r$ such that $n > r$ the following holds:
$$F_{n}^2 - F_{n + r}F_{n - r}=( - 1)^{n - r}F_{r}^2.$$
But in order to prove this we first need two lemmas: \begin{lem}
For all positive integers $a$ and $b$ such that $a>1$, the identity
$$F_{a + b} = F_{a}F_{b + 1} + F_{a - 1}F_{b}$$ is true. \end{lem}
\begin{proof} 
We will prove this by induction on $b$. When $b = 1$, the identity states that 
$$F_{a + 1} = F_{a}F_{2} + F_{a - 1}F_{1} \quad \Longleftrightarrow \quad F_{a + 1} = F_{a}\cdot 1 + F_{a - 1}\cdot 1$$
which is true by the definition  of the Fibonacci numbers.  Now, assume for possible values of $b$ less than some positive integer $b_{0}$ such that $b_{0}>1$, the proposition is true. Then
\begin{alignat*}{2}
& F_{a}F_{b_{0} + 1} + F_{a - 1}F_{b_{0}} & \\ 
=\ & F_{a}(F_{b_{0}} + F_{b_{0} - 1}) + F_{a - 1}(F_{b_{0} - 1} + F_{b_{0} - 2}) & \\
=\ & (F_{a}F_{b_{0}} + F_{a - 1}F_{b_{0} - 1}) + (F_{a}F_{b_{0} - 1} + F_{a - 1}F_{b_{0} - 2}) & \\
=\ & F_{a + (b_{0} - 1)} + F_{(a - 1) + (b_{0} - 1)} &  \mbox{(induction hypothesis)} \\
=\ & F_{a + b_{0}} & \mbox{(definition of the Fibonacci numbers)}
\end{alignat*}
This concludes the proof. \end{proof}
\begin{lem}
For all positive integers $t$ such that $t>1$, the following holds:
$$F_{t - 1}^2 + F_{t}F_{t - 1} - F_{t}^2 = ( - 1)^t.$$
\end{lem}
\begin{proof}
We will (again) proceed by induction. First, when $t=2$, we have
$$F_{1}^2 + F_{2}F_{1} - F_{2}^2=1 \quad \Longleftrightarrow \quad 1 + 1\cdot 1 - 1^2=1$$
which is true. Now let us assume that the proposition is true for
all positive integers which are greater than 1 and less than some
positive integer $t_{0}$ $(t_{0}>2)$. Then
\begin{alignat*}{2}
& F_{t_{0} - 1}^2 + F_{t_{0}}F_{t_{0} - 1} - F_{t_{0}}^2 & \\
=\ & F_{t_{0} - 1}^2  + (F_{t_{0} - 1} + F_{t_{0} - 2})F_{t_{0} - 1} - (F_{t_{0} - 1} + F_{t_{0} - 2})^2 & \\
=\ & F_{t_{0} - 1}^2  + F_{t_{0} - 1}^2 + F_{t_{0} - 2}F_{t_{0} - 1} - F_{t_{0} - 1}^2 - F_{t_{0} - 2}^2 - 2F_{t_{0} - 1}F_{t_{0} - 2} & \\
=\ & F_{t_{0} - 1}^2 - 2F_{t_{0} - 1}F_{t_{0} - 2} - F_{t_{0} - 2}^2 & \\
=\ & - (F_{t_{0} - 2}^2 - 2F_{t_{0} - 2}F_{t_{0} - 1} - F_{t_{0} - 1}^2) & \\
=\ &( - 1)( - 1)^{t_{0} - 1} & \mbox{(by induction hypothesis)} \\
=\ & ( - 1)^{t_{0}} 
\end{alignat*} and the proof is complete.
\end{proof}
Now to the main proposition. Let us make the substitutions $x = n - r$
and $a=r$ so that the theorem now states:
\begin{thm}
For all positive integers $x$ and $a$, the following identity holds:
$$F_{x + a}^2 - F_{x + 2a}F_{x}=( - 1)^{x}F_{a}^2.$$
\end{thm}
\begin{proof}  We follow a series of calculations
\begin{alignat*}{2}
& F_{x + a}^2 - F_{x + 2a}F_{x} & \\ 
=\ & (F_{x}F_{a + 1} + F_{x - 1}F_{a})^2  - (F_{x}F_{2a + 1} + F_{x - 1}F_{2a})F_{x} & \text{(by lemma 1)} \\
=\ & F_{x}^2 F_{a + 1}^2 + 2F_{x} F_{a + 1} F_{x - 1} F_{a} + F_{x - 1}^2 F_{a}^2 - & \\ 
 & \quad F_{x}(F_{x}(F_{a + 1}^2 + F_{a}^2) + F_{x - 1}(F_{a} F_{a + 1} + F_{a - 1} F_{a})) & \text{(by lemma 1 again)} \\
=\ & F_{x}F_{x - 1}F_{a}(F_{a + 1} - F_{a - 1}) + F_{a}^2(F_{x - 1}^2 - F_{x}^2) & \\
=\ & F_{x}F_{x - 1}F_{a}(F_{a}) + F_{a}^2(F_{x - 1}^2 - F_{x}^2) & \text{(by the definition of Fibonacci numbers)} \\
=\ & F_{a}^2(F_{x - 1}^2 + F_{x}F_{x - 1} - F_{x}^2) & \\
=\ & F_{a}^2( - 1)^x & \text{(by lemma 2)} 
\end{alignat*}
completing our proof of the theorem.
\end{proof}
%%%%%
%%%%%
\end{document}
