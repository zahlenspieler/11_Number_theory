\documentclass[12pt]{article}
\usepackage{pmmeta}
\pmcanonicalname{SolutionsOf1xx2x3y2}
\pmcreated{2013-03-22 17:05:09}
\pmmodified{2013-03-22 17:05:09}
\pmowner{rm50}{10146}
\pmmodifier{rm50}{10146}
\pmtitle{solutions of $1+x+x^2+x^3=y^2$}
\pmrecord{10}{39378}
\pmprivacy{1}
\pmauthor{rm50}{10146}
\pmtype{Theorem}
\pmcomment{trigger rebuild}
\pmclassification{msc}{11F80}
\pmclassification{msc}{14H52}
\pmclassification{msc}{11D41}

% this is the default PlanetMath preamble.  as your knowledge
% of TeX increases, you will probably want to edit this, but
% it should be fine as is for beginners.

% almost certainly you want these
\usepackage{amssymb}
\usepackage{amsmath}
\usepackage{amsfonts}

% used for TeXing text within eps files
%\usepackage{psfrag}
% need this for including graphics (\includegraphics)
%\usepackage{graphicx}
% for neatly defining theorems and propositions
\usepackage{amsthm}
% making logically defined graphics
%%%\usepackage{xypic}

% there are many more packages, add them here as you need them

% define commands here

\begin{document}
This article shows that the only solutions in integers to the equation
\[1+x+x^2+x^3=y^2\]
are the obvious trivial solutions $x=0,\pm 1$ together with $x=7, y=20$. This result was known to Fermat.

First, note that the equation is $(1+x)(1+x^2)=y^2$, so we have immediately that $x\ge -1$. So, noting the solutions for $x=0,\pm 1$, assume in what follows that $x>1$.

Let $d=\gcd(1+x,1+x^2)$. Then $x\equiv -1\pmod d$, so that $1+x^2\equiv 2\pmod d$. But $d\mid 1+x^2$ so that $2\equiv 0\pmod d$ and $d$ is either $1$ or $2$. If $d=1$, so that $1+x$ and $1+x^2$ are coprime, then $1+x$ and $1+x^2$ must both be squares. But $1+x^2$ is not a square for $x>1$. Thus the $\gcd$ of $1+x$ and $1+x^2$ is $2$, so each must be twice a square, say
\begin{gather*}
1+x=2r^2\\
1+x^2=2s^2
\end{gather*}
and then
\[(r^2)^2+(r^2-1)^2=\left(\frac{1+x}{2}\right)^2+\left(\frac{-1+x}{2}\right)^2=\frac{1+x^2}{2}=s^2\]
so that $r^2, r^2-1, s$ form a primitive Pythagorean triple (note that $r>1$ since $x>1$).

Recall that if $(a,b,c)$ is a primitive Pythagorean triple, then precisely one of $a$ and $b$ is even, and we can choose coprime integers $p,q$ such that $a=p^2-q^2, b=2pq, c=p^2+q^2$ or $a=2pq, b=p^2-q^2, c=p^2+q^2$ depending on the parity of $a$.

Suppose first that $r$ is odd. Then
\begin{gather*}
r^2=p^2-q^2\\
r^2-1=2pq\\
s=p^2+q^2
\end{gather*}
Then $1=r^2-(r^2-1)=(p-q)^2-2q^2$. Now, note that $p-q$ must be a square, say $p-q=t^2$, since $\gcd(p-q,p+q)=1$ and $(p-q)(p+q)$ is a square. Then
\[t^4=(p-q)^2=2q^2+1=(q^2+1)^2-q^4\]
so that
\[t^4+q^4=(q^2+1)^2\]
But we know that \PMlinkname{the sum of two fourth powers can be a square}{ExampleOfFermatsLastTheorem} only for the trivial case where all are zero. So $r$ cannot be odd.

So suppose that $r$ is even. Then
\begin{gather*}
r^2=2pq\\
r^2-1=p^2-q^2\\
s=p^2+q^2
\end{gather*}
From the second of these formulas, we see that $p$ must be even (consider both sides $\pmod 4$), say $p=2t^2$. Now,
\begin{multline*}(p+q-1)(p+q+1)=(p+q)^2-1=p^2+2pq+q^2-1=\\
p^2+2pq+q^2-(r^2-(r^2-1))=p^2+2pq+q^2-2pq+p^2-q^2=2p^2=8t^4
\end{multline*}
Since $p$ and $q$ have opposite parity, $p+q\pm 1$ are even, so that
\[2t^4=\frac{p+q-1}{2}\cdot\left(\frac{p+q-1}{2}+1\right)=u(u+1)\]
Thus one of $u,u+1$ is a fourth power and the other is twice a fourth power, say $b^4$ and $2c^4$.

Now,
\begin{gather*}
u=b^4\Rightarrow u+1=2c^4\Rightarrow b^4-2c^4=-1\\
u=2c^4\Rightarrow u+1=b^4\Rightarrow b^4-2c^4=1
\end{gather*}
so that $b^4-2c^4=\pm 1$.

If $b^4-2c^4=1$, then
\[((c^2)^2+1)^2=(c^2)^4+b^4\]
and again we have a square being the sum of two fourth powers. So this case is impossible.

If $b^4-2c^4=-1$, write $e=c^2$, then $(e^2-1)^2=e^4-b^4$. It follows (see \PMlinkname{here}{X4Y4z2HasNoSolutionsInPositiveIntegers}) that either $b=0$ or $e^2-1=0$. If $b=0$, we get the impossibility $e^4-2e^2+1=e^4$, while if $(e^2-1)^2=0$, then $e^4=b^4, e=\pm 1$ and so $b=\pm 1$. Then $\displaystyle \frac{p+q-1}{2}=b^4=1$, so that $p+q=3$. Thus $r^2=4$, so $r=2$ and, finally, $1+x=2r^2=8$, so that $x=7$.

Thus, the only nontrivial solution to the equation given is
\[1+7+49+343=400\]
%%%%%
%%%%%
\end{document}
