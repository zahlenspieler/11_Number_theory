\documentclass[12pt]{article}
\usepackage{pmmeta}
\pmcanonicalname{ConwaysConstant}
\pmcreated{2013-03-22 18:02:36}
\pmmodified{2013-03-22 18:02:36}
\pmowner{PrimeFan}{13766}
\pmmodifier{PrimeFan}{13766}
\pmtitle{Conway's constant}
\pmrecord{4}{40567}
\pmprivacy{1}
\pmauthor{PrimeFan}{13766}
\pmtype{Definition}
\pmcomment{trigger rebuild}
\pmclassification{msc}{11A63}

\endmetadata

% this is the default PlanetMath preamble.  as your knowledge
% of TeX increases, you will probably want to edit this, but
% it should be fine as is for beginners.

% almost certainly you want these
\usepackage{amssymb}
\usepackage{amsmath}
\usepackage{amsfonts}

% used for TeXing text within eps files
%\usepackage{psfrag}
% need this for including graphics (\includegraphics)
%\usepackage{graphicx}
% for neatly defining theorems and propositions
%\usepackage{amsthm}
% making logically defined graphics
%%%\usepackage{xypic}

% there are many more packages, add them here as you need them

% define commands here

\begin{document}
{\em Conway's constant} $\lambda \approx 1.303577296$ gives the asymptotic rate of growth in the length between $a_i$ and $a_{i + 1}$ in most look and say sequences. That is, given a function $d(n)$ that gives us the number of digits of $n$ in base 10, then $$\lim_{i \to \infty} \frac{a_{i + 1}}{a_i} = \lambda.$$

For example, starting with $n = 1$ and skipping ahead to $a_7$, we observe

\begin{tabular}{|r|r|l|}
$i$ & $a_i$ & $\frac{a_{i + 1}}{a_i}$ \\
 7 &                           13112221 & 1.333333333... \\
 8 &                         1113213211 & 1.25 \\
 9 &                     31131211131221 & 1.4 \\
10 &               13211311123113112211 & 1.428571428... \\
11 &         11131221133112132113212221 & 1.3 \\
12 & 3113112221232112111312211312113211 & 1.307692307... \\
\end{tabular}

Conway's constant is the largest zero of this degree 71 polynomial:

$x^{71} - x^{69} - 2x^{68} - x^{67} + 2x^{66} + 2x^{65} + x^{64} - x^{63} - x^{62} - x^{61} - x^{60} - x^{59}$
$+  2x^{58} + 5x^{57} + 3x^{56} - 2x^{55} - 10x^{54} - 3x^{53} - 2x^{52} + 6x^{51} + 6x^{50} + x^{49} + 9x^{48}$
$- 3x^{47} -  7x^{46} - 8x^{45} - 8x^{44} + 10x^{43} + 6x^{42} + 8x^{41} - 5x^{40} - 12x^{39} + 7x^{38} - 7x^{37}$
$+ 7x^{36} + x^{35} -  3x^{34} + 10x^{33} + x^{32} - 6x^{31} - 2x^{30} - 10x^{29} - 3x^{28} + 2x^{27} + 9x^{26}$
$- 3x^{25} + 14x^{24} - 8x^{23} -  7x^{21} + 9x^{20} + 3x^{19} - 4x^{18} - 10x^{17} - 7x^{16} + 12x^{15}$
$+ 7x^{14} + 2x^{13} - 12x^{12} - 4x^{11} -  2x^{10} + 5x^9 + x^7 - 7x^6 + 7x^5 - 4x^4 + 12x^3 - 6x^2 + 3x - 6$

\begin{thebibliography}{1}
\bibitem{sf} Steven R. Finch, {\it Mathematical Constants}. Cambridge: Cambridge University Press (2003): 453
\end{thebibliography}
%%%%%
%%%%%
\end{document}
