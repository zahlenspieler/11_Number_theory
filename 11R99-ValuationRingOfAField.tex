\documentclass[12pt]{article}
\usepackage{pmmeta}
\pmcanonicalname{ValuationRingOfAField}
\pmcreated{2013-03-22 19:03:25}
\pmmodified{2013-03-22 19:03:25}
\pmowner{rm50}{10146}
\pmmodifier{rm50}{10146}
\pmtitle{valuation ring of a field}
\pmrecord{4}{41937}
\pmprivacy{1}
\pmauthor{rm50}{10146}
\pmtype{Theorem}
\pmcomment{trigger rebuild}
\pmclassification{msc}{11R99}
\pmclassification{msc}{12J20}
\pmclassification{msc}{13A18}
\pmclassification{msc}{13F30}
%\pmkeywords{valuation}
%\pmkeywords{discrete}
\pmrelated{HenselianField}
\pmrelated{RingOfExponent}
\pmdefines{valuation ring}
\pmdefines{discrete valuation}

\endmetadata

\usepackage{amssymb}
\usepackage{amsmath}
\usepackage{amsfonts}

% used for TeXing text within eps files
%\usepackage{psfrag}
% need this for including graphics (\includegraphics)
%\usepackage{graphicx}
% for neatly defining theorems and propositions
\usepackage{amsthm}
% making logically defined graphics
%%%\usepackage{xypic}

% there are many more packages, add them here as you need them

% define commands here
\newcommand{\Abs}[1]{\left\lvert #1\right\rvert}
\newcommand{\BQ}{\mathbb{Q}}
\newcommand{\BR}{\mathbb{R}}
\newcommand{\BZ}{\mathbb{Z}}
\newcommand{\BN}{\mathbb{N}}
\newcommand{\suchthat}{\ \mid\ }
\newcommand{\smm}{\mathfrak{m}}
\newcommand{\U}[1]{{#1}^*}

\newtheorem{thm}{Theorem}
\newtheorem{cor}[thm]{Corollary}
\newtheorem{lem}[thm]{Lemma}
\newtheorem{prop}[thm]{Proposition}
\newtheorem{defn}{Definition}
\begin{document}
In this article, $K$ is a field with a nontrivial nonarchimedean absolute value (valuation) $\Abs\cdot$ and $\U{K}$ its multiplicative group of units (nonzero elements).

\begin{prop} ~
\newline
\begin{enumerate} \item $A=_{df}\{x\in K\suchthat \Abs x\leq 1\}$ is a ring, called the \emph{valuation ring} of $(K,\Abs{\cdot})$,
\item $\smm =_{df}\{x\in K\suchthat \Abs x < 1\}$ is the unique maximal ideal of $A$, and $\U{A}=\{x\in K\suchthat \Abs x =1\}$,
\item $K$ is the fraction field of $A$.
\end{enumerate}
\end{prop}

\begin{proof} For (1), note that $1\in A$, that $x,y\in A\Rightarrow \Abs x\leq 1, \Abs y \leq 1\Rightarrow\Abs{xy} \leq 1\Rightarrow xy\in A$, and that $x,y\in A\Rightarrow \Abs{x-y}\leq\max(\Abs x,\Abs {-y})\leq 1\Rightarrow x-y\in A$.

For (2), it is obvious that $\smm+\smm\subset\smm$ and that $\smm A\subset\smm$ so that $\smm$ is an ideal. Clearly $A-\smm = \{x\in K\suchthat \Abs x = 1\}$ which is obviously $\U{A}$ and the result follows from general considerations regarding units in a local ring.

Finally, to prove (3), choose some $x\in K$ with $\Abs x < 1$ (to do this, choose any $z$ whose valuation is not $1$; then either $z$ or $z^{-1}$ will suffice). Given $y\in\U{K}$, there is some $n$ such that $\Abs y\cdot\Abs{x}^n<1$, so that $yx^n\in A$ and thus
\[\frac{yx^n}{x^n}= y\]
is in the fraction field of $A$.
\end{proof}

We say that the absolute value $\Abs{\cdot}$ is \emph{discrete} if $\Abs{\U{K}}$ is a discrete subgroup of $\BR_{>0}$. Note that $\BR_{>0}\cong (\BR,+)$ via $\log$, so discrete subgroups are isomorphic to $\BZ$ (are a lattice in $\BR$), and thus a discrete absolute value is of the form $\Abs{\U{K}} = \alpha^{\BZ}$ for some $\alpha\geq 1$, and $\alpha=1$ corresponds to the trivial absolute value.

\begin{prop} In the notation of the preceding theorem, TFAE:
\begin{enumerate}
\item $A$ is principal
\item $\Abs\cdot$ is discrete
\item $A$ is Noetherian
\end{enumerate}
If any of these hold, $A$ is a discrete valuation ring (DVR).
\end{prop}
\begin{proof} ($1 \Rightarrow 2$): If $A$ is principal, then $\smm=(\pi)$ with $\Abs{\pi} < 1$. Since $A$ is a UFD, any element $x\in A-\{0\}$ can be written uniquely as $x=u\pi^n$ for $u\in\U{A}, n\geq 0$, and then $\Abs x = \Abs u \cdot \Abs{\pi}^n = \Abs{\pi}^n$. Thus $\Abs{A-\{0\}} = \Abs{\pi}^{\BN}$ and $\Abs{\U{K}} = \Abs{\pi}^{\BZ}$ so that $\Abs \cdot$ is discrete.

($2 \Rightarrow 1$): If the absolute value is discrete, we may choose $\pi\in \U{K}$ with $\Abs{\pi}<1$ but with the largest possible absolute value strictly less than $1$. Then for $x\in\smm$, we have $\Abs x < 1$, so $\Abs x\leq \Abs{\pi}$ and thus $\displaystyle \Abs{\frac{x}{\pi}}\leq 1$ so that $\displaystyle \frac{x}{\pi}\in A$. It follows that $x\in \pi A = (\pi)$, so $A$ is principal.

Clearly principal implies Noetherian, so it suffices to prove that $3\Rightarrow 2$: if $\Abs\cdot$ is not discrete, then $A$ is not Noetherian. But if the absolute value is not discrete, we can choose a convergent sequence of absolute values and, using the fact that the valuations form an additive subgroup of $\BR$, we can find a convergent sequence $(r_n)$ with $r_{n+1}>r_n$, $\lim r_n=1$, and a sequence of elements of $A$ with $\Abs{x_n} = r_n$. Now consider $I_n=\{x\in A, \Abs x\leq r_n\}$. Then
\[I_1\subset\cdots\subset I_n\subset I_{n+1}\subset\cdots\]
and $x_{n+1}\in I_{n+1}\backslash I_n$, so that $A$ is not Noetherian.

The fact that $A$ is a DVR follows trivially if any of these conditions holds.
\end{proof}

%%%%%
%%%%%
\end{document}
