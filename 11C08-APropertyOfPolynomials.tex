\documentclass[12pt]{article}
\usepackage{pmmeta}
\pmcanonicalname{APropertyOfPolynomials}
\pmcreated{2013-03-22 19:34:45}
\pmmodified{2013-03-22 19:34:45}
\pmowner{akdevaraj}{13230}
\pmmodifier{akdevaraj}{13230}
\pmtitle{A property of polynomials}
\pmrecord{17}{42566}
\pmprivacy{1}
\pmauthor{akdevaraj}{13230}
\pmtype{Definition}
\pmcomment{trigger rebuild}
\pmclassification{msc}{11C08}

% this is the default PlanetMath preamble.  as your knowledge
% of TeX increases, you will probably want to edit this, but
% it should be fine as is for beginners.

% almost certainly you want these
\usepackage{amssymb}
\usepackage{amsmath}
\usepackage{amsfonts}

% used for TeXing text within eps files
%\usepackage{psfrag}
% need this for including graphics (\includegraphics)
%\usepackage{graphicx}
% for neatly defining theorems and propositions
%\usepackage{amsthm}
% making logically defined graphics
%%%\usepackage{xypic}

% there are many more packages, add them here as you need them

% define commands here

\begin{document}
$A   PROPERTY  OF  POLYNOMIALS$
  
Let  $ f(x)$ \hspace{.5 in}\ be  a  polynomial  in  $x$  where  $x$$  and  the  coefficients $ $ a) are integers , $ b) $\hspace{.5in}\,$ x $ is  a  square  matrix and $k$\hspace[.5in}\ is a natural  number.$
$ The property:$  $ f(x + k*f(x)) is  congruent   to  $0$ mod $f(x)$.
$Proof: $ $\hspace{.5in}$$\hspace{.5in}\$\hspace{.5in}\ There  is  no loss  of  generality in taking $ k  = 1$.$ By  Taylor's  theorem  we  get $ $

f(x + f(x))  =  f(x)  +  f(x)f'(x)  +  (f(x)^2)/2f''(x)  +  (f(x)^3)/3!f'''(x)....
                  =  f(x){1  +  f'(x)   +  f(x)/2*f''(x)  +  f(x)^2/3!..}$
i.e. $ f(x + f(x)) $ is  congruent  to 0 (mod( f(x))).

%%%%%
%%%%%
\end{document}
