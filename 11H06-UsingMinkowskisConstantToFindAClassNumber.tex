\documentclass[12pt]{article}
\usepackage{pmmeta}
\pmcanonicalname{UsingMinkowskisConstantToFindAClassNumber}
\pmcreated{2013-03-22 15:05:38}
\pmmodified{2013-03-22 15:05:38}
\pmowner{alozano}{2414}
\pmmodifier{alozano}{2414}
\pmtitle{using Minkowski's constant to find a class number}
\pmrecord{4}{36822}
\pmprivacy{1}
\pmauthor{alozano}{2414}
\pmtype{Example}
\pmcomment{trigger rebuild}
\pmclassification{msc}{11H06}
\pmclassification{msc}{11R29}
%\pmkeywords{computing class numbers}
\pmrelated{ClassNumbersAndDiscriminantsTopicsOnClassGroups}

% this is the default PlanetMath preamble.  as your knowledge
% of TeX increases, you will probably want to edit this, but
% it should be fine as is for beginners.

% almost certainly you want these
\usepackage{amssymb}
\usepackage{amsmath}
\usepackage{amsthm}
\usepackage{amsfonts}

% used for TeXing text within eps files
%\usepackage{psfrag}
% need this for including graphics (\includegraphics)
%\usepackage{graphicx}
% for neatly defining theorems and propositions
%\usepackage{amsthm}
% making logically defined graphics
%%%\usepackage{xypic}

% there are many more packages, add them here as you need them

% define commands here

\newtheorem*{thm}{Theorem}
\newtheorem{defn}{Definition}
\newtheorem{prop}{Proposition}
\newtheorem{lemma}{Lemma}
\newtheorem{cor}{Corollary}

\theoremstyle{definition}
\newtheorem{exa}{Example}

% Some sets
\newcommand{\Nats}{\mathbb{N}}
\newcommand{\Ints}{\mathbb{Z}}
\newcommand{\Reals}{\mathbb{R}}
\newcommand{\Complex}{\mathbb{C}}
\newcommand{\Rats}{\mathbb{Q}}
\newcommand{\Cl}{\operatorname{Cl}}
\begin{document}
We will use the theorem of Minkowski (see \PMlinkname{the parent entry}{MinkowskisConstant}).

\begin{thm}[Minkowski's Theorem]
\label{thm1}
Let $K$ be a number field and let $D_K$ be its discriminant. Let $n=r_1+2r_2$ be the degree of $K$ over $\Rats$, where $r_1$ and $r_2$ are the number of real and complex embeddings, respectively. The class group of $K$ is denoted by $\Cl(K)$. In any ideal class $C\in \Cl(K)$, there exists an ideal $\mathfrak{A}\in C$ such that:
$$|{\bf N}(\mathfrak{A})| \leq M_K \sqrt{|D_K|}$$
where ${\bf N}(\mathfrak{A})$ denotes the absolute norm of $\mathfrak{A}$ and 
$$M_K=\frac{n!}{n^n} \left(\frac{4}{\pi}\right)^{r_2}.$$
\end{thm}

\begin{exa}
The discriminants of the quadratic fields $K_2=\Rats(\sqrt{2}),\ K_3=\Rats(\sqrt{3})$ and $K_{13}=\Rats(\sqrt{13})$ are $D_{K_2}=8,\ D_{K_3}=12$ and $D_{K_{13}}=13$ respectively. For all three $n=2=r_1$ and $r_2=0$. Therefore, the Minkowski's constants are:
$$M_{K_i}=\frac{1}{2}\sqrt{|D_{K_i}|},\quad i=2,3,13$$
so in the three cases:
$$M_{K_i}\leq \frac{1}{2}\sqrt{13}=1.802\ldots$$
Now, suppose that $C$ is an arbitrary class in $\Cl(K_i)$. By the theorem, there exists an ideal $\mathfrak{A}$, representative of $C$, such that:
$$|{\bf N}(\mathfrak{A})|<1.802\ldots <2$$ 
and therefore ${\bf N}(\mathfrak{A})=1$. Since the only ideal of norm one is the trivial ideal $\mathcal{O}_{K_i}$, which is principal, the class $C$ is also the trivial class in $\Cl(K_i)$. Hence there is only one class in the class group, and the class number is one for the three fields $K_2,\ K_3$ and $K_{13}$.\\
\end{exa}

\begin{exa}
Let $K=\Rats(\sqrt{17})$. The discriminant is $D_K=17$ and the Minkowski's bound reads:
$$M_K=\frac{1}{2}\sqrt{17}=2.06\ldots$$
Suppose that $C$ is an arbitrary class in $\Cl(K)$. By the theorem, there exists an ideal $\mathfrak{A}$, representative of $C$, such that:
$$|{\bf N}(\mathfrak{A})|<2.06\ldots$$ 
and therefore ${\bf N}(\mathfrak{A})=1$ or $2$. However,
$$2=\frac{-3+\sqrt{17}}{2}\cdot \frac{3+\sqrt{17}}{2}$$
so the ideal $2\mathcal{O}_K$ is split in $K$ and the prime ideals 
$$\left(\frac{-3+\sqrt{17}}{2} \right), \quad \left( \frac{3+\sqrt{17}}{2}\right)$$
are the only ones of norm $2$. Since they are principal, the class $C$ is the trivial class, and the class group $\Cl(K)$ is trivial. Hence, the class number of $\Rats(\sqrt{17})$ is one.
\end{exa}
%%%%%
%%%%%
\end{document}
