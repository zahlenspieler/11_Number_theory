\documentclass[12pt]{article}
\usepackage{pmmeta}
\pmcanonicalname{ProofThatTheSetOfSumproductNumbersInBase2IsFinite}
\pmcreated{2013-03-22 15:46:56}
\pmmodified{2013-03-22 15:46:56}
\pmowner{Mravinci}{12996}
\pmmodifier{Mravinci}{12996}
\pmtitle{proof that the set of sum-product numbers in base 2 is finite}
\pmrecord{6}{37739}
\pmprivacy{1}
\pmauthor{Mravinci}{12996}
\pmtype{Proof}
\pmcomment{trigger rebuild}
\pmclassification{msc}{11A63}
\pmsynonym{Proof that the set of sum-product numbers in binary is finite}{ProofThatTheSetOfSumproductNumbersInBase2IsFinite}

% this is the default PlanetMath preamble.  as your knowledge
% of TeX increases, you will probably want to edit this, but
% it should be fine as is for beginners.

% almost certainly you want these
\usepackage{amssymb}
\usepackage{amsmath}
\usepackage{amsfonts}

% used for TeXing text within eps files
%\usepackage{psfrag}
% need this for including graphics (\includegraphics)
%\usepackage{graphicx}
% for neatly defining theorems and propositions
%\usepackage{amsthm}
% making logically defined graphics
%%%\usepackage{xypic}

% there are many more packages, add them here as you need them

% define commands here
\begin{document}
For any positive integer with 0's in its binary representation, the equality

$$n = \sum_{i = 1}^m d_i \prod_{i = 1}^m d_i$$

is false, or we can rewrite it as the inequality $n > 0$. Obviously, for $n = 0$ the equality given above is true, thus 0 is the only sum-product number in binary that has 0's in its binary representation.

This leaves the Mersenne numbers (numbers whose significant digits are all 1's), an infinite set but just as manageable. The Mersenne numbers are usually expressed as $2^k - 1$, but for our purposes we express them as $\displaystyle \sum_{i = 0}^{k - 1} 2^i$, which shows that for $n = 2^k - 1$, $\displaystyle \sum_{i = 1}^m d_i = k$. Regardless of how many digits a Mersenne number has, it is obvious that $\displaystyle \prod_{i = 1}^m d_i = 1$ because of the multiplicative identity. Thus, for a Mersenne number, the right hand side of our equality above will be $k$. For any nonnegative $k$, the inequality $k < 2^k - 1$ will always hold except of course for $k = 1$.

This proves that 0 and 1 are the only sum-product numbers in binary.
%%%%%
%%%%%
\end{document}
