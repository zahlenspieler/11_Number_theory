\documentclass[12pt]{article}
\usepackage{pmmeta}
\pmcanonicalname{StormerNumber}
\pmcreated{2013-03-22 17:52:16}
\pmmodified{2013-03-22 17:52:16}
\pmowner{PrimeFan}{13766}
\pmmodifier{PrimeFan}{13766}
\pmtitle{St{\o}rmer number}
\pmrecord{5}{40350}
\pmprivacy{1}
\pmauthor{PrimeFan}{13766}
\pmtype{Definition}
\pmcomment{trigger rebuild}
\pmclassification{msc}{11A51}
\pmsynonym{Stormer number}{StormerNumber}
\pmsynonym{St\"ormer number}{StormerNumber}
\pmsynonym{arc-cotangent irreducible number}{StormerNumber}

\endmetadata

% this is the default PlanetMath preamble.  as your knowledge
% of TeX increases, you will probably want to edit this, but
% it should be fine as is for beginners.

% almost certainly you want these
\usepackage{amssymb}
\usepackage{amsmath}
\usepackage{amsfonts}

% used for TeXing text within eps files
%\usepackage{psfrag}
% need this for including graphics (\includegraphics)
%\usepackage{graphicx}
% for neatly defining theorems and propositions
%\usepackage{amsthm}
% making logically defined graphics
%%%\usepackage{xypic}

% there are many more packages, add them here as you need them

% define commands here
\newcommand{\gpf}{\textrm{gpf}}
\begin{document}
A {\em St{\o}rmer number} or {\em arc-cotangent irreducible number} is a positive integer $n$ for which the greatest prime factor of $n^2 + 1$ exceeds $2n$. The first few St{\o}rmer numbers are 2, 4, 5, 6, 9, 10, 11, 12, 14, 15, 16, 19, 20, 22, 23, 24, 25, 26, 27, 28, 29, 33, 34, 35, 36, 37, 39, 40, 42, 44, 45, 48, 49, etc., listed in A005528 of Sloane's OEIS. Weakening the inequality from $\gpf(n^2 + 1) > 2n$ to $\gpf(n^2 + 1) \geq 2n$ makes no difference other than admitting 1 to the list (and possibly changing index offsets accordingly).

The St{\o}rmer numbers arise in connection with the problem of representing Gregory numbers $t_{\frac{a}{b}}$ as sums of Gregory numbers for integers. Conway and Guy explain in their book thus: ``To find St{\o}rmer's decomposition for $t_{a / b}$, you repeatedly multiply $a + bi$ by numbers $n \pm i$ for which $n$ is a St{\o}rmer number and the sign is chosen so that you can cancel the corresponding prime number $p$ ($n$ is the smallest number for which $n^2 + 1$ is divisible by $p$).''

St{\o}rmer numbers are named after the Norwegian physicist \PMlinkname{Carl St{\o}rmer}{CarlStormer}.

\begin{thebibliography}{2}
\bibitem{jc} John H. Conway \& R. K. Guy, {\it The Book of Numbers}. New York: Copernicus Press (1996): 245 - 248. 
\bibitem{jt} J. Todd, ``A problem on arc tangent relations'', {\it Amer. Math. Monthly}, {\bf 56} (1949): 517 - 528.
\end{thebibliography}
%%%%%
%%%%%
\end{document}
