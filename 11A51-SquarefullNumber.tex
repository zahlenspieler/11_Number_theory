\documentclass[12pt]{article}
\usepackage{pmmeta}
\pmcanonicalname{SquarefullNumber}
\pmcreated{2013-03-22 11:45:26}
\pmmodified{2013-03-22 11:45:26}
\pmowner{KimJ}{5}
\pmmodifier{KimJ}{5}
\pmtitle{squarefull number}
\pmrecord{10}{30204}
\pmprivacy{1}
\pmauthor{KimJ}{5}
\pmtype{Definition}
\pmcomment{trigger rebuild}
\pmclassification{msc}{11A51}
\pmclassification{msc}{55-00}
\pmclassification{msc}{82-00}
\pmclassification{msc}{83-00}
\pmclassification{msc}{81-00}
\pmclassification{msc}{39A99}
\pmclassification{msc}{39B72}
\pmclassification{msc}{33E30}
\pmsynonym{powerful number}{SquarefullNumber}
%\pmkeywords{number theory}
\pmrelated{NFullNumber}

\endmetadata

\usepackage{amssymb}
\usepackage{amsmath}
\usepackage{amsfonts}
\usepackage{graphicx}
%%%%\usepackage{xypic}
\begin{document}
A natural number $n$ is called squarefull (or powerful) if for every prime $p | n$ we have $p^2 | n$. In 1978 Erd\H{o}s conjectured that we cannot have three consecutive squarefull natural numbers. If we assume the ABC Conjecture, there are only finitely many such consecutive triples.
%%%%%
%%%%%
%%%%%
%%%%%
\end{document}
