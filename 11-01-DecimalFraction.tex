\documentclass[12pt]{article}
\usepackage{pmmeta}
\pmcanonicalname{DecimalFraction}
\pmcreated{2013-03-22 17:27:15}
\pmmodified{2013-03-22 17:27:15}
\pmowner{CWoo}{3771}
\pmmodifier{CWoo}{3771}
\pmtitle{decimal fraction}
\pmrecord{10}{39836}
\pmprivacy{1}
\pmauthor{CWoo}{3771}
\pmtype{Definition}
\pmcomment{trigger rebuild}
\pmclassification{msc}{11-01}
\pmrelated{RationalNumber}
\pmdefines{decimal number}

\endmetadata

\usepackage{amssymb,amscd}
\usepackage{amsmath}
\usepackage{amsfonts}
\usepackage{mathrsfs}

% used for TeXing text within eps files
%\usepackage{psfrag}
% need this for including graphics (\includegraphics)
%\usepackage{graphicx}
% for neatly defining theorems and propositions
\usepackage{amsthm}
% making logically defined graphics
%%\usepackage{xypic}
\usepackage{pst-plot}
\usepackage{psfrag}

% define commands here
\newtheorem{prop}{Proposition}
\newtheorem{thm}{Theorem}
\newtheorem{ex}{Example}
\newcommand{\real}{\mathbb{R}}
\newcommand{\pdiff}[2]{\frac{\partial #1}{\partial #2}}
\newcommand{\mpdiff}[3]{\frac{\partial^#1 #2}{\partial #3^#1}}
\begin{document}
A rational number $d$ is called a \emph{decimal fraction} if $10^kd$ is an integer for some non-negative integer $k$.  For example, any integer, as well as rationals such as 
$$0.23123,\qquad \frac{3}{4},\qquad \frac{236}{125}$$
are all decimal fractions.  Rational numbers such as $$\frac{1}{3},\qquad -\frac{227}{12}, \qquad 2.\overline{312}$$ are not.

There are two other ways of characterizing a decimal fraction: for a rational number $d$,
\begin{enumerate}
\item $d$ is as in the above definition;
\item $d$ can be written as a fraction $\displaystyle{\frac{p}{q}}$, where $p$ and $q$ are integers, and $q=2^m5^n$ for some non-negative integers $m$ and $n$;
\item $d$ has a terminating decimal expansion, meaning that it has a decimal representation $$a.d_1d_2\cdots d_n000\cdots$$
where $a$ is a nonnegative integer and $d_i$ is any one of the digits $0,\ldots,9$.
\end{enumerate}

A decimal fraction is sometimes called a \emph{decimal number}, although a decimal number in the most general sense may have non-terminating decimal expansions.

\textbf{Remarks}.  Let $D\subset \mathbb{Q}$ be the set of all decimal fractions.
\begin{itemize}
\item
If $a,b \in D$, then $a\cdot b$ and $a+b\in D$ as well.  Also, $-a \in D$ whenever $a\in D$.  In other words, $D$ is a subring of $\mathbb{Q}$.  Furthermore, as an abelian group, $D$ is $2$-divisible and $5$-divisible.  However, unlike $\mathbb{Q}$, $D$ is not \PMlinkname{divisible}{DivisibleGroup}.
\item
As inherited from $\mathbb{Q}$, $D$ has a total order structure.  It is easy to see that $D$ is \PMlinkname{dense}{DenseTotalOrder}: for any $a,b\in D$ with $a< b$, there is $c\in D$ such that $a<c<b$.  Simply take $c=\displaystyle{\frac{a+b}{2}}$.
\item
From a topological point of view, $D$, as a subset of $\mathbb{R}$, is dense in $\mathbb{R}$.  This is essentially the fact that every real number has a decimal expansion, so that every real number can be ``approximated'' by a decimal fraction to any degree of accuracy.
\item
We can associate each decimal fraction $d$ with the least non-negative integer $k(d)$ such that $10^{k(d)}d$ is an integer.  This integer is uniquely determined by $d$.  In fact, $k(d)$ is the last decimal place where its corresponding digit is non-zero in its decimal representation.  For example, $k(1.41243)=5$ and $k(7/25)=2$.  It is not hard to see that if we write $d=\displaystyle{\frac{p}{2^m5^n}}$, where $p$ and $2^m5^n$ are coprime, then $k(d)=\max(m,n)$.
\item
For each non-negative integer $i$, let $D(i)$ be the set of all $d\in D$ such that $k(d)=i$.  Then $D$ can be partitioned into sets $$D=D(0)\cup D(1) \cup \cdots \cup D(n) \cup \cdots.$$  Note that $D(0)=\mathbb{Z}$.  Another basic property is that if $a\in D(i)$ and $b\in D(j)$ with $i<j$, then $a+b\in D(j)$.
\end{itemize}
%%%%%
%%%%%
\end{document}
