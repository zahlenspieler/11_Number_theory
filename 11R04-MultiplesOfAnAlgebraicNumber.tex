\documentclass[12pt]{article}
\usepackage{pmmeta}
\pmcanonicalname{MultiplesOfAnAlgebraicNumber}
\pmcreated{2014-05-16 19:58:36}
\pmmodified{2014-05-16 19:58:36}
\pmowner{pahio}{2872}
\pmmodifier{pahio}{2872}
\pmtitle{multiples of an algebraic number}
\pmrecord{10}{37126}
\pmprivacy{1}
\pmauthor{pahio}{2872}
\pmtype{Theorem}
\pmcomment{trigger rebuild}
\pmclassification{msc}{11R04}
\pmrelated{TheoryOfAlgebraicNumbers}
\pmrelated{AlgebraicSinesAndCosines}
\pmrelated{SomethingRelatedToAlgebraicInteger}
\pmrelated{RationalAlgebraicIntegers}

% this is the default PlanetMath preamble.  as your knowledge
% of TeX increases, you will probably want to edit this, but
% it should be fine as is for beginners.

% almost certainly you want these
\usepackage{amssymb}
\usepackage{amsmath}
\usepackage{amsfonts}

% used for TeXing text within eps files
%\usepackage{psfrag}
% need this for including graphics (\includegraphics)
%\usepackage{graphicx}
% for neatly defining theorems and propositions
 \usepackage{amsthm}
% making logically defined graphics
%%%\usepackage{xypic}

% there are many more packages, add them here as you need them

% define commands here

\theoremstyle{definition}
\newtheorem*{thmplain}{Theorem}
\begin{document}
\textbf{Theorem.}\, If $\alpha$ is an algebraic number, then there exists a non-zero \PMlinkname{multiple}{GeneralAssociativity} of $\alpha$ which is an algebraic integer.


{\em Proof.}\, Let $\alpha$ be a \PMlinkescapetext{root} of the equation
$$x^n\!+\!r_1x^{n-1}\!+\!r_2x^{n-2}\!+\cdots+\!r_n = 0,$$
where $r_1$, $r_2$, \ldots, $r_n$ are rational numbers ($n > 0$).\, Let $l$ be the least common multiple of the denominators of the $r_j$'s.\, Then we have
$$0 = l^n(\alpha^n\!+\!r_1\alpha^{n-1}\!+\!r_2\alpha^{n-2}\!+\cdots+\!r_n) =
 (l\alpha)^n\!+\!lr_1(l\alpha)^{n-1}\!+\!l^2r_2(l\alpha)^{n-2}\!+\cdots+\!l^nr_n,$$
i.e. the \PMlinkescapetext{multiple \,$l\alpha$\, of $\alpha$ satisfies} the algebraic equation
$$x^n\!+\!lr_1x^{n-1}\!+\!l^2r_2x^{n-2}\!+\cdots+\!l^nr_n = 0$$
with rational integer coefficients.\\

According to the theorem, any algebraic number $\xi$ is a 
\PMlinkname{quotient}{Division} of an algebraic integer 
(of the field $\mathbb{Q}(\xi)$) and a rational integer.
%%%%%
%%%%%
\end{document}
