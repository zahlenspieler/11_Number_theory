\documentclass[12pt]{article}
\usepackage{pmmeta}
\pmcanonicalname{UnitaryDivisor}
\pmcreated{2013-03-22 16:53:20}
\pmmodified{2013-03-22 16:53:20}
\pmowner{CompositeFan}{12809}
\pmmodifier{CompositeFan}{12809}
\pmtitle{unitary divisor}
\pmrecord{4}{39144}
\pmprivacy{1}
\pmauthor{CompositeFan}{12809}
\pmtype{Definition}
\pmcomment{trigger rebuild}
\pmclassification{msc}{11A51}

\endmetadata

% this is the default PlanetMath preamble.  as your knowledge
% of TeX increases, you will probably want to edit this, but
% it should be fine as is for beginners.

% almost certainly you want these
\usepackage{amssymb}
\usepackage{amsmath}
\usepackage{amsfonts}

% used for TeXing text within eps files
%\usepackage{psfrag}
% need this for including graphics (\includegraphics)
%\usepackage{graphicx}
% for neatly defining theorems and propositions
%\usepackage{amsthm}
% making logically defined graphics
%%%\usepackage{xypic}

% there are many more packages, add them here as you need them

% define commands here

\begin{document}
Given the divisors $d_i$ of an integer $n$, if it is the case that for a particular $d_i$ the equality $\gcd(d_i, \frac{n}{d_i}) = 1$ holds true, then $d_i$ is called a {\em unitary divisor} of $n$. For example, the unitary divisors of 120 are 1, 3, 5, 8, 15, 24, 40, 120 (while 2, 4, 6, 10, 12, 20, 30, 60 are proper divisors but not unitary divisors). All the divisors of squarefree numbers are also unitary divisors.
%%%%%
%%%%%
\end{document}
