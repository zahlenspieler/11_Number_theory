\documentclass[12pt]{article}
\usepackage{pmmeta}
\pmcanonicalname{FraenkelsPartitionTheorem}
\pmcreated{2013-03-22 13:40:09}
\pmmodified{2013-03-22 13:40:09}
\pmowner{Kevin OBryant}{1315}
\pmmodifier{Kevin OBryant}{1315}
\pmtitle{Fraenkel's partition theorem}
\pmrecord{6}{34333}
\pmprivacy{1}
\pmauthor{Kevin OBryant}{1315}
\pmtype{Theorem}
\pmcomment{trigger rebuild}
\pmclassification{msc}{11B83}
\pmsynonym{Fraenkel's theorem}{FraenkelsPartitionTheorem}
\pmrelated{BeattySequence}
\pmrelated{BeattysTheorem}
\pmrelated{DataStream}
\pmrelated{WideraInterlaceAndDeinterlace}

\endmetadata

\usepackage{amsmath,amssymb,amsthm}
\newcommand{\alpr}{{\alpha^\prime}}
\newcommand{\bepr}{\beta^\prime}
\newcommand{\floor}[1]{\left\lfloor #1 \right\rfloor}
\newcommand{\ceiling}[1]{\left\lceil #1 \right\rceil}
\newenvironment{namedtheorem}[1]{\medskip \noindent {\bf #1:}\begin{em}}{\end{em}\medskip}
\begin{document}
Fraenkel's partition theorem is a generalization of Beatty's Theorem. Set
    $$ {\cal B}(\alpha,\alpha^\prime) := \left( \floor{\frac{n-\alpha^\prime}{\alpha}} \right)_{n=1}^\infty. $$
We say that two sequences partition $\mathbb{N}=\{1,2,3,\ldots\}$ if the sequences are disjoint and their union is $\mathbb{N}$.


\begin{namedtheorem}{Fraenkel's Partition Theorem}
The sequences ${\cal B}(\alpha,\alpr)$ and ${\cal B}(\beta,\bepr)$ partition $\mathbb{N}$ if and only if the following five conditions are
satisfied.
\begin{enumerate}
 \item
    $0<\alpha<1$.
 \item
    $\alpha+\beta=1$.
 \item
    $0\le\alpha+\alpr \le 1$.
 \item
    If $\alpha$ is irrational, then $\alpr+\bepr=0$ and $k\alpha+\alpr\not\in\mathbb{Z}$ for $2\le k\in \mathbb{N}$.
 \item
    If $\alpha$ is rational (say $q\in \mathbb{N}$ is minimal with $q\alpha \in \mathbb{N}$), then
    $\frac1q \le \alpha+\alpr$ and $\ceiling{q\alpr}+\ceiling{q\bepr}=1.$
\end{enumerate}
\end{namedtheorem}

{\Huge \bf References}

\begin{description}
\item[ [1] ] Aviezri S. Fraenkel, {\em The bracket function and complementary sets of integers}, Canad. J.
    Math. {\bf 21} (1969), 6--27. {\bf \PMlinkexternal{MR
    38:3214}{http://www.ams.org/mathscinet-getitem?mr=38:3214}}
\item[ [2] ] Kevin O'Bryant, {\em Fraenkel's partition and Brown's decomposition},
\PMlinkexternal{arXiv:math.NT/0305133}{http://lanl.arxiv.org/abs/math.NT/0305133}.
\end{description}
%%%%%
%%%%%
\end{document}
