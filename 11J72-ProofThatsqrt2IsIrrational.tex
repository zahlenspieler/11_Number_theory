\documentclass[12pt]{article}
\usepackage{pmmeta}
\pmcanonicalname{ProofThatsqrt2IsIrrational}
\pmcreated{2013-03-22 12:39:13}
\pmmodified{2013-03-22 12:39:13}
\pmowner{Wkbj79}{1863}
\pmmodifier{Wkbj79}{1863}
\pmtitle{proof that $\sqrt{2}$ is irrational}
\pmrecord{11}{32920}
\pmprivacy{1}
\pmauthor{Wkbj79}{1863}
\pmtype{Proof}
\pmcomment{trigger rebuild}
\pmclassification{msc}{11J72}
\pmrelated{Irrational}
\pmrelated{Surd}

\endmetadata

% this is the default PlanetMath preamble.  as your knowledge
% of TeX increases, you will probably want to edit this, but
% it should be fine as is for beginners.

% almost certainly you want these
\usepackage{amssymb}
\usepackage{amsmath}
\usepackage{amsfonts}

% used for TeXing text within eps files
%\usepackage{psfrag}
% need this for including graphics (\includegraphics)
%\usepackage{graphicx}
% for neatly defining theorems and propositions
%\usepackage{amsthm}
% making logically defined graphics
%%%\usepackage{xypic}

% there are many more packages, add them here as you need them

% define commands here
\def\N{{\mathbb N}}
\begin{document}
\PMlinkescapeword{argument}
\PMlinkescapeword{hypothesis}
\PMlinkescapeword{order}

Assume that the \PMlinkname{square root of $2$}{SquareRootOf2} is rational.  Then we can write

\begin{equation*}
 \sqrt{2} = \frac{a}{b},
\end{equation*}
where $a,b\in\N$ and $a$ and $b$ are relatively prime.  Then $\displaystyle 2=(\sqrt{2})^2=\left(\frac{a}{b}\right)^2=\frac{a^2}{b^2}$.  Thus, $a^2=2b^2$.  Therefore, $2\mid a^2$.  Since $2$ is prime, it must divide $a$.  Then $a=2c$ for some $c\in\N$.  Thus, $2b^2=a^2=(2c)^2=4c^2$, yielding that $b^2=2c^2$.  Therefore, $2\mid b^2$.  Since $2$ is prime, it must divide $b$.

Since $2\mid a$ and $2\mid b$, we have that $a$ and $b$ are not relatively prime, which contradicts the hypothesis.  Hence, the initial assumption is false.  It follows $\sqrt{2}$ is irrational.

With a little bit of work, this argument can be generalized to any positive integer that is not a square.  Let $n$ be such an integer.  Then there must exist a prime $p$ and $k,m\in\N$ such that $n=p^km$, where $p\nmid m$ and $k$ is odd. Assume that $\sqrt{n}=a/b$, where $a,b\in\N$ and are relatively prime. Then $\displaystyle p^km=n=(\sqrt{n})^2=\left(\frac{a}{b}\right)^2=\frac{a^2}{b^2}$. Thus, $a^2=p^kmb^2$.  From the fundamental theorem of arithmetic, it is clear that the maximum powers of $p$ that divides $a^2$ and $b^2$ are even.  Since $k$ is odd and $p$ does not divide $m$, the maximum power of $p$ that divides $p^kmb^2$ is also odd.  Thus, the same should be true for $a^2$.  Hence, we have reached a contradiction and $\sqrt{n}$ must be irrational.

The same argument can be generalized even more, for example to the case of nonsquare irreducible fractions and to higher order roots.
%%%%%
%%%%%
\end{document}
