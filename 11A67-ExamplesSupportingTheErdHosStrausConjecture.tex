\documentclass[12pt]{article}
\usepackage{pmmeta}
\pmcanonicalname{ExamplesSupportingTheErdHosStrausConjecture}
\pmcreated{2013-03-22 16:54:10}
\pmmodified{2013-03-22 16:54:10}
\pmowner{PrimeFan}{13766}
\pmmodifier{PrimeFan}{13766}
\pmtitle{examples supporting the Erd\H{o}s-Straus conjecture}
\pmrecord{9}{39160}
\pmprivacy{1}
\pmauthor{PrimeFan}{13766}
\pmtype{Example}
\pmcomment{trigger rebuild}
\pmclassification{msc}{11A67}

\endmetadata

% this is the default PlanetMath preamble.  as your knowledge
% of TeX increases, you will probably want to edit this, but
% it should be fine as is for beginners.

% almost certainly you want these
\usepackage{amssymb}
\usepackage{amsmath}
\usepackage{amsfonts}

% used for TeXing text within eps files
%\usepackage{psfrag}
% need this for including graphics (\includegraphics)
%\usepackage{graphicx}
% for neatly defining theorems and propositions
%\usepackage{amsthm}
% making logically defined graphics
%%%\usepackage{xypic}

% there are many more packages, add them here as you need them

% define commands here

\begin{document}
As with any conjecture, a million examples are not enough to prove the Erd\H{o}s-Straus conjecture, but a single counterexample is enough to disprove. But if these examples at least provide a slightly better understanding of the problem at hand, the effort is not entirely wasted.

Most users are well aware that a computer algebra system (and even fraction-capable scientific calculators) will automatically express a fraction by the lowest common denominator. This presents no problem for smaller instances, such as $\frac{4}{8}$, but for larger denominators it might not always be obvious that, for example, $\frac{2}{1729} = \frac{4}{3458}$. If one is unsure, one can always enter the fraction $\frac{4}{n}$ by itself and the CAS will dutifully respond with the LCD expression, which will hopefully match the sum of three unit fractions entered earlier.

I say we start with $n = 2$ if for no other reason than to start at the beginning. The only possible solution is $$\frac{4}{2} = 1 + \frac{1}{2} + \frac{1}{2}.$$ Similarly, for $n = 3$ we have $$\frac{4}{3} = \frac{1}{2} + \frac{1}{2} + \frac{1}{3}.$$ It is with $n = 4$ that solutions with distinct denominators become available: $$\frac{4}{4} = \frac{1}{2} + \frac{1}{3} + \frac{1}{6},$$ easily suggested by the study of perfect numbers. Some may consider solutions with distinct denominators more elegant, some are just happy to find any solution for a given $n$.

Since addition is commutative, it does not matter in what order we list the unit fractions that add up to our desired $\frac{4}{n}$. However, by tradition, they are listed in descending order: the biggest fraction (the one with the smallest denominator) is listed first, the smallest last.

The following table lists some distinct denominator solutions for $4 < n < 21$, with the numerators omitted for compactness:

\begin{tabular}{|r|l|}
5 & 2, 5, 10 \\
6 & 2, 8, 24; 3, 4, 12 \\
7 & 2, 21, 42; 3, 6, 14 \\
8 & 3, 6, 42; 3, 8, 24; 4, 6, 12 \\
9 & 3, 12, 36; 4, 6, 36 \\
10 & 3, 20, 60; 5, 6, 30; 4, 10, 20 \\
11 & 3, 44, 132; 4, 11, 44; 4, 12, 33 \\
12 & 4, 18, 36; 4, 16, 48; 6, 8, 24 \\
13 & 4, 26, 52 \\
14 & 4, 42, 84; 5, 70, 140; 6, 14, 21; 7, 8, 56; 6, 12, 28 \\
15 & 4, 90, 180; 5, 18, 90; 6, 15, 30; 7, 10, 42 \\
16 & 5, 30, 60; 6, 12, 84; 8, 12, 24; 6, 20, 30 \\
17 & 5, 30, 510; 6, 17, 102 \\
18 & 6, 24, 72; 8, 12, 72 \\
19 & 6, 38, 57 \\
20 & 7, 20, 140 \\
\end{tabular}

As the table shows, solutions for prime $n$ are harder to come by than for composite $n$. When $n = pq$, with $p$ prime and $q$ any other integer, solutions for $n$ can be simply derived from those for $p$ by multiplying those denominators by $q$. For example, for $n = 42$, we can take the solutions for $n = 6$, multiply those denominators by 7 and voil\`{a}: $$\frac{4}{42} = \frac{1}{12} + \frac{1}{126} + \frac{1}{252}.$$
%%%%%
%%%%%
\end{document}
