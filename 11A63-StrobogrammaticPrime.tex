\documentclass[12pt]{article}
\usepackage{pmmeta}
\pmcanonicalname{StrobogrammaticPrime}
\pmcreated{2013-03-22 16:19:24}
\pmmodified{2013-03-22 16:19:24}
\pmowner{PrimeFan}{13766}
\pmmodifier{PrimeFan}{13766}
\pmtitle{strobogrammatic prime}
\pmrecord{5}{38451}
\pmprivacy{1}
\pmauthor{PrimeFan}{13766}
\pmtype{Definition}
\pmcomment{trigger rebuild}
\pmclassification{msc}{11A63}

% this is the default PlanetMath preamble.  as your knowledge
% of TeX increases, you will probably want to edit this, but
% it should be fine as is for beginners.

% almost certainly you want these
\usepackage{amssymb}
\usepackage{amsmath}
\usepackage{amsfonts}

% used for TeXing text within eps files
%\usepackage{psfrag}
% need this for including graphics (\includegraphics)
%\usepackage{graphicx}
% for neatly defining theorems and propositions
%\usepackage{amsthm}
% making logically defined graphics
%%%\usepackage{xypic}

% there are many more packages, add them here as you need them

% define commands here

\begin{document}
A {\em strobogrammatic prime} is a prime number that, given a base and given a set of glyphs, is a strobogrammatic number (it appears the same whether viewed normally or upside down). In base 10, given a set of glyphs where 0, 1 and 8 are symmetrical around the horizontal axis, and 6 and 9 are the same as each other upside down, (such as on the seven-segment display of a calculator), the first few strobogrammatic primes are: 11, 101, 181, 619, 16091, 18181 (listed in A007597 of Sloane's OEIS).

In binary, given a glyph for 1 consisting of a single line without hooks or serifs, all Mersenne primes are strobogrammatic primes. Palindromic primes in binary are also strobogrammatic.

In base 10, dihedral primes that don't use 2 or 5 are also strobogrammatic primes.
%%%%%
%%%%%
\end{document}
