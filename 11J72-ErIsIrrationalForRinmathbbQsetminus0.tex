\documentclass[12pt]{article}
\usepackage{pmmeta}
\pmcanonicalname{ErIsIrrationalForRinmathbbQsetminus0}
\pmcreated{2013-03-22 15:07:46}
\pmmodified{2013-03-22 15:07:46}
\pmowner{Cosmin}{8605}
\pmmodifier{Cosmin}{8605}
\pmtitle{$e^r$ is irrational for $r\in\mathbb{Q}\setminus\{0\}$}
\pmrecord{12}{36872}
\pmprivacy{1}
\pmauthor{Cosmin}{8605}
\pmtype{Theorem}
\pmcomment{trigger rebuild}
\pmclassification{msc}{11J72}
\pmsynonym{$e^r$ is irrational for non-zero rational r}{ErIsIrrationalForRinmathbbQsetminus0}
\pmsynonym{irrationality of the exponential function on $\mathbb{Q}$}{ErIsIrrationalForRinmathbbQsetminus0}
%\pmkeywords{e}
%\pmkeywords{e irrational}
%\pmkeywords{e^r}
%\pmkeywords{rational}
%\pmkeywords{irrational}
%\pmkeywords{irrational exponential}
\pmrelated{Irrational}
\pmrelated{EIsIrrationalProof}
\pmrelated{EIsIrrational}
\pmrelated{EIsTranscendental}

% This is Cosmin's PlanetMath preamble.

% Packages
\usepackage{amssymb}
\usepackage{amsmath}
\usepackage{amsfonts}
\usepackage{graphicx}
\usepackage{amsthm}
\usepackage{mathrsfs}
%%%\usepackage{xypic}

% Theorem Environments
\newtheorem*{thm}{Theorem}
\newtheorem*{lem}{Lemma}
\newtheorem*{cor}{Corollary}

% New Commands
  %Sets
    \newcommand{\bbP}{\mathbb{P}}
    \newcommand{\bbN}{\mathbb{N}}
    \newcommand{\bbZ}{\mathbb{Z}}
    \newcommand{\bbQ}{\mathbb{Q}}
    \newcommand{\bbR}{\mathbb{R}}
    \newcommand{\bbC}{\mathbb{C}}
    \newcommand{\bbK}{\mathbb{K}}
    \newcommand{\bbB}{\mathbb{B}}
    \newcommand{\bbS}{\mathbb{S}}
    \newcommand{\bbA}{\mathbb{A}}
    \newcommand{\bbT}{\mathbb{T}}
  %Script and Cal Letters
    \newcommand{\scP}{\mathscr{P}}
    \newcommand{\scF}{\mathscr{F}}
    \newcommand{\scC}{\mathscr{C}}
    \newcommand{\scL}{\mathscr{L}}
    \newcommand{\caR}{\mathcal{R}}
    \newcommand{\caP}{\mathcal{P}}
    \newcommand{\caM}{\mathcal{M}}
    \newcommand{\caS}{\mathcal{S}}
    \newcommand{\caT}{\mathcal{T}}
    \newcommand{\caU}{\mathcal{U}}
    \newcommand{\caX}{\mathcal{X}}
    \newcommand{\caY}{\mathcal{Y}}
    \newcommand{\caZ}{\mathcal{Z}}
  %Other Commands
    \newcommand{\vect}[1]{\boldsymbol{#1}}
\begin{document}
\PMlinkescapeword{observation}
\PMlinkescapeword{order}
\PMlinkescapeword{times}
\PMlinkescapeword{simple}
We here present a proof of the following theorem:
\begin{thm} \( e^r \) is irrational for all \( r\in\bbQ\setminus\{0\} \) \end{thm}
To begin with, note that it is sufficient to show that \( e^u \) is irrational for any \PMlinkname{positive integer}{NaturalNumber}\footnote{In this entry, \( \bbN := \{1,2,3,\dotsc\} \) and \(\bbN_0 := \bbN \cup \{0\}\).} \( u \) (for if \( e^r = e^{\frac{u}{v}} \) were rational, so would \(( e^{\frac{u}{v}})^v = e^u \)). Next, we look at some simple properties of polynomial \( \displaystyle f_n(x) := \frac{x^n (1-x)^n}{n!} \): 
\begin{itemize}
\item \( \displaystyle f_n(x)=\frac{1}{n!}\sum_{i=n}^{2n} c_i x^i \), with \( c_i \in \bbZ \) for all \( i \). 
\item \(f_n^{(k)}(0)\) and \(f_n^{(k)}(1)\) are integers for all \( k\in\bbN_0 \): as \( 0 \) is a \PMlinkname{root}{Root} of order \( n \), \( f_n^{(k)}(0)=0 \) unless \( n\leq k\leq 2n \), in which case \( f_n^{(k)}(0)=\frac{k!}{n!} c_k \), an integer. Since \( f_n^{(k)}(x)=(-1)^k f_n^{(k)}(1-x) \), the same is true for \( f_n^{(k)}(1) \).
\item For all \( 0<x<1 \) we have \( 0<f_n(x)<\frac{1}{n!} \). 
\end{itemize}
Now we can readily prove the theorem:
\begin{proof}
Assume that \( e^u=\frac{a}{b} \) for some \( (a,b)\in\bbN^2 \) and let
\[ F_n(x) := \sum_{k=0}^\infty (-1)^k u^{2n-k} f_n^{(k)}(x), \] 
which is actually a finite sum since \( f_n^{(k)}(x)=0 \) for all \( k>2n \). Differentiating \( F_n(x) \) yields \( F_n'(x)= u^{2n+1} f_n(x) - u F_n(x) \) and thus:
\[ \frac{d}{dx}\left[e^{ux} F_n(x)\right] = ue^{ux} F_n(x) + e^{ux} F'_n(x) = u^{2n+1} e^{ux} f_n(x). \]
Now consider the sequence
\[ (w_n)_{n\in\bbN} := b\int_0^1 u^{2n+1}e^{ux}f_n(x)\,dx = b\left[ e^{ux} F_n(x) \right]^1_0 = a F_n(1) - b F_n(0). \]
Given the remarks on \( f_n(x) \), \( w_n \) should be an integer for all \( n\in\bbN \), yet it is clear that \( w_n < b u^{2n+1}\frac{1}{n!} = \frac{a}{n!} u^{2n+1} \) and so \( \lim\limits_{n\to\infty}w_n = 0 \), a contradiction.
\end{proof}

The result could also easily have been obtained by starting with \( w_n \) and integrating by parts \( 2n \) times. Note also that much stronger statements are known, such as ``\( e^a \) is transcendental for all \( a\in\bbA\setminus\{0\} \)''\footnote{\(\bbA\) denotes the set of algebraic numbers.}. We conclude this entry with the following evident corollary:
\begin{cor} For all \( r\in\bbQ^{+},\: \log r\) is irrational. \end{cor}

\begin{thebibliography}{2}
\bibitem{AZ} {\sc M. Aigner \& G. M. Ziegler}: \emph{Proofs from THE BOOK}, 3\(^\mathrm{rd}\) edition (2004), Springer-Verlag, 30--31.
\bibitem{HW} {\sc G. H. Hardy \& E. M. Wright}: \emph{An Introduction to the Theory of Numbers}, 5\(^\mathrm{th}\) edition (1979), Oxford University Press, 46--47.
\end{thebibliography}
%%%%%
%%%%%
\end{document}
