\documentclass[12pt]{article}
\usepackage{pmmeta}
\pmcanonicalname{ConvolutionInversesForArithmeticFunctions}
\pmcreated{2013-03-22 15:58:32}
\pmmodified{2013-03-22 15:58:32}
\pmowner{Wkbj79}{1863}
\pmmodifier{Wkbj79}{1863}
\pmtitle{convolution inverses for arithmetic functions}
\pmrecord{27}{37992}
\pmprivacy{1}
\pmauthor{Wkbj79}{1863}
\pmtype{Theorem}
\pmcomment{trigger rebuild}
\pmclassification{msc}{11A25}
\pmrelated{ArithmeticFunction}
\pmrelated{MultiplicativeFunction}
\pmrelated{ArithmeticFunctionsFormARing}
\pmrelated{ElementaryResultsAboutMultiplicativeFunctionsAndConvolution}

\endmetadata

\usepackage{amssymb}
\usepackage{amsmath}
\usepackage{amsfonts}

\usepackage{psfrag}
\usepackage{graphicx}
\usepackage{amsthm}
%%\usepackage{xypic}

\newtheorem*{thm*}{Theorem}
\begin{document}
\PMlinkescapeword{convolution}

\begin{thm*}
An arithmetic function $f$ has a convolution inverse if and only if $f(1) \neq 0$.
\end{thm*}

\begin{proof}
If $f$ has a convolution inverse $g$, then $f*g=\varepsilon$, where $\varepsilon$ denotes the convolution identity function.  Thus, $1=\varepsilon(1)=(f*g)(1)=f(1)g(1)$, and it follows that $f(1) \neq 0$.

Conversely, if $f(1) \neq 0$, then an arithmetic function $g$ must be constructed such that $(f*g)(n)=\varepsilon(n)$ for all $n \in \mathbb{N}$.  This will be done by induction on $n$.

Since $f(1) \neq 0$, we have that $\displaystyle \frac{1}{f(1)} \in \mathbb{C}$.  Define $\displaystyle g(1)=\frac{1}{f(1)}$.

Now let $k \in \mathbb{N}$ with $k>1$ and $g(1), \dots, g(k-1)$ be such that $(f*g)(n)=\varepsilon(n)$ for all $n \in \mathbb{N}$ with $n<k.$  Define

$$g(k)=-\frac{1}{f(1)}\sum_{d|k \text{ and } d<k} f \left( \frac{k}{d} \right) g(d).$$

Then

\begin{center}
\begin{tabular}{ll}
$\displaystyle (f*g)(k)$ & $\displaystyle = \sum_{d|k} f \left( \frac{k}{d} \right) g(d)$ \\
& $\displaystyle = f(1)g(k) + \sum_{d|k \text{ and } d<k} f \left( \frac{k}{d} \right) g(d)$ \\
& $\displaystyle = f(1) \left( -\frac{1}{f(1)}\sum_{d|k \text{ and } d<k} f \left( \frac{k}{d} \right) g(d) \right) + \sum_{d|k \text{ and } d<k} f \left( \frac{k}{d} \right) g(d)$ \\
& $\displaystyle =0$ \\
& $\displaystyle =\varepsilon(k).$ \end{tabular}
\end{center}
\end{proof}

In the entry titled arithmetic functions form a ring, it is proven that convolution is associative and commutative.  Thus, $G=\{ f \colon \mathbb{N} \to \mathbb{C} \, | f(1) \neq 0 \}$ is an abelian group under convolution.  The set of all multiplicative functions is a subgroup of $G$.
%%%%%
%%%%%
\end{document}
