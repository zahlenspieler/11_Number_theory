\documentclass[12pt]{article}
\usepackage{pmmeta}
\pmcanonicalname{ExistenceOfPrimitiveRootsForPowersOfAnOddPrime}
\pmcreated{2013-03-22 16:21:01}
\pmmodified{2013-03-22 16:21:01}
\pmowner{alozano}{2414}
\pmmodifier{alozano}{2414}
\pmtitle{existence of primitive roots for powers of an odd prime}
\pmrecord{4}{38484}
\pmprivacy{1}
\pmauthor{alozano}{2414}
\pmtype{Theorem}
\pmcomment{trigger rebuild}
\pmclassification{msc}{11-00}
\pmrelated{EveryPrimeHasAPrimitiveRoot}

% this is the default PlanetMath preamble.  as your knowledge
% of TeX increases, you will probably want to edit this, but
% it should be fine as is for beginners.

% almost certainly you want these
\usepackage{amssymb}
\usepackage{amsmath}
\usepackage{amsthm}
\usepackage{amsfonts}

% used for TeXing text within eps files
%\usepackage{psfrag}
% need this for including graphics (\includegraphics)
%\usepackage{graphicx}
% for neatly defining theorems and propositions
%\usepackage{amsthm}
% making logically defined graphics
%%%\usepackage{xypic}

% there are many more packages, add them here as you need them

% define commands here

\newtheorem*{thm}{Theorem}
\newtheorem{defn}{Definition}
\newtheorem{prop}{Proposition}
\newtheorem{lemma}{Lemma}
\newtheorem{cor}{Corollary}

\theoremstyle{definition}
\newtheorem{exa}{Example}

% Some sets
\newcommand{\Nats}{\mathbb{N}}
\newcommand{\Ints}{\mathbb{Z}}
\newcommand{\Reals}{\mathbb{R}}
\newcommand{\Complex}{\mathbb{C}}
\newcommand{\Rats}{\mathbb{Q}}
\newcommand{\Gal}{\operatorname{Gal}}
\newcommand{\Cl}{\operatorname{Cl}}
\begin{document}
The following theorem gives a way of finding a primitive root for $p^k$, for an odd prime $p$ and $k\geq 1$, given a primitive root of $p$. Recall that every prime has a primitive root.

\begin{thm}Suppose that $p$ is an odd prime. Then $p^k$ also has a primitive root, for all $k\geq 1$. Moreover:
\begin{enumerate}
\item If $g$ is a primitive root of $p$ and $g^{p-1}\neq 1 \mod p^2$ then $g$ is a primitive root of $p^2$. Otherwise, if $g^{p-1}\equiv 1 \mod p^2$ then $g+p$ is a primitive root of $p^2$.

\item If $k\geq 2$ and $h$ is a primitive root of $p^k$ then $h$ is a primitive root of $p^{k+1}$.
\end{enumerate}
\end{thm}
%%%%%
%%%%%
\end{document}
