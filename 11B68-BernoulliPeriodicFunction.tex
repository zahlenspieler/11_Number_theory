\documentclass[12pt]{article}
\usepackage{pmmeta}
\pmcanonicalname{BernoulliPeriodicFunction}
\pmcreated{2013-03-22 11:45:54}
\pmmodified{2013-03-22 11:45:54}
\pmowner{KimJ}{5}
\pmmodifier{KimJ}{5}
\pmtitle{Bernoulli periodic function}
\pmrecord{10}{30218}
\pmprivacy{1}
\pmauthor{KimJ}{5}
\pmtype{Definition}
\pmcomment{trigger rebuild}
\pmclassification{msc}{11B68}
%\pmkeywords{number theory}

\endmetadata

\usepackage{amssymb}
\usepackage{amsmath}
\usepackage{amsfonts}
\usepackage{graphicx}
%%%%\usepackage{xypic}
\begin{document}
Let $b_r $ be the $r\text{th}$ Bernoulli polynomial. Then the $r\mbox{th}$ Bernoulli periodic function $B_r(x)$ is defined as the periodic function of period 1 which coincides with $b_r$ on $[0,1]$.
%%%%%
%%%%%
%%%%%
%%%%%
\end{document}
