\documentclass[12pt]{article}
\usepackage{pmmeta}
\pmcanonicalname{PrimeIdealDecompositionInCyclotomicExtensionsOfmathbbQ}
\pmcreated{2013-03-22 13:53:49}
\pmmodified{2013-03-22 13:53:49}
\pmowner{alozano}{2414}
\pmmodifier{alozano}{2414}
\pmtitle{prime ideal decomposition in cyclotomic extensions of $\mathbb{Q}$}
\pmrecord{5}{34644}
\pmprivacy{1}
\pmauthor{alozano}{2414}
\pmtype{Theorem}
\pmcomment{trigger rebuild}
\pmclassification{msc}{11R18}
%\pmkeywords{cyclotomic}
%\pmkeywords{prime ideal}
%\pmkeywords{decomposition}
\pmrelated{PrimeIdealDecompositionInQuadraticExtensionsOfMathbbQ}
\pmrelated{CalculatingTheSplittingOfPrimes}
\pmrelated{KroneckerWeberTheorem}
\pmrelated{ExamplesOfPrimeIdealDecompositionInNumberFields}
\pmrelated{SplittingAndRamificationInNumberFieldsAndGaloisExtensions}

% this is the default PlanetMath preamble.  as your knowledge
% of TeX increases, you will probably want to edit this, but
% it should be fine as is for beginners.

% almost certainly you want these
\usepackage{amssymb}
\usepackage{amsmath}
\usepackage{amsthm}
\usepackage{amsfonts}

% used for TeXing text within eps files
%\usepackage{psfrag}
% need this for including graphics (\includegraphics)
%\usepackage{graphicx}
% for neatly defining theorems and propositions
%\usepackage{amsthm}
% making logically defined graphics
%%\usepackage{xypic}

% there are many more packages, add them here as you need them

% define commands here

\newtheorem{thm}{Theorem}
\newtheorem{defn}{Definition}
\newtheorem{prop}{Proposition}
\newtheorem{lemma}{Lemma}
\newtheorem{cor}{Corollary}

% Some sets
\newcommand{\Nats}{\mathbb{N}}
\newcommand{\Ints}{\mathbb{Z}}
\newcommand{\Reals}{\mathbb{R}}
\newcommand{\Complex}{\mathbb{C}}
\newcommand{\Rats}{\mathbb{Q}}
\begin{document}
Let $q\in\Ints$ be a prime greater than $2$, let $\zeta_q=e^{2\pi
i/q}$ and write $L=\Rats(\zeta_q)$ for the cyclotomic extension.
The ring of integers of $L$ is $\mathcal{O}_L=\Ints[\zeta_q]$. The
discriminant of $L/\Rats$ is:
$$D_{L/\Rats}=\pm q^{q-2}$$
and it is $+$ exactly when $q-1\equiv 0,1\ \operatorname{mod}\ 4$.

\begin{prop}
$\sqrt{\pm q}\in \Rats(\zeta_q)$, with $+$ exactly when $q-1\equiv
0,1\ \operatorname{mod}\ 4$.
\end{prop}
\begin{proof}
It can be proved that:
$$D_{L/\Rats}=\pm q^{q-2}=\prod_{1\leq i < j \leq
q-1}(\zeta_q^i-\zeta_q^j)^2$$ Taking square roots we obtain
$$q^{\frac{q-3}{2}}\sqrt{\pm q}=\prod_{1\leq i < j \leq
q-1}(\zeta_q^i-\zeta_q^j)\in \Rats(\zeta_q)$$ Hence the result
holds (and the sign depends on whether $q-1\equiv 0,1\
\operatorname{mod}\ 4$).
\end{proof}

Let $K=\Rats(\sqrt{\pm q})$ with the corresponding sign. Thus, by
the proposition we have a tower of fields:
$\xymatrix{
 & L=\Rats(\zeta_q) \ar@{-}[d]\\
 & K \ar@{-}[d]\\
 & \Rats}$

For a prime ideal $p\Ints$ the decomposition in the quadratic
extension $K/\Rats$ is well-known (see \PMlinkexternal{this entry}{http://planetmath.org/encyclopedia/PrimeIdealDecompositionInQuadraticExtensionsOfMathbbQ.html}). The next theorem
characterizes the decomposition in the extension $L/\Rats$:

\begin{thm}
Let $p\in\Ints$ be a prime.
\begin{enumerate}
\item If $p=q$, $q\mathcal{O}_L=\left(1-\zeta_q\right)^{q-1}$. In
other words, the prime $q$ is totally ramified in $L$.

\item If $p\neq q$ then $p\Ints$ splits into $(q-1)/f$ distinct
primes in $\mathcal{O}_L$, where $f$ is the order of $p\
\operatorname{mod}\ q$ (i.e. $p^f\equiv 1\ \operatorname{mod}\ q$,
and for all $1< n<f, p^n\neq 1\ \operatorname{mod}\ q$).
\end{enumerate}
\end{thm}

\begin{thebibliography}{9}
\bibitem{marcus} Daniel A.Marcus, {\em Number Fields}. Springer, New York.
\end{thebibliography}
%%%%%
%%%%%
\end{document}
