\documentclass[12pt]{article}
\usepackage{pmmeta}
\pmcanonicalname{PythagoreanPrime}
\pmcreated{2013-03-22 16:53:53}
\pmmodified{2013-03-22 16:53:53}
\pmowner{PrimeFan}{13766}
\pmmodifier{PrimeFan}{13766}
\pmtitle{Pythagorean prime}
\pmrecord{4}{39155}
\pmprivacy{1}
\pmauthor{PrimeFan}{13766}
\pmtype{Definition}
\pmcomment{trigger rebuild}
\pmclassification{msc}{11A41}

% this is the default PlanetMath preamble.  as your knowledge
% of TeX increases, you will probably want to edit this, but
% it should be fine as is for beginners.

% almost certainly you want these
\usepackage{amssymb}
\usepackage{amsmath}
\usepackage{amsfonts}

% used for TeXing text within eps files
%\usepackage{psfrag}
% need this for including graphics (\includegraphics)
%\usepackage{graphicx}
% for neatly defining theorems and propositions
%\usepackage{amsthm}
% making logically defined graphics
%%%\usepackage{xypic}

% there are many more packages, add them here as you need them

% define commands here

\begin{document}
A {\em Pythagorean prime} $p$ is a prime number of the form $4n + 1$. The first few are 5, 13, 17, 29, 37, 41, 53, 61, 73, 89, 97, etc., listed in A002144 of Sloane's OEIS. Because of its form, a Pythagorean prime is the sum of two squares, e.g., 29 = 25 + 4. In fact, with the exception of 2, these are the only primes that can be represented as the sum of two squares (thus, in Waring's problem, all other primes require three or four squares).

Though Pythagorean primes are primes on the line of real integers, they are not Gaussian primes in the complex plane. Expressing a Pythagorean prime as $a^2 + b^2$ (it doesn't matter whether $a < b$ or viceversa) leads to the complex factorization by simple plugging in of the values thus: $p = (a + bi)(a - bi)$, where $i$ is the imaginary unit.
%%%%%
%%%%%
\end{document}
