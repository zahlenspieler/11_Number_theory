\documentclass[12pt]{article}
\usepackage{pmmeta}
\pmcanonicalname{IdealNorm}
\pmcreated{2013-03-22 15:43:23}
\pmmodified{2013-03-22 15:43:23}
\pmowner{pahio}{2872}
\pmmodifier{pahio}{2872}
\pmtitle{ideal norm}
\pmrecord{17}{37672}
\pmprivacy{1}
\pmauthor{pahio}{2872}
\pmtype{Definition}
\pmcomment{trigger rebuild}
\pmclassification{msc}{11R04}
\pmsynonym{norm of an ideal}{IdealNorm}
\pmsynonym{norm of ideal}{IdealNorm}
%\pmkeywords{residue class}
%\pmkeywords{congruence}
\pmrelated{NormAndTraceOfAlgebraicNumber}
\pmrelated{Congruences}
\pmrelated{MultiplicativeCongruence}
\pmrelated{BasisOfIdealInAlgebraicNumberField}
\pmrelated{IdealClassGroupIsFinite}
\pmrelated{RationalIntegersInIdeals}
\pmdefines{congruence modulo an ideal}
\pmdefines{congruent modulo the ideal}
\pmdefines{residue classes modulo the ideal}
\pmdefines{absolute norm of ideal}

\endmetadata

% this is the default PlanetMath preamble.  as your knowledge
% of TeX increases, you will probably want to edit this, but
% it should be fine as is for beginners.

% almost certainly you want these
\usepackage{amssymb}
\usepackage{amsmath}
\usepackage{amsfonts}

% used for TeXing text within eps files
%\usepackage{psfrag}
% need this for including graphics (\includegraphics)
%\usepackage{graphicx}
% for neatly defining theorems and propositions
 \usepackage{amsthm}
% making logically defined graphics
%%%\usepackage{xypic}

% there are many more packages, add them here as you need them

% define commands here

\theoremstyle{definition}
\newtheorem*{thmplain}{Theorem}

\def\N{\mathrm{N}}
\begin{document}
Let $\alpha$ and $\beta$ be algebraic integers in an algebraic number field $K$ and $\mathfrak{m}$ a non-zero ideal in the ring of integers of $K$.\, We say that $\alpha$ and $\beta$ are {\em congruent modulo the ideal} $\mathfrak{m}$ in the case that\, $\alpha\!-\!\beta \in \mathfrak{m}$.\, This is denoted by
    $$\alpha \equiv \beta \pmod{\mathfrak{m}}.$$
This congruence relation \PMlinkescapetext{divides} the ring of integers of $K$ into equivalence classes, which are called the {\em residue classes modulo the ideal} $\mathfrak{m}$.

\textbf{Definition.}\, Let $K$ be an algebraic number field and\, $\mathfrak{a}$\, a non-zero ideal in $K$.\, The {\em absolute norm of ideal} $\mathfrak{a}$ means the number of all residue classes modulo $\mathfrak{a}$.

\textbf{Remark.}\, The \PMlinkescapetext{norm} of any ideal $\mathfrak{a}$ of $K$ is finite --- it has the expression
  $$\N(\mathfrak{a}) = \sqrt{\frac{\Delta(\mathfrak{a})}{d}}$$
where $\Delta(\mathfrak{a})$ is the discriminant of the ideal and $d$ the fundamental number of the field.

\PMlinkescapetext{\textbf{Some properties}}
\begin{itemize}
\item $\N(\mathfrak{ab}) 
  = \N(\mathfrak{a})\!\cdot\!\N(\mathfrak{b})$ 
\item $\N(\mathfrak{a}) = 1 \quad\Leftrightarrow\quad \mathfrak{a} = (1)$
\item $\N((\alpha)) = |\N(\alpha)|$
\item $\N(\mathfrak{a}) \in \mathfrak{a}$
\item If $\N(\mathfrak{p})$ is a rational prime, then $\mathfrak{p}$ is a prime ideal.
\end{itemize}
%%%%%
%%%%%
\end{document}
