\documentclass[12pt]{article}
\usepackage{pmmeta}
\pmcanonicalname{EulersLuckyNumber}
\pmcreated{2013-03-22 16:55:33}
\pmmodified{2013-03-22 16:55:33}
\pmowner{PrimeFan}{13766}
\pmmodifier{PrimeFan}{13766}
\pmtitle{Euler's lucky number}
\pmrecord{4}{39189}
\pmprivacy{1}
\pmauthor{PrimeFan}{13766}
\pmtype{Definition}
\pmcomment{trigger rebuild}
\pmclassification{msc}{11A41}
\pmsynonym{lucky number of Euler}{EulersLuckyNumber}
\pmsynonym{Eulerian lucky number}{EulersLuckyNumber}

\endmetadata

% this is the default PlanetMath preamble.  as your knowledge
% of TeX increases, you will probably want to edit this, but
% it should be fine as is for beginners.

% almost certainly you want these
\usepackage{amssymb}
\usepackage{amsmath}
\usepackage{amsfonts}

% used for TeXing text within eps files
%\usepackage{psfrag}
% need this for including graphics (\includegraphics)
%\usepackage{graphicx}
% for neatly defining theorems and propositions
%\usepackage{amsthm}
% making logically defined graphics
%%%\usepackage{xypic}

% there are many more packages, add them here as you need them

% define commands here

\begin{document}
\PMlinkescapeword{lucky number}

A prime number $p$ is one of {\em Euler's lucky numbers} if $n^2 - n + p$ for each $0 < n < p$ is also a prime. Put another way, a lucky number of Euler's plus the $n$th oblong number produces a list of primes $p$-long. There are only six of them: 2, 3, 5, 11, 17 and 41, these are listed in A014556 of Sloane's OEIS.

41 is perhaps the most famous of these. We can verify that 2 + 41 is 43, a prime, that 47 is also prime, so are 53, 61, 71, 83, 97, and so on to 1601, giving a list of 41 primes. Predictably, 1681 is divisible by 41, being its square. For $n > p$ the formula does not consistently give only composites or only primes.
%%%%%
%%%%%
\end{document}
