\documentclass[12pt]{article}
\usepackage{pmmeta}
\pmcanonicalname{AntonsCongruence}
\pmcreated{2013-03-22 13:22:49}
\pmmodified{2013-03-22 13:22:49}
\pmowner{Thomas Heye}{1234}
\pmmodifier{Thomas Heye}{1234}
\pmtitle{Anton's congruence}
\pmrecord{10}{33914}
\pmprivacy{1}
\pmauthor{Thomas Heye}{1234}
\pmtype{Theorem}
\pmcomment{trigger rebuild}
\pmclassification{msc}{11A07}
%\pmkeywords{relative prime}
\pmrelated{Factorial}

\endmetadata

% this is the default PlanetMath preamble.  as your knowledge
% of TeX increases, you will probably want to edit this, but
% it should be fine as is for beginners.

% almost certainly you want these
\usepackage{amssymb}
\usepackage{amsmath}
\usepackage{amsfonts}

% used for TeXing text within eps files
%\usepackage{psfrag}
% need this for including graphics (\includegraphics)
%\usepackage{graphicx}
% for neatly defining theorems and propositions
\usepackage{amsthm}
% making logically defined graphics
%%%\usepackage{xypic}

% there are many more packages, add them here as you need them

% define commands here
\newcommand{\pfac}[1]{\left(#1\underline{!}\right)_p}
\begin{document}
For every $n \in \mathbb{N}$ $\pfac{n}$ stands for the product of numbers
between $1$ and $n$ which are not divisible by a given prime $p$. And we set
$\pfac{0} =1$.

The corollary below generalizes a result first found by Anton, Stickelberger,
and Hensel:

Let $N_0$ be the least non-negative residue of $n \pmod{p^s}$ where $p$ is a
prime number and $n \in \mathbb{N}$. Then
\begin{displaymath}
\pfac{n} \equiv \left(\pm 1\right)^{\left\lfloor n/p^s
\right\rfloor}\cdot \pfac{N_0} \pmod{p^s}.
\end{displaymath}

\begin{proof}
We write each $r$ in the product below as $ip^s +j$ to get
\begin{eqnarray*}
\pfac{n} &=& \prod\limits_{\substack{1 \le r \le n\\ p^s \not\div r}} r \\ &=&\left( 
\prod\limits_{\substack{0 \le i \le \left\lfloor n/p^s\right\rfloor -1 \\ 1 \le j < p^s \\ p^s
\not\div j}} ip^s +j\right)\left( \prod\limits_{\substack{i=\left\lfloor
n/p^s\right\rfloor \\ 1\le j \le N_0 \\ p^s \not\div j}} ip^s +j\right) \\
 &\equiv& \prod\limits_{i=0}^{\left\lfloor n/p^s \right\rfloor -1}
\prod\limits_{\substack{1 \le j < p^s \\ p^s \not\div j }} j\cdot
	\prod\limits_{\substack{j=1 \\ p^s \not\div j}}^{N_0} j) \\
 &\equiv&
\pfac{p^s}^{\left\lfloor n/p^s\right\rfloor}\cdot \pfac{N_0} \pmod{p^s}.
\end{eqnarray*}
From Wilson's theorem for prime powers it follows that
\begin{displaymath}
\pfac{n} \equiv
\begin{cases} 
\pfac{N_0} \text{if} & p=2, s \ge 3 \\
(-1)^{\left\lfloor n/p^s\right\rfloor}\cdot\pfac{N_0} & \text{otherwise.}
\end{cases} \pmod{p^s}.
\end{displaymath}
\end{proof}
%%%%%
%%%%%
\end{document}
