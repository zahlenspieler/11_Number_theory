\documentclass[12pt]{article}
\usepackage{pmmeta}
\pmcanonicalname{WilsonPrime}
\pmcreated{2013-03-22 16:40:39}
\pmmodified{2013-03-22 16:40:39}
\pmowner{PrimeFan}{13766}
\pmmodifier{PrimeFan}{13766}
\pmtitle{Wilson prime}
\pmrecord{5}{38885}
\pmprivacy{1}
\pmauthor{PrimeFan}{13766}
\pmtype{Definition}
\pmcomment{trigger rebuild}
\pmclassification{msc}{11A41}

% this is the default PlanetMath preamble.  as your knowledge
% of TeX increases, you will probably want to edit this, but
% it should be fine as is for beginners.

% almost certainly you want these
\usepackage{amssymb}
\usepackage{amsmath}
\usepackage{amsfonts}

% used for TeXing text within eps files
%\usepackage{psfrag}
% need this for including graphics (\includegraphics)
%\usepackage{graphicx}
% for neatly defining theorems and propositions
%\usepackage{amsthm}
% making logically defined graphics
%%%\usepackage{xypic}

% there are many more packages, add them here as you need them

% define commands here

\begin{document}
If for a given prime $p$ it is true that $(p - 1)! \equiv -1 \mod p^2$, then $p$ is called a {\em Wilson prime}. Like all other primes, Wilson primes satisfy the primality condition of Wilson's theorem, but they also divide the Wilson quotient. Only three are known as of 2007, namely: 5, 13, 563 (A007540 in Sloane's OEIS, which normally requires at least four terms before accepting sequences into the table). It is not known if there are infinitely many Wilson primes exist; the fourth Wilson prime would have to be greater than $10^9$.
%%%%%
%%%%%
\end{document}
