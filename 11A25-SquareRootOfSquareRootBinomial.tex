\documentclass[12pt]{article}
\usepackage{pmmeta}
\pmcanonicalname{SquareRootOfSquareRootBinomial}
\pmcreated{2013-03-22 15:21:21}
\pmmodified{2013-03-22 15:21:21}
\pmowner{pahio}{2872}
\pmmodifier{pahio}{2872}
\pmtitle{square root of square root binomial}
\pmrecord{10}{37178}
\pmprivacy{1}
\pmauthor{pahio}{2872}
\pmtype{Topic}
\pmcomment{trigger rebuild}
\pmclassification{msc}{11A25}
%\pmkeywords{nested square roots}
\pmrelated{TakingSquareRootAlgebraically}
\pmrelated{SquareRootsOfRationals}
\pmdefines{square root binomial}
\pmdefines{square root polynomial}

% this is the default PlanetMath preamble.  as your knowledge
% of TeX increases, you will probably want to edit this, but
% it should be fine as is for beginners.

% almost certainly you want these
\usepackage{amssymb}
\usepackage{amsmath}
\usepackage{amsfonts}

% used for TeXing text within eps files
%\usepackage{psfrag}
% need this for including graphics (\includegraphics)
%\usepackage{graphicx}
% for neatly defining theorems and propositions
 \usepackage{amsthm}
% making logically defined graphics
%%%\usepackage{xypic}

% there are many more packages, add them here as you need them

% define commands here

\theoremstyle{definition}
\newtheorem*{thmplain}{Theorem}
\begin{document}
Some people call the expressions of the form\, $a\!+\!b\sqrt{c}$\, the \PMlinkescapetext{{\em square root binomials}}, especially when $c$ is an square-free integer greater than 1 (and $a$ and $b$  rational numbers).\, On the high school \PMlinkescapetext{level one may learn to perform arithmetic operations between such binomials} (see e.g. division), or also polynomials containing several square root \PMlinkescapetext{terms}.\, Taking the square root of a square root binomial is more difficult and usually results nested square roots.\, However, there are some exceptions if the numbers are appropriate.\, We have the \PMlinkescapetext{formulae}
$$\sqrt{a+\sqrt{b}} = \sqrt{\frac{a+\sqrt{a^2-b}}{2}}+\sqrt{\frac{a-\sqrt{a^2-b}}{2}}$$
and
$$\sqrt{a-\sqrt{b}} = \sqrt{\frac{a+\sqrt{a^2-b}}{2}}-\sqrt{\frac{a-\sqrt{a^2-b}}{2}}.$$
If\, $a^2\!-\!b$\, happens to be square of a rational number, then the \PMlinkescapetext{formulae allow to convert the square roots on the left side} into expressions without nested square roots.

For example, because\, $6^2\!-\!20 = 16 = 4^2$,\, we obtain
$$\sqrt{6\!+\!2\sqrt{5}} = \sqrt{6\!+\!\sqrt{20}} = 
\sqrt{\frac{6\!+\!4}{2}}+\sqrt{\frac{6\!-\!4}{2}} = 1\!+\!\sqrt{5},$$
and because\, $4^2\!-\!7 = 9 = 3^2$,\, we get
$$\sqrt{4\!-\!\sqrt{7}} = \sqrt{\frac{4\!+\!3}{2}}-\sqrt{\frac{4\!-\!3}{2}}
= \frac{\sqrt{7}\!-\!1}{\sqrt{2}} = \frac{\sqrt{14}\!-\!\sqrt{2}}{2}.$$

\begin{thebibliography}{9}
\bibitem{VA}{\sc K. V\"ais\"al\"a:} {\em Algebran oppi- ja esimerkkikirja} I.  \, -- Werner S\"oderstr\"om osakeyhti\"o, Porvoo \& Helsinki (1952).
\end{thebibliography}
%%%%%
%%%%%
\end{document}
