\documentclass[12pt]{article}
\usepackage{pmmeta}
\pmcanonicalname{EssentialComponent}
\pmcreated{2013-03-22 13:19:42}
\pmmodified{2013-03-22 13:19:42}
\pmowner{bbukh}{348}
\pmmodifier{bbukh}{348}
\pmtitle{essential component}
\pmrecord{7}{33840}
\pmprivacy{1}
\pmauthor{bbukh}{348}
\pmtype{Definition}
\pmcomment{trigger rebuild}
\pmclassification{msc}{11B05}
\pmclassification{msc}{11B13}
\pmrelated{SchnirlemannDensity}
\pmrelated{Basis2}

\endmetadata

\usepackage{amssymb}
\usepackage{amsmath}
\usepackage{amsfonts}

%%%\usepackage{xypic}
\makeatletter
\@ifundefined{bibname}{}{\renewcommand{\bibname}{References}}
\makeatother
\begin{document}
If $A$ is a set of nonnegative integers such that 
\begin{equation}
\sigma(A+B)>\sigma B
\end{equation}
for every set $B$ with Schnirelmann density $0<\sigma B<1$, then $A$ is an \emph{essential component}.

Erd\H{o}s proved that every \PMlinkid{basis}{3831} is an essential component. In fact he proved that
\begin{equation*}
\sigma(A+B)\geq \sigma B+\frac{1}{2h}(1-\sigma B)\sigma B,
\end{equation*}
where $h$ denotes the \PMlinkid{order}{3831} of $A$.

Pl\"unnecke improved that to
\begin{equation*}
\sigma(A+B)\geq \sigma B^{1-1/h}.
\end{equation*}

There are non-basic essential components. Linnik constructed non-basic essential component for which $A(n)=O(n^\epsilon)$ for every $\epsilon>0$.

\begin{thebibliography}{1}

\bibitem{cite:halberstam_sequences}
Heini Halberstam and Klaus~Friedrich Roth.
\newblock {\em Sequences}.
\newblock Springer-Verlag, second edition, 1983.
\newblock \PMlinkexternal{Zbl 0498.10001}{http://www.emis.de/cgi-bin/zmen/ZMATH/en/quick.html?type=html&an=0498.10001}.

\end{thebibliography}

%@BOOK{cite:halberstam_sequences,
% author    = {Heini Halberstam and Klaus Friedrich Roth},
% title     = "Sequences",
% publisher = {Springer-Verlag},
% year      = {1983},
% edition   = {Second}
%}
%%%%%
%%%%%
\end{document}
