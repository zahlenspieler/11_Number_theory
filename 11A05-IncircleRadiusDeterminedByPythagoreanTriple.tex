\documentclass[12pt]{article}
\usepackage{pmmeta}
\pmcanonicalname{IncircleRadiusDeterminedByPythagoreanTriple}
\pmcreated{2013-03-22 17:45:46}
\pmmodified{2013-03-22 17:45:46}
\pmowner{pahio}{2872}
\pmmodifier{pahio}{2872}
\pmtitle{incircle radius determined by Pythagorean triple}
\pmrecord{14}{40216}
\pmprivacy{1}
\pmauthor{pahio}{2872}
\pmtype{Feature}
\pmcomment{trigger rebuild}
\pmclassification{msc}{11A05}
\pmsynonym{incircle radius of right triangle}{IncircleRadiusDeterminedByPythagoreanTriple}
\pmrelated{Triangle}
\pmrelated{PythagoreanTriple}
\pmrelated{DifferenceOfSquares}
\pmrelated{FirstPrimitivePythagoreanTriplets}
\pmrelated{X4Y4z2HasNoSolutionsInPositiveIntegers}

\endmetadata

% this is the default PlanetMath preamble.  as your knowledge
% of TeX increases, you will probably want to edit this, but
% it should be fine as is for beginners.

% almost certainly you want these
\usepackage{amssymb}
\usepackage{amsmath}
\usepackage{amsfonts}

% used for TeXing text within eps files
%\usepackage{psfrag}
% need this for including graphics (\includegraphics)
%\usepackage{graphicx}
% for neatly defining theorems and propositions
 \usepackage{amsthm}
% making logically defined graphics
%%%\usepackage{xypic}

% there are many more packages, add them here as you need them

% define commands here

\theoremstyle{definition}
\newtheorem*{thmplain}{Theorem}

\begin{document}
If the sides of a right triangle are integers, then so is the radius of the incircle of this triangle.\\
For example, the incircle radius of the Egyptian triangle is 1.\\

{\em Proof.}\, The sides of such a right triangle may be expressed by the integer parametres $m,\,n$ with\,
$m > n > 0$\, as 
\begin{align}
       a \;=\; 2mn, \quad b \;=\; m^2\!-\!n^2, \quad c \;=\; m^2\!+\!n^2;
\end{align}
the \PMlinkname{radius of the incircle}{Incircle} is
\begin{align}
       r \;=\; \frac{2A}{a\!+\!b\!+\!c},
\end{align}
where $A$ is the area of the triangle.\, Using (1) and (2) we obtain
$$r \;=\; \frac{2\cdot2mn\cdot(m^2\!-\!n^2)/2}{2mn\!+\!(m^2\!-\!n^2)\!+\!(m^2\!+\!n^2)} \;=\; 
      \frac{2mn(m\!+\!n)(m\!-\!n)}{2m(m\!+\!n)} \;=\; (m\!-\!n)n,$$
which is a positive integer.\\


\textbf{Remark.}\, The corresponding radius of the circumcircle need not to be integer, since by Thales' theorem, the radius is always half of the hypotenuse which may be odd (e.g. 5).



%%%%%
%%%%%
\end{document}
