\documentclass[12pt]{article}
\usepackage{pmmeta}
\pmcanonicalname{FareyPair}
\pmcreated{2013-03-22 14:54:42}
\pmmodified{2013-03-22 14:54:42}
\pmowner{drini}{3}
\pmmodifier{drini}{3}
\pmtitle{Farey pair}
\pmrecord{6}{36597}
\pmprivacy{1}
\pmauthor{drini}{3}
\pmtype{Definition}
\pmcomment{trigger rebuild}
\pmclassification{msc}{11A55}
\pmrelated{ContinuedFraction}
\pmdefines{mediant}
\pmdefines{Farey interval}

\endmetadata

\usepackage{graphicx}
%%%\usepackage{xypic} 
\usepackage{bbm}
\newcommand{\Z}{\mathbbmss{Z}}
\newcommand{\C}{\mathbbmss{C}}
\newcommand{\R}{\mathbbmss{R}}
\newcommand{\Q}{\mathbbmss{Q}}
\newcommand{\mathbb}[1]{\mathbbmss{#1}}
\newcommand{\figura}[1]{\begin{center}\includegraphics{#1}\end{center}}
\newcommand{\figuraex}[2]{\begin{center}\includegraphics[#2]{#1}\end{center}}
\newtheorem{dfn}{Definition}
\begin{document}
Two nonnegative reduced fractions $a/b$ and $c/d$ make a \emph{Farey pair} (with $a/b < c/d$) whenever $bc-ad=1$, in other words, they are a Farey pair if their difference is $1/bd$. The interval $[a/b, c/d]$ is known as a \emph{Farey interval.}

Given a Farey pair $a/b,c/d$, their \emph{mediant} is $(a+c)/(b+d)$. The mediant has the following property:

{\sl If $[a,b,c/d]$ is a Farey interval, then the two subintervals obtained when inserting the mediant are also Farey pairs. Besides, between all fractions that are strictly between $a/b,c/d$, the mediant is the one having the smallest denominator.}

{\bf Example.}\\
Notice that $3/8$ and $5/11$ form a Farey pair, since 
$8\cdot 5 - 3\cdot 13 =40-391$. The mediant here is $8/21$.

Then $3/8$ and $8/21$ form a Farey pair: $8\cdot 8 - 3\cdot 21 = 64-63=1$.
No fraction between $3/8$ and $5/11$ other than $8/21$ has a denominator smaller or equal than $21$.
%%%%%
%%%%%
\end{document}
