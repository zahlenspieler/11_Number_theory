\documentclass[12pt]{article}
\usepackage{pmmeta}
\pmcanonicalname{RayClassField}
\pmcreated{2013-03-22 13:54:01}
\pmmodified{2013-03-22 13:54:01}
\pmowner{alozano}{2414}
\pmmodifier{alozano}{2414}
\pmtitle{ray class field}
\pmrecord{5}{34648}
\pmprivacy{1}
\pmauthor{alozano}{2414}
\pmtype{Definition}
\pmcomment{trigger rebuild}
\pmclassification{msc}{11R37}
\pmsynonym{conductor}{RayClassField}
%\pmkeywords{abelian extension}
%\pmkeywords{ray class}
%\pmkeywords{class field}
\pmrelated{ArtinMap}
\pmrelated{ExistenceOfHilbertClassField}
\pmrelated{NumberField}
\pmrelated{AnExactSequenceForRayClassGroups}
\pmdefines{conductor of an extension}

% this is the default PlanetMath preamble.  as your knowledge
% of TeX increases, you will probably want to edit this, but
% it should be fine as is for beginners.

% almost certainly you want these
\usepackage{amssymb}
\usepackage{amsmath}
\usepackage{amsthm}
\usepackage{amsfonts}

% used for TeXing text within eps files
%\usepackage{psfrag}
% need this for including graphics (\includegraphics)
%\usepackage{graphicx}
% for neatly defining theorems and propositions
%\usepackage{amsthm}
% making logically defined graphics
%%%\usepackage{xypic}

% there are many more packages, add them here as you need them

% define commands here

\newtheorem{thm}{Theorem}
\newtheorem{defn}{Definition}
\newtheorem{prop}{Proposition}
\newtheorem{lemma}{Lemma}
\newtheorem{cor}{Corollary}

% Some sets
\newcommand{\Nats}{\mathbb{N}}
\newcommand{\Ints}{\mathbb{Z}}
\newcommand{\Reals}{\mathbb{R}}
\newcommand{\Complex}{\mathbb{C}}
\newcommand{\Rats}{\mathbb{Q}}
\begin{document}
\begin{prop}
Let $L/K$ be a finite abelian extension of number fields, and let
$\mathcal{O}_K$ be the ring of integers of $K$. There exists an
integral ideal $\mathcal{C}\subset \mathcal{O}_K$, divisible by
precisely the prime ideals of $K$ that ramify in $L$, such that
$$\left((\alpha),L/K\right)=1,\quad \forall \alpha \in K^{\ast},\
\alpha\equiv1\ \operatorname{mod}\ \mathcal{C}
$$ where
$\left((\alpha),L/K\right)$ is the Artin map.
\end{prop}

\begin{defn}
The \emph{conductor} of a finite abelian extension $L/K$ is the
largest ideal $\mathcal{C}_{L/K}\subset \mathcal{O}_K$ satisfying
the above properties.
\end{defn}

Note that there is a ``largest ideal'' with this condition because
if proposition 1 is true for $\mathcal{C}_1,\mathcal{C}_2$ then it
is also true for $\mathcal{C}_1+\mathcal{C}_2$.

\begin{defn}
Let $\mathcal{I}$ be an integral ideal of $K$. A \emph{ray class
field} of $K$ (modulo $\mathcal{I}$) is a finite abelian extension
$K_{\mathcal{I}}/K$ with the property that for any other finite
abelian extension $L/K$ with conductor $\mathcal{C}_{L/K}$,
$$\mathcal{C}_{L/K} \mid \mathcal{I}\Rightarrow
L\subset K_{\mathcal{I}}$$
\end{defn}

Note: It can be proved that there is a unique ray class field with
a given conductor. In words, the ray class field is the biggest
abelian extension of $K$ with a given conductor (although the
conductor of $K_{\mathcal{I}}$ does not necessarily equal
$\mathcal{I}$ !, see example $2$).

{\bf Remark}: Let $\mathfrak{p}$ be a prime of $K$ unramified in $L$, and let $\mathfrak{P}$ be a prime above $\mathfrak{p}$. Then $(\mathfrak{p},L/K)=1$ if and only if the extension of residue fields is of degree 1
$$[\mathcal{O}_L/\mathfrak{P}\colon \mathcal{O}_K/\mathfrak{p}]=1$$
if and only if $\mathfrak{p}$ splits completely in $L$. Thus we obtain a characterization of the ray class field of conductor $\mathcal{C}$ as the abelian extension of $K$ such that a prime of $K$ splits completely if and only if it is of the form
$$(\alpha),\quad \alpha \in K^{\ast},\
\alpha\equiv1\ \operatorname{mod}\ \mathcal{C}$$

{\bf Examples}:
\begin{enumerate}
\item The ray class field of $\Rats$ of conductor $N\Ints$ is the
$N^{th}$-cyclotomic extension of $\Rats$. More concretely, let
$\zeta_N$ be a primitive $N^{th}$ root of unity. Then
$$\Rats_{N\Ints}=\Rats(\zeta_N)$$

\item $$\Rats(i)_{(2)}=\Rats(i)$$ so the conductor of
$\Rats(i)_{(2)}/\Rats$ is $(1)$.

\item $K_{(1)}$, the ray class field of conductor $(1)$, is the
maximal abelian extension of $K$ which is unramified everywhere.
It is, in fact, the Hilbert class field of $K$.
\end{enumerate}

\begin{thebibliography}{9}
\bibitem{tate} Artin/Tate, {\em Class Field Theory}. W.A.Benjamin Inc., New York.
\end{thebibliography}
%%%%%
%%%%%
\end{document}
