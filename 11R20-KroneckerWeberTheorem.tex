\documentclass[12pt]{article}
\usepackage{pmmeta}
\pmcanonicalname{KroneckerWeberTheorem}
\pmcreated{2013-03-22 13:52:41}
\pmmodified{2013-03-22 13:52:41}
\pmowner{alozano}{2414}
\pmmodifier{alozano}{2414}
\pmtitle{Kronecker-Weber theorem}
\pmrecord{6}{34620}
\pmprivacy{1}
\pmauthor{alozano}{2414}
\pmtype{Theorem}
\pmcomment{trigger rebuild}
\pmclassification{msc}{11R20}
\pmclassification{msc}{11R37}
\pmclassification{msc}{11R18}
\pmrelated{ComplexMultiplication}
\pmrelated{AbelianExtension}
\pmrelated{PrimeIdealDecompositionInCyclotomicExtensionsOfMathbbQ}
\pmrelated{NumberField}
\pmrelated{CyclotomicExtension}
\pmrelated{ArithmeticOfEllipticCurves}
\pmdefines{abelian extensions of quadratic imaginary number fields}
\pmdefines{Weber function}

\endmetadata

% this is the default PlanetMath preamble.  as your knowledge
% of TeX increases, you will probably want to edit this, but
% it should be fine as is for beginners.

% almost certainly you want these
\usepackage{amssymb}
\usepackage{amsmath}
\usepackage{amsthm}
\usepackage{amsfonts}

% used for TeXing text within eps files
%\usepackage{psfrag}
% need this for including graphics (\includegraphics)
%\usepackage{graphicx}
% for neatly defining theorems and propositions
%\usepackage{amsthm}
% making logically defined graphics
%%%\usepackage{xypic}

% there are many more packages, add them here as you need them

% define commands here

\newtheorem{thm}{Theorem}
\newtheorem{defn}{Definition}
\newtheorem{prop}{Proposition}
\newtheorem{lemma}{Lemma}
\newtheorem{cor}{Corollary}

% Some sets
\newcommand{\Nats}{\mathbb{N}}
\newcommand{\Ints}{\mathbb{Z}}
\newcommand{\Reals}{\mathbb{R}}
\newcommand{\Complex}{\mathbb{C}}
\newcommand{\Rats}{\mathbb{Q}}
\begin{document}
The following theorem classifies the possible \PMlinkid{abelian extensions}{AbelianExtension}
of $\Rats$.

\begin{thm}[Kronecker-Weber Theorem]
Let $L/\Rats$ be a finite \PMlinkid{abelian extension}{AbelianExtension}, then $L$ is contained
in a cyclotomic extension, i.e. there is a root of unity $\zeta$
such that $L \subseteq \Rats(\zeta)$.
\end{thm}

In a similar fashion to this result, the theory of elliptic curves
with complex multiplication provides a classification of \PMlinkid{abelian
extensions}{AbelianExtension} of quadratic imaginary number fields:

\begin{thm}
Let $K$ be a quadratic imaginary number field with ring of
integers $\mathcal{O}_K$. Let $E$ be an elliptic curve with
complex multiplication by $\mathcal{O}_K$ and let $j(E)$ be the
$j$-invariant of $E$. Then:
\begin{enumerate}
\item $K(j(E))$ is the Hilbert class field of $K$.

\item If $j(E)\neq 0,1728$ then the maximal \PMlinkid{abelian extension}{AbelianExtension} of
$K$ is given by:
$$K^{ab}=K(j(E),h(E_{\operatorname{torsion}}))$$
where $h(E_{\operatorname{torsion}})$ is the set of
$x$-coordinates of all the torsion points of $E$.
\end{enumerate}
\end{thm}

Note: The map $h\colon E \to \Complex$ is called a \emph{Weber
function} for $E$. We can define a Weber function for the cases
$j(E)=0,1728$ so the theorem holds true for those two cases as
well. Assume $E\colon y^2=x^3+Ax+B$, then:
$$ h(P)=
\begin{cases}
x(P) ,\text{ if $j(E)\neq 0, 1728$};\\
x^2(P) ,\text{ if $j(E)=1728$};\\
x^3(P) ,\text{ if $j(E)=0$}.
\end{cases}
$$

\begin{thebibliography}{9}
\bibitem{lang} S. Lang, {\em Algebraic Number Theory}, Springer-Verlag, New York.
\bibitem{silverman} Joseph H. Silverman, {\em Advanced Topics in the Arithmetic of Elliptic Curves}. Springer-Verlag, New
York.
\end{thebibliography}
%%%%%
%%%%%
\end{document}
