\documentclass[12pt]{article}
\usepackage{pmmeta}
\pmcanonicalname{FactorialModuloPrimePowers}
\pmcreated{2013-03-22 13:22:52}
\pmmodified{2013-03-22 13:22:52}
\pmowner{Thomas Heye}{1234}
\pmmodifier{Thomas Heye}{1234}
\pmtitle{factorial modulo prime powers}
\pmrecord{7}{33915}
\pmprivacy{1}
\pmauthor{Thomas Heye}{1234}
\pmtype{Result}
\pmcomment{trigger rebuild}
\pmclassification{msc}{11A07}

% this is the default PlanetMath preamble.  as your knowledge
% of TeX increases, you will probably want to edit this, but
% it should be fine as is for beginners.

% almost certainly you want these
\usepackage{amssymb}
\usepackage{amsmath}
\usepackage{amsfonts}

% used for TeXing text within eps files
%\usepackage{psfrag}
% need this for including graphics (\includegraphics)
%\usepackage{graphicx}
% for neatly defining theorems and propositions
\usepackage{amsthm}
% making logically defined graphics
%%%\usepackage{xypic}

% there are many more packages, add them here as you need them

% define commands here
\newcommand{\pfac}[1]{\left(#1!\right)_p}
\begin{document}
For $n \in \mathbb{N}$ and any prime number $p$, $\pfac{n}$ is the product of numbers $1 \le m \le n$, where $p \not\vert m$.

For natural numbers $n, s$ and a given prime number $p$, we have the congruence
\begin{displaymath} \frac{n!}{p^{\sum\limits_{i=0}^d \left\lfloor
\frac{n}{p^i}\right\rfloor}} \equiv\prod\limits_{i=0}^d\left((\pm
1)^{\left\lfloor\frac{n}{p^s+i}\right\rfloor} \pfac{N_i}\right) \pmod{p^s},
\end{displaymath}
where $N_i$ is the least non-negative residue of $\left\lfloor \frac{n}{p^i}\right\rfloor
\pmod{p^s}$. Here $d+1$ denotes the number of digits in the representation
of $n$ in base $p$. More precisely, $\pm 1$ is $-1$ unless $p=2, s \ge 3$.

\begin{proof}
Let $i \ge 0$. Then the set of numbers between 1 and
$\left\lfloor \frac{n}{p^i}\right\rfloor$ is
\[\left\{kp, k \ge 1, k \le \left\lfloor\frac{n}{p^{i+1}}\right\rfloor\right\}.\]
This is true for every integer $i$ with $p^{i+1} \le n$. So we have
\begin{equation}
\label{F1}
\frac{\left\lfloor \frac{n}{p^i}\right\rfloor!}{\left\lfloor \frac{n}{p^{i+1}}\right\rfloor
p^{\left\lfloor\frac{n}{p^{i+1}}\right\rfloor}}
=\pfac{\left\lfloor\frac{n}{p^i}\right\rfloor}.
\end{equation}
Multiplying all terms with $0 \le i \le d$, where $d$ is the largest power of
$p$ not greater than $n$, the statement follows from the generalization of
Anton's congruence.
\end{proof}
%%%%%
%%%%%
\end{document}
