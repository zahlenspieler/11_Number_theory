\documentclass[12pt]{article}
\usepackage{pmmeta}
\pmcanonicalname{MostSignificantDigit}
\pmcreated{2013-03-22 16:52:20}
\pmmodified{2013-03-22 16:52:20}
\pmowner{PrimeFan}{13766}
\pmmodifier{PrimeFan}{13766}
\pmtitle{most significant digit}
\pmrecord{4}{39122}
\pmprivacy{1}
\pmauthor{PrimeFan}{13766}
\pmtype{Definition}
\pmcomment{trigger rebuild}
\pmclassification{msc}{11A63}

% this is the default PlanetMath preamble.  as your knowledge
% of TeX increases, you will probably want to edit this, but
% it should be fine as is for beginners.

% almost certainly you want these
\usepackage{amssymb}
\usepackage{amsmath}
\usepackage{amsfonts}

% used for TeXing text within eps files
%\usepackage{psfrag}
% need this for including graphics (\includegraphics)
%\usepackage{graphicx}
% for neatly defining theorems and propositions
%\usepackage{amsthm}
% making logically defined graphics
%%%\usepackage{xypic}

% there are many more packages, add them here as you need them

% define commands here

\begin{document}
The {\em most significant digit} of a number $n$ written in a given positional base $b$ is the digit in the most significant place value, and has to be in the range $-1 < d_k < b$. In the case of an integer, the most significant digit is the $b^k$'s place value, where $k$ is the total number of digits, or $k = \lfloor log_{b} n \rfloor$.

In an array of digits $k$ long meant for mathematical manipulation, it might be convenient to index the least significant digit with index 1 or 0, and the more significant digits with larger integers. (This enables the calculation of the value of a given digit as $d_ib^i$  rather than $d_ib^{k - i}$.) For an array of digits meant for text string manipulation, however, the most significant digit might be placed at position 0 or 1 (for example, by Mathematica's IntegerDigits function).

In binary, the most significant digit is often called the {\em most significant bit}.
%%%%%
%%%%%
\end{document}
