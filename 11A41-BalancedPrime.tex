\documentclass[12pt]{article}
\usepackage{pmmeta}
\pmcanonicalname{BalancedPrime}
\pmcreated{2013-03-22 16:36:50}
\pmmodified{2013-03-22 16:36:50}
\pmowner{PrimeFan}{13766}
\pmmodifier{PrimeFan}{13766}
\pmtitle{balanced prime}
\pmrecord{4}{38811}
\pmprivacy{1}
\pmauthor{PrimeFan}{13766}
\pmtype{Definition}
\pmcomment{trigger rebuild}
\pmclassification{msc}{11A41}
\pmrelated{StrongPrime}
\pmrelated{WeakPrime}

% this is the default PlanetMath preamble.  as your knowledge
% of TeX increases, you will probably want to edit this, but
% it should be fine as is for beginners.

% almost certainly you want these
\usepackage{amssymb}
\usepackage{amsmath}
\usepackage{amsfonts}

% used for TeXing text within eps files
%\usepackage{psfrag}
% need this for including graphics (\includegraphics)
%\usepackage{graphicx}
% for neatly defining theorems and propositions
%\usepackage{amsthm}
% making logically defined graphics
%%%\usepackage{xypic}

% there are many more packages, add them here as you need them

% define commands here

\begin{document}
If for the given $n$th prime $p_n$ the equality $$p_n = {1 \over 3} \sum_{i = n - 1}^{n + 1} p_i$$ is true, then $p_n$ is said to be a {\em balanced prime}. That is, the arithmetic mean of the given prime, the prime immediately below and the one immediately above, is equal to the middle prime. The first few are 5, 53, 157, 173, 211, 257, 263, 373, listed in A006562 of Sloane's OEIS. As of 2006, the largest known balanced prime is $197418203 \times 2^{25000} - 1$, discovered by David Broadhurst and François Morain using FastECPP and PrimeForm.
%%%%%
%%%%%
\end{document}
