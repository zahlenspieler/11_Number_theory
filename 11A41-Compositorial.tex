\documentclass[12pt]{article}
\usepackage{pmmeta}
\pmcanonicalname{Compositorial}
\pmcreated{2013-03-22 19:20:40}
\pmmodified{2013-03-22 19:20:40}
\pmowner{Kausthub}{26471}
\pmmodifier{Kausthub}{26471}
\pmtitle{Compositorial}
\pmrecord{4}{42293}
\pmprivacy{1}
\pmauthor{Kausthub}{26471}
\pmtype{Definition}
\pmcomment{trigger rebuild}
\pmclassification{msc}{11A41}

\endmetadata

% this is the default PlanetMath preamble.  as your knowledge
% of TeX increases, you will probably want to edit this, but
% it should be fine as is for beginners.

% almost certainly you want these
\usepackage{amssymb}
\usepackage{amsmath}
\usepackage{amsfonts}

% used for TeXing text within eps files
%\usepackage{psfrag}
% need this for including graphics (\includegraphics)
%\usepackage{graphicx}
% for neatly defining theorems and propositions
%\usepackage{amsthm}
% making logically defined graphics
%%%\usepackage{xypic}

% there are many more packages, add them here as you need them

% define commands here

\begin{document}
The compositorial of a number n equals the product of composite numbers less than or equal to n. For example, the compositorial of 6 is 24 since the composite numbers less than or equal to 6 are 4 and 6 and 4 * 6 = 24. Similarly the distinct compositorial numbers are 1, 4, 24, 192, 1728, 17280, 207360 and so on (Sequence A036991 of OEIS).

The numbers whose compositorial minus 1 is a prime number are 1, 2, 3 , 4, 5, 8, 14, 20, 26 ... (Sequence A140294 of OEIS). The numbers whose compositorial plus 1 is a prime number are 4, 5, 6, 7, 8, 16, 17, 21, 24 ... (Sequence 140293 of OEIS).
%%%%%
%%%%%
\end{document}
