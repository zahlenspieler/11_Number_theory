\documentclass[12pt]{article}
\usepackage{pmmeta}
\pmcanonicalname{DivisionFactorial}
\pmcreated{2013-10-30 22:15:55}
\pmmodified{2013-10-30 22:15:55}
\pmowner{Paulo Fernandesky}{1000738}
\pmmodifier{Paulo Fernandesky}{1000738}
\pmtitle{ Division Factorial}
\pmrecord{7}{87602}
\pmprivacy{1}
\pmauthor{Paulo Fernandesky}{1000738}
\pmtype{Theorem}
\pmclassification{msc}{11A51 Division Factorial}

\endmetadata

% this is the default PlanetMath preamble.  as your knowledge
% of TeX increases, you will probably want to edit this, but
% it should be fine as is for beginners.

% almost certainly you want these
\usepackage{amssymb}
\usepackage{amsmath}
\usepackage{amsfonts}

% need this for including graphics (\includegraphics)
\usepackage{graphicx}
% for neatly defining theorems and propositions
\usepackage{amsthm}

% making logically defined graphics
%\usepackage{xypic}
% used for TeXing text within eps files
%\usepackage{psfrag}

% there are many more packages, add them here as you need them

% define commands here

\begin{document}
\documentclass{article}
\usepackage{amssymb}
\usepackage{amsmath}
\usepackage{amsfonts}
\usepackage{psfrag}
\usepackage{graphicx}
\usepackage{amsthm}
\usepackage{xypic}
\begin{document}

1    \section{ Division Factorial}
The prospect of the {Division Factorial} follows the same concept of multiplicative factor. The same operations known since the twelfth century. The same logical properties that would be studied later in 1808 by the mathematician {Christian Kramp} which introduced the notation {n!}. Stirling also presented a formula approach to these results.
However, it is urgent to talk about {divisive factor}. As you think of the beginning, not the multiplicative inversion extensive. It is elementary that the corporeality of Factor Theory arises simply. About natural number {n} which is the product of all positive integers less than or equal to {n}.
Demonstrate exemplary one truth and accuracy, which supports more than a demonstration \cite {PF}. Likewise, presents the multiplicative factor, which has similar properties thereto. Therefore, evidenced in the following series:
\[
1 : 2 : 3 : 4 : 5 = 8333...\times10^3
\]
applied by;
\[
\frac{n+1}{(n+1)!}= if $ {n} $is growing,
\]
Otherwise, decreasing the following sequence:
\[
5 : 4 : 3 : 2 : 1 = 208333...\times10^1
\]
It is applied by:

\[
\frac{n}{(n-1)!}=if  $ {n} $ is decreasing,
\]


\begin{thebibliography}{99}

\bibitem{PF} FERNÂNDESKY, PAULO., 2013. \"Os Teoremas. N. (Ed.). Escrytos of the distribution. Lisboa, 2013. p.22-38. (Statistics, Kindle, Artigo.
\end{thebibliography}

\end{document}

\end{document}
