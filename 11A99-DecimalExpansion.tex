\documentclass[12pt]{article}
\usepackage{pmmeta}
\pmcanonicalname{DecimalExpansion}
\pmcreated{2013-03-22 15:04:01}
\pmmodified{2013-03-22 15:04:01}
\pmowner{pahio}{2872}
\pmmodifier{pahio}{2872}
\pmtitle{decimal expansion}
\pmrecord{28}{36787}
\pmprivacy{1}
\pmauthor{pahio}{2872}
\pmtype{Result}
\pmcomment{trigger rebuild}
\pmclassification{msc}{11A99}
\pmsynonym{decadic expansion}{DecimalExpansion}
\pmsynonym{digital expansion}{DecimalExpansion}
\pmrelated{Base3}
\pmrelated{ExistenceAndUniquenessOfDecimalExpansion}
\pmrelated{0ne1AsRealNumbers}
\pmrelated{Factoradic}
\pmrelated{FactorialBase}
\pmrelated{MatheRealism}
\pmrelated{JosephLiouville}
\pmdefines{decadic}
\pmdefines{hexadic}
\pmdefines{dyadic}

% this is the default PlanetMath preamble.  as your knowledge
% of TeX increases, you will probably want to edit this, but
% it should be fine as is for beginners.

% almost certainly you want these
\usepackage{amssymb}
\usepackage{amsmath}
\usepackage{amsfonts}

% used for TeXing text within eps files
%\usepackage{psfrag}
% need this for including graphics (\includegraphics)
%\usepackage{graphicx}
% for neatly defining theorems and propositions
%\usepackage{amsthm}
% making logically defined graphics
%%%\usepackage{xypic}

% there are many more packages, add them here as you need them

% define commands here
\begin{document}
Every rational number $\frac{m}{n}$, where $m$ and $n$ are positive integers, has an endless \PMlinkescapetext{periodic} {\em decimal expansion} (or {\em decadic expansion} --- according to Greek).\, The decimal expansion of $\frac{m}{n}$ means the series \PMlinkescapetext{presentation}
\begin{align}
  \nu.\nu_1\nu_2\nu_3\ldots = \nu+10^{-1}\nu_1+10^{-2}\nu_2+10^{-3}\nu_3+\ldots
\end{align}
where \,$\nu = \lfloor\frac{m}{n}\rfloor$\, is the \PMlinkname{integer part}{Floor} of $\frac{m}{n}$ and the integers $\nu_j$ are the remainders of\, $\lfloor 10^j\cdot\!\frac{m}{n}\rfloor$\, when divided by 10; thus\, $0 \leqq \nu_j < 10$.

We may suppose that $m$ and $n$ are coprime (if necessary, reduce the fraction).\, Then the \PMlinkescapetext{length $l$ of the period} depends only on the denominator $n$.\, In the case that\, $\gcd(n,\,10) = 1$,\, the \PMlinkescapetext{period length} is the least positive integer $l$ such that $10^l\equiv 1 \pmod{n}$ (the \PMlinkescapetext{period length} does not change if we multiply the fraction by a suitable power of 10 and then reduce all prime factors of 10 from the denominator).\, In every case, the \PMlinkescapetext{period length} is a factor of the number $\varphi(n)$, where $\varphi$ is Euler's totient function.

\textbf{Examples}

$\frac{1}{8} = 0.125000\ldots = 0.124999\ldots$ (one-digit \PMlinkescapetext{periods};  N.B. two possibilities),

$\frac{1}{12} = 0.08333\ldots$ (one-digit per.),

$\frac{1}{37} = 0.'027'027'027'\ldots$ (three-digit per.),

$\frac{1}{82} = 0.0'12195'12195'12195'\ldots$ (five-digit per.), 

$\frac{1}{25351} = 0.000039446\ldots$ (hundred-digit per.)

The tail of infinitely many 0's (as in 0.125000\ldots) is of course usually not written out.\, Such a tail is possible only when $n$ has no other prime factors except prime factors of the base of the digit system in question. 

If the tails of 0's are not accepted, then the digital expansion of every positive rational is unique (then e.g. 0.124999\ldots\, is the only \PMlinkescapetext{expansion} for $\frac{1}{8}$ in the decimal system).

Completely similar results concern the digital expansions in any other positional digit system.\, Let the fraction $\frac{1}{31}$ be an example ($\varphi(31) = 30$); its \PMlinkescapetext{presentation} is

in the decadic (decimal) digit system\, $\frac{1}{31} = 0.'032258064516129'\ldots_{\mathrm{ten}}$ \quad(15-digit per.),

in the hexadic (senary) digit system\, $\frac{1}{51} = 0.'010545'010545'010545'\ldots_{\mathrm{six}}$ \quad(6-digit per.),

in the dyadic (\PMlinkescapetext{binary}) digit system\, $\frac{1}{11111} = 0.000010000100001\ldots_{\mathrm{two}}$ \quad(5-digit per.).

\textbf{Note.} \,Also any irrational number has a unique decimal expansion, but it is non-periodic; for example  \PMlinkname{Liouville's number}{ExampleOfTranscendentalNumber}
    $$0.110001\,000000\,000000\,000001\,000000\,\ldots$$
which is transcendental over $\mathbb{Q}$.
%%%%%
%%%%%
\end{document}
