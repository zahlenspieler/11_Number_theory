\documentclass[12pt]{article}
\usepackage{pmmeta}
\pmcanonicalname{ThainesTheorem}
\pmcreated{2013-03-22 14:12:34}
\pmmodified{2013-03-22 14:12:34}
\pmowner{mathcam}{2727}
\pmmodifier{mathcam}{2727}
\pmtitle{Thaine's theorem}
\pmrecord{6}{35643}
\pmprivacy{1}
\pmauthor{mathcam}{2727}
\pmtype{Theorem}
\pmcomment{trigger rebuild}
\pmclassification{msc}{11R29}

\endmetadata

% this is the default PlanetMath preamble.  as your knowledge
% of TeX increases, you will probably want to edit this, but
% it should be fine as is for beginners.

% almost certainly you want these
\usepackage{amssymb}
\usepackage{amsmath}
\usepackage{amsfonts}
\usepackage{amsthm}

% used for TeXing text within eps files
%\usepackage{psfrag}
% need this for including graphics (\includegraphics)
%\usepackage{graphicx}
% for neatly defining theorems and propositions
%\usepackage{amsthm}
% making logically defined graphics
%%%\usepackage{xypic}

% there are many more packages, add them here as you need them

% define commands here
\newtheorem{theorem}{Theorem}
\newcommand{\mc}{\mathcal}
\newcommand{\mb}{\mathbb}
\newcommand{\mf}{\mathfrak}
\newcommand{\ol}{\overline}
\newcommand{\ra}{\rightarrow}
\newcommand{\la}{\leftarrow}
\newcommand{\La}{\Leftarrow}
\newcommand{\Ra}{\Rightarrow}
\newcommand{\nor}{\vartriangleleft}
\newcommand{\Gal}{\text{Gal}}
\newcommand{\GL}{\text{GL}}
\newcommand{\Z}{\mb{Z}}
\newcommand{\R}{\mb{R}}
\newcommand{\Q}{\mb{Q}}
\newcommand{\C}{\mb{C}}
\newcommand{\<}{\langle}
\renewcommand{\>}{\rangle}
\begin{document}
Let $F/\Q$ be a totally real abelian number field.  By the Kronecker-Weber theorem, there exists an $m$ such that $F\subset \Q(\zeta_m)$. Let $G$ be the Galois group of the extension $F/Q$.  Let $\mc{O}_F^\times$ denote the group of units in the ring of integers of $F$, let $C$ be the subgroup of $\mc{O}_F^\times$ consisting of units $\eta$ of the form 
\begin{align*}
\eta=\pm N_{\Q(\zeta_m)/F}\left(\prod_{a\in(\Z/m\Z)^\times}(\zeta_m^a-1)^{b_a}\right)
\end{align*}
for some collection of $b_a\in\Z$.  (Here, $N$ denotes the norm operator and $\zeta_m$ is a primitive $m$-th root of unity.)  Finally, let $A$ denote the ideal class group of $F$.

\begin{theorem}[Thaine]
Suppose $p$ is a rational prime not dividing the degree $[F:\Q]$ and suppose $\theta\in\Z[G]$ annihilates the Sylow $p$-subgroup of $E/C'$.  Then $2\theta$ annihilates the Sylow $p$-subgroup of $A$.
\end{theorem}

This is one of the most sophisticated results concerning the annihilators of an ideal class group.  It is a direct, but more complicated, version of Stickelberger's theorem, applied to totally real fields (for which Stickelberger's theorem gives no information).
%%%%%
%%%%%
\end{document}
