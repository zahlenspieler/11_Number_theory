\documentclass[12pt]{article}
\usepackage{pmmeta}
\pmcanonicalname{AndricasConjecture}
\pmcreated{2013-03-22 16:41:22}
\pmmodified{2013-03-22 16:41:22}
\pmowner{PrimeFan}{13766}
\pmmodifier{PrimeFan}{13766}
\pmtitle{Andrica's conjecture}
\pmrecord{8}{38901}
\pmprivacy{1}
\pmauthor{PrimeFan}{13766}
\pmtype{Conjecture}
\pmcomment{trigger rebuild}
\pmclassification{msc}{11A41}
\pmsynonym{Andrica conjecture}{AndricasConjecture}

\endmetadata

% this is the default PlanetMath preamble.  as your knowledge
% of TeX increases, you will probably want to edit this, but
% it should be fine as is for beginners.

% almost certainly you want these
\usepackage{amssymb}
\usepackage{amsmath}
\usepackage{amsfonts}

% used for TeXing text within eps files
%\usepackage{psfrag}
% need this for including graphics (\includegraphics)
%\usepackage{graphicx}
% for neatly defining theorems and propositions
%\usepackage{amsthm}
% making logically defined graphics
%%%\usepackage{xypic}

% there are many more packages, add them here as you need them

% define commands here

\begin{document}
Conjecture (Dorin Andrica). Given the $n$th prime $p_n$, it is always the case that $1 > \sqrt{p_{n + 1}} - \sqrt{p_n}$.

This conjecture remains unproven as of 2007. The conjecture has been checked up to $n = 10^5$ by computers.

The largest known difference of square roots of consecutive primes happens for the small $n = 4$, being approximately 0.67087347929081. The difference of the square roots of the primes $10^{314} - 1929$ and $10^{314} + 2318$ (which cap a prime gap of 4247 consecutive composite numbers discovered in 1992 by Baugh \& O'Hara) is a very small number which is obviously much smaller than 1.

\begin{thebibliography}{1}
\bibitem{db} D. Baugh \& F. O'Hara, ``Large Prime Gaps'' {\it J. Recr. Math.} {\bf 24} (1992): 186 - 187.
\end{thebibliography}
%%%%%
%%%%%
\end{document}
