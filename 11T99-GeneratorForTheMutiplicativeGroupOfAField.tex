\documentclass[12pt]{article}
\usepackage{pmmeta}
\pmcanonicalname{GeneratorForTheMutiplicativeGroupOfAField}
\pmcreated{2013-03-22 16:53:17}
\pmmodified{2013-03-22 16:53:17}
\pmowner{polarbear}{3475}
\pmmodifier{polarbear}{3475}
\pmtitle{generator for the mutiplicative group of a field}
\pmrecord{16}{39143}
\pmprivacy{1}
\pmauthor{polarbear}{3475}
\pmtype{Result}
\pmcomment{trigger rebuild}
\pmclassification{msc}{11T99}
\pmclassification{msc}{12E20}

% this is the default PlanetMath preamble.  as your knowledge
% of TeX increases, you will probably want to edit this, but
% it should be fine as is for beginners.

% almost certainly you want these
\usepackage{amssymb}
\usepackage{amsmath}
\usepackage{amsfonts}

% used for TeXing text within eps files
%\usepackage{psfrag}
% need this for including graphics (\includegraphics)
%\usepackage{graphicx}
% for neatly defining theorems and propositions
%\usepackage{amsthm}
% making logically defined graphics
%%%\usepackage{xypic}

% there are many more packages, add them here as you need them

\newtheorem{proposition}{Proposition}


\begin{document}
\begin{proposition} The multiplicative group $K^{*}$ of a finite field $K$ is cyclic.\end{proposition}
Theorem 3.1 in the \PMlinkname{finite fields}{FiniteField} entry proves
this proposition along with a more general result:\newline
\begin{proposition} If for every natural number $d$, the equation $x^d = 1$ has at most $d$ solutions in a finite group $G$ then $G$ is cyclic. Equivalently, for any positive divisor $d$ of $|G|$. \end{proposition} This last proposition implies that every finite subgroup of the multiplicative group of a field (finite or not) is cyclic.
\newline
  We will give an alternative constructive proof of Proposition 1:\newline
 We first factorize $q-1 = \prod_{i=1}^n p_i^{e_i}$. There exists an element $y_i$ in $K^{*}$ such that $y_i$ is not root of $x^{(q-1)/p_i} -1$, since the polynomial has degree less than $q-1$ . Let $z_i = y_i^{(q-1)/p_i^{e_i}}$. We note that $z_i$ has order $p_i^{e_i}$. In fact $z_i^{{p_i}^{e_i}} = 1$ and $z_i^{{p_i}^{e_i-1}} = y_i^{(q-1)/p_i} \ne 1$.\newline
 Finally we choose the element $z = \prod_{i=1}^n z_i$. By the Theorem 1 \PMlinkname{here}{OrderOfElementsInFiniteGroups}, we obtain that the order of $z$ is $q-1$ i.e. $z$ is a generator of the cyclic group $K^{*}$.
%%%%%
%%%%%
\end{document}
