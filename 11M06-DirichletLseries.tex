\documentclass[12pt]{article}
\usepackage{pmmeta}
\pmcanonicalname{DirichletLseries}
\pmcreated{2013-03-22 13:22:28}
\pmmodified{2013-03-22 13:22:28}
\pmowner{mathcam}{2727}
\pmmodifier{mathcam}{2727}
\pmtitle{Dirichlet L-series}
\pmrecord{14}{33905}
\pmprivacy{1}
\pmauthor{mathcam}{2727}
\pmtype{Definition}
\pmcomment{trigger rebuild}
\pmclassification{msc}{11M06}
\pmsynonym{Dirichlet L-function}{DirichletLseries}
\pmrelated{LSeriesOfAnEllipticCurve}
\pmrelated{DirichletSeries}

% this is the default PlanetMath preamble.  as your knowledge
% of TeX increases, you will probably want to edit this, but
% it should be fine as is for beginners.

% almost certainly you want these
\usepackage{amssymb}
\usepackage{amsmath}
\usepackage{amsfonts}

% used for TeXing text within eps files
%\usepackage{psfrag}
% need this for including graphics (\includegraphics)
%\usepackage{graphicx}
% for neatly defining theorems and propositions
%\usepackage{amsthm}
% making logically defined graphics
%%%\usepackage{xypic}

% there are many more packages, add them here as you need them

% define commands here
\begin{document}
The \emph{Dirichlet L-series} associated to a Dirichlet character $\chi$ is the series 
\begin{equation}
L(\chi,s)=\sum_{n=1}^\infty \frac{\chi(n)}{n^s}.
\end{equation}
It converges absolutely and uniformly in the domain $\Re(s) \geq 1 + \delta$ for any positive $\delta$, and admits the Euler product identity
\begin{equation}
L(\chi,s)=\prod_p \frac{1}{1-\chi(p)p^{-s}}
\end{equation}
where the product is over all primes $p$, by virtue of the multiplicativity of $\chi$.  In the case where $\chi=\chi_0$ is the trivial character mod m, we have
\begin{equation}
L(\chi_0,s)=\zeta(s)\prod_{p|m} (1-p^{-s}),
\end{equation}
where $\zeta(s)$ is the Riemann Zeta function.  If $\chi$ is non-primitive, and $C_\chi$ is the conductor of $\chi$, we have
\begin{equation}
L(\chi,s)=L(\chi\prime,s)\prod_{p|m\atop p\nmid C_\chi}(1-\chi(p)p^{-s}),
\end{equation}
where $\chi\prime$ is the primitive character which induces $\chi$.  For non-trivial, primitive characters $\chi$ mod m, $L(\chi,s)$ admits an analytic continuation to all of $\mathbb{C}$ and satsfies the symmetric functional equation
\begin{equation}
L(\chi,s)\left(\frac{m}{\pi}\right)^{s/2}\Gamma\left(\frac{s+e_\chi}{2}\right)=\frac{g_1(\chi)}{i^{e_\chi}\sqrt{m}}L(\chi^{-1},1-s)\left(\frac{m}{\pi}\right)^{\frac{1-s}{2}}\Gamma\left(\frac{1-s+e_\chi}{2}\right).
\end{equation}
Here, $e_\chi\in \{0,1\}$ is defined by $\chi(-1)=(-1)^{e_\chi}\chi(1)$, $\Gamma$ is the gamma function, and $g_1(\chi)$ is a Gauss sum.
(3),(4), and (5) combined show that $L(\chi,s)$ admits a meromorphic continuation to all of $\mathbb{C}$ for all Dirichlet characters $\chi$, and an analytic one for non-trivial $\chi$.
Again assuming that $\chi$ is non-trivial and primitive character mod m, if $k$ is a positive integer, we have
\begin{equation}
L(\chi,1-k)=-\frac{B_{k,\chi}}{k},
\end{equation}
where $B_{k,\chi}$ is a generalized Bernoulli number.  By (5), taking into account the poles of $\Gamma$, we get for $k$ positive, $k \equiv e_\chi$ mod 2,
\begin{equation}
L(\chi,k)=(-1)^{1+\frac{k-e_\chi}{2}}\frac{g_1(\chi)}{2i^{e_\chi}}\left(\frac{2\pi}{m}\right)^k\frac{B_{k,\chi^{-1}}}{k!}.
\end{equation}
This series was first investigated by Dirichlet (for whom they were named), who used the non-vanishing of $L(\chi,1)$ for non-trivial $\chi$ to prove his famous Dirichlet's theorem on primes in arithmetic progression.  This is probably the first instance of using complex analysis to prove a purely number theoretic result.
%%%%%
%%%%%
\end{document}
