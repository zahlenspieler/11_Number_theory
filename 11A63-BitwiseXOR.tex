\documentclass[12pt]{article}
\usepackage{pmmeta}
\pmcanonicalname{BitwiseXOR}
\pmcreated{2013-03-22 17:02:52}
\pmmodified{2013-03-22 17:02:52}
\pmowner{PrimeFan}{13766}
\pmmodifier{PrimeFan}{13766}
\pmtitle{bitwise XOR}
\pmrecord{4}{39337}
\pmprivacy{1}
\pmauthor{PrimeFan}{13766}
\pmtype{Definition}
\pmcomment{trigger rebuild}
\pmclassification{msc}{11A63}
\pmrelated{BitwiseAND}
\pmrelated{BitwiseOR}
\pmrelated{BitwiseNOT}

\endmetadata

% this is the default PlanetMath preamble.  as your knowledge
% of TeX increases, you will probably want to edit this, but
% it should be fine as is for beginners.

% almost certainly you want these
\usepackage{amssymb}
\usepackage{amsmath}
\usepackage{amsfonts}

% used for TeXing text within eps files
%\usepackage{psfrag}
% need this for including graphics (\includegraphics)
%\usepackage{graphicx}
% for neatly defining theorems and propositions
%\usepackage{amsthm}
% making logically defined graphics
%%%\usepackage{xypic}

% there are many more packages, add them here as you need them

% define commands here

\begin{document}
{\em Bitwise XOR} or {\em bitwise exclusive OR} is a bit-level operation on two binary values which indicates which bits are set in only one value. For each position $i$, if the bit $d_i$ in one value is 1 and the other is 0, then $d_i$ of the result is 1, otherwise it's 0. If both input $d_i$ are 1, the output $d_i$ is 0. then For example, given 50 and 163 in two unsigned bytes, a bitwise XOR returns 145.

\begin{tabular}{|r|c|c|c|c|c|c|c|c|}
    & 0 & 0 & 1 & 1 & 0 & 0 & 1 & 0 \\
XOR & 1 & 0 & 1 & 0 & 0 & 0 & 1 & 1 \\
  = & 1 & 0 & 0 & 1 & 0 & 0 & 0 & 1 \\
\end{tabular}

Given a Mersenne number of the form $2^k - 1$ (where $k$ is the bit size of the data type in use, e.g., 8 for bytes, 16 for words, 32 for double words, etc.) and some smaller integer $n$, XORing that Mersenne number with $n$ has the same effect as performing a bitwise NOT on $n$. Or, given $m$ and $n$, XORing them has the same effect as $m - n$ (subject to some caveats about the sign bit, or if the values are unsigned, the effect is then the same as $|m - n|$). XORing two values that are the same gives 0.
%%%%%
%%%%%
\end{document}
