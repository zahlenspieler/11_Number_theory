\documentclass[12pt]{article}
\usepackage{pmmeta}
\pmcanonicalname{displaystylesumnleXYOmeganOleftfracxlogXy12yrightFor1leY2}
\pmcreated{2013-03-22 16:09:15}
\pmmodified{2013-03-22 16:09:15}
\pmowner{Wkbj79}{1863}
\pmmodifier{Wkbj79}{1863}
\pmtitle{$\displaystyle \sum_{n \le x} y^{\Omega(n)}=O\left( \frac{x(\log x)^{y-1}}{2-y} \right)$ for $1 \le y<2$}
\pmrecord{18}{38235}
\pmprivacy{1}
\pmauthor{Wkbj79}{1863}
\pmtype{Theorem}
\pmcomment{trigger rebuild}
\pmclassification{msc}{11N37}
\pmrelated{AsymptoticEstimate}
\pmrelated{ConvolutionMethod}
\pmrelated{DisplaystyleXlog2xOleftsum_nLeX2OmeganRight}
\pmrelated{DisplaystyleSum_nLeXYomeganO_yxlogXy1ForYGe0}
\pmrelated{DisplaystyleSum_nLeXTaunaO_axlogX2a1ForAGe0}
\pmrelated{2omeganLeTaunLe2Omegan}

\endmetadata

\usepackage{amssymb}
\usepackage{amsmath}
\usepackage{amsfonts}

\usepackage{psfrag}
\usepackage{graphicx}
\usepackage{amsthm}
%%\usepackage{xypic}

\newtheorem*{thm*}{Theorem}
\begin{document}
Within this entry, $\Omega$ refers to the \PMlinkname{number of (nondistinct) prime factors function}{NumberOfNondistinctPrimeFactorsFunction}, $\mu$ refers to the M\"{o}bius function, $\log$ refers to the natural logarithm, $p$ refers to a prime, and $d$, $k$, $m$, and $n$ refer to positive integers.

\begin{thm*}
For $1 \le y<2$, $\displaystyle \sum_{n \le x} y^{\Omega(n)}=O\left( \frac{x(\log x)^{y-1}}{2-y} \right)$.
\end{thm*}

\begin{proof}
Let $g$ be a function such that $y^{\Omega}=1*g$.  Then $g$ is multiplicative and $g=\mu*y^{\Omega}$.  Thus:

\begin{center}
\begin{tabular}{ll}
$\displaystyle \sum_{n \le x} y^{\Omega(n)}$ & $\displaystyle =\sum_{d \le x} \sum_{m \le \frac{x}{d}} g(d)$ by the convolution method \\
& $\displaystyle =O\left( \sum_{d \le x} g(d) \cdot \frac{x}{d} \right)$ \\
& $\displaystyle =O\left( x \prod_{p \le x} \left( 1+\sum_{k} \frac{g(p^k)}{p^k} \right) \right)$ \\
& $\displaystyle =O\left( x \prod_{p \le x} \left( 1+\sum_{k} \frac{\mu(1)y^k+\mu(p)y^{k-1}}{p^k} \right) \right)$ \\
& $\displaystyle =O\left( x \prod_{p \le x} \left( 1+\frac{y-1}{p} \sum_{k} \left( \frac{y}{p} \right)^{k-1} \right) \right)$ \\
& $\displaystyle =O\left( x \prod_{p \le x} \left( 1+\frac{y-1}{p} \cdot \frac{1}{1-\frac{y}{p}} \right) \right)$ \\
& $\displaystyle =O\left( x \prod_{p \le x} \left( 1+\frac{y-1}{p-y} \right) \right)$ \\
& $\displaystyle =O\left( x \left(1+\frac{y-1}{2-y} \right) \prod_{3 \le p \le x} \left( 1+\frac{y-1}{p-y} \right) \right)$ \\
& $\displaystyle =O\left( x \left(\frac{2-y+y-1}{2-y} \right) \prod_{3 \le p \le x} \exp \left(\frac{y-1}{p-y} \right) \right)$ \\
& $\displaystyle =O\left( \frac{x}{2-y} \left( \exp \left(\sum_{3 \le p \le x} \frac{y-1}{p-y} \right) \right) \right)$ \\
& $\displaystyle =O\left( \frac{x}{2-y} \left( \exp \left( \sum_{3 \le p \le x} \frac{1}{p-y} \right) \right)^{y-1} \right)$ \\
& $\displaystyle =O\left( \frac{x}{2-y} \left( \exp \left( \log\log x+O(1) \right) \right)^{y-1} \right)$ \\
& $\displaystyle =O\left( \frac{x}{2-y} \left(ce^{\log\log x} \right)^{y-1} \right)$ for some $c>0$ \\
& $\displaystyle =O\left( \frac{x}{2-y} \left(\max\{1,c\}\log x\right)^{y-1} \right)$ \\
& $\displaystyle =O\left( \frac{x}{2-y} \left(\max\{1,c\}\right)^{2-1} \left( \log x \right)^{y-1} \right)$ \\
& $\displaystyle =O\left( \frac{x \left( \log x \right)^{y-1}}{2-y} \right)$. \end{tabular}
\end{center}
\end{proof}

Note that a \PMlinkescapetext{similar} result for $y=2$ (and therefore for $y \ge 2$), such as $\displaystyle \sum_{n \le x} 2^{\Omega(n)}=O(x\log x)$, is unobtainable, as evidenced by \PMlinkname{this theorem}{DisplaystyleXlog2xOleftsum_nLeX2OmeganRight}.  On the other hand, the asymptotic estimates $\displaystyle \sum_{n \le x} 2^{\omega(n)}=O(x\log x)$ and $\displaystyle \sum_{n \le x} \tau(n)=O(x\log x)$ are true.
%%%%%
%%%%%
\end{document}
