\documentclass[12pt]{article}
\usepackage{pmmeta}
\pmcanonicalname{PrimeConstant}
\pmcreated{2013-03-22 15:02:17}
\pmmodified{2013-03-22 15:02:17}
\pmowner{mathcam}{2727}
\pmmodifier{mathcam}{2727}
\pmtitle{prime constant}
\pmrecord{12}{36749}
\pmprivacy{1}
\pmauthor{mathcam}{2727}
\pmtype{Definition}
\pmcomment{trigger rebuild}
\pmclassification{msc}{11A41}

\endmetadata

% this is the default PlanetMath preamble.  as your knowledge
% of TeX increases, you will probably want to edit this, but
% it should be fine as is for beginners.

% almost certainly you want these
\usepackage{amssymb}
\usepackage{amsmath}
\usepackage{amsfonts}

% used for TeXing text within eps files
%\usepackage{psfrag}
% need this for including graphics (\includegraphics)
%\usepackage{graphicx}
% for neatly defining theorems and propositions
%\usepackage{amsthm}
% making logically defined graphics
%%%\usepackage{xypic}

% there are many more packages, add them here as you need them

% define commands here
\begin{document}
The number $\rho$ defined by 
\[ \rho = \sum_{p} \frac{1}{2^p} \]
is known as the {\sl prime constant}. It is simply the number whose binary expansion corresponds to the characteristic function of the set of prime numbers.    That is, its $n$th binary digit is $1$ if $n$ is prime and $0$ if $n$ is composite.

The beginning of the decimal expansion of $\rho$ is:
\[ \rho = 0.414682509851111660248109622... \]

The number $\rho$ is easily shown to be irrational.  To see why, suppose it were rational.  Denote the $k$th digit of the binary expansion of $\rho$ by $r_k$.  Then, since $\rho$ is assumed rational, there must exist $N$, $k$ positive integers such that $r_n=r_{n+ik}$ for all $n>N$ and all $i \in \mathbb{N}$.

Since there are an infinite number of primes, we may choose a prime $p>N$.  By definition we see that $r_p=1$.  As noted, we have $r_p=r_{p+ik}$ for all $i \in \mathbb{N}$.  Now consider the case $i=p$.  We have $r_{p+i \cdot k}=r_{p+p \cdot k}=r_{p(k+1)}=0$, since $p(k+1)$ is composite because $k+1 \geq 2$.  Since $r_p \neq r_{p(k+1)}$ we see that $\rho$ is irrational.

The partial continued fractions of the prime constant can be found \PMlinkexternal{here}{http://www.research.att.com/cgi-bin/access.cgi/as/njas/sequences/eisA.cgi?Anum=A051007}.
%%%%%
%%%%%
\end{document}
