\documentclass[12pt]{article}
\usepackage{pmmeta}
\pmcanonicalname{IdealDecompositionInDedekindDomain}
\pmcreated{2015-05-05 19:05:43}
\pmmodified{2015-05-05 19:05:43}
\pmowner{pahio}{2872}
\pmmodifier{pahio}{2872}
\pmtitle{ideal decomposition in Dedekind domain}
\pmrecord{13}{37219}
\pmprivacy{1}
\pmauthor{pahio}{2872}
\pmtype{Topic}
\pmcomment{trigger rebuild}
\pmclassification{msc}{11R37}
\pmclassification{msc}{11R04}
\pmrelated{ProductOfFinitelyGeneratedIdeals}
\pmrelated{PolynomialCongruence}
\pmrelated{CancellationIdeal}
\pmrelated{DivisibilityInRings}
\pmrelated{IdealsOfADiscreteValuationRingArePowersOfItsMaximalIdeal}
\pmrelated{DivisorTheory}
\pmrelated{GreatestCommonDivisorOfSeveralIntegers}

% this is the default PlanetMath preamble.  as your knowledge
% of TeX increases, you will probably want to edit this, but
% it should be fine as is for beginners.

% almost certainly you want these
\usepackage{amssymb}
\usepackage{amsmath}
\usepackage{amsfonts}

% used for TeXing text within eps files
%\usepackage{psfrag}
% need this for including graphics (\includegraphics)
%\usepackage{graphicx}
% for neatly defining theorems and propositions
 \usepackage{amsthm}
% making logically defined graphics
%%%\usepackage{xypic}

% there are many more packages, add them here as you need them

% define commands here

\theoremstyle{definition}
\newtheorem*{thmplain}{Theorem}
\begin{document}
According to the entry ``\PMlinkname{fractional ideal}{FractionalIdeal}'', we can \PMlinkescapetext{state} that in a Dedekind domain $R$, each non-zero integral ideal $\mathfrak{a}$ may be written as a product of finitely many prime ideals $\mathfrak{p}_i$ of $R$, 
 $$\mathfrak{a} = \mathfrak{p}_1\mathfrak{p}_2...\mathfrak{p}_k.$$
The product decomposition is unique up to the order of the factors.\, This is stated and proved, with more general assumptions, in the entry ``\PMlinkname{prime ideal factorisation is unique}{PrimeIdealFactorizationIsUnique}''.\\

\textbf{Corollary.}\, If $\alpha_1$, $\alpha_2$, ..., $\alpha_m$ are elements of a Dedekind domain $R$ and $n$ is a positive integer, then one has
\begin{align}
  (\alpha_1,\,\alpha_2,\,...,\,\alpha_m)^n = 
  (\alpha_1^n,\,\alpha_2^n,\,...,\,\alpha_m^n)
\end{align}
for the ideals of $R$.\\

This corollary may be proven by induction on the number $m$ of the \PMlinkescapetext{generators (not on the exponent} $n$).
%%%%%
%%%%%
\end{document}
