\documentclass[12pt]{article}
\usepackage{pmmeta}
\pmcanonicalname{SufficientConditionOfIdenticalCongruence}
\pmcreated{2013-03-22 18:56:03}
\pmmodified{2013-03-22 18:56:03}
\pmowner{pahio}{2872}
\pmmodifier{pahio}{2872}
\pmtitle{sufficient condition of identical congruence}
\pmrecord{8}{41788}
\pmprivacy{1}
\pmauthor{pahio}{2872}
\pmtype{Theorem}
\pmcomment{trigger rebuild}
\pmclassification{msc}{11C08}
\pmclassification{msc}{11A07}
\pmrelated{Sufficient}
\pmrelated{CongruenceOfArbitraryDegree}
\pmrelated{PolynomialCongruence}

% this is the default PlanetMath preamble.  as your knowledge
% of TeX increases, you will probably want to edit this, but
% it should be fine as is for beginners.

% almost certainly you want these
\usepackage{amssymb}
\usepackage{amsmath}
\usepackage{amsfonts}

% used for TeXing text within eps files
%\usepackage{psfrag}
% need this for including graphics (\includegraphics)
%\usepackage{graphicx}
% for neatly defining theorems and propositions
 \usepackage{amsthm}
% making logically defined graphics
%%%\usepackage{xypic}

% there are many more packages, add them here as you need them

% define commands here

\theoremstyle{definition}
\newtheorem*{thmplain}{Theorem}

\begin{document}
\textbf{Theorem.}\, Let\, $f(X) := a_nX^n+\ldots+a_1X+a_0$\, be a polynomial in $X$ with integer coefficients $a_i$ and $m$ a positive integer.\, If the congruence
\begin{align}
f(x) \;\equiv\; 0 \pmod{m}
\end{align}
is satisfied by $m$ successive integers $x$, then it is satisfied by all integers $x$, in other words it is an identical congruence.\\


\emph{Proof.}\, There is an integer $x_0$ such that (1) is satisfied by
$$x \;:=\; x_0\!+\!1,\,x_0\!+\!2,\,\ldots,\,x_0\!+\!m.$$
But these values form a complete residue system modulo $m$.\, Thus, if $x$ is an arbitrary integer, one has
$$x \;\equiv\; x_0\!+\!r \pmod{m} \quad\mbox{where}\;\; 1\leqq r \leqq m.$$
This implies
$$a_ix^i \;\equiv\; a_i(x_0\!+\!r)^i \pmod{m} \quad\mbox{for}\;\; i = 0,\,1,\,\ldots,\,n$$
and consequently
$$\underbrace{\sum_{i=0}^na_ix^i}_{f(x)} \;\equiv\; \sum_{i=0}^na_i(x_0\!+\!r)^i \;=\; f(x_0\!+\!r) \;\equiv\; 0 
\pmod{m}.$$
Accordingly, (1) is true for any integer $x$, Q.E.D.\\

\textbf{Note.}\, Though the congruence (1) is identical, it need not be a question of a formal congruence
\begin{align}
f(X) \;\underline{\equiv}\; 0 \pmod{m},
\end{align}
i.e. all coefficients $a_i$ need not be congruent to 0 modulo $m$.



%%%%%
%%%%%
\end{document}
