\documentclass[12pt]{article}
\usepackage{pmmeta}
\pmcanonicalname{GelfondsConstant}
\pmcreated{2013-03-22 18:54:34}
\pmmodified{2013-03-22 18:54:34}
\pmowner{PrimeFan}{13766}
\pmmodifier{PrimeFan}{13766}
\pmtitle{Gelfond's constant}
\pmrecord{4}{41758}
\pmprivacy{1}
\pmauthor{PrimeFan}{13766}
\pmtype{Definition}
\pmcomment{trigger rebuild}
\pmclassification{msc}{11J81}

\endmetadata

% this is the default PlanetMath preamble.  as your knowledge
% of TeX increases, you will probably want to edit this, but
% it should be fine as is for beginners.

% almost certainly you want these
\usepackage{amssymb}
\usepackage{amsmath}
\usepackage{amsfonts}

% used for TeXing text within eps files
%\usepackage{psfrag}
% need this for including graphics (\includegraphics)
%\usepackage{graphicx}
% for neatly defining theorems and propositions
%\usepackage{amsthm}
% making logically defined graphics
%%%\usepackage{xypic}

% there are many more packages, add them here as you need them

% define commands here

\begin{document}
{\em Gelfond's constant} $e^\pi$ was one of the first numbers to be proven to be transcendental by appying Gelfond's theorem. However, naming the constant after Gelfond comes from Eric Weisstein, with many people preferring to refer to it simply as $e$ to the power of $\pi$.

Its value in base 10 is approximately 23.1406926327792690057290863679485473802661062426. Its continued fraction representation is neither terminating nor periodic, and begins $$23 + \frac{1}{7 + \frac{1}{9 + \frac{1}{3 + \, \cdots}}}$$
%%%%%
%%%%%
\end{document}
