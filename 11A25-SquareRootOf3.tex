\documentclass[12pt]{article}
\usepackage{pmmeta}
\pmcanonicalname{SquareRootOf3}
\pmcreated{2013-03-22 17:29:46}
\pmmodified{2013-03-22 17:29:46}
\pmowner{PrimeFan}{13766}
\pmmodifier{PrimeFan}{13766}
\pmtitle{square root of 3}
\pmrecord{6}{39884}
\pmprivacy{1}
\pmauthor{PrimeFan}{13766}
\pmtype{Definition}
\pmcomment{trigger rebuild}
\pmclassification{msc}{11A25}
\pmsynonym{Theodorus's constant}{SquareRootOf3}
\pmsynonym{Theodorus' constant}{SquareRootOf3}

\endmetadata

% this is the default PlanetMath preamble.  as your knowledge
% of TeX increases, you will probably want to edit this, but
% it should be fine as is for beginners.

% almost certainly you want these
\usepackage{amssymb}
\usepackage{amsmath}
\usepackage{amsfonts}

% used for TeXing text within eps files
%\usepackage{psfrag}
% need this for including graphics (\includegraphics)
%\usepackage{graphicx}
% for neatly defining theorems and propositions
%\usepackage{amsthm}
% making logically defined graphics
%%%\usepackage{xypic}

% there are many more packages, add them here as you need them

% define commands here

\begin{document}
The \emph{square root of 3}, also known as \emph{Theodorus's constant}, is the number the square of which is equal to the integer 3. It is an irrational number, one of the first few to have been proved irrational. Theodorus of Cyrene proved that the square roots of the integers 3, 5 to 8, 10 to 15 and 17 are all irrational. The decimal expansion of $\sqrt{3}$ is 1.7320508075688772935... (sequence \PMlinkexternal{A002194}{http://www.research.att.com/~njas/sequences/A002194} in Sloane's OEIS). Its simple continued fraction is $$1 + \frac{1}{1 + \frac{1}{2 + \frac{1}{1 + \frac{1}{2 + \ldots}}}},$$ repeating 1 and 2 periodically (Sloane's \PMlinkexternal{A040001}{http://www.research.att.com/~njas/sequences/A040001}).

Given a unit cube, the diagonal from the vertex joining three sides to the other vertex joining the three other sides is $\sqrt{3}$. Given a unit hexagon, the distance from one side to the parallel opposite side is $\sqrt{3}$. More generally, the ratio of the length of a side of a hexagon to the distance from that side to the opposing parallel side is $1 : \sqrt{3}$, and the same ratio applies to the length of the side of a cube to the diagonal of that cube.

\begin{thebibliography}{1}
\bibitem{mj} M. F. Jones, ``22900D approximations to the square roots of the primes less than 100'', {\it Math. Comp} {\bf 22} (1968): 234 - 235. 
\bibitem{hu} H. S. Uhler, ``Approximations exceeding 1300 decimals for $\sqrt{3}$, $\frac{1}{\sqrt{3}}$, $\sin(\frac{\pi}{3})$ and distribution of digits in them'' {\it Proc. Nat. Acad. Sci. U. S. A.} {\bf 37} (1951): 443 - 447.
\end{thebibliography}
%%%%%
%%%%%
\end{document}
