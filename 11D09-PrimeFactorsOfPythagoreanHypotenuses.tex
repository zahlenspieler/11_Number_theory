\documentclass[12pt]{article}
\usepackage{pmmeta}
\pmcanonicalname{PrimeFactorsOfPythagoreanHypotenuses}
\pmcreated{2014-01-31 10:48:20}
\pmmodified{2014-01-31 10:48:20}
\pmowner{pahio}{2872}
\pmmodifier{pahio}{2872}
\pmtitle{prime factors of Pythagorean hypotenuses}
\pmrecord{12}{87626}
\pmprivacy{1}
\pmauthor{pahio}{2872}
\pmtype{Theorem}
\pmclassification{msc}{11D09}
\pmclassification{msc}{51M05}
\pmclassification{msc}{11E25}

% this is the default PlanetMath preamble.  as your knowledge
% of TeX increases, you will probably want to edit this, but
% it should be fine as is for beginners.

% almost certainly you want these
\usepackage{amssymb}
\usepackage{amsmath}
\usepackage{amsfonts}

% need this for including graphics (\includegraphics)
\usepackage{graphicx}
% for neatly defining theorems and propositions
\usepackage{amsthm}

% making logically defined graphics
%\usepackage{xypic}
% used for TeXing text within eps files
%\usepackage{psfrag}

% there are many more packages, add them here as you need them

% define commands here

\begin{document}
The possible hypotenuses of the 
\PMlinkname{Pythagorean triangles}{PythagoreanTriangle} form the
infinite sequence
$$5,\,10,\,13,\,15,\,17,\,20,\,25,\,26,\,29,\,30,\,34,\,35,\,
37,\,39,\,40,\,41,\,45,\,\ldots$$
the mark of which is 
\PMlinkexternal{A009003}{http://oeis.org/search?q=a009003&language=english&go=Search} in the corpus of the integer 
sequences of \PMlinkexternal{OEIS}{http://oeis.org/}.\, This sequence has the subsequence A002144
$$5,\,13,\,17,\,29,\,37,\,41,\,53,\,61,\,73,\,89,\,97,\,
101,\,109,\,113,\,137,\,\ldots$$
of the odd Pythagorean primes.\\


Generally, the hypotenuse $c$ of a Pythagorean triangle 
(Pythagorean triple) may be characterised by being the 
contraharmonic mean 
$$c \;=\; \frac{u^2\!+\!v^2}{u\!+\!v}$$
of some two different integers $u$ and $v$ (as has been shown 
in the parent entry), but also by the

\textbf{Theorem.}\, A positive integer $c$ is the length of the 
hypotenuse of a Pythagorean triangle if and only if at least 
one of the prime factors of $c$ is of the form $4n\!+\!1$.\\


\textbf{Lemma 1.}\, All prime factors of the hypotenuse $c$ in 
a primitive Pythagorean triple are of the form $4n\!+\!1$.  

This can be proved here by making the antithesis that there 
exists a prime $4n\!-\!1$ dividing $c$.\, Then also
$$4n\!-\!1 \mid c^2 \;=\; a^2\!+\!b^2 
\;=\; (a\!+\!ib)(a\!-\!ib)$$
where $a$ and $b$ are the catheti in the triple.\, But $4n\!-\!1$ 
is prime also in the ring $\mathbb{Z}[i]$ of the Gaussian 
integers, whence it must divide at least one of the factors 
$a\!+\!ib$ and $a\!-\!ib$.\, Apparently, that would imply that 
$4n\!-\!1$ divides both $a$ and $b$.\, This means that the 
triple $(a, b, c)$ were not primitive, whence 
the antithesis is wrong and the lemma true.\, $\Box$\\

Also the converse is true in the following form:

\textbf{Lemma 2.}\, If all prime factors of a positive integer 
$c$ are of the form $4n\!+\!1$, then $c$ is the hypotenuse in 
a Pythagorean triple.\, (Especially, any prime $4n\!+\!1$ is 
found as the hypotenuse in a primitive Pythagorean triple.)

{\it Proof}.\, For proving this, one can start from Fermat{'}s 
theorem, by which the prime numbers of such form are sums of 
two squares (see the 
\PMlinkexternal{Theorem on sums of two squares by Fermat}
{http://en.wikipedia.org/wiki/Proofs_of_Fermat's_theorem_on_sums_of_two_squares}).\, 
Since the sums of two squares form a set closed under 
multiplication, now also the product $c$ is a sum of two 
squares, and similarly is $c^2$, i.e. $c$ is the hypotenuse 
in a Pythagorean triple.\, $\Box$\\

{\it Proof of the Theorem.}\, Suppose that $c$ is the 
hypotenuse of a Pythagorean triple $(a, b, c)$; dividing the 
triple members by their greatest common factor we get a 
primitive triple $(a'\!, b'\!, c')$ where\; $c'\mid\;c$.\; By 
Lemma 1, the prime factors of $c'$, being also prime factors of 
$c$, are of the form $4n\!+\!1$.

On the contrary, let's suppose that a prime factor $p$ of\, 
$c = pd$\, is of the form $4n\!+\!1$.\, Then Lemma 2 guarantees 
a Pythagorean triple $(r, s, p)$, whence also $(rd, sd, c)$ is 
Pythagorean and $c$ thus a hypotenuse.\, $\Box$\\




\end{document}
