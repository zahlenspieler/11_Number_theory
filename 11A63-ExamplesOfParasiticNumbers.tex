\documentclass[12pt]{article}
\usepackage{pmmeta}
\pmcanonicalname{ExamplesOfParasiticNumbers}
\pmcreated{2013-03-22 16:22:38}
\pmmodified{2013-03-22 16:22:38}
\pmowner{PrimeFan}{13766}
\pmmodifier{PrimeFan}{13766}
\pmtitle{examples of parasitic numbers}
\pmrecord{4}{38520}
\pmprivacy{1}
\pmauthor{PrimeFan}{13766}
\pmtype{Example}
\pmcomment{trigger rebuild}
\pmclassification{msc}{11A63}

\endmetadata

% this is the default PlanetMath preamble.  as your knowledge
% of TeX increases, you will probably want to edit this, but
% it should be fine as is for beginners.

% almost certainly you want these
\usepackage{amssymb}
\usepackage{amsmath}
\usepackage{amsfonts}

% used for TeXing text within eps files
%\usepackage{psfrag}
% need this for including graphics (\includegraphics)
%\usepackage{graphicx}
% for neatly defining theorems and propositions
%\usepackage{amsthm}
% making logically defined graphics
%%%\usepackage{xypic}

% there are many more packages, add them here as you need them

% define commands here

\begin{document}
When using some scientific calculators in the course of searching for parasitic numbers, it might be useful to work in base 10 first, as such calculators might be incapable of dealing with values other than integer values in the other bases.

In base 10, the following numbers are $d_1$-parasitic (where $d_1$ is the least significant digit): 1, 11, 111, 1111, 11111, 102564, 111111,  1111111, 11111111, 111111111, 1111111111, 11111111111, 102564102564, 111111111111,
1012658227848, 1111111111111, 11111111111111, 111111111111111, 1111111111111111, 11111111111111111, 102564102564102564, 105263157894736842,
111111111111111111, 1111111111111111111, 11111111111111111111,
111111111111111111111, 1014492753623188405797, 1111111111111111111111,
11111111111111111111111, 102564102564102564102564, 111111111111111111111111,
1111111111111111111111111, 11111111111111111111111111, 
111111111111111111111111111, 1034482758620689655172413793,
1111111111111111111111111111, etc. In all honesty, however, the repunits are parasitic in a trivial sense because of the multiplicative identity. That's why they're omitted from the list of $d_1$-parasitic numbers in sequence A081463 of Sloane's OEIS.

Sequence A092697 lists the smallest $d_i$-parasitic numbers for $1 < i < 10$: 1, 102564, 1012658227848, 105263157894736842, 1014492753623188405797, 1034482758620689655172413793, 102040816326530612244897959183673469387755, 10112359550561797752808988764044943820224719, 1016949152542372881355932203389830508474576271186440677966.



%%%%%
%%%%%
\end{document}
