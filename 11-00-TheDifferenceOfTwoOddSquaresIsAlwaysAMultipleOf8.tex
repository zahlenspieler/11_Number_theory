\documentclass[12pt]{article}
\usepackage{pmmeta}
\pmcanonicalname{TheDifferenceOfTwoOddSquaresIsAlwaysAMultipleOf8}
\pmcreated{2013-03-22 18:47:06}
\pmmodified{2013-03-22 18:47:06}
\pmowner{PrimeFan}{13766}
\pmmodifier{PrimeFan}{13766}
\pmtitle{the difference of two odd squares is always a multiple of 8}
\pmrecord{5}{41580}
\pmprivacy{1}
\pmauthor{PrimeFan}{13766}
\pmtype{Theorem}
\pmcomment{trigger rebuild}
\pmclassification{msc}{11-00}
\pmclassification{msc}{30-00}
\pmclassification{msc}{26-00}

% this is the default PlanetMath preamble.  as your knowledge
% of TeX increases, you will probably want to edit this, but
% it should be fine as is for beginners.

% almost certainly you want these
\usepackage{amssymb}
\usepackage{amsmath}
\usepackage{amsfonts}

% used for TeXing text within eps files
%\usepackage{psfrag}
% need this for including graphics (\includegraphics)
%\usepackage{graphicx}

% for neatly defining theorems and propositions
\usepackage{amsthm}

% making logically defined graphics
%%%\usepackage{xypic}

% there are many more packages, add them here as you need them

% define commands here

\begin{document}
{\bf Theorem.} Given any odd real integers $m$ and $n$, the difference of their squares is always a multiple of 8. That is, $8 |  (m^2 - n^2)$.

For example, given $m = 9$ and $n$ varied from $-1$ to 13 in steps of 2, calculating $m^2 - n^2$, we get the sequence 80, 80, 72, 56, 32, 0, $-40$, $-88$.

Proving this theorem was one of the problems in the 1851 math exam Lewis Carroll took. The difficulty of this problem is one of perception: so often a simple-sounding theorem in number theory turns out to have a rather complicated proof. But for this simple theorem, the proof turns out to be quite simple, almost bordering on triviality.

\begin{proof}
Since it was stipulated that $m$ and $n$ are both odd, we can rewrite them as $m = 2a + 1$ and $n = 2b + 1$. The square of $m$ is then $m^2 = (2a + 1)(2a + 1) = 4a^2 + 2a + 2a + 1$, which simplifies to $4a^2 + 4a + 1$. Likewise, $n^2 = 4b^2 + 4b + 1$. Their difference is then $(4a^2 + 4a + 1) - (4b^2 + 4b + 1) = (4a^2 + 4a) - (4b^2 + 4b)$. This can't be reduced to fewer terms. However, either side can be rewritten so as to not explicitly use squaring: $4a^2 + 4a = 4a(a + 1)$. This reveals that either the left-hand or right-hand side of our subtraction is a number that is four times a pronic number, that is, a number of the form $a(a + 1)$. All pronic numbers are even, but at this point we can't distinguish between singly even numbers and doubly even numbers. But, as it turns out, each pronic number is twice a triangular number, which is an integer. So $4a(a + 1)$ is eight times some triangular number $T_c$, and $4b^2 + 4b = 8T_d$. We rewrite our subtraction yet again: $m^2 - n^2 = 8T_c - 8T_d$. By redistributing, we get $m^2 - n^2 = 8(T_c - T_d)$, proving the theorem.
\end{proof}

The exam paper which Lewis Carroll turned in has not survived, so we don't know in what way he solved this particular problem or any of the other problems on that test. But we do know that he passed the test with flying colors.

% \begin{thebibliography}{1}
% \bibitem{rw} Robin Wilson, {\it Lewis Carroll in numberland: his fantastical mathematical logical life, an agony in eight fits}. New York: W.W. Norton (2008) 52 - 55
% \end{thebibliography}
%%%%%
%%%%%
\end{document}
