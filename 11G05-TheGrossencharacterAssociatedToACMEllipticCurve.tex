\documentclass[12pt]{article}
\usepackage{pmmeta}
\pmcanonicalname{TheGrossencharacterAssociatedToACMEllipticCurve}
\pmcreated{2013-03-22 15:45:29}
\pmmodified{2013-03-22 15:45:29}
\pmowner{alozano}{2414}
\pmmodifier{alozano}{2414}
\pmtitle{the Gr\"ossencharacter associated to a CM elliptic curve}
\pmrecord{4}{37712}
\pmprivacy{1}
\pmauthor{alozano}{2414}
\pmtype{Definition}
\pmcomment{trigger rebuild}
\pmclassification{msc}{11G05}
\pmrelated{Grossencharacter}
\pmrelated{EllipticCurve}
\pmdefines{grossencharacter associated to an elliptic curve}

\endmetadata

% this is the default PlanetMath preamble.  as your knowledge
% of TeX increases, you will probably want to edit this, but
% it should be fine as is for beginners.

% almost certainly you want these
\usepackage{amssymb}
\usepackage{amsmath}
\usepackage{amsthm}
\usepackage{amsfonts}

% used for TeXing text within eps files
%\usepackage{psfrag}
% need this for including graphics (\includegraphics)
%\usepackage{graphicx}
% for neatly defining theorems and propositions
%\usepackage{amsthm}
% making logically defined graphics
%%\usepackage{xypic}

% there are many more packages, add them here as you need them

% define commands here

\newtheorem*{thm}{Theorem}
\newtheorem{defn}{Definition}
\newtheorem{prop}{Proposition}
\newtheorem{lemma}{Lemma}
\newtheorem{cor}{Corollary}

\theoremstyle{definition}
\newtheorem{exa}{Example}

% Some sets
\newcommand{\Nats}{\mathbb{N}}
\newcommand{\Ints}{\mathbb{Z}}
\newcommand{\Reals}{\mathbb{R}}
\newcommand{\Complex}{\mathbb{C}}
\newcommand{\Rats}{\mathbb{Q}}
\newcommand{\Gal}{\operatorname{Gal}}
\newcommand{\Cl}{\operatorname{Cl}}
\newcommand{\RInts}{\mathcal{O}_K}
\begin{document}
Let $K$ be a quadratic imaginary field and let $A/F$ be an
elliptic curve defined over a number field $F$ (such that
$K\subset F$), with complex multiplication by $K$. The so-called
`Main Theorem of Complex Multiplication' (\cite{sil2}, Thm. 8.2)
implies the existence of a Gr\"ossencharacter of $F$,
$\psi_{A/F}:\mathcal{A}_F^\ast\to \Complex^\ast$ associated to the
curve $A/F$ satisfying several interesting properties which we
collect in the following statement.

\begin{thm}[\cite{sil2}, Thm. 9.1, Prop. 10.4, Cor. 10.4.1]
\label{grossen} Let $\wp$ be a prime of $F$ of good reduction for
$A/F$, i.e. the reduction $\widetilde{A}/F$ of $A$ modulo $\wp$ is
smooth. There exists a Gr\"ossencharacter of $F$,
$\psi_{A/F}:\mathcal{A}_F^\ast \to \Complex^\ast$, such that:
\begin{enumerate}
\item $\psi_{A/F}$ is unramified at a prime $\mathfrak{Q}$ of $F$
if and only if $A/F$ has good reduction at $\mathfrak{Q}$;

\item $\psi_{A/F}(\wp)$ belongs to $\RInts$, thus multiplication
by $[\psi_{A/F}(\wp)]$ is a well defined endomorphism of $A/F$.
Moreover $N_\Rats^F(\wp)=N_\Rats^K(\psi_{A/F}(\wp))$;

\item the following diagram is commutative
\begin{center}
$\xymatrix{ 
A \ar@{->}[d] \ar@{->}[r]^{[\psi_{A/F}(\wp)]} & A \ar@{->}[d] &\\
\widetilde{A} \ar@{->}[r]^{\phi_\wp} & \widetilde{A} &
}$
\end{center}
where $\phi_\wp:\widetilde{A}\to\widetilde{A}$ be the
$N_\Rats^F(\wp)$-power Frobenius map and the vertical maps are
reduction mod $\wp$;

\item let $|\widetilde{A}(\mathcal{O}_F/\wp)|$ be the number of
points in $\widetilde{A}$ over the finite field
$\mathcal{O}_F/\wp$ and put $a_\wp=N_{\Rats}^F(\wp) +1 -
|\widetilde{A}(\mathcal{O}_F/\wp)|$. Then
$$a_{\wp}=\psi_{A/F}(\wp)+\overline{\psi_{A/F}(\wp)}=2\cdot \Re
(\psi_{A/F}(\wp)).$$

\item (due to Deuring) let $L(A/F,s)$ be the $L$-function
associated to the elliptic curve $A/F$. If $K\subset F$ then
$L(A/F,s)=L(\psi_{A/F},s)L(\overline{\psi_{A/F}},s)$. If
$K\nsubseteq F$, and $F'=FK$, then $L(E/F,s)=L(\psi_{A/F'},s)$.
\end{enumerate}
\end{thm}

In particular, if $h_K=1$ then $A$ is defined over $K$ (actually,
it may be defined over $\Rats$), $\psi_{A/K}(\wp)$ is a generator
of $\wp$ (by part (2), and the explicit generator can be pinned
down using part (4)). Thus, if $e$ is the number of roots of unity
in $K$, then $\psi_{A/K}^k(\wp)=\alpha^k$ where $\alpha$ is {\it
any} generator of $\wp$. Also, by part (5),
$L(A/\Rats,s)=L(\psi_{A/K},s)$.

\begin{thebibliography}{00}
\bibitem{silverman} J. H. Silverman, {\em The Arithmetic of
Elliptic Curves}, Springer-Verlag, New York.
\bibitem{sil2} J. H. Silverman, {\em Advanced Topics in
the Arithmetic of Elliptic Curves}. Springer-Verlag, New York,
1994.


\end{thebibliography}
%%%%%
%%%%%
\end{document}
