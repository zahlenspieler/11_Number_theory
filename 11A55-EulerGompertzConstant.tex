\documentclass[12pt]{article}
\usepackage{pmmeta}
\pmcanonicalname{EulerGompertzConstant}
\pmcreated{2013-03-22 18:49:06}
\pmmodified{2013-03-22 18:49:06}
\pmowner{PrimeFan}{13766}
\pmmodifier{PrimeFan}{13766}
\pmtitle{Euler-Gompertz constant}
\pmrecord{7}{41620}
\pmprivacy{1}
\pmauthor{PrimeFan}{13766}
\pmtype{Definition}
\pmcomment{trigger rebuild}
\pmclassification{msc}{11A55}
\pmsynonym{Gompertz constant}{EulerGompertzConstant}

% this is the default PlanetMath preamble.  as your knowledge
% of TeX increases, you will probably want to edit this, but
% it should be fine as is for beginners.

% almost certainly you want these
\usepackage{amssymb}
\usepackage{amsmath}
\usepackage{amsfonts}

% used for TeXing text within eps files
%\usepackage{psfrag}
% need this for including graphics (\includegraphics)
%\usepackage{graphicx}
% for neatly defining theorems and propositions
%\usepackage{amsthm}
% making logically defined graphics
%%%\usepackage{xypic}

% there are many more packages, add them here as you need them

% define commands here

\begin{document}
The {\em Euler-Gompertz constant} is the value of the continued fraction $$C_2 = 0 + \frac{1}{1 + \frac{1}{1 + \frac{1}{1 + \frac{2}{1 + \ldots}}}},$$ 

in which after three appearances of 1 in the numerator position, follow the integers from 2 forward each given twice consecutively; the value of this constant is approximately 0.596347362323194074341078499369279376074... Finch gives two formulas for this constant: 

$$C_2 = -e\textrm{Ei}(-1) = \int_1^\infty \frac{\textrm{exp}(1 - x)}{x} dx,$$ with $e$ being the natural log base and $\textrm{Ei}$ being the exponential integral. 

The constant can also be expressed as a formula involving an infinite sum: $$e \left( \left( \sum_{i = 1}^\infty \frac{(-1)^{i - 1}}{i!i} \right) - \gamma \right),$$ with $\gamma$ being the Euler-Mascheroni constant.

\begin{thebibliography}{1}
\bibitem{sr} Steven R. Finch, {\it Mathematical Constants}. Cambridge: Cambridge University Press (2003): 424
\end{thebibliography}
%%%%%
%%%%%
\end{document}
