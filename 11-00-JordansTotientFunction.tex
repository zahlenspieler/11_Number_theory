\documentclass[12pt]{article}
\usepackage{pmmeta}
\pmcanonicalname{JordansTotientFunction}
\pmcreated{2013-03-22 11:42:21}
\pmmodified{2013-03-22 11:42:21}
\pmowner{akrowne}{2}
\pmmodifier{akrowne}{2}
\pmtitle{Jordan's totient function}
\pmrecord{39}{30010}
\pmprivacy{1}
\pmauthor{akrowne}{2}
\pmtype{Definition}
\pmcomment{trigger rebuild}
\pmclassification{msc}{11-00}
\pmclassification{msc}{46M15}
\pmclassification{msc}{18C15}

\endmetadata

\usepackage{amssymb}
\usepackage{amsmath}
\usepackage{amsfonts}
\usepackage{graphicx}
%%%%%%%%%%%%%%%%%%%%%%%%%%%\usepackage{xypic}
\begin{document}
Let $p$ be a prime, $k$ and $n$ natural numbers.  Then

$$
  J_k(n) = n^k \prod_{p|n} ( 1-p^{-k})
$$

where the product is over divisors of $n$. 

This is a generalization of Euler's Totient Function.
%%%% 1
%%%%%
%%%%%
%%%%%
%%%%%
%%%%%
%%%%%
%%%%%
%%%%%
%%%%%
%%%%%
%%%%%
%%%%%
%%%%%
%%%%%
%%%%%
%%%%%
%%%%%
%%%%%
%%%%%
%%%%%
%%%%%
%%%%%
%%%%%
%%%%%
%%%%%
%%%%%
%%%%%
\end{document}
