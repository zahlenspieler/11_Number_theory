\documentclass[12pt]{article}
\usepackage{pmmeta}
\pmcanonicalname{TwinPrimeConjecture}
\pmcreated{2013-03-22 13:21:32}
\pmmodified{2013-03-22 13:21:32}
\pmowner{alozano}{2414}
\pmmodifier{alozano}{2414}
\pmtitle{twin prime conjecture}
\pmrecord{11}{33883}
\pmprivacy{1}
\pmauthor{alozano}{2414}
\pmtype{Conjecture}
\pmcomment{trigger rebuild}
\pmclassification{msc}{11N05}
\pmrelated{PrimeTriplesConjecture}
\pmrelated{BrunsConstant}
\pmdefines{twin prime constant}
\pmdefines{twin primes}

% this is the default PlanetMath preamble.  as your knowledge
% of TeX increases, you will probably want to edit this, but
% it should be fine as is for beginners.

% almost certainly you want these
\usepackage{amssymb}
\usepackage{amsmath}
\usepackage{amsfonts}

% used for TeXing text within eps files
%\usepackage{psfrag}
% need this for including graphics (\includegraphics)
%\usepackage{graphicx}
% for neatly defining theorems and propositions
%\usepackage{amsthm}
% making logically defined graphics
%%%\usepackage{xypic}

% there are many more packages, add them here as you need them

% define commands here
\begin{document}
Two consecutive odd numbers which are both prime are called {\it twin primes}, e.g. 5 and 7, or 41 and 43, or 1,000,000,000,061 and 1,000,000,000,063. But is there an infinite number of twin primes ?

In 1849 de Polignac made the more general conjecture that for every natural number $n$, there are infinitely many prime pairs which have a distance of $2n$. The case $n=1$ is the twin prime conjecture.

In 1940, Erd\H{o}s showed that there is a constant $c<1$ and infinitely many primes $p$ such that $q-p<c \ln{p}$ where $q$ denotes the next prime after $p$.
This result was improved in 1986 by Maier; he showed that a constant $c < 0.25$ can be used. The constant $c$ is called the \emph{twin prime constant.}

In 1966, Chen Jingrun showed that there are infinitely many primes $p$ such that $p+2$ is either a prime or a semiprime.
%%%%%
%%%%%
\end{document}
