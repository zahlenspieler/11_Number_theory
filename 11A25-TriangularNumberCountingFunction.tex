\documentclass[12pt]{article}
\usepackage{pmmeta}
\pmcanonicalname{TriangularNumberCountingFunction}
\pmcreated{2013-03-22 18:03:03}
\pmmodified{2013-03-22 18:03:03}
\pmowner{PrimeFan}{13766}
\pmmodifier{PrimeFan}{13766}
\pmtitle{triangular number counting function}
\pmrecord{5}{40576}
\pmprivacy{1}
\pmauthor{PrimeFan}{13766}
\pmtype{Definition}
\pmcomment{trigger rebuild}
\pmclassification{msc}{11A25}

\endmetadata

% this is the default PlanetMath preamble.  as your knowledge
% of TeX increases, you will probably want to edit this, but
% it should be fine as is for beginners.

% almost certainly you want these
\usepackage{amssymb}
\usepackage{amsmath}
\usepackage{amsfonts}

% used for TeXing text within eps files
%\usepackage{psfrag}
% need this for including graphics (\includegraphics)
%\usepackage{graphicx}
% for neatly defining theorems and propositions
%\usepackage{amsthm}
% making logically defined graphics
%%%\usepackage{xypic}

% there are many more packages, add them here as you need them

% define commands here

\begin{document}
For a given nonnegative number $x$, the {\em triangular number counting function} counts how many triangular numbers are not greater than $x$. The formula is simple: $$\lfloor \frac{-1 + \sqrt{8x + 1}}{2} \rfloor ,$$ in sharp contrast to the lack of a formula for the prime counting function $\pi(x)$. If one accepts 0 as a triangular number, the formula easily accommodates this by the mere addition of 1 after flooring the fraction.

\begin{thebibliography}{1}
\bibitem{zs} Zhi-Wei Sun, ``On Sums of Primes and Triangular Numbers'' ArXiv preprint, 10 April (2008): 1
\end{thebibliography}
%%%%%
%%%%%
\end{document}
